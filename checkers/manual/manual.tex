\documentclass[10pt]{article}
\usepackage{pslatex}
\usepackage{fullpage}
\usepackage{graphicx}
\usepackage{hevea}
\usepackage{listings}
\usepackage{url}
\usepackage{alltt}

\usepackage{relsize}
% \def\codesize{\smaller}
\def\codesize{\relax}           % for "pslatex"
%HEVEA \def\codesize{\relax}
\newcommand{\code}[1]{\ifmmode{\mbox{\codesize\ttfamily{#1}}}\else{\codesize\ttfamily #1}\fi}
% This can't handle a URL with an embedded "#" -- at least at UW CSE
\newcommand{\myurl}[1]{{\codesize\url{#1}}}
%HEVEA \def\myurl{\url}
\def\<#1>{\code{#1}}

\usepackage{fancyvrb}
%BEGIN LATEX
\RecustomVerbatimEnvironment{Verbatim}{Verbatim}{fontsize=\codesize}
%END LATEX

%HEVEA \footerfalse    % Disable hevea advertisement in footer

\newcommand{\htmlhr}{\relax}
%HEVEA \renewcommand{\htmlhr}{\@hr{}{}}

%HEVEA \newcommand{\discretionary}[3]{\relax}

%HEVEA \newstyle{.lstframe}{margin:auto;margin-bottom:2em}

% Left and right curly braces in tt font
\newcommand{\ttlcb}{\texttt{\char "7B}}
\newcommand{\ttrcb}{\texttt{\char "7D}}
\newcommand{\ttbs}{\texttt{\char "5C}}


\title{The Checker Framework Manual}
\author{% MIT Program Analysis Group \\
\url{http://types.cs.washington.edu/checker-framework/}}
\newcommand{\ReleaseInfo}{0.9.6 (29 Jul 2009)}
\date{Version \ReleaseInfo{}}

\begin{document}

\maketitle

% %BEGIN LATEX
% \tableofcontents
% %END LATEX

%BEGIN LATEX
  %% Bring items closer together in list environments
  % Prevent infinite loops
  \let\Itemize =\itemize
  \let\Enumerate =\enumerate
  \let\Description =\description
  % Zero the vertical spacing parameters
  \def\Nospacing{\itemsep=0pt\topsep=0pt\partopsep=0pt\parskip=0pt\parsep=0pt}
  % Redefine the environments in terms of the original values
  \renewenvironment{itemize}{\Itemize\Nospacing}{\endlist}
  \renewenvironment{enumerate}{\Enumerate\Nospacing}{\endlist}
  \renewenvironment{description}{\Description\Nospacing}{\endlist}

  % Add line between figure and text
  \makeatletter
  \def\topfigrule{\kern3\p@ \hrule \kern -3.4\p@} % the \hrule is .4pt high
  \def\botfigrule{\kern-3\p@ \hrule \kern 2.6\p@} % the \hrule is .4pt high
  \def\dblfigrule{\kern3\p@ \hrule \kern -3.4\p@} % the \hrule is .4pt high
  \makeatother
%END LATEX


% Reference to Checker Framework Javadoc for a class (not a method, etc.).
% Arg 1: directory under checkers/, including internal "/", but no leading
% or trailing "/".
% Arg 2: class name.
% In the printed version, only the base class name appears.
% In the HTML version, it's a link to the Javadoc.
\newcommand{\refclass}[2]{\ahref{doc/checkers/#1/#2.html}{\<#2>}}
% Reference to Checker Framework Javadoc for a method or field.).
% Arg 1: directory under checkers/, including internal "/", but no leading
% or trailing "/".
% Arg 2: class name.
% Arg 3: method name.
% Arg 4: fully-qualified arguments.  Example: "(T)"
% In the printed version, only "class.method" appears.
% In the HTML version, it's a link to the Javadoc.
\newcommand{\refmethod}[4]{\ahref{doc/checkers/#1/#2.html\##3#4}{\<#2.#3>}}
% Reference to Sun Javadoc.
% Arg 1: .html reference, without the .../api/ prefix
% Arg 2: What will appear in the formatted manual.
% Problem:  the "?is-external=true" must appear before any "#".  But why is
% it necessary at all?
% \newcommand{\sunjavadoc}[2]{\ahref{http://java.sun.com/javase/6/docs/api/#1?is-external=true}{\<#2>}}
\newcommand{\sunjavadoc}[2]{\ahref{http://java.sun.com/javase/6/docs/api/#1}{\<#2>}}

\noindent
\textbf{For the impatient:}
Section~\ref{installation} describes how to \textbf{install and use} pluggable
type-checkers.

\newcommand{\sectionpageref}[1]{Section~\ref{#1}, page~\pageref{#1}}
%HEVEA \renewcommand{\sectionpageref}[1]{Section~\ref{#1}}

%BEGIN LATEX
\medskip
%END LATEX

\noindent
You can also jump directly to the documentation for a particular checker:
\begin{itemize}
\item Nullness checker: \sectionpageref{nullness-checker}
\item Interning checker: \sectionpageref{interning-checker}
\item IGJ (immutability) checker: \sectionpageref{igj-checker}
\item Javari (immutability) checker: \sectionpageref{javari-checker}
\item Lock checker: \sectionpageref{lock-checker}
\end{itemize}


%HEVEA \setcounter{tocdepth}{2}
\tableofcontents
\newpage

\htmlhr
\chapter{Introduction\label{introduction}}

The Checker Framework enhances Java's type system to make it more powerful
and useful.
This lets software developers detect and
prevent errors in their Java programs.

A ``checker'' is a tool that warns you about certain errors or gives you a
guarantee that those errors do not occur.
The Checker Framework comes with checkers for specific types of errors:

\begin{enumerate}

\item
  \ahrefloc{nullness-checker}{Nullness Checker} for null pointer errors
  (see \chapterpageref{nullness-checker})
\item
  \ahrefloc{initialization-checker}{Initialization Checker} to ensure all
  fields are set in the constructor (see
  \chapterpageref{initialization-checker})
\item
  \ahrefloc{map-key-checker}{Map Key Checker} to track which values are
  keys in a map (see \chapterpageref{map-key-checker})
\item
  \ahrefloc{interning-checker}{Interning Checker} for errors in equality
  testing and interning (see \chapterpageref{interning-checker})
\item
  \ahrefloc{lock-checker}{Lock Checker} for concurrency and lock errors
  (see \chapterpageref{lock-checker})
\item
  \ahrefloc{fenum-checker}{Fake Enum Checker} to allow type-safe fake enum
  patterns (see \chapterpageref{fenum-checker})
\item
  \ahrefloc{tainting-checker}{Tainting Checker} for trust and security errors
  (see \chapterpageref{tainting-checker})
\item
  \ahrefloc{regex-checker}{Regex Checker} to prevent use of syntactically
  invalid regular expressions (see \chapterpageref{regex-checker})
\item
  \ahrefloc{formatter-checker}{Format String Checker} to ensure that format
  strings have the right number and type of \<\%> directives (see
  \chapterpageref{formatter-checker})
\item
  \ahrefloc{propkey-checker}{Property File Checker} to ensure that valid
  keys are used for property files and resource bundles (see
  \chapterpageref{propkey-checker})
\item
  \ahrefloc{i18n-checker}{Internationalization Checker} to
  ensure that code is properly internationalized (see
  \chapterpageref{i18n-checker}).
  %% Not interesting enough to go at the beginning of the introduction:
  % a checker for compiler message keys as used in the Checker Framework.
\item
  \ahrefloc{signature-checker}{Signature String Checker} to ensure that the
  string representation of a type is properly used, for example in
  \<Class.forName> (see \chapterpageref{signature-checker}).
\item
  \ahrefloc{units-checker}{Units Checker} to ensure operations are
  performed on correct units of measurement
  (see \chapterpageref{units-checker})
\item
  \ahrefloc{linear-checker}{Linear Checker} to control aliasing and prevent
  re-use (see \chapterpageref{linear-checker})
\item
  \ahrefloc{igj-checker}{IGJ Checker} for mutation errors (incorrect
  side effects), based on the IGJ type system (see
  \chapterpageref{igj-checker})
\item
  \ahrefloc{javari-checker}{Javari Checker} for mutation errors
  (incorrect side effects), based on the Javari type system (see
  \chapterpageref{javari-checker})
\item
  \ahrefloc{subtyping-checker}{Subtyping Checker} for customized checking without
  writing any code (see \chapterpageref{subtyping-checker})
% \item
%   \ahrefloc{typestate-checker}{Typestate checker} to ensure operations are
%   performed on objects that are in the right state, such as only opened
%   files being read (see \chapterpageref{typestate-checker})
\item
  \ahrefloc{third-party-checkers}{Third-party checkers} that are distributed
  separately from the Checker Framework
  (see \chapterpageref{third-party-checkers})

\end{enumerate}

\noindent
These checkers are easy to use and are invoked as arguments to \<javac>.


The Checker Framework also enables you to write new checkers of your
own; see Chapters~\ref{subtyping-checker} and~\ref{writing-a-checker}.


\section{How to read this manual\label{how-to-read-this-manual}}

If you wish to get started using some particular type system from the list
above, then the most effective way to read this manual is:

\begin{itemize}
\item
  Read all of the introductory material
  (Chapters~\ref{introduction}--\ref{using-a-checker}).
\item
  Read just one of the descriptions of a particular type system and its
  checker (Chapters~\ref{nullness-checker}--\ref{third-party-checkers}).
\item
  Skim the advanced material that will enable you to make more effective
  use of a type system
  (Chapters~\ref{polymorphism}--\ref{troubleshooting}), so that you will
  know what is available and can find it later.  Skip
  Chapter~\ref{writing-a-checker} on creating a new checker.
\end{itemize}


\section{How it works:  Pluggable types\label{pluggable-types}}

The Checker Framework supports adding
pluggable type systems to the Java language in a backward-compatible way.
Java's built-in type-checker finds and prevents many errors --- but it
doesn't find and prevent \emph{enough} errors.  The Checker Framework lets you
run an additional type-checker as a plug-in to the javac compiler.  Your
code stays completely backward-compatible:  your code compiles with any
Java compiler, it runs on any JVM, and your coworkers don't have to use the
enhanced type system if they don't want to.  You can check only part of
your program.  Type inference tools exist to help you annotate your
code.


A type system designer uses the Checker Framework to define type qualifiers
and their semantics, and a
compiler plug-in (a ``checker'') enforces the semantics.  Programmers can
write the type qualifiers in their programs and use the plug-in to detect
or prevent errors.  The Checker Framework is useful both to programmers who
wish to write error-free code, and to type system designers who wish to
evaluate and deploy their type systems.



% This manual is organized as follows.
% \begin{itemize}
% \item Chapter~\ref{introduction} overviews the Checker Framework and
%   describes how to \ahrefloc{installation}{install} it (Chapter~\ref{installation}).
% \item Chapter~\ref{using-a-checker} describes how to \ahrefloc{using-a-checker}{use a checker}.
% \item
%   The next chapters are user manuals for the \ahrefloc{nullness-checker}{Nullness}
%   (Chapter~\ref{nullness-checker}), \ahrefloc{interning-checker}{Interning}
%   (Chapter~\ref{interning-checker}), \ahrefloc{javari-checker}{Javari} (Chapter~\ref{javari-checker}),
%   \ahrefloc{igj-checker}{IGJ} (Chapter~\ref{igj-checker}), and \ahrefloc{subtyping-checker}{Basic}
%   (Chapter~\ref{subtyping-checker}) checkers.
% \item Chapter~\ref{annotating-libraries} describes an approach for \ahrefloc{annotating-libraries}{annotating external
% libraries}.
% \item Chapter~\ref{writing-a-checker} describes how to
%   \ahrefloc{writing-a-checker}{write a new checker} using the Checker Framework.
% \end{itemize}






This document uses the terms ``checker'', ``checker plugin'',
``type-checking compiler plugin'', and ``annotation processor'' as
synonyms.

\section{Installation\label{installation}}

This section describes how to install the Checker Framework.
(If you wish to
use the Checker Framework from Eclipse, see the Checker Framework Eclipse
Plugin webpage: \ahrefurl{http://types.cs.washington.edu/checker-framework/eclipse/}.)
The Checker Framework release contains everything that you need, both to
run checkers and to write your own checkers.  As an alternative, you can
build the latest development version from source
(Section~\refwithpage{build-source}).

% Not "\ahrefurl" because it looks bad in the printed manual.
\textbf{Requirement:}
You must have \textbf{JDK 7} or later installed.  You can get JDK 7 from
\ahref{\url{http://www.oracle.com/technetwork/java/javase/downloads/index.html}}{Oracle}
or elsewhere.  If you are using Apple Mac OS X, you can use
\ahref{\url{http://developer.apple.com/search/index.php?q=java}}{Apple's implementation},
\ahref{\url{http://landonf.bikemonkey.org/static/soylatte/}}{SoyLatte},
or the \ahref{\url{http://openjdk.java.net/}}{OpenJDK}.

The installation process is simple!  It has two required steps and one
optional step.
\begin{enumerate}
\item
  Download the Checker Framework distribution:
  %BEGIN LATEX
  \\
  %END LATEX
  \ahrefurl{http://types.cs.washington.edu/checker-framework/current/checker-framework.zip}

\item 
  Unzip it to create a \code{checker-framework} directory.

\item
  \label{installation-configure-step}
  Configure your IDE, build system, or command shell to use the Checker
  Framework compiler.  Choose the appropriate section of
  Chapter~\ref{external-tools} for
% Keep this list up to date with the list at the beginning
% of file external-tools.tex .
javac (Section~\ref{javac-installation}),
Ant (Section~\ref{ant-task}),
Maven (Section~\ref{maven-plugin}),
Gradle (Section~\ref{gradle}),
IntelliJ IDEA (Section~\ref{intellij}),
Eclipse (Section~\ref{eclipse}),
or
tIDE (Section~\ref{tide}).


\end{enumerate}

That's all there is to it!  Now you are ready to start using the checkers.

We recommend that you work through the
\ahreforurl{http://types.cs.washington.edu/checker-framework/tutorial/}{Checker
Framework tutorial}, which walks you through how to use the Checker
Framework in Eclipse or on
the command line.

Section~\ref{example-use} walks you through a simple example.  More detailed
instructions for using a checker appear in Chapter~\ref{using-a-checker}.



\section{Example use:  detecting a null pointer bug\label{example-use}}

This section gives a very simple example of running the Checker Framework
from the command line.  There is also a \ahreforurl{http://types.cs.washington.edu/checker-framework/tutorial/}{tutorial}
that gives more extensive instructions for using the Checker Framework in
Eclipse or on the command line.


To run a checker on a source file, just compile as usual, but pass the
\<-processor> flag to the compiler.

For instance, if you usually run the javac compiler like
this:

\begin{Verbatim}
  javac Foo.java Bar.java
\end{Verbatim}

\noindent
then you will instead use the command line:

\begin{alltt}
  javac \textbf{-processor \textit{ProcessorName}} Foo.java Bar.java
\end{alltt}

\noindent
but take note that the \code{javac} command must refer to the
Checker Framework compiler (see Section~\ref{javac-installation}).

If you usually do your coding within an IDE, you will need to configure
the IDE; see Chapter~\ref{external-tools}.


\begin{enumerate}
\item
  Let's consider this very simple Java class.  The local variable \<ref> is
  annotated as \<@NonNull>, indicating that \<ref> must be a reference to a
  non-null object.  Save the file as \<GetStarted.java>.

\begin{Verbatim}
import org.checkerframework.checker.nullness.qual.*;

public class GetStarted {
    void sample() {
        @NonNull Object ref = new Object();
    }
}
\end{Verbatim}

\item
  Run the Nullness Checker on the class.
  Either run this command:
\begin{Verbatim}
  javac -processor org.checkerframework.checker.nullness.NullnessChecker GetStarted.java
\end{Verbatim}

\noindent
or compile from within your IDE, which you have customized (see
Chapter~\ref{external-tools} to use the
Checker Framework compiler and to pass the extra arguments.

  The compilation should complete without any errors.

\item
  Let's introduce an error now.  Modify \<ref>'s assignment to:
\begin{alltt}
  @NonNull Object ref = \textbf{null};
\end{alltt}

\item
  Run the Nullness Checker again, just as before.  This run should emit
  the following error:
\begin{Verbatim}
GetStarted.java:5: incompatible types.
found   : @Nullable <nulltype>
required: @NonNull Object
        @NonNull Object ref = null;
                              ^
1 error
\end{Verbatim}

\end{enumerate}

The type qualifiers (e.g., \<@NonNull>) are permitted anywhere
that you can write a type, including generics and casts; see
Section~\ref{writing-annotations}.  Here are some examples:

\begin{alltt}
  \underline{@Interned} String intern() \ttlcb{} ... \ttrcb{}             // return value
  int compareTo(\underline{@NonNull} String other) \ttlcb{} ... \ttrcb{}  // parameter
  \underline{@NonNull} List<\underline{@Interned} String> messages;     // non-null list of interned Strings
\end{alltt}


\section{What comes with the Checker Framework distribution\label{release-content}}

The Checker Framework distribution contains the following notable
directories and files:

\begin{itemize}
\item \<changelog-checkerframework.txt> The changelog.
\item \<checker/bin/javac> A replacement for the \<javac> compiler that
  enables use of the Checker Framework.
\item \<checker/manual/> A local copy of this manual in PDF and HTML formats.
\item \<tutorial/> The Checker Framework tutorial.
\item \<checker/dist/> Contains jar files for use by advanced users:
  \begin{itemize}
  \item \<javac.jar> A Java 9 javac with additional support for
  Checker Framework extensions.
  \item \<checker.jar> The Checker Framework classes.
  \item \<jdk7.jar> and \<jdk8.jar> Annotations for the JDK classes for
  Java~7 and Java~8 (but no class bodies).
  \item \<checker-qual.jar> The annotation types defined by the
  Checker Framework. This jar file is useful to distribute with code
  that uses Checker Framework annotations. See Section~\ref{distributing}.
  \item \<checker-source.jar> The Checker Framework source code for
  use by IDEs.
  \item \<checker-javadoc.jar> The Checker Framework Javadoc for use
  by IDEs.
  \end{itemize}
\end{itemize}

% The source distribution contains a superset of these files.
% See the developer manual for details.


\htmlhr
\chapter{Using a checker\label{using-a-checker}}

A pluggable type-checker enables you to detect certain bugs in your code,
or to prove that they are not present.  The verification happens at compile
time.


Finding bugs, or verifying their absence, with a checker plugin is a two-step process, whose steps are
described in Sections~\ref{writing-annotations} and \ref{running}.

\begin{enumerate}

\item The programmer writes annotations, such as \refqualclass{checker/nullness/qual}{NonNull} and
  \refqualclass{checker/interning/qual}{Interned}, that specify additional information about Java types.
  (Or, the programmer uses an inference tool to automatically insert
  annotations in his code:  see Sections~\ref{nullness-inference} and~\ref{javari-inference}.)
  It is possible to annotate only part of your code:  see
  Section~\ref{unannotated-code}.

\item The checker reports whether the program contains any erroneous code
  --- that is, code that is inconsistent with the annotations.

\end{enumerate}

This chapter is structured as follows:
\begin{itemize}
\item Section~\ref{writing-annotations}: How to write annotations
\item Section~\ref{running}:  How to run a checker
\item Section~\ref{checker-guarantees}: What the checker guarantees
\item Section~\ref{tips-about-writing-annotations}: Tips about writing annotations
\end{itemize}

Additional topics that apply to all checkers are covered later in the manual:
\begin{itemize}
\item Chapter~\ref{advanced-type-system-features}: Advanced type system features
\item Chapter~\ref{warnings-and-legacy}: Handling warnings and legacy code
\item Chapter~\ref{annotating-libraries}: Annotating libraries
\item Chapter~\ref{writing-a-checker}: How to create a new checker
\item Chapter~\ref{external-tools}: Integration with external tools
\end{itemize}


Finally, there is a 
\ahreforurl{http://types.cs.washington.edu/checker-framework/tutorial/}{tutorial}
that walks you through using the Checker Framework in Eclipse or on the
command line.

% The annotations have to be on your classpath even when you are not using
% the -processor, because of the existence of the import statement for
% the annotations.


\section{Writing annotations\label{writing-annotations}}

The syntax of type annotations in Java is specified by
\ahref{\url{http://types.cs.washington.edu/jsr308/}}{JSR 308}~\cite{JSR308-2008-09-12}.  Ordinary
Java permits annotations on declarations.  JSR 308 permits annotations
anywhere that you would write a type, including generics and casts.  You
can also write annotations to indicate type qualifiers for array levels and
receivers.  Here are a few examples:

\begin{alltt}
  \underline{@Interned} String intern() \ttlcb{} ... \ttrcb{}               // return value
  int compareTo(\underline{@NonNull} String other) \ttlcb{} ... \ttrcb{}    // parameter
  String toString(\underline{@ReadOnly} MyClass this) \ttlcb{} ... \ttrcb{} // receiver ("this" parameter)
  \underline{@NonNull} List<\underline{@Interned} String> messages;       // generics:  non-null list of interned Strings
  \underline{@Interned} String \underline{@NonNull} [] messages;          // arrays:  non-null array of interned Strings
  myDate = (\underline{@ReadOnly} Date) readonlyObject;       // cast
\end{alltt}

You can also write the annotations within comments, as in
\code{List</*@NonNull*/ String>}.  The Checker Framework compiler, which is
distributed with the Checker Framework, will still process
the annotations.
However, your code will remain compilable by people who are not using the
Checker Framework compiler.  For more details, see
Section~\ref{annotations-in-comments}.



\subsection{Distributing your annotated project\label{distributing}}

If your code contains annotations, then your code has a dependency on the
annotation declarations.  People who want to compile or run your code may
need declarations of the annotations on their classpath.

\begin{itemize}
\item
To perform pluggable type-checking, all of the Checker Framework (which
also contains the annotation declarations) is needed.
\item
To compile the code:
\begin{itemize}
\item
  If you wrote annotations in comments (see
  Section~\ref{annotations-in-comments}) and/or used implicit import
  statements (see Section~\ref{implicit-import-statements}), then the code
  can be compiled by any Java compiler, without needing declarations of the
  annotations.
\item
  Otherwise, compiling the code requires a declaration of the annotations.
  These appear in the full Checker Framework.  Additionally, the Checker
  Framework distribution \code{.zip} file contains a small jar file,
  \code{checker-qual.jar}, that only contains the definitions of the
  distributed qualifiers and some run-time utilities, without any support for type-checking.
\end{itemize}
\item
To run the code:
\begin{itemize}
\item
  If you compiled the code without using the annotation declarations, then
  no annotation declarations are needed.
\item
  If you compiled the code using the annotation declarations, then users
  may need to have the annotation declarations on their classpath.
\end{itemize}
\end{itemize}

A simple rule of thumb is as follows.  When distributing your source code,
you may wish to include either the Checker Framework jar file or the
\code{checker-qual.jar} file.  When distributing compiled binaries, you
may wish to compile them without using the annotations, or include the
contents of \code{checker-qual.jar} in your distribution.


\section{Running a checker\label{running}}

To run a checker plugin, run the compiler \code{javac} as usual,
but pass the \code{-processor \emph{plugin\_class}} command-line
option.
(You can run a checker from within your favorite IDE or build system.  See
Chapter~\ref{external-tools} for details about
Ant (Section~\ref{ant-task}),
Maven (Section~\ref{maven-plugin}),
Gradle (Section~\ref{gradle}),
IntelliJ IDEA (Section~\ref{intellij}),
Eclipse (Section~\ref{eclipse}),
and
tIDE (Section~\ref{tide}), and about customizing other IDEs and build tools.)
Remember that you must be using the
Type Annotations version of \<javac>, which you already installed (see Section~\ref{installation}).

A concrete examples (using the Nullness Checker) is:

%BEGIN LATEX
\begin{smaller}
%END LATEX
\begin{Verbatim}
  javac -processor org.checkerframework.checker.nullness.NullnessChecker MyFile.java
\end{Verbatim}
%BEGIN LATEX
\end{smaller}
%END LATEX


\noindent
Note that the two invocations above are equivalent.  Each one invokes the
Checker Framework compiler, automatically adds 
the annotated JDK to the classpath.  For more information on annotated
JDKs, see Section~\ref{skeleton-using}.

The checker is run on only the Java files that javac compiles.
This includes all Java files specified on the command line (or
created by another annotation processor).  It may also include other of
your Java files (but not if a more recent \code{.class} file exists).
Even when the checker does not analyze a class (say, the class was
already compiled, or source code is not available), it does check
the \emph{uses} of those classes in the source code being compiled.

You can always compile the code without the \code{-processor}
command-line option, but in that case no checking of the type
annotations is performed.  The annotations are still written to the
resulting \<.class> files, however.



\subsection{Summary of command-line options\label{checker-options}}

You can pass command-line arguments to a checker via javac's standard \<-A>
option (``\<A>'' stands for ``annotation'').  All of the distributed
checkers support the following command-line options.

Unsound checking: ignore some errors
\begin{itemize}
\item \<-AskipUses>, \<-AonlyUses> 
  Suppress all errors and warnings at all uses of a given class --- or at all
  uses except those of a given class.  See Section~\ref{askipuses}
\item \<-AskipDefs>, \<-AonlyDefs>
  Suppress all errors and warnings within the definition of a given class
  --- or everywhere except within the definition of a given class.  See
  Section~\ref{askipdefs}
\item \<-AsuppressWarnings>
  Suppress all warnings matching the given key; see
  Section~\ref{suppresswarnings-command-line}
\item \<-AignoreRawTypeArguments>
  Ignore subtype tests for type arguments that were inferred for a raw
  type.  If possible, it is better to write the type arguments.  See
  Section~\ref{generics-raw-types}.
\item \<-AassumeSideEffectFree>
  Unsoundly assume that every method is side-effect-free; see
  Section~\ref{type-refinement-purity}.
\item \<-AassumeAssertionsAreEnabled>, \<-AassumeAssertionsAreDisabled>
  Whether to assume that assertions are enabled or disabled; see Section~\ref{type-refinement-assertions}.
\end{itemize}

\label{unsound-by-default}
More sound (strict) checking: enable errors that are disabled by default
\begin{itemize}
\item \<-AenablePurity>
  Check the bodies of methods marked \<@SideEffectFree>, \<@Deterministic>,
  and \<@Pure> to ensure the method satisfies the annotation.  By default,
  the Checker Framework unsoundly trusts the method annotation.  See
  Section~\ref{type-refinement-purity}.
\item \<-AinvariantArrays>
  Make array subtyping invariant; that is, two arrays are subtypes of one
  another only if they have exactly the same element type.  By default,
  the Checker Framework unsoundly permits covariant array subtyping, just
  as Java does.  See Section~\ref{invariant-arrays}.
\item \<-AconcurrentSemantics>
  Whether to assume concurrent semantics (field values may change at any
  time) or sequential semantics; see Section~\ref{faq-concurrency}.
\end{itemize}

Type-checking modes:  enable/disable functionality
\begin{itemize}
\item \<-Alint>
  Enable or disable optional checks; see Section~\ref{lint-options}.
\item \<-AshowSuppressWarningKeys>
  With each warning, show all possible keys to suppress that warning;
  see Section~\ref{suppresswarnings-command-line}
\item \<-AsuggestPureMethods>
  Suggest methods that could be marked \<@SideEffectFree>,
  \<@Deterministic>, or \<@Pure>; see Section~\ref{type-refinement-purity}.
\item \<-AcheckCastElementType>
  In a cast, require that parameterized type arguments and array elements
  are the same.  By default, the Checker Framework unsoundly permits them
  to differ, just as Java does.  See Section~\ref{covariant-type-parameters}
  and Section~\ref{invariant-arrays}.
\item \<-Awarns>
  Treat checker errors as warnings.  If you use this, you
  may wish to also supply \code{-Xmaxwarns 10000}, because by default
  \<javac> prints at most 100 warnings.
\end{itemize}

Stub libraries
\begin{itemize}
\item \<-Astubs>
  List of stub files or directories; see Section~\ref{stub-using}.
\item \<-AstubWarnIfNotFound>
  Warn if a stub file entry could not be found; see Section~\ref{stub-using}.
\end{itemize}

Debugging
\begin{itemize}
\item
 \<-AprintAllQualifiers>,
 \<-Adetailedmsgtext>,
 \<-AprintErrorStack>,
 \<-Anomsgtext>
Amount of detail in messages; see Section~\ref{debugging-options-detail}.

\item
 \<-Aignorejdkastub>,
 \<-Anocheckjdk>
 \<-AstubDebug>,
Stub and JDK libraries; see Section~\ref{debugging-options-libraries}

\item
 \<-Afilenames>,
 \<-Ashowchecks>
Progress tracing; see Section~\ref{debugging-options-progress}

\item
 \<-Aflowdotdir>,
 \<-AresourceStats>
Miscellaneous debugging options; see Section~\ref{debugging-options-misc}
\end{itemize}


\noindent
Some checkers support additional options, which are described in that
checker's manual section.
% Search for "@SupportedOptions" in the implementation to find them all.
For example, \<-Aquals> tells
the Subtyping Checker (see Chapter~\ref{subtyping-checker}) and the Fenum Checker
(see Chapter~\ref{fenum-checker}) which annotations to check.


Here are some standard javac command-line options that you may find useful.
Many of them contain the word ``processor'', because in javac jargon, a
checker is a type of ``annotation processor''.

\begin{itemize}
\item \<-processor> Names the checker to be
  run; see Section~\ref{running}
\item \<-processorpath> Indicates where to search for the
  checker; should also contain any qualifiers used by the Subtyping
  Checker; see Section~\ref{subtyping-example}
\item \<-proc:>\{\<none>,\<only>\} Controls whether checking
  happens; \<-proc:none>
  means to skip checking; \<-proc:only> means to do only
  checking, without any subsequent compilation; see
  Section~\ref{checker-auto-discovery}
\item \<-Xbootclasspath/p:> Indicates where to find the annotated JDK classes;
  see Section~\ref{skeleton-using}
\item \<-implicit:class> Suppresses warnings about implicitly compiled files
  (not named on the command line); see Section~\ref{ant-task}
\item \<-XDTA:noannotationsincomments> and \<-XDTA:spacesincomments>
  to turn off parsing annotation comments and
  to turn on parsing annotation comments even when they
  contain spaces; applicable only to the Checker Framework compiler;
  see Section~\ref{annotations-in-comments}
\item \<-J> Supply an argument to the JVM that is running javac; example: \\
  \<-J-Djsr308\_imports=org.checkerframework.checker.nullness.qual.*:org.checkerframework.dataflow.qual.*> \\
 See Section~\ref{implicit-import-statements}
\end{itemize}


\subsection{Checker auto-discovery\label{checker-auto-discovery}}

``Auto-discovery'' makes the \code{javac} compiler always run a checker
plugin, even if you do not explicitly pass the \code{-processor}
command-line option.  This can make your command line shorter, and ensures
that your code is checked even if you forget the command-line option.

\begin{sloppypar}
To enable auto-discovery, place a configuration file named
\code{META-INF/services/javax.annotation.processing.Processor}
in your classpath.  The file contains the names of the checker plugins to
be used, listed one per line.  For instance, to run the Nullness Checker and the
Interning Checker automatically, the configuration file should contain:
\end{sloppypar}

%BEGIN LATEX
\begin{smaller}
%END LATEX
\begin{Verbatim}
  org.checkerframework.checker.nullness.NullnessChecker
  org.checkerframework.checker.interning.InterningChecker
\end{Verbatim}
%BEGIN LATEX
\end{smaller}
%END LATEX

You can disable this auto-discovery mechanism by passing the
\code{-proc:none} command-line option to \<javac>, which disables all
annotation processing including all pluggable type-checking.

%% Auto-discovering all the distributed checkers by default would be
%% problematic.  So, leave it up to the user to enable auto-discovery.
%%  1. We don't want to auto-discover both the Javari & IGJ type-checkers,
%%     as then the user would see multiple, possibly contradictory, types
%%     of mutability diagnostics.
%%  2. The nullness and mutability checkers would issue lots of errors for
%%     unannotated code, and that would be irritating.



\section{What the checker guarantees\label{checker-guarantees}}

A checker can guarantee that a particular property holds throughout the
code.  For example, the Nullness Checker (Chapter~\ref{nullness-checker})
guarantees that every expression whose type is a \refqualclass{checker/nullness/qual}{NonNull} type never
evaluates to null.  The Interning Checker (Chapter~\ref{interning-checker})
guarantees that every expression whose type is an \refqualclass{checker/interning/qual}{Interned} type
evaluates to an interned value.  The checker makes its guarantee by
examining every part of your program and verifying that no part of the
program violates the guarantee.

There are some limitations to the guarantee.


\begin{itemize}

\item
  A compiler plugin can check only those parts of your program that you run
  it on.  If you compile some parts of your program without running the
  checker, then there is no guarantee that the entire program satisfies the
  property being checked.  Some examples of un-checked code are:

  \begin{itemize}
  \item
    Code compiled without the \code{-processor} switch, including any
    external library supplied as a \code{.class} file.
  \item
    Code compiled with the \code{-AskipUses}, \code{-AonlyUses}, \code{-AskipDefs} or \code{-AonlyDefs}
    properties (see Section~\ref{suppressing-warnings}).
  \item
    Suppression of warnings, such as via the \code{@SuppressWarnings}
    annotation (see Section~\ref{suppressing-warnings}).
  \item
    Native methods (because the implementation is not Java code, it cannot
    be checked).
  \end{itemize}

  In each of these cases, any \emph{use} of the code is checked --- for
  example, a call to a native method must be compatible with any
  annotations on the native method's signature.
  However, the annotations on the un-checked code are trusted; there is no
  verification that the implementation of the native method satisfies the
  annotations.

\item
  The Checker Framework is, by default, unsound in a few places where a
  conservative analysis would issue too many false positive warnings.
  These are listed in Section~\ref{unsound-by-default}; as of January 2014,
  they are: (1) purity annotations are trusted, (2) array types are
  covariant, and (3) programs are assumed to be single-threaded.
  You can supply a command-line argument to make the Checker Framework
  sound for each of these cases.

%% This isn't an unsoundness in the Checker Framework:  for any type system
%% that does not include a conservative library annotation for
%% Method.invoke, it is a bug in that particular type-checker.
% \item
%   Reflection can violate the Java type system, and
%   the checkers are not sophisticated enough to reason about the possible
%   effects of reflection.  Similarly, deserialization and cloning can
%   create objects that could not result from normal constructor calls, and
%   that therefore may violate the property being checked.

\item
  Specific checkers may have other limitations; see their documentation for
  details.

\end{itemize}

A checker can be useful in finding bugs or in verifying part of a
program, even if the checker is unable to verify the correctness of an
entire program.

In order to avoid a flood of unhelpful warnings, many of the checkers avoid
issuing the same warning multiple times.  For example, in this code:

\begin{Verbatim}
  @Nullable Object x = ...;
  x.toString();                 // warning
  x.toString();                 // no warning
\end{Verbatim}

\noindent
In this case, the second call to \<toString> cannot possibly throw a null
pointer warning --- \<x> is non-null if control flows to the second
statement.
In other cases, a checker avoids issuing later warnings with the same cause
even when later code in a method might also fail.
This does not
affect the soundness guarantee, but a user may need to examine more
warnings after fixing the first ones identified.  (More often, at least in
our experience to date, a single fix corrects all the warnings.)

% It might be worthwhile to permit a user to see every warning -- though I
% would not advocate this setting for daily use.

If you find that a checker fails to issue a warning that it
should, then please report a bug (see Section~\ref{reporting-bugs}).


\section{Tips about writing annotations\label{tips-about-writing-annotations}}


\subsection{How to get started annotating legacy code\label{get-started-with-legacy-code}}

Annotating an entire existing program may seem like a daunting task.  But,
if you approach it systematically and do a little bit at a time, you will
find that it is manageable.

You should start with a property that matters to you, to achieve the best
benefits.  It is easiest to add annotations if you know the code or the
code contains documentation; you will find that you spend most of your time
understanding the code, and very little time actually writing annotations
or running the checker.

Don't get discouraged if you see many type-checker warnings at first.
Often, adding just a few missing annotations will eliminate many warnings,
and you'll be surprised how fast the process goes overall.

It is best to annotate one package at a time,
% Upcoming fix that applies different defaults to annotated and unannotated
% code will eliminate this reason.
and to annotate the entire package so that you don't forget any classes
(failing to annotate a class can lead to unexpected results).
Start as close to the leaves of the call tree as possible, such as with
libraries --- that is,
start with methods/classes/packages that have few dependencies on other
code or, equivalently, start with code that a lot of your other code
depends on.  The reason for this is that it is
easiest to annotate a class if the code it calls has already been
annotated.

For each class, read its Javadoc.  For instance, if you are adding
annotations for the Nullness Checker (Section~\ref{nullness-checker}), then
you can search the documentation for ``null'' and then add \<@Nullable>
anywhere appropriate.  Do not annotate the method bodies yet ---
first, get the signatures and fields annotated.  The only reason to even
\emph{read} the method bodies yet is to determine signature annotations for
undocumented methods ---
for example, if the method returns null, you know its return type should be
annotated \<@Nullable>, and a parameter that is compared against \<null>
may need to be annotated \<@Nullable>.  If you are only annotating
signatures (say, for a library you do not maintain and do not wish to
check), you are now done.

If you wish to check the implementation, then after the signatures are
annotated, run the checker.  Then, add method body annotations (usually,
few are necessary), fix bugs in code, and add annotations to signatures
where necessary.  If signature annotations are necessary, then you may want
to fix the documentation that did not indicate the property; but this isn't
strictly necessary, since the annotations that you wrote provide that
documentation.

You may wonder about the effect of adding a given annotation --- how many
other annotations it will require, or whether it conflicts with other code.
Suppose you have added an annotation to a method parameter.  You could
manually examine all callees.  A better way can be to save the checker
output before adding the annotation, and to compare it to the checker
output after adding the annotation.  This helps you to focus on the
specific consequences of your change.

Also see Chapter~\ref{warnings-and-legacy}, which tells you what to do when
you are unable to eliminate checker warnings.



\subsection{Do not annotate local variables unless necessary\label{tips-local-inference}}

The checker infers annotations for local variables (see
Section~\ref{type-refinement}).  Usually, you only need to annotate fields
and method signatures.  After doing those, you can add annotations inside
method bodies if the checker is unable to infer the correct annotation, if
you need to suppress a warning (see Section~\ref{suppressing-warnings}),
etc.


\subsection{Annotations indicate normal behavior\label{annotate-normal-behavior}}

You should use annotations to specify \emph{normal} behavior.  The
annotations indicate all the values that you \emph{want} to flow to
reference --- not every value that might possibly flow there if your
program has a bug.

Many methods are guaranteed to throw an exception if they are passed \code{null}
as an argument.  Examples include

\begin{Verbatim}
  java.lang.Double.valueOf(String)
  java.lang.String.contains(CharSequence)
  org.junit.Assert.assertNotNull(Object)
  com.google.common.base.Preconditions.checkNotNull(Object)
\end{Verbatim}

\refqualclass{checker/nullness/qual}{Nullable} (see Section~\ref{nullness-annotations})
might seem like a reasonable annotation for the parameter,
for two reasons.  First, \code{null} is a legal argument with a
well-defined semantics:  throw an exception.  Second, \code{@Nullable}
describes a possible program execution:  it might be possible for
\code{null} to flow there, if your program has a bug.

% (Checking for such a bug is the whole purpose of the \code{assertNotNull}
% and \code{checkNotNull} methods.)

However, it is never useful for a programmer to pass \code{null}.  It is
the programmer's intention that \code{null} never flows there.  If
\code{null} does flow there, the program will not continue normally
(whether or not it throws a NullPointerException).

Therefore, you should mark such parameters as
\refqualclass{checker/nullness/qual}{NonNull}, indicating
the intended use of the method.  When you use the \code{@NonNull}
annotation, the checker is able to issue compile-time warnings about
possible run-time exceptions, which is its purpose.  Marking the parameter
as \code{@Nullable} would suppress such warnings, which is undesirable.

% (The note at
% http://google-collections.googlecode.com/svn/trunk/javadoc/com/google/common/base/Preconditions.html
% argues that the parameter could be marked as @Nullable, since it is
% possible for null to flow there at run time.  However, since that is an
% erroneous case, the annotation would be counterproductive rather than
% useful.)


\subsection{Subclasses must respect superclass annotations\label{annotations-are-a-contract}}

An annotation indicates a guarantee that a client can depend upon.  A subclass
is not permitted to \emph{weaken} the contract; for example,
if a method accepts \code{null} as an argument, then every overriding
definition must also accept \code{null}.
A subclass is permitted to \emph{strengthen} the contract; for example,
if a method does \emph{not} accept \code{null} as an argument, then an
overriding definition is permitted to accept \code{null}.

\begin{sloppypar}
As a bad example, consider an erroneous \code{@Nullable} annotation at
line 141 of \ahref{\url{http://code.google.com/p/google-collections/source/browse/trunk/src/com/google/common/collect/Multiset.java}}{\code{com/google/common/collect/Multiset.java}}, version r78:
\end{sloppypar}

\begin{Verbatim}
101  public interface Multiset<E> extends Collection<E> {
...
122    /**
123     * Adds a number of occurrences of an element to this multiset.
...
129     * @param element the element to add occurrences of; may be {@code null} only
130     *     if explicitly allowed by the implementation
...
137     * @throws NullPointerException if {@code element} is null and this
138     *     implementation does not permit null elements. Note that if {@code
139     *     occurrences} is zero, the implementation may opt to return normally.
140     */
141    int add(@Nullable E element, int occurrences);
\end{Verbatim}

There exist implementations of Multiset that permit \code{null} elements,
and implementations of Multiset that do not permit \code{null} elements.  A
client with a variable \code{Multiset ms} does not know which variety of
Multiset \code{ms} refers to.  However, the \code{@Nullable} annotation
promises that \code{ms.add(null, 1)} is permissible.  (Recall from
Section~\ref{annotate-normal-behavior} that annotations should indicate
normal behavior.)

If parameter \code{element} on line 141 were to be annotated, the correct
annotation would be \code{@NonNull}.  Suppose a client has a reference to
same Multiset \code{ms}.  The only way the client can be sure not to throw an exception is to pass
only non-\code{null} elements to \code{ms.add()}.  A particular class
that implements Multiset could declare \code{add} to take a
\code{@Nullable} parameter.  That still satisfies the original contract.
It strengthens the contract by promising even more:  a client with such a
reference can pass any non-\code{null} value to \code{add()}, and may also
pass \code{null}.

\textbf{However}, the best annotation for line 141 is no annotation at all.
The reason is that each implementation of the Multiset interface should
specify its own nullness properties when it specifies the type parameter
for Multiset.  For example, two clients could be written as

\begin{Verbatim}
  class MyNullPermittingMultiset implements Multiset<@Nullable Object> { ... }
  class MyNullProhibitingMultiset implements Multiset<@NonNull Object> { ... }
\end{Verbatim}

\noindent
or, more generally, as

\begin{Verbatim}
  class MyNullPermittingMultiset<E extends @Nullable Object> implements Multiset<E> { ... }
  class MyNullProhibitingMultiset<E extends @NonNull Object> implements Multiset<E> { ... }
\end{Verbatim}

Then, the specification is more informative, and the Checker Framework is
able to do more precise checking, than if line 141 has an annotation.

It is a pleasant feature of the Checker Framework that in many cases, no
annotations at all are needed on type parameters such as \code{E} in \<MultiSet>.


\subsection{Annotations on constructor invocations\label{annotations-on-constructor-invocations}}

%% I want to get rid of this syntax.
%% However, @Linear provides a compelling use case.

In the checkers distributed with the Checker Framework, an annotation on a
constructor invocation is equivalent to a cast on a constructor result.
That is, the following two expressions have identical semantics:  one is
just shorthand for the other.

\begin{Verbatim}
  new @ReadOnly Date()
  (@ReadOnly Date) new Date()
\end{Verbatim}

However, you should rarely have to use this.  The Checker Framework will
determine the qualifier on the result, based on the ``return value''
annotation on the constructor definition.  The ``return value'' annotation
appears before the constructor name, for example:

\begin{Verbatim}
  class MyClass {
    @ReadOnly MyClass() { ... }
  }
\end{Verbatim}

In general, you should only use an annotation on a constructor invocation
when you know that the cast is
guaranteed to succeed.  An example from the IGJ checker
(Chapter~\ref{igj-checker}) is \<new @Immutable MyClass()> or \<new
@Mutable MyClass()>, where you know that every other reference to the class
is annotated \<@ReadOnly>.


\subsection{When to use (and not use) type qualifiers\label{when-to-use-type-qualifiers}}

For some programming tasks, you can use either a Java subclass or a type
qualifier.  For instance, suppose that your code currently uses
\code{String} to represent an address.  You could create a new \code{Address}
class and refactor your code to use it, or you could create a
\code{@Address} annotation and apply it to some uses of \code{String} in
your code.  If both of these are truly possible, then it is probably more
foolproof to use the Java class.  We do not encourage you to use type
qualifiers as a poor substitute for classes.  However, sometimes type
qualifiers are a better choice.

Using a new class may make your code incompatible with existing libraries or
clients.  Brian Goetz expands on this issues in an article on the
pseudo-typedef antipattern~\cite{Goetz2006:typedef}.  Even if compatibility
is not a concern, a code change may introduce bugs, whereas adding
annotations does not change the run-time behavior.  It is possible to add
annotations to existing code, including code you do not maintain or cannot
change (for code that strictly cannot be changed, it is recommended to add
annotations in comments --- see Section~\ref{annotations-in-comments}).
It is possible to annotate primitive types without converting them
to wrappers, which would make the code both uglier and slower.

Type qualifiers can be applied to any type, including final classes that
cannot be subclassed.

Type qualifiers permit you to remove operations, with a compile-time
guarantee.  An example is mutating methods that are forbidden by immutable
types (see Chapters~\ref{igj-checker} and~\ref{javari-checker}).  More
generally, type qualifiers permit creating a new supertype, not just a
subtype, of an existing Java type.

% This is the least important reason.
A final reason is efficiency.  Type qualifiers can be more
efficient, since there is no run-time representation such as a wrapper
or a separate class, nor introduction of dynamic dispatch for methods that
could otherwise be statically dispatched.


\subsection{What to do if a checker issues a warning about your code\label{handling-warnings}}

When you first run a type-checker on your code, it is likely to issue
warnings or errors.  For each warning, try to understand why the checker
issues it.  For example, if you are using the
\ahrefloc{nullness-checker}{Nullness Checker}
(\chapterpageref{nullness-checker}), try to understand why it cannot prove
that no null pointer exception ever occurs.  There are three general
reasons, listed below.  You will need to
examine your code, and possibly write test cases, to understand the reason.

\begin{enumerate}
\item
There is a bug in your code, such as an actual possible null dereference.
Fix your code to prevent that crash.

\item
There is a weakness in the annotations.  Improve the annotations.
For example, continuing the Nullness Checker example, if a particular
variable is annotated as \refqualclass{checker/nullness/qual}{Nullable} but it
actually never contains \<null> at run time, then change the annotation to 
\refqualclass{checker/nullness/qual}{NonNull}.  The weakness might be in the
annotations in your code, or in the annotations in a library that your code
calls.  Another possible problem is that a library is unannotated (see
\chapterpageref{annotating-libraries}).

\item
There is a weakness in the type-checker.  Then your code is safe --- it never
suffers the error at run time --- but the checker cannot prove this
fact.  The checker is not omniscient, and some
tricky coding paradigms are beyond its analysis capabilities.  In this
case, you should suppress the warning; see
\chapterpageref{suppressing-warnings}.  (Alternatively, if the weakness is
a bug in the checker, then  please report the bug; see
\chapterpageref{reporting-bugs}.)
\end{enumerate}

If you have trouble understanding a Checker Framework warning message, you
can search for its text in this manual.
Oftentimes there is an explanation of what to do.


% LocalWords:  NonNull zipfile processor classfiles annotationname javac htoc
% LocalWords:  SuppressWarnings un skipUses java plugins plugin TODO cp igj
% LocalWords:  nonnull javari langtools sourcepath classpath OpenJDK pre jsr
% LocalWords:  Djsr qual Alint javac's dotequals nullable supertype JLS Papi
% LocalWords:  deserialization Mahmood Telmo Correa changelog txt nullness ESC
% LocalWords:  Nullness Xspacesincomments unselect checkbox unsetting PolyNull
% LocalWords:  bashrc IDE xml buildfile PolymorphicQualifier enum API elts INF
% LocalWords:  type-checker proc discoverable Xlint util QualifierDefaults Foo
% LocalWords:  DefaultQualifier DefaultQualifiers SoyLatte GetStarted Formatter
% LocalWords:  Dcheckers Warski MyClass ProcessorName compareTo toString myDate
% LocalWords:  ReadOnly readonlyObject int XDTA spacesincomments newdir Awarns
% LocalWords:  subpackages bak tIDE Multiset NullPointerException AskipUses
% LocalWords:  html JCIP MultiSet Astubs Afilenames Anomsgtext Ashowchecks tex
% LocalWords:  Aquals processorpath regex RegEx Xmaxwarns Xbootclasspath com
% LocalWords:  IntelliJ assertNotNull checkNotNull Goetz antipattern subclassed
% LocalWords:  callees Xmx unconfuse fenum propkey forName jsr308 Djsr308
% LocalWords:  bootclasspath AonlyUses AskipDefs AonlyDefs AenablePurity
%  LocalWords:  AsuppressWarnings AassumeSideEffectFree Adetailedmsgtext
%  LocalWords:  AignoreRawTypeArguments AsuggestPureMethods Anocheckjdk
%  LocalWords:  AassumeAssertionsAreEnabled AassumeAssertionsAreDisabled
%  LocalWords:  AconcurrentSemantics AstubWarnIfNotFound AprintErrorStack
%  LocalWords:  AprintAllQualifiers Aignorejdkastub AstubDebug Aflowdotdir
%  LocalWords:  AresourceStats noannotationsincomments jls r78 JDKs i18n
%  LocalWords:  AinvariantArrays AcheckCastElementType formatter pathname
%  LocalWords:  typedef

\htmlhr
\chapter{Nullness Checker\label{nullness-checker}}

If the Nullness Checker issues no warnings for a given program, then
running that program will never throw a null pointer exception.  This
guarantee enables a programmer to prevent errors from occurring when a
program is run.  See Section~\ref{nullness-checks} for more details about
the guarantee and what is checked.

The most important annotations supported by the Nullness Checker are 
\refqualclass{checker/nullness/qual}{NonNull} and 
\refqualclass{checker/nullness/qual}{Nullable}.
\refqualclass{checker/nullness/qual}{NonNull} is rarely written, because it is
the default.  All of the annotations are explained in
Section~\ref{nullness-annotations}.

To run the Nullness Checker, supply the
\code{-processor org.checkerframework.checker.nullness.NullnessChecker}
command-line option to javac.  For
examples, see Section~\ref{nullness-example}.

The NullnessChecker is actually an ensemble of three pluggable
type-checkers that work together: the Nullness Checker proper (which is the
main focus of this chapter), the Initialization Checker
(Section~\ref{initialization-checker}), and the Map Key Checker
(Section~\ref{map-key-checker}).
Their type hierarchies are completely independent, but they work together
to provide precise nullness checking.


\section{What the Nullness Checker checks\label{nullness-checks}}

The checker issues a warning in these cases:

\begin{enumerate}

\item
  When an expression of non-\refqualclass{checker/nullness/qual}{NonNull} type
  is dereferenced, because it might cause a null pointer exception.
  Dereferences occur not only when a field is accessed, but when an array
  is indexed, an exception is thrown, a lock is taken in a synchronized
  block, and more.  For a complete description of all checks performed by
  the Nullness Checker, see the Javadoc for
  \refclass{checker/nullness}{NullnessVisitor}.

\item
  When an expression of \refqualclass{checker/nullness/qual}{NonNull} type
  might become null, because it
  is a misuse of the type:  the null value could flow to a dereference that
  the checker does not warn about.

  As a special case of an of \refqualclass{checker/nullness/qual}{NonNull}
  type becoming null, the checker also warns whenever a field of
  \refqualclass{checker/nullness/qual}{NonNull} type is not initialized in a
  constructor.  Also see the discussion of the \code{-Alint=uninitialized}
  command-line option below.

\end{enumerate}

This example illustrates the programming errors that the checker detects:

\begin{Verbatim}
  @Nullable Object   obj;  // might be null
  @NonNull  Object nnobj;  // never null
  ...
  obj.toString()         // checker warning:  dereference might cause null pointer exception
  nnobj = obj;           // checker warning:  nnobj may become null
  if (nnobj == null)     // checker warning:  redundant test
\end{Verbatim}

Parameter passing and return values are checked analogously to assignments.

The Nullness Checker also checks the correctness, and correct use, of
rawness annotations for checking initialization (see
Section~\ref{initialization-rawness-checker}) and of map key annotations (see
Section~\ref{map-key-checker}).


The checker performs additional checks if certain \code{-Alint}
command-line options are provided.  (See
Section~\ref{alint} for more details about the \code{-Alint}
command-line option.)

\label{lint-nulltest-section}\label{lint-uninitialized-section}
\begin{enumerate}
\item
  \label{lint-nulltest-item}%
  If you supply the \code{-Alint=redundantNullComparison} command-line option, then the
  checker warns when a null check is performed against a value that is
  guaranteed to be non-null, as in \code{("m" == null)}.  Such a check is
  unnecessary and might indicate a programmer error or misunderstanding.
  The lint option is disabled by default because sometimes such checks are
  part of ordinary defensive programming.  

\item
  \label{lint-uninitialized-item}%
  If you supply the \code{-Alint=uninitialized} command-line option, then
  the checker warns if a constructor fails to initialize any field,
  including \refqualclass{checker/nullness/qual}{Nullable} types and primitive
  types.  Such a warning is unrelated to whether your code might throw a
  null pointer exception.  However, you might want to enable this warning
  because it is better code style to supply an explicit initializer, even
  if there is a default value such as \code{0} or \code{false}.
  This command-line option does not affect the Nullness Checker's tests
  that fields of \refqualclass{checker/nullness/qual}{NonNull} type are
  initialized --- such initialization is mandatory, not optional.

\end{enumerate}


\section{Nullness annotations\label{nullness-annotations}}

The Nullness Checker uses three separate type hierarchies:  one for nullness,
one for rawness (Section~\ref{initialization-rawness-checker}),
and one for map keys (Section~\ref{map-key-checker})
The Nullness Checker has four varieties of annotations:  nullness
type qualifiers, nullness method annotations, rawness type qualifiers, and
map key type
qualifiers.

\subsection{Nullness qualifiers\label{nullness-qualifiers}}

The nullness hierarchy contains these qualifiers:

\begin{description}

\item[\refqualclass{checker/nullness/qual}{Nullable}]
  indicates a type that includes the null value.  For example, the type \code{Boolean}
  is nullable:  a variable of type \code{Boolean} always has one of the
  values \code{TRUE}, \code{FALSE}, or \code{null}.

\item[\refqualclass{checker/nullness/qual}{NonNull}]
  indicates a type that does not include the null value.  The type
  \code{boolean} is non-null; a variable of type \code{boolean} always has
  one of the values \code{true} or \code{false}.  The type \code{@NonNull
    Boolean} is also non-null:  a variable of type \code{@NonNull Boolean}
  always has one of the values \code{TRUE} or \code{FALSE} --- never
  \code{null}.  Dereferencing an expression of non-null type can never cause
  a null pointer exception.

  The \<@NonNull> annotation is rarely written in a program, because it is
  the default (see Section~\ref{null-defaults}).

\item[\refqualclass{checker/nullness/qual}{PolyNull}]
  indicates qualifier polymorphism.  For a description of
  \refqualclass{checker/nullness/qual}{PolyNull}, see
  Section~\ref{qualifier-polymorphism}.

\item[\refqualclass{checker/nullness/qual}{MonotonicNonNull}]
  indicates a reference that may be \code{null}, but if it ever becomes
  non-\code{null}, then it never becomes \code{null} again.  This is
  appropriate for lazily-initialized fields, among other uses.  When the
  variable is read, its type is treated as
  \refqualclass{checker/nullness/qual}{Nullable}, but when the variable is
  assigned, its type is treated as
  \refqualclass{checker/nullness/qual}{NonNull}.

  Because the Nullness Checker works intraprocedurally (it analyzes one
  method at a time), when a \code{MonotonicNonNull} field is first read within a
  method, the field cannot be assumed to be non-null.  The benefit of
  MonotonicNonNull over Nullable is its different interaction with
  flow-sensitive type qualifier refinement (Section~\ref{type-refinement}).
  After a check of a MonotonicNonNull
  field, all subsequent accesses \emph{within that method} can be assumed
  to be NonNull, even after arbitrary external method calls that have
  access to the given field.

  It is permitted to initialize a MonotonicNonNull field to null, but the
  field may not be assigned to null anywhere else in the program.  If you
  supply the \<noInitForMonotonicNonNull> lint flag (for example, supply 
  \<-Alint=noInitForMonotonicNonNull> on the command line), then
  @MonotonicNonNull fields are not allowed to have initializers.

  Use of \<@MonotonicNonNull> on a static field is a code smell:  it may
  indicate poor design.  You should consider whether it is possible to make
  the field a member field that is set in the constructor.

\end{description}

Figure~\ref{fig-nullness-hierarchy} shows part of the type hierarchy for the
Nullness type system.
(The annotations exist only at compile time; at run time, Java has no
multiple inheritance.)

\begin{figure}
\includeimage{nullness}{3.5cm}
\caption{Partial type hierarchy for the Nullness type system.
Java's \<Object> is expressed as \<@Nullable Object>.  Programmers can omit
most type qualifiers, because the default annotation
(Section~\ref{null-defaults}) is usually correct.
The Nullness Checker verifies three type hierarchies:  this one for
nullness, one for initialization (Section~\ref{initialization-checker}),
and one for map keys (Section~\ref{map-key-checker}).}
\label{fig-nullness-hierarchy}
\end{figure}


\subsection{Nullness method annotations\label{nullness-method-annotations}}

The Nullness Checker supports several annotations that specify method
behavior.  These are declaration annotations, not type annotations:  they
apply to the method itself rather than to some particular type.

\begin{description}

\item[\refqualclass{checker/nullness/qual}{RequiresNonNull}]
  indicates a method precondition:  The annotated method expects the
  specified variables (typically field references) to be non-null when the
  method is invoked.

\item[\refqualclass{checker/nullness/qual}{EnsuresNonNull}]
\item[\refqualclass{checker/nullness/qual}{EnsuresNonNullIf}]
  indicates a method postcondition.  With \<@EnsuresNonNull>, the given
  expressions are non-null after the method returns; this is useful for a
  method that initializes a field, for example.  With
  \<@EnsuresNonNullIf>, if the annotated
  method returns the given boolean value (true or false), then the given
  expressions are non-null.  See Section~\ref{conditional-nullness} and the
  Javadoc for examples of their use.

\end{description}


\subsection{Initialization qualifiers\label{initialization-qualifiers-overview}}

The Nullness Checker invokes an Initialization Checker, whose annotations indicate whether
an object is fully initialized --- that is, whether all of its fields have
been assigned.

\begin{description}
\item[\refqualclass{checker/initialization/qual}{Initialized}]
\item[\refqualclass{checker/initialization/qual}{UnknownInitialization}]
\item[\refqualclass{checker/initialization/qual}{UnderInitialization}]
\end{description}

\noindent
Use of these annotations can help you to type-check more
code.  Figure~\ref{fig-initialization-hierarchy} shows its type hierarchy.  For
details, see Section~\ref{initialization-checker}.

A slightly simpler variant, called the Rawness Initialization Checker, is also available:

\begin{description}
\item[\refqualclass{checker/nullness/qual}{Raw}]
\item[\refqualclass{checker/nullness/qual}{NonRaw}]
\item[\refqualclass{checker/nullness/qual}{PolyRaw}]
\end{description}

\noindent
Figure~\ref{fig-rawness-hierarchy} shows its type hierarchy.  For
details, see Section~\ref{initialization-rawness-checker}.


\subsection{Map key qualifiers\label{map-key-qualifiers}}

The Nullness Checker invokes a Map Key Checker, whose annotation,
\refqualclass{checker/nullness/qual}{KeyFor}, indicates that
a value is a key for a given map.  This indicates whether
\code{map.containsKey(value)} would evaluate to \code{true}.

\begin{description}
\item[\refqualclass{checker/nullness/qual}{KeyFor}]
\end{description}

Use of this annotation can help you to type-check more code.  For details,
see Section~\ref{map-key-checker}.


\section{Writing nullness annotations\label{writing-nullness-annotations}}

\subsection{Implicit qualifiers\label{nullness-implicit-qualifiers}}

As described in Section~\ref{effective-qualifier}, the Nullness Checker
adds implicit qualifiers, reducing the number of annotations that must
appear in your code.
For example, enum types are implicitly non-null, so you never need to write
\<@NonNull MyEnumType>.

For a complete description of all implicit nullness qualifiers, see the
Javadoc for \refclass{checker/nullness}{NullnessAnnotatedTypeFactory}.



\subsection{Default annotation\label{null-defaults}}

Unannotated references are treated as if they had a default annotation.
The standard defaulting rule is  ``CLIMB to top'',  described in
Section~\ref{climb-to-top}.  Its effect is to default all types to
\<@NonNull>, except that \<@Nullable> is used for casts, locals,
instanceof, and implicit bounds.  A user can choose a different defaulting
rule.

%% Cut this to shorten the section.  Most users won't care about it.
% %BEGIN LATEX
% \begin{sloppy}
% %END LATEX
% Here are three possible default rules you may wish to use.  Other rules are
% possible but are not as useful.
% \begin{itemize}
% \item
%   \refqualclass{checker/nullness/qual}{Nullable}:  Unannotated types are regarded as possibly-null, or
%   nullable.  This default is backward-compatible with Java, which permits
%   any reference to be null.  You can activate this default by writing
%   a \code{@DefaultQualifier(Nullable.class)} annotation on a
%   % package/
%   class or method
%   % /variable
%   declaration.
% \item
%   \refqualclass{checker/nullness/qual}{NonNull}:  Unannotated types are treated as non-null.
%   % This may leads to fewer annotations written in your code.
%   You can activate this
%   default via the
%   \code{@DefaultQualifier(NonNull.class)} annotation.
% \item
%   Non-null except locals (NNEL):  Unannotated types are treated as
%   \refqualclass{checker/nullness/qual}{NonNull}, \emph{except} that the
%   unannotated raw type of a local variable is treated as
%   \refqualclass{checker/nullness/qual}{Nullable}.  (Any generic arguments to a
%   local variable still default to
%   \refqualclass{checker/nullness/qual}{NonNull}.)  This is the standard
%   behavior.  You can explicitly activate this default via the
%   \code{@DefaultQualifier(value=NonNull.class,
%     locations=\discretionary{}{}{}\{DefaultLocation\discretionary{}{}{}.ALL\_EXCEPT\_LOCALS\})}
%   annotation.
% 
%   The NNEL default leads to the smallest number of explicit annotations in
%   your code~\cite{PapiACPE2008}.  It is what we recommend.  If you do not
%   explicitly specify a different default, then NNEL is the default.
% \end{itemize}
% %BEGIN LATEX
% \end{sloppy}
% %END LATEX


\subsection{Conditional nullness\label{conditional-nullness}}

The Nullness Checker supports a form of conditional nullness types, via the
\refqualclass{checker/nullness/qual}{EnsuresNonNullIf} method annotations.
The annotation on a method declares that some expressions are non-null, if
the method returns true (false, respectively).

Consider \sunjavadoc{java/lang/Class.html}{java.lang.Class}.
Method
\sunjavadoc{java/lang/Class.html#getComponentType()}{Class.getComponentType()}
may return null, but it is specified to return a non-null value if
\sunjavadoc{java/lang/Class.html#isArray()}{Class.isArray()} is
true.
You could declare this relationship in the following way (this particular
example is already
done for you in the annotated JDK that comes with the Checker Framework):

\begin{Verbatim}
  class Class {
    @EnsuresNonNullIf(expression="getComponentType()", result=true)
    public native boolean isArray();

    public native @Nullable Class<?> getComponentType();
  }
\end{Verbatim}

A client that checks that a \code{Class} reference is indeed that of an array,
can then de-reference the result of \code{Class.getComponentType} safely
without any nullness check.  The Checker Framework source code itself
uses such a pattern:

\begin{Verbatim}
    if (clazz.isArray()) {
      // no possible null dereference on the following line
      TypeMirror componentType = typeFromClass(clazz.getComponentType());
      ...
    }
\end{Verbatim}

Another example is \sunjavadoc{java/util/Queue.html#peek()}{Queue.peek}
and \sunjavadoc{java/util/Queue.html#poll()}{Queue.poll}, which return
non-null if \sunjavadoc{java/util/Collection.html#isEmpty()}{isEmpty}
returns false.

The argument to \code{@EnsuresNonNullIf} is a Java expression, including method calls
(as shown above), method formal parameters, fields, etc.; for details, see
Section~\ref{java-expressions-as-arguments}.
More examples of the use of these annotations appear in the Javadoc for
\refqualclass{checker/nullness/qual}{EnsuresNonNullIf}.


\subsection{Nullness and arrays\label{nullness-arrays}}

The components of a newly created object of reference type are all
null. Only after initialization can the array actually be considered
to contain non-null components.
Therefore, the following is not allowed:

\begin{Verbatim}
  @NonNull Object [] oa = new @NonNull Object[10]; // error
\end{Verbatim}

Instead, one creates a nullable or lazy-nonnull array, initializes
each component, and then assigns the result to a non-null array:

\begin{Verbatim}
  @MonotonicNonNull Object [] temp = new @MonotonicNonNull Object[10];
  for (int i = 0; i < temp.length; ++i) {
    temp[i] = new Object();
  }
  @SuppressWarnings("nullness")
  @NonNull Object [] oa = temp;
\end{Verbatim}

Note that the checker is currently not powerful enough to ensure that
each array component was initialized. Therefore, the last assignment
needs to be trusted:  that is, a programmer must verify that it is safe,
then write a \<@SuppressWarnings> annotation.

% TODO: explain more aspects, give more examples.


\subsection{Run-time checks for nullness\label{nullness-runtime-checks}}

When you perform a run-time check for nullness, such as \<if (x != null)
...>, then the Nullness Checker refines the type of \<x> to
\<@NonNull> within the scope of the test.  For more details, see
Section~\ref{type-refinement}.


\subsection{Additional details\label{nullness-additional-details}}

The Nullness Checker does some special checks in certain circumstances, in
order to soundly reduce the number of warnings that it produces.

For example, a call to 
\sunjavadoc{java/lang/System.html#getProperty(java.lang.String)}{System.getProperty(String)}
can return null in general, but it will not return null if the argument is
one of the built-in-keys listed in the documentation of 
\sunjavadoc{java/lang/System.html#getProperties()}{System.getProperties()}.
The Nullness Checker is aware of this fact, so you do not have to suppress
a warning for a call like \<System.getProperty("line.separator")>.  The
warning is still issued for code like this:

\begin{Verbatim}
  final String s = "line.separator";
  nonNullvar = System.getProperty(s);
\end{Verbatim}

\noindent
though that case could be handled as well, if desired.
(Suppression of the warning is, strictly speaking, not sound, because a
library that your code calls, or your code itself, could perversely change
the system properties; the Nullness Checker assumes this bizarre coding
pattern does not happen.)


\subsection{Inference of \code{@NonNull} and \code{@Nullable} annotations\label{nullness-inference}}

It can be tedious to write annotations in your code.  Tools exist that
can automatically infer annotations and insert them in your source code.
(This is different than type qualifier refinement for local variables
(Section~\ref{type-refinement}), which infers a more specific type for
local variables and uses them during type-checking but does not insert them
in your source code.  Type qualifier refinement is always enabled, no
matter how annotations on signatures got inserted in your source code.)

Your choice of tool depends on what default annotation (see
Section~\ref{null-defaults}) your code uses.  You only need one of these tools.

\begin{itemize}

\item
  Inference of \refqualclass{checker/nullness/qual}{Nullable}:
  %
  If your code uses the standard CLIMB-to-top default (Section~\ref{climb-to-top}) or
  the \refclass{checker/nullness/qual}{NonNull} default, then use the
  \ahref{\url{http://plse.cs.washington.edu/daikon/download/doc/daikon.html#AnnotateNullable}}{AnnotateNullable}
  tool of the \ahref{\url{http://plse.cs.washington.edu/daikon/}}{Daikon invariant
    detector}.

\item
  Inference of \refqualclass{checker/nullness/qual}{NonNull}:
  %
  If your code uses the Nullable default, use one of these tools:
\begin{itemize}
\item
  \ahref{\url{http://julia.scienze.univr.it/}}{Julia analyzer},
\item
  \ahref{\url{http://nit.gforge.inria.fr}}{Nit: Nullability Inference Tool},
\item
  \ahref{\url{http://jastadd.org/jastadd-tutorial-examples/non-null-types-for-java/}}{Non-null
    checker and inferencer} of the \ahref{\url{http://jastadd.org/}}{JastAdd
    Extensible Compiler}.
\end{itemize}

\end{itemize}



\section{Suppressing nullness warnings\label{suppressing-warnings-nullness}}

The Checker Framework supplies several ways to suppress warnings, most
notably the \<@SuppressWarnings("nullness")> annotation (see
Section~\ref{suppressing-warnings}).  An example use is

\begin{Verbatim}
    // might return null
    @Nullable Object getObject(...) { ... }

    void myMethod() {
      // The programmer knows that this particular call never returns null,
      // perhaps based on the arguments or the state of other objects.
      @SuppressWarnings("nullness")
      @NonNull Object o2 = getObject(...);
\end{Verbatim}


The Nullness Checker supports an additional warning suppression key,
\<nullness:generic.argument>.
Use of \<@SuppressWarnings("nullness:generic.argument")> causes the Nullness
Checker to suppress warnings related to misuse of generic type
arguments.  One use for this key is when a class is declared to take only
\<@NonNull> type arguments, but you want to instantiate the class with a
\<@Nullable> type argument, as in \code{List<@Nullable Object>}.  For a more
complete explanation of this example, see
Section~\refwithpage{faq-list-map-nonnull-typeargs}.

The Nullness Checker also permits you to use assertions or method calls to
suppress warnings; see below.

% TODO: check whether the SuppressWarnings keys are correct.


\subsection{Suppressing warnings with assertions and method calls\label{suppressing-warnings-with-assertions}}

Occasionally, it is inconvenient or
verbose to use the \<@SuppressWarnings> annotation.  For example, Java does
not permit annotations such as \<@SuppressWarnings> to appear on statements.
In such cases, you may be able to use the \<@AssumeAssertion> string in
an \<assert> message (see Section~\ref{assumeassertion}).

For situations when all of these approaches are inconvenient,
the Nullness Checker provides an additional way to suppress warnings:
via the \<castNonNull> method.  This is
appropriate when the Nullness Checker issues a warning, but the programmer
knows for sure that the warning is a false positive, because the value
cannot ever be null at run time.

\begin{enumerate}
\item
  Use an assertion.  If the string ``\<@AssumeAssertion(nullness)>''
  appears in the message, then the Nullness Checker treats the
  assertion as suppressing a warning and assumes that the assertion always
  succeeds.  For example, the checker assumes that no null pointer
  exception can occur in code such as
\begin{Verbatim}
  assert x != null : "@AssumeAssertion(nullness)";
  ... x.f ...
\end{Verbatim}

  If the string ``\<@AssumeAssertion(nullness)>'' does not appear in the
  assertion message, then the Nullness Checker treats the assertion as being
  used for defensive programming, and it warns if the method might throw a
  nullness-related exception.

  A downside of putting the string in the assertion message is that if the
  assertion ever fails, then a user might see the string and be confused.
  But the string should only be used if the programmer has reasoned that
  the assertion can never fail.

% (Another way of stating the Nullness Checker's use of assertions is as an
% additional caveat to the guarantees provided by a checker
% (Section~\ref{checker-guarantees}).  The Nullness Checker prevents null
% pointer errors in your code under the assumption that assertions are
% enabled, and it does not guarantee that all of your assertions succeed.)


\item
  Use the \refmethod{checker/nullness}{NullnessUtils}{castNonNull}{-T-} method.

The Nullness
 Checker considers both the return value, and also the argument, to
 be non-null after the method call.  Therefore, the
 \<castNonNull> method can be used either as a cast expression or
 as a statement.  The Nullness Checker issues no warnings in any of
the following code:

\begin{Verbatim}
  // one way to use castNonNull as a cast:
  @NonNull String s = castNonNull(possiblyNull1);

  // another way to use castNonNull as a cast:
  castNonNull(possiblyNull2).toString();

  // one way to use castNonNull as a statement:
  castNonNull(possiblyNull3);
  possiblyNull3.toString();
\end{Verbatim}

  The method also throws \<AssertionError> if Java assertions are enabled and
  the argument is \<null>.  However, it is not intended for general defensive
  programming; see Section~\ref{defensive-programming}.

  A potential disadvantage of using the \<castNonNull> method is that your
  code becomes dependent on the Checker Framework at run time as well as at
  compile time.  You can avoid this by copying the implementation of
  \<castNonNull> into your own code, and possibly renaming it if you do not
  like the name.  Be sure to retain the documentation that indicates that
  your copy is intended for use only to suppress warnings and not for
  defensive programming.  See Section~\ref{defensive-programming} for an
  explanation of the distinction.

%% AssertParametersNonNull was removed. Use EnsuresNonNull instead.
%% \item
%%   Use the \refqualclass{checker/nullness/qual}{AssertParametersNonNull}
%%   annotation.  It is used on \<castNonNull>, and may be used on other
%%   methods with the same semantics; it should probably never be used in any
%%   other situation.

\end{enumerate}


\subsection{Suppressing warnings on nullness-checking routines and defensive programming\label{defensive-programming}}

%% Work this in
% As explained in Section~\ref{annotate-normal-behavior}, annotations should
% indicate normal behavior that will not cause an exception.
%
% TODO: discuss how to write your own, and why the default doesn't have
% assert or checking methods suppress warnings.


One way to suppress warnings in the Nullness Checker is to use
method \code{castNonNull}.
(Section~\ref{suppressing-warnings-with-assertions} gives other techniques.)

This section explains why the Nullness Checker introduces a new method
rather than re-using the \<assert> statement (as in
\<assert x != null>) or an existing method such as:

\begin{Verbatim}
  org.junit.Assert.assertNotNull(Object)
  com.google.common.base.Preconditions.checkNotNull(Object)
\end{Verbatim}

In each case, the assertion or method indicates an application invariant --- a
fact that should always be true.  There are two distinct reasons a
programmer may have written the invariant, depending on whether the
programmer is 100\% sure that the application invariant holds.

\begin{enumerate}
\item
  A programmer might write it as \textbf{defensive programming}.  This causes
  the program to throw an exception, which is useful for debugging because
  it gives an earlier run-time indication of the error.
  A programmer would use an assertion in this way if the programmer is not
  100\% sure that the application invariant holds.

  % , or even to document what the program
  % is intended to do.

\item
  A programmer might write it to \textbf{suppress} false positive
  \textbf{warning messages} from a checker.  A programmer would use an
  assertion this way if the programmer is 100\% sure that the application
  invariant holds, and the reference can never be null at run time.

\end{enumerate}

With assertions and existing methods like JUnit's \<assertNotNull>, there
is no way of knowing the programmer's intent in using the method.
Different programmers or codebases may use them in different ways.
Guessing wrong would make the Nullness Checker less useful, because it
would either miss real errors or issue warnings where there is no real
error.  Also, different checking tools issue different false warnings that
need to be suppressed, so warning suppression needs to be customized for
each tool rather than inferred from general-purpose code.


As an example of using assertions for defensive programming, some style
guides suggest using assertions or method calls to indicate nullness.  A
programmer might write

\begin{Verbatim}
    String s = ...
    assert s != null;    // or:  assertNotNull(s);   or: checkNotNull(s);
    ... Double.valueOf(s) ...
\end{Verbatim}

A programming error might cause \<s> to be null, in which case the code
would throw an exception at run time.
If the assertion caused the Nullness Checker to assume that \<s> is not
\<null>, then the Nullness Checker would issue no warning for this code.
That would be undesirable, because the whole purpose of the Nullness
Checker is to give a compile-time warning about possible run-time
exceptions.  Furthermore, if the programmer uses assertions for defensive
programming systematically throughout the codebase, then many useful
Nullness Checker warnings would be suppressed.


Because it is important to distinguish between the two uses of assertions
(defensive programming vs.~suppressing warnings), the Checker Framework
introduces the \refmethod{checker/nullness}{NullnessUtils}{castNonNull}{-T-} method.
Unlike existing assertions and
methods, \<castNonNull> is intended only to suppress false warnings that are
issued by the Nullness Checker, not for defensive programming.

If you know that a particular codebase uses
% the \<assert> statement or
a nullness-checking method not for defensive programming but to indicate
facts that are guaranteed to be true (that is, these assertions will never
fail at run time), then you can cause the Nullness Checker to suppress
warnings related to them, just as it does for \<castNonNull>.
Annotate its definition just as
\refmethod{checker/nullness}{NullnessUtils}{castNonNull}{-T-} is annotated (see the
source code for the Checker Framework).
% TODO:
% For an assert statement, XXXXX.
Also, be sure to document the intention in the method's Javadoc, so that
programmers do not
accidentally misuse it for defensive programming.


If you are annotating a codebase that already contains precondition checks,
such as:

\begin{Verbatim}
  public String get(String key, String def) {
    checkNotNull(key, "key"); //NOI18N
    ...
  }
\end{Verbatim}

\noindent
then you should mark the appropriate parameter as \<@NonNull> (which is the
default).  This will prevent the checker from issuing a warning about the
\<checkNotNull> call.


\section{Examples\label{nullness-example}}

\subsection{Tiny examples\label{nullness-tiny-examples}}

To try the Nullness Checker on a source file that uses the \refqualclass{checker/nullness/qual}{NonNull} qualifier,
use the following command (where \code{javac} is the Checker Framework compiler that
is distributed with the Checker Framework):

\begin{Verbatim}
  javac -processor org.checkerframework.checker.nullness.NullnessChecker examples/NullnessExample.java
\end{Verbatim}

\noindent
Compilation will complete without warnings.

To see the checker warn about incorrect usage of annotations (and therefore the
possibility of a null pointer exception at run time), use the following command:

\begin{Verbatim}
  javac -processor org.checkerframework.checker.nullness.NullnessChecker examples/NullnessExampleWithWarnings.java
\end{Verbatim}

\noindent
The compiler will issue two warnings regarding violation of the semantics of
\refqualclass{checker/nullness/qual}{NonNull}.
% in the \code{NonNullExampleWithWarnings.java} file.


\subsection{Annotated library\label{nullness-annotated-library}}

Some libraries that are annotated with nullness qualifiers are:

\begin{itemize}
\item
The Nullness Checker itself.

\item
The
\ahref{\url{http://code.google.com/p/plume-lib/}}{Plume-lib library}.
Run the command \code{make check-nullness}.


\item
The
\ahref{\url{http://plse.cs.washington.edu/daikon/}}{Daikon invariant detector}.
Run the command \code{make check-nullness}.

% \item
% The annotation scene library.
% To run the Nullness Checker on the annotation scene library,
% % TODO: how does one do this?
% first download the scene library suite (which includes build
% dependencies for the scene library as well as its source code) and extract it
% into your Checker Framework installation. The checker can then be run on the annotation
% scene library with Apache Ant using the following commands:
%
% \begin{Verbatim}
%   cd checkers
%   ant -f scene-lib-test.xml
% \end{Verbatim}
%
% % \noindent
% % where \code{checkers} is the location of the Checker Framework installation.
%
% You can view the annotated source code, which contains \refqualclass{checker/nullness/qual}{NonNull} annotations, in
% the
% %BEGIN LATEX
% \begin{smaller}
% %END LATEX
% \code{checkers/scene-lib-test/src/annotations/}
% %BEGIN LATEX
% \end{smaller}
% %END LATEX
% directory.

\end{itemize}


\section{Tips for getting started\label{nullness-getting-started}}

Here are some tips about getting started using the Nullness Checker on a
legacy codebase.  For more generic advice (not specific to the Nullness
Checker), see Section~\ref{get-started-with-legacy-code}.

Your goal is to add \refqualclass{checker/nullness/qual}{Nullable} annotations
to the types of any variables that can be null.  (The default is to assume
that a variable is non-null unless it has a \code{@Nullable} annotation.)
Then, you will run the Nullness Checker.  Each of its errors indicates
either a possible null pointer exception, or a wrong/missing annotation.
When there are no more warnings from the checker, you are done!

We recommend that you start by searching the code for occurrences of
\code{null} in the following locations; when you find one, write the
corresponding annotation:

\begin{itemize}
\item
  in Javadoc:  add \code{@Nullable} annotations to method signatures (parameters and return types).
% Search for "\*.*\bnull\b"
\item
  \code{return null}:  add a \code{@Nullable} annotation to the return type
  of the given method.
% Search for "return null" and "return.*? null" and "return.*: null"
\item
  \code{\emph{param} == null}:  when a formal parameter is compared to
  \code{null}, then in most cases you can add a \code{@Nullable} annotation
  to the formal parameter's type
\item
  \code{\emph{TypeName} \emph{field} = null;}:  when a field is initialized to
  \code{null} in its declaration, then it needs either a
  \refqualclass{checker/nullness/qual}{Nullable} or a
  \refqualclass{checker/nullness/qual}{MonotonicNonNull} annotation.  If the field
  is always set to a non-null value in the constructor, then you can just
  change the declaration to \code{\emph{Type} \emph{field};}, without an
  initializer, and write no type annotation (because the default is
  \<@NonNull>).
\item
  declarations of \<contains>, \<containsKey>, \<containsValue>, \<equals>,
  \<get>, \<indexOf>, \<lastIndexOf>, and \<remove> (with \<Object> as the
  argument type):
  change the argument type to \<@Nullable Object>; for \<remove>, also change
  the return type to \<@Nullable Object>.
% Emacs code for the argument types:
% ;;NOT: (tags-query-replace " apply(Object " " apply(/*@Nullable*/ Object ")
% (tags-query-replace " \\(get\\|equals\\|remove\\|contains\\|containsValue\\|containsKey\\|indexOf\\|lastIndexOf\\)(Object " " \\1(/*@Nullable*/ Object ")

\end{itemize}

\noindent
You should ignore all other occurrences of \code{null} within a method
body.  In particular, you (almost) never need to annotate local variables.

Only after this step should you run \code{ant} to invoke
the Nullness Checker.  The reason is that it is quicker to search for
places to change than to repeatedly run the checker and fix the errors it
tells you about, one at a time.

Here are some other tips:
\begin{itemize}
\item
    \begin{sloppypar}
    In any file where you write an annotation such as \code{@Nullable},
    don't forget to add \code{import org.checkerframework.checker.nullness.qual.*;}.
    \end{sloppypar}
\item
    To indicate an array that can be null, write, for example: \code{int
      @Nullable []}. \\
    By contrast, \code{@Nullable Object []} means a non-null array that
    contains possibly-null objects.
\item
    If you know that a particular variable is definitely not null, but the
    Nullness Checker cannot figure it out, then you can tell it by writing
    an assertion (see Section~\ref{suppressing-warnings}):
\begin{Verbatim}
assert var != null : "@SuppressWarnings(nullness)";
\end{Verbatim}
\item
    To indicate that a routine returns the same value every time it is
    called, use \refqualclass{dataflow/qual}{Pure} (see Section~\ref{type-refinement-purity}).
\item
    To indicate a method precondition (a contract stating the conditions
    under which a client is allowed to call it), you can use annotations
    such as \refqualclass{checker/nullness/qual}{RequiresNonNull} (see Section~\ref{nullness-method-annotations}).
\end{itemize}



\section{Other tools for nullness checking\label{nullness-related-work}}

\newcommand{\linktoNonNull}{\refclass{checker/nullness/qual}{NonNull}}
\newcommand{\linktoNullable}{\refclass{checker/nullness/qual}{Nullable}}

The Checker Framework's nullness annotations are similar to annotations used
in IntelliJ IDEA, FindBugs, JML, the JSR 305 proposal, NetBeans, and other tools.  Also
see Section~\ref{other-tools} for a comparison to other tools.

You might prefer to use the Checker Framework because it has a more
powerful analysis that can warn you about more null pointer errors in your
code.

If your code is already annotated with a different nullness
annotation, you can reuse that effort.  The Checker Framework comes with
cleanroom re-implementations of annotations from other tools.  It treats
them exactly as if you had written the corresponding annotation from the
Nullness Checker, as described in Figure~\ref{fig-nullness-refactoring}.


% These lists should be kept in sync with NullnessAnnotatedTypeFactory.java .
\begin{figure}
\begin{center}
% The ~ around the text makes things look better in Hevea (and not terrible
% in LaTeX).
\begin{tabular}{ll}
\begin{tabular}{|l|}
\hline
 ~com.sun.istack.internal.NotNull~ \\ \hline
 ~edu.umd.cs.findbugs.annotations.NonNull~ \\ \hline
 ~javax.annotation.Nonnull~ \\ \hline
 ~javax.validation.constraints.NotNull~ \\ \hline
 ~org.eclipse.jdt.annotation.NonNull~ \\ \hline
 ~org.jetbrains.annotations.NotNull~ \\ \hline
 ~org.netbeans.api.annotations.common.NonNull~ \\ \hline
 ~org.jmlspecs.annotation.NonNull~ \\ \hline
 ~android.support.annotation.NonNull~ \\ \hline
\end{tabular}
&
$\Rightarrow$
~org.checkerframework.checker.nullness.qual.NonNull~
\\
\
\\
\begin{tabular}{|l|l|}
\hline
 ~com.sun.istack.internal.Nullable~ \\ \hline
 ~edu.umd.cs.findbugs.annotations.Nullable~ \\ \hline
 ~edu.umd.cs.findbugs.annotations.CheckForNull~ \\ \hline
 ~edu.umd.cs.findbugs.annotations.UnknownNullness~ \\ \hline
 ~javax.annotation.Nullable~ \\ \hline
 ~javax.annotation.CheckForNull~ \\ \hline
 ~org.eclipse.jdt.annotation.Nullable~ \\ \hline
 ~org.jetbrains.annotations.Nullable~ \\ \hline
 ~org.netbeans.api.annotations.common.CheckForNull~ \\ \hline
 ~org.netbeans.api.annotations.common.NullAllowed~ \\ \hline
 ~org.netbeans.api.annotations.common.NullUnknown~ \\ \hline
 ~org.jmlspecs.annotation.Nullable~ \\ \hline
 ~android.support.annotation.Nullable~ \\ \hline
\end{tabular}
&
$\Rightarrow$
~org.checkerframework.checker.nullness.qual.Nullable~
\end{tabular}
\end{center}
%BEGIN LATEX
\vspace{-1.5\baselineskip}
%END LATEX
\caption{Correspondence between other nullness annotations and the
  Checker Framework's annotations.}
\label{fig-nullness-refactoring}
\end{figure}

%% Removed, because it's tedious and should be obvious to a decent programmer.
% Your IDE may be able to do that for you.  Alternately, do the following:
% \begin{enumerate}
% \item
%   replace \<@Nonnull> by \<@NonNull> (note capitalization difference)
% \item
%   replace \<@CheckForNull> by \<@Nullable>
% \item
%   replace \<@UnknownNullness> by \<@Nullable>
% \item
%   convert each single-type import statement (without a ``\<*>'' character)
%    according to the table above.
% \item
%   convert each on-demand import statements, such as ``\<import
%    edu.umd.cs.findbugs.annotations.*;>''.
% \begin{itemize}
%    \item
%   One approach is to change it into a set of single-type imports,
%       then convert the relevant ones.
%    \item
%   Another approach is to change it according to the table above, then
%       try to compile and re-introduce the single-type imports as necessary.
% \end{itemize}
%    These approaches let you continue to use other annotations in the
%    \<edu.umd.cs.findbugs.annotations> package, even though you are not using
%    its nullness annotations.
% \end{enumerate}


Alternately, the Checker Framework can process those other annotations (as
well as its own, if they also appear in your program).  The Checker
Framework has its own definition of the annotations on the left side of
Figure~\ref{fig-nullness-refactoring}, so that they can be used as type
qualifiers.  The Checker Framework interprets them according to the right
side of Figure~\ref{fig-nullness-refactoring}.

The Checker Framework may issue more or fewer errors than another tool.
This is expected, since each tool uses a different analysis.  Remember that
the Checker Framework aims at soundness:  it aims to never miss a possible
null dereference, while at the same time limiting false reports.  Also,
note FindBugs's non-standard meaning for \<@Nullable>
(Section~\ref{findbugs-nullable}).

Because some of the names are the same (\<NonNull>, \<Nullable>), you can
import at most one of the annotations with
conflicting names; the other(s) must be written out fully rather than
imported.

Note that some older tools interpret array and vararg declarations
inconsistently with the Java specification.  For example, they might
interpret \<@NonNull Object []> as ``non-null array of objects'', rather
than as ``array of non-null objects'' which is the correct Java
interpretation.  Such an interpretation is unfortunate and confusing.  See
Section~\ref{faq-array-syntax-meaning} for some more details about this
issue.


\subsection{Which tool is right for you?\label{choosing-nullness-tool}}

Different tools are appropriate in different circumstances.  Here is a
brief comparison with FindBugs, but similar points apply to other tools.

The Checker Framework has a more powerful nullness analysis; FindBugs misses
some real
errors.  However, FindBugs does not require you to annotate your code as
thoroughly as the Checker Framework does.  Depending on the importance of
your code, you may desire:  no nullness checking, the cursory checking of
FindBugs, or the thorough checking of the Checker Framework.  You might
even want to ensure that both tools run, for example if your coworkers or
some other organization are still using FindBugs.  If you know that you
will eventually want to use the Checker Framework, there is no point using
FindBugs first; it is easier to go straight to using the Checker Framework.

FindBugs can find other errors in addition to nullness errors; here
we focus on its nullness checks.  Even if you use FindBugs for its other
features, you may want to use the Checker Framework for analyses that can
be expressed as pluggable type-checking, such as detecting nullness errors.

Regardless of whether you wish to use the FindBugs nullness analysis, you
may continue running all of the other FindBugs analyses at the same time as
the Checker Framework; there are no interactions among them.

If FindBugs (or any other tool) discovers a nullness error that the Checker
Framework does not, please report it to us (see
Section~\ref{reporting-bugs}) so that we can enhance the Checker Framework.



\subsection{Incompatibility note about FindBugs \tt{@Nullable}\label{findbugs-nullable}}

FindBugs has a non-standard definition of \<@Nullable>.  FindBugs's treatment is not
documented in its own
\ahref{\url{http://findbugs.sourceforge.net/api/edu/umd/cs/findbugs/annotations/Nullable.html}}{Javadoc};
it is different from the definition of \<@Nullable> in every other tool for
nullness analysis; it means the same thing as \<@NonNull> when applied to a
formal parameter; and it invariably surprises programmers.  Thus, FindBugs's
\<@Nullable> is detrimental rather than useful as documentation.
In practice, your best bet is to not rely on FindBugs for nullness analysis,
even if you find FindBugs useful for other purposes.

You can skip the rest of this section unless you wish to learn more details.

FindBugs suppresses all warnings at uses of a \<@Nullable> variable.
(You have to use \<@CheckForNull> to
indicate a nullable variable that FindBugs should check.)  For example:

\begin{Verbatim}
     // declare getObject() to possibly return null
     @Nullable Object getObject() { ... }

     void myMethod() {
       @Nullable Object o = getObject();
       // FindBugs issues no warning about calling toString on a possibly-null reference!
       o.toString();
     }
\end{Verbatim}

\noindent
The Checker Framework does not emulate this non-standard behavior of
FindBugs, even if the code uses FindBugs annotations.

With FindBugs, you annotate a declaration, which suppresses checking at
\emph{all} client uses, even the places that you want to check.
It is better to suppress warnings at only the specific client uses
where the value is known to be non-null; the Checker Framework supports
this, if you write \<@SuppressWarnings> at the client uses.
The Checker Framework also supports suppressing checking at all client uses,
by writing a \<@SuppressWarnings> annotation at the declaration site.
Thus, the Checker Framework supports both use cases, whereas FindBugs
supports only one and gives the programmer less flexibility.

In general, the Checker Framework will issue more warnings than FindBugs,
and some of them may be about real bugs in your program.
See Section~\ref{suppressing-warnings-nullness} for information about
suppressing nullness warnings.

(FindBugs made a poor choice of names.  The choice of names should make a
clear distinction between annotations that specify whether a reference is
null, and annotations that suppress false warnings.  The choice of names
should also have been consistent for other tools, and intuitively clear to
programmers.  The FindBugs choices make the FindBugs annotations less
helpful to people, and much less useful for other tools.  As a separate
issue, the FindBugs
analysis is also very imprecise.  For type-related analyses, it is best to
stay away from the FindBugs nullness annotations, and use a more capable
tool like the Checker Framework.)



% As background, here is an explanation of the (sometimes surprising)
% semantics of the FindBugs nullness annotations.
%
%  * edu.umd.cs.findbugs.annotations.NonNull     javax.annotation.Nonnull
%    These mean the obvious thing:   the reference is never null.
%
%  * edu.umd.cs.findbugs.annotations.Nullable     javax.annotation.Nullable
%    This means that the value may be null, but that *all warnings should be
%    suppressed* regarding its use.  In other words, the value is really
%    nullable, but clients should treat it as non-null.  For example:
%
%      // declare getObject() to possibly return null
%      @Nullable Object getObject() { ... }
%
%      // FindBugs issues no warning about calling toString on a possibly-null reference
%      getObject().toString();
%
%    In the Checker Framework, this corresponds to declaring the method
%    return value as @Nullable, then suppressing warnings at client uses
%    that are known to be non-null.  An easy way to suppress the warning
%    is by adding an assert statement about the return value.
%
%    (Alternately, you could declare the method return value as @NonNull, and
%    suppress warnings within the method definition where it returns null,
%    but this approach is not recommended because the @NonNull annotation on
%    the return value would be misleading, and warnings should be suppressed
%    at particular sites where they are known to be unnecessary, not at all
%    call sites whatsoever.)
%
%  * edu.umd.cs.findbugs.annotations.CheckForNull      javax.annotation.CheckForNull
%    This means that the value may be null.  To avoid a NullPointerException,
%    every client should check nullness before dereferencing the value.
%    In the Checker Framework, this corresponds to @Nullable.


\section{Initialization Checker\label{initialization-checker}}

Every object's fields start out as null.  By the time the constructor
finishes executing, the \<@NonNull> fields have been set to a different
value.  Your code can suffer a NullPointerException when using a
\<@NonNull> field, if your code uses the field during initialization.
The Nullness Checker prevents this problem by warning you anytime that you
may be accessing an uninitialized field.  This check is useful because it
prevents errors in your code.  However, the analysis can be confusing to
understand.  If you wish to disable the initialization checks, pass the
command-line argument \<-AsuppressWarnings=uninitialized> when running the
Nullness Checker.  You will no longer get a guarantee of no null pointer
exceptions, but you can still use the Nullness Checker to find most of the
null pointer problems in your code.


An object is partially initialized from the time that its constructor starts until its constructor
finishes.  This is relevant to the Nullness Checker because while the
constructor is executing --- that is, before initialization completes ---
a \<@NonNull>
field may be observed to be null, until that field is set.  In
particular, the Nullness Checker issues a warning for code like this:

\begin{Verbatim}
  public class MyClass {
    private @NonNull Object f;
    public MyClass(int x, int y) {
      // Error because constructor contains no assignment to this.f.
      // By the time the constructor exits, f must be initialized to a non-null value.
    }
    public MyClass(int x) {
      // Error because this.f is accessed before f is initialized.
      // At the beginning of the constructor's execution, accessing this.f
      // yields null, even though field f has a non-null type.
      this.f.toString();
    }
    public MyClass(int x, int y, int z) {
      m();
    }
    public void m() {
      // Error because this.f is accessed before f is initialized,
      // even though the access is not in a constructor.
      // When m is called from the constructor, accessing f yields null,
      // even though field f has a non-null type.
      this.f.toString();
    }
\end{Verbatim}

\noindent
When a field \<f> is declared with a \refqualclass{checker/nullness/qual}{NonNull}
type, then code can depend on the fact that the field is not \<null>.
However, this guarantee does not hold for a partially-initialized object.

The Nullness Checker uses three annotations to indicate whether an object
is initialized (all its \<@NonNull> fields have been assigned), under
initialization (its constructor is currently executing), or its
initialization state is unknown.

These distinctions are mostly relevant within the constructor, or for
references to \code{this} that escape the constructor (say, by being stored
in a field or passed to a method before initialization is complete).  
Use of initialization annotations is rare in most code.

The most common use for the \<@UnderInitialization> annotation is for a
helper routine that is called by constructor.  For example:

\begin{Verbatim}
  class MyClass {
    Object field1;
    Object field2;
    Object field3;

    public MyClass(String arg1) {
      this.field1 = arg1;
      init_other_fields();
    }

    // A helper routine that initializes all the fields other than field1.
    @EnsuresNonNull({"field2", "field3"})
    private void init_other_fields(@UnderInitialization(MyClass.class) MyClass this) {
      field2 = new Object();
      field3 = new Object();
    }
  }
\end{Verbatim}

For compatibility with Java 6 and 7, you can write the receiver
parameter in comments (see Section~\ref{annotations-in-comments}):
\begin{Verbatim}
    private void init_other_fields(/*>>>@UnderInitialization(MyClass.class) MyClass this*/) {
\end{Verbatim}

% Most readers can
% skip this section on first reading; you can return to it once you have
% mastered the rest of the Nullness Checker.

\subsection{Initialization qualifiers\label{initialization-qualifiers}}

\begin{figure}
\includeimage{initialization}{4cm}
\caption{Partial type hierarchy for the Initialization type system.
  \<@UnknownInitialization> and \<@UnderInitialization> each take an
  optional parameter indicating how far initialization has proceeded, and
  the right side of the figure illustrates its type hierarchy in more detail.}
\label{fig-initialization-hierarchy}
\end{figure}

The initialization hierarchy is shown in Figure~\ref{fig-initialization-hierarchy}.
The initialization hierarchy contains these qualifiers:

\begin{description}

\item[\refqualclass{checker/initialization/qual}{Initialized}]
  indicates a type that contains a fully-initialized object.  \code{Initialized}
  is the default, so there is little need for a programmer to write this
  explicitly.

\item[\refqualclass{checker/initialization/qual}{UnknownInitialization}]
  indicates a type that may contain a partially-initialized object.  In a
  partially-initialized object, fields that are annotated as
  \refqualclass{checker/nullness/qual}{NonNull} may be null because the field
  has not yet been assigned.

  \<@UnknownInitialization> takes a parameter that is the class the object
  is definitely initialized up to.  For instance, the type
  \<@UnknownInitialization(Foo.class)> denotes an object in which every
  fields declared in \<Foo> or its superclasses is initialized, but other
  fields might not be.
  Just \<@UnknownInitialization> is equivalent to
  \<@UnknownInitialization(Object.class)>.

\item[\refqualclass{checker/initialization/qual}{UnderInitialization}]
  indicates a type that contains a partially-initialized object that is
  under initialization --- that is, its constructor is currently executing.
  It is otherwise the same as \<@UnknownInitialization>.  Within the
  constructor, \code{this} has
  \refqualclass{checker/initialization/qual}{UnderInitialization} type until
  all the \code{@NonNull} fields have been assigned.

\end{description}

  A partially-initialized object (\code{this} in a constructor) may be
  passed to a helper method or stored in a variable; if so, the method
  receiver, or the field, would have to be annotated as
  \<@UnknownInitialization> or as \<@UnderInitialization>.

% However, if the constructor makes
% a method call (passing \code{this} as a parameter or the receiver), then
% the called method could observe the object in an illegal state.

If a reference has
\code{@UnknownInitialization} or \code{@UnderInitialization} type, then all of its \code{@NonNull} fields are treated as
\refqualclass{checker/nullness/qual}{MonotonicNonNull}:  when read, they are
treated as being \refqualclass{checker/nullness/qual}{Nullable}, but when
written, they are treated as being
\refqualclass{checker/nullness/qual}{NonNull}.

The initialization hierarchy is orthogonal to the nullness hierarchy.  It
is legal for a reference to be \<@NonNull @UnderInitialization>, \<@Nullable @UnderInitialization>,
\<@NonNull @Initialized>, or \<@Nullable @Initialized>.  The nullness hierarchy tells
you about the reference itself:  might the reference be null?  The initialization
hierarchy tells you about the \<@NonNull> fields in the referred-to object:
might those fields be temporarily null in contravention of their
type annotation?
% It's a figure rather than appearing inline so as not to span page breaks
% in the printed manual.
Figure~\ref{fig-initialization-examples} contains some examples.

\begin{figure}
\begin{tabular}{l|l|l}
Declarations & Expression & Expression's nullness type, or checker error \\ \hline
\begin{minipage}{1.5in}
\begin{Verbatim}
class C {
  @NonNull Object f;
  @Nullable Object g;
  ...
}
\end{Verbatim}
\end{minipage} & & \\ \cline{2-3}
\<@NonNull @Initialized C a;>
& \<a> & \<@NonNull> \\
& \<a.f> & \<@NonNull> \\
& \<a.g> & \<@Nullable> \\ \cline{2-3}
\<@NonNull @UnderInitialization C b;>
& \<b> & \<@NonNull> \\
& \<b.f> & \<@MonotonicNonNull> \\
& \<b.g> & \<@Nullable> \\ \cline{2-3}
\<@Nullable @Initialized C c;>
& \<c> & \<@Nullable> \\
& \<c.f> & error: deref of nullable \\
& \<c.g> & error: deref of nullable \\ \cline{2-3}
\<@Nullable @UnderInitialization C d;>
& \<d> & \<@Nullable> \\
& \<d.f> & error: deref of nullable \\
& \<d.g> & error: deref of nullable \\
\end{tabular}
\caption{Examples of the interaction between nullness and initialization.
  Declarations are shown at the left for reference, but the focus of the
  table is the expressions and their nullness type or error.}
\label{fig-initialization-examples}
\end{figure}


% Does our implementation handle static fields soundly?  NO!  See issue
% #105.  Maybe document this?


\subsection{How an object becomes initialized\label{becoming-initialized}}

Within the constructor,
\code{this} starts out with \refqualclass{checker/initialization/qual}{UnderInitialization} type.
As soon as all of the \refqualclass{checker/nullness/qual}{NonNull} fields
have been initialized, then \code{this} is treated as initialized.
(See
Section~\ref{becoming-initialized-clarification} for a slight clarification of
this rule.)

The Initialization Checker issues an error if the constructor fails to initialize
any \code{@NonNull} field.  This ensures that the object is in a legal (initialized)
state by the time that the constructor exits.
\urldef{\jlsdefiniteassignmenturl}{\url}{https://docs.oracle.com/javase/specs/jls/se7/html/jls-16.html}
\urldef{\jlsfinalvariablesurl}{\url}{https://docs.oracle.com/javase/specs/jls/se7/html/jls-4.html#jls-4.12.4}
This is different than Java's test for definite assignment (see
\ahref{\jlsdefiniteassignmenturl}{JLS ch.16}),
% , which requires that local
% variables (and blank \code{final} fields) must be assigned.  Java does not
% require that non-\code{final} fields be assigned, since
which does not apply to fields (except blank final ones, defined in
\ahref{\jlsdefiniteassignmenturl}{JLS \S 4.12.4}) because fields
have a default value of null.


All \code{@NonNull} fields must either have a
default in the field declaration, or be assigned in the constructor or in a
helper method that the constructor calls.  If
your code initializes (some) fields in a helper method, you will need to
annotate the helper method with an annotation such as
\refqualclass{checker/nullness/qual}{EnsuresNonNull}\code{(\{"field1", "field2"\})}
for all the fields that the helper method assigns.
It's a bit odd, but you use that same annotation, \code{@EnsuresNonNull},
to indicate that a primitive field has its value set in a helper method,
which is relevant when you supply the \code{-Alint=uninitialized}
command-line option (see Section~\ref{lint-uninitialized-section}).

% TODO:
% We need an
%   @EnsuresInitialized("b")
% that is analogous to
%   @EnsuresNonNull("b")
% when we are dealing with a field of primitive type.
% But, for now just use the same annotation, @EnsuresNonNull, for both purposes.


\subsection{Partial initialization\label{partial-initialization}}

So far, we have discussed initialization as if it is an all-or-nothing property:
an object is non-initialized until initialization completes, and then it is initialized.  The full truth is a bit more complex:  during the
initialization process an object can be partially initialized, and as the
object's superclass constructors complete, its initialization status is updated.  The
Initialization Checker lets you express such properties when necessary.

Consider a simple example:

\begin{Verbatim}
class A {
  Object a;
  A() {
    a = new Object();
  }
}
class B extends A {
  Object b;
  B() {
    super();
    b = new Object();
  }
}
\end{Verbatim}

Consider what happens during execution of \<new B()>.

\begin{enumerate}
\item \<B>'s constructor begins to execute.  At this point, neither the
  fields of \<A> nor those of \<B> have been initialized yet.
\item \<B>'s constructor calls \<A>'s constructor, which begins to execute.
  No fields of \<A> nor of \<B> have been initialized yet.
\item \<A>'s constructor completes.  Now, all the fields of \<A> have been
  initialized, and their invariants (such as that field \<a> is non-null) can be
  depended on.  However, because \<B>'s constructor has not yet completed
  executing, the object being constructed is not yet fully initialized.
  When treated as an \<A> (e.g., if only the \<A> fields are accessed), the
  object is initialized, but when treated as a \<B>, the object
  is still non-initialized.
\item \<B>'s constructor completes.  The object is initialized when treated
  as an \<A> or a \<B>.  (And, the object is fully initialized
   if \<B>'s constructor was invoked via a \<new B()>.  But the type system
   cannot assume that -- there might be a \<class C extends B \{
  ... \}>, and \<B>'s constructor might have been invoked from that.)
\end{enumerate}

At any moment during initialization, the superclasses of a given class
can be divided into those that have completed initialization and those that
have not yet completed initialization.  More precisely, at any moment there
is a point in the class hierarchy such that all the classes above that
point are fully initialized, and all those below it are not yet
initialized.  As initialization proceeds, this dividing line between the
initialized and uninitialized classes moves down the type hierarchy.

The Nullness Checker lets you indicate where the dividing line is between
the initialized and non-initialized classes.
The \<@UnderInitialization(\emph{classliteral})>
indicates the first class that is known to be fully initialized.
When you write \refqualclass{checker/initialization/qual}{UnderInitialization}\code{(OtherClass.class) MyClass x;}, that
means that variable \<x> is initialized for \<OtherClass> and its
superclasses, and \<x> is (possibly) uninitialized for \<MyClass> and all subclasses.

\label{becoming-initialized-clarification}

We can now state a clarification of Section~\ref{becoming-initialized}'s rule
for an object becoming initialized.
As soon as all of the \refqualclass{checker/nullness/qual}{NonNull} fields
in class $C$ have been initialized, then \code{this} is treated as
\refqualclass{checker/initialization/qual}{UnderInitialization}\code{(\emph{C})}, rather than
treated as simply 
\refqualclass{checker/initialization/qual}{Initialized}.

The example above lists 4 moments during construction.  At those moments,
the type of the object being constructed is:

\begin{enumerate}
\item
  \<@UnderInitialization B>
\item
  \<@UnderInitialization A>
\item
  \<@UnderInitialization(A.class) A>
  %% Not quite equivalent because the Java (non-qualified) type differs
  % ; equivalently, \<@UnderInitialization B>
\item
  \<@UnderInitialization(B.class) B>
\end{enumerate}

% \paragraph{Example}
% 
% As another example, consider the following 12 declarations, where class A
% extends Object and class B extends A:
% 
% \begin{Verbatim}
%     @UnderInitialization(Object.class) Object uOo;
%     @Initialized Object o;
% 
%     @UnderInitialization(Object.class) A uOa;
%     @UnderInitialization(A.class) A uAa;
%     @Initialized A nraA;
% 
%     @UnderInitialization(Object.class) B uOb;
%     @UnderInitialization(A.class) B uAb;
%     @UnderInitialization(B.class) B uBb;
%     @Initialized B b;
% \end{Verbatim}
% 
% In the following table, the type in cell C1 is a supertype of the type in
% cell C2 if:  C1 is at least as high and at least as far left in the table
% as C2 is.  For example, \<nraA>'s type is a supertype of those of \<rB>,
% \<nraB>, \<nrbB>, \<a>, and \<b>.  (The empty cells on the top row are real
% types, but are not expressible.  The other empty cells are not interesting
% types.)
% 
% \noindent
% \begin{tabular}{|c|c|c|}
% 
% \hline
%     \<@UnderInitialization Object rO;>
% &
% & 
% \\
% \hline
% 
%     \<@Initialized(Object.class) Object nroO;>
% &
% \begin{minipage}{2in}
% \begin{Verbatim}
% @UnderInitialization A rA;
% @Initialized(Object.class) A nroA;
% \end{Verbatim}
% \end{minipage}
% &
%     \<@Initialized(Object.class) B nroB;>
% \\
% \hline
% 
% &
%     \<@Initialized(A.class) A nraA;>
% &
% \begin{minipage}{1.75in}
% \begin{Verbatim}
% @UnderInitialization B rB;
% @Initialized(A.class) B nraB;
% \end{Verbatim}
% \end{minipage}
% \\
% \hline
% 
% &
% &
%     \<@Initialized(B.class) B nrbB;>
% \\
% \hline
% 
%     \<Object o;>
% &
%     \<A a;>
% &
%     \<B b;>
% \\
% \hline
% \end{tabular}



% \urldef{\jlsconstructorbodyurl}{\url}{https://docs.oracle.com/javase/specs/jls/se7/html/jls-8.html#jls-8.8.7}
% (Recall that the superclass constructor is called on the first line, or is
% inserted automatically by the compiler before the first line, see
% \ahref{\jlsconstructorbodyurl}{JLS \S8.8.7}.)



\subsection{How to handle warnings\label{initialization-warnings}}

There are several ways to address a warning ``error:  the constructor does
not initialize fields: \ldots''.
\begin{itemize}
\item
  Declare the field as \refqualclass{checker/nullness/qual}{Nullable}.  Recall
  that if you did not write an annotation, the field defaults to
  \refqualclass{checker/nullness/qual}{NonNull}.
\item
  Declare the field as \refqualclass{checker/nullness/qual}{MonotonicNonNull}.
  This is appropriate if the field starts out as \<null> but is later set
  to a non-null value.  You may then wish to use the
  \refqualclass{checker/nullness/qual}{EnsuresNonNull} annotation to indicate
  which methods set the field, and the
  \refqualclass{checker/nullness/qual}{RequiresNonNull} annotation to indicate
  which methods require the field to be non-null.
\item
  Initialize the field in the constructor or in the field's initializer, if 
  the field should be initialized.  (In this case, the Initialization
  Checker has found a bug!)

  Do \emph{not} initialize the field to an arbitrary non-null value just to
  eliminate the warning.  Doing so degrades your code:  it introduces a
  value that will confuse other programmers, and it converts a clear
  NullPointerException into a more obscure error.
\end{itemize}

If your code calls an instance method from a constructor, you may see a
message such as the following:

\begin{Verbatim}
Foo.java:123: error: call to initHelper() not allowed on the given receiver.
    initHelper();
              ^
  found   : @UnderInitialization(com.google.Bar.class) @NonNull MyClass
  required: @Initialized @NonNull MyClass
\end{Verbatim}

The problem is that the current object (\<this>) is under initialization,
but the receiver formal parameter (Section~\ref{faq-receiver}) of method
\<initHelper()> is implicitly annotated as
\refqualclass{checker/initialization/qual}{Initialized}.  If
\<initHelper()> doesn't depend on its receiver being initialized --- that
is, it's OK to call \<x.initHelper> even if \<x> is not initialized ---
then you can indicate that:

\begin{Verbatim}
class MyClass {
  void initHelper(@UnknownInitialization MyClass this, String param1) { ... }
}
\end{Verbatim}

\noindent
If you are using annotations in comments, you would write:

\begin{Verbatim}
class MyClass {
  void initHelper(/*>>>@UnknownInitialization MyClass this,*/ String param1) { ... }
}
\end{Verbatim}

\noindent
You are likely to want to annotate \<initHelper()> with
\refqualclass{checker/nullness/qual}{EnsuresNonNull} as well; see
Section~\ref{nullness-method-annotations}.


You may get the ``call to \ldots\ is not allowed on the given receiver''
error even if your constructor has already initialized all the fields.  
For this code:

\begin{Verbatim}
public class MyClass {
  @NonNull Object field;
  public MyClass() {
    field = new Object();
    helperMethod();
  }
  private void helperMethod() {
  }
}
\end{Verbatim}

\noindent
the Nullness Checker issues the following warning:

\begin{Verbatim}
MyClass.java:7: error: call to helperMethod() not allowed on the given receiver.
    helperMethod();
                ^
  found   : @UnderInitialization(MyClass.class) @NonNull MyClass
  required: @Initialized @NonNull MyClass
1 error
\end{Verbatim}

\begin{sloppypar}
The reason is that even though the object under construction has had all
the fields declared in \<MyClass> initialized, there might be a subclass of
\<MyClass>.  Thus, the receiver of \<helperMethod> should be declared as 
\<@UnderInitialization(MyClass.class)>, which says that initialization has
completed for all the \<MyClass> fields but may not have been completed
overall.  If \<helperMethod> had been a public method that could also be called after
initialization was actually complete, then the receiver should have type
\<@UnknownInitialization>, which is the supertype of
\<@UnknownInitialization> and \<@UnderInitialization>.
\end{sloppypar}


\subsection{More details about initialization checking\label{initialization-checking}}


\paragraph{Suppressing warnings}

\begin{sloppypar}
You can suppress warnings related to partially-initialized objects with
\<@SuppressWarnings("initialization")>.
\end{sloppypar}

\paragraph{Checking initialization of all fields, not just \code{@NonNull} ones}

When the \code{-Alint=uninitialized} command-line option is provided, then
an object is considered uninitialized until \emph{all} its fields are assigned, not
just the \code{@NonNull} ones.  See Section~\ref{lint-uninitialized-section}.


\paragraph{Use of method annotations}

A method with a non-initialized receiver may assume that a few fields (but not all
of them) are non-null, and it sometimes sets some more fields to non-null
values.  To express these concepts, use the
\refqualclass{checker/nullness/qual}{RequiresNonNull},
\refqualclass{checker/nullness/qual}{EnsuresNonNull}, and
\refqualclass{checker/nullness/qual}{EnsuresNonNullIf} method annotations;
see Section~\ref{nullness-method-annotations}.


\paragraph{Source of the type system}

The type system enforced by the Initialization Checker is known as
``Freedom Before Commitment''~\cite{SummersM2011}.  Our implementation
changes its initialization modifiers (``committed'', ``free'', and
``unclassified'') to ``initialized'', ``unknown initialization'', and
``under initialization''.  Our implementation also has several
enhancements.  For example, it supports partial initialization (the
argument to the \<@UnknownInitialization> and \<@UnderInitialization>
annotations.



\subsection{Rawness Initialization Checker\label{initialization-rawness-checker}}

The Checker Framework supports two different initialization checkers that
are integrated with the Nullness Checker.
You can use whichever one you prefer.

One (described in most of Section~\ref{initialization-checker}) uses the three annotations 
\refqualclass{checker/initialization/qual}{Initialized},
\refqualclass{checker/initialization/qual}{UnknownInitialization}, and
\refqualclass{checker/initialization/qual}{UnderInitialization}.
We recommend that you use it.

The other (described here in Section~\ref{initialization-rawness-checker})
uses the two annotations
\refqualclass{checker/nullness/qual}{Raw} and
\refqualclass{checker/nullness/qual}{NonRaw}.
The rawness type system is slightly easier to use but slightly less
expressive.

To run the Nullness Checker with the rawness variant of the Initialization
Checker, invoke the NullnessRawnessChecker rather than the NullnessChecker;
that is, supply the \code{-processor org.checkerframework.checker.nullness.NullnessRawnessChecker}
command-line option to javac.
Although \code{@Raw} roughly corresponds to \code{@UnknownInitialization}
and \code{@NonRaw} roughly corresponds to \code{@Initialized}, the
annotations are not aliased and you must use the ones that correspond to
the type-checker that you are running.


%% TODO: cut much of this, which is redundant with the text above.



An object is
\emph{raw} from the time that its constructor starts until its constructor
finishes.  This is relevant to the Nullness Checker because while the
constructor is executing --- that is, before initialization completes ---
a \<@NonNull>
field may be observed to be null, until that field is set.  In
particular, the Nullness Checker issues a warning for code like this:

\begin{Verbatim}
  public class MyClass {
    private @NonNull Object f;
    public MyClass(int x, int y) {
      // Error because constructor contains no assignment to this.f.
      // By the time the constructor exits, f must be initialized to a non-null value.
    }
    public MyClass(int x) {
      // Error because this.f is accessed before f is initialized.
      // At the beginning of the constructor's execution, accessing this.f
      // yields null, even though field f has a non-null type.
      this.f.toString();
    }
    public MyClass(int x, int y, int z) {
      m();
    }
    public void m() {
      // Error because this.f is accessed before f is initialized,
      // even though the access is not in a constructor.
      // When m is called from the constructor, accessing f yields null,
      // even though field f has a non-null type.
      this.f.toString();
    }
\end{Verbatim}

\noindent
In general, code can depend that field \<f> is not \<null>, because the
field is declared with a \refqualclass{checker/nullness/qual}{NonNull} type.
However, this guarantee does not hold for a partially-initialized object.

The Nullness Checker uses the \refqualclass{checker/nullness/qual}{Raw} annotation to indicate that an object
is not yet fully initialized --- that is, not all its \<@NonNull> fields have been
assigned.  Rawness is mostly relevant within the constructor, or for
references to \code{this} that escape the constructor (say, by being stored
in a field or passed to a method before initialization is complete).  
Use of rawness annotations is rare in most code.

The most common use for the \<@Raw> annotation is for a helper routine that
is called by constructor.  For example:

\begin{Verbatim}
  class MyClass {
    Object field1;
    Object field2;
    Object field3;

    public MyClass(String arg1) {
      this.field1 = arg1;
      init_other_fields();
    }

    // A helper routine that initializes all the fields other than field1
    @EnsuresNonNull({"field2", "field3"})
    private void init_other_fields(@Raw MyClass this) {
      field2 = new Object();
      field3 = new Object();
    }
  }
\end{Verbatim}

For compatibility with Java 6 and 7, you can write the receiver
parameter in comments (see Section~\ref{annotations-in-comments}):
\begin{Verbatim}
    private void init_other_fields(/*>>> @Raw MyClass this*/) {
\end{Verbatim}

% Most readers can
% skip this section on first reading; you can return to it once you have
% mastered the rest of the Nullness Checker.

\subsubsection{Rawness qualifiers\label{rawness-qualifiers}}

\begin{figure}
\includeimage{rawness}{3.5cm}
\caption{Partial type hierarchy for the Rawness Initialization type system.}
\label{fig-rawness-hierarchy}
\end{figure}

The rawness hierarchy is shown in Figure~\ref{fig-rawness-hierarchy}.
The rawness hierarchy contains these qualifiers:

\begin{description}

\item[\refqualclass{checker/nullness/qual}{Raw}]
  indicates a type that may contain a partially-initialized object.  In a
  partially-initialized object, fields that are annotated as
  \refqualclass{checker/nullness/qual}{NonNull} may be null because the field
  has not yet been assigned.  Within the constructor,
  \code{this} has \refqualclass{checker/nullness/qual}{Raw} type until all
  the \code{@NonNull} fields have been assigned.
  A partially-initialized object (\code{this} in a constructor) may be
  passed to a helper method or stored in a variable; if so, the method
  receiver, or the field, would have to be annotated as \<@Raw>.

% Cut this?
\item[\refqualclass{checker/nullness/qual}{NonRaw}]
  indicates a type that contains a fully-initialized object.  \code{NonRaw}
  is the default, so there is little need for a programmer to write this
  explicitly.

\item[\refqualclass{checker/nullness/qual}{PolyRaw}]
  indicates qualifier polymorphism over rawness (see
  Section~\ref{qualifier-polymorphism}).

\end{description}

% However, if the constructor makes
% a method call (passing \code{this} as a parameter or the receiver), then
% the called method could observe the object in an illegal state.

If a reference has
\code{@Raw} type, then all of its \code{@NonNull} fields are treated as
\refqualclass{checker/nullness/qual}{MonotonicNonNull}:  when read, they are
treated as being \refqualclass{checker/nullness/qual}{Nullable}, but when
written, they are treated as being
\refqualclass{checker/nullness/qual}{NonNull}.


The rawness hierarchy is orthogonal to the nullness hierarchy.  It
is legal for a reference to be \<@NonNull @Raw>, \<@Nullable @Raw>,
\<@NonNull @NonRaw>, or \<@Nullable @NonRaw>.  The nullness hierarchy tells
you about the reference itself:  might the reference be null?  The rawness
hierarchy tells you about the \<@NonNull> fields in the referred-to object:
might those fields be temporarily null in contravention of their
type annotation?
% It's a figure rather than appearing inline so as not to span page breaks
% in the printed manual.
Figure~\ref{fig-rawness-examples} contains some examples.

\begin{figure}
\begin{tabular}{l|l|l}
Declarations & Expression & Expression's nullness type, or checker error \\ \hline
\begin{minipage}{1.5in}
\begin{Verbatim}
class C {
  @NonNull Object f;
  @Nullable Object g;
  ...
}
\end{Verbatim}
\end{minipage} & & \\ \cline{2-3}
\<@NonNull @NonRaw C a;>
& \<a> & \<@NonNull> \\
& \<a.f> & \<@NonNull> \\
& \<a.g> & \<@Nullable> \\ \cline{2-3}
\<@NonNull @Raw C b;>
& \<b> & \<@NonNull> \\
& \<b.f> & \<@MonotonicNonNull> \\
& \<b.g> & \<@Nullable> \\ \cline{2-3}
\<@Nullable @NonRaw C c;>
& \<c> & \<@Nullable> \\
& \<c.f> & error: deref of nullable \\
& \<c.g> & error: deref of nullable \\ \cline{2-3}
\<@Nullable @Raw C d;>
& \<d> & \<@Nullable> \\
& \<d.f> & error: deref of nullable \\
& \<d.g> & error: deref of nullable \\
\end{tabular}
\caption{Examples of the interaction between nullness and rawness.
  Declarations are shown at the left for reference, but the focus of the
  table is the expressions and their nullness type or error.}
\label{fig-rawness-examples}
\end{figure}


% Does our implementation handle static fields soundly?  NO!  See issue
% #105.  Maybe document this?


\subsubsection{How an object becomes non-raw\label{becoming-non-raw}}

Within the constructor,
\code{this} starts out with \refqualclass{checker/nullness/qual}{Raw} type.
As soon as all of the \refqualclass{checker/nullness/qual}{NonNull} fields
have been initialized, then \code{this} is treated as non-raw.
% TODO:  (See
% Section~\ref{becoming-non-raw-clarification} for a slight clarification of
% this rule.)

The Nullness Checker issues an error if the constructor fails to initialize
any \code{@NonNull} field.  This ensures that the object is in a legal (non-raw)
state by the time that the constructor exits.
\urldef{\jlsdefiniteassignmenturl}{\url}{https://docs.oracle.com/javase/specs/jls/se7/html/jls-16.html}
\urldef{\jlsfinalvariablesurl}{\url}{https://docs.oracle.com/javase/specs/jls/se7/html/jls-4.html#jls-4.12.4}
This is different than Java's test for definite assignment (see
\ahref{\jlsdefiniteassignmenturl}{JLS ch.16}),
% , which requires that local
% variables (and blank \code{final} fields) must be assigned.  Java does not
% require that non-\code{final} fields be assigned, since
which does not apply to fields (except blank final ones, defined in
\ahref{\jlsdefiniteassignmenturl}{JLS \S 4.12.4}) because fields
have a default value of null.


% and can only be passed to methods when the corresponding parameter is
% annotated with \refqualclass{checker/nullness/qual}{Raw}.  Similar
% restrictions apply to assigning \code{this} to a field.

All \code{@NonNull} fields must either have a
default in the field declaration, or be assigned in the constructor or in a
helper method that the constructor calls.  If
your code initializes (some) fields in a helper method, you will need to
annotate the helper method with an annotation such as
\refqualclass{checker/nullness/qual}{EnsuresNonNull}\code{(\{"field1", "field2"\})}
for all the fields that the helper method assigns.
It's a bit odd, but you use that same annotation, \code{@EnsuresNonNull},
to indicate that a primitive field has its value set in a helper method,
which is relevant when you supply the \code{-Alint=uninitialized}
command-line option (see Section~\ref{lint-uninitialized-section}).

% TODO:
% We need an
%   @EnsuresInitialized("b")
% that is analogous to
%   @EnsuresNonNull("b")
% when we are dealing with a field of primitive type.
% But, for now just use the same annotation, @EnsuresNonNull, for both purposes.


\subsubsection{Partial initialization\label{rawness-partial-initialization}}

So far, we have discussed rawness as if it is an all-or-nothing property:
an object is fully raw until initialization completes, and then it is no
longer raw.  The full truth is a bit more complex:  during the
initialization process, an object can be partially initialized, and as the
object's superclass constructors complete, its rawness changes.  The
Nullness Checker lets you express such properties when necessary.

Consider a simple example:

\begin{Verbatim}
class A {
  Object a;
  A() {
    a = new Object();
  }
}
class B extends A {
  Object b;
  B() {
    super();
    b = new Object();
  }
}
\end{Verbatim}

Consider what happens during execution of \<new B()>.

\begin{enumerate}
\item \<B>'s constructor begins to execute.  At this point, neither the
  fields of \<A> nor those of \<B> have been initialized yet.
\item \<B>'s constructor calls \<A>'s constructor, which begins to execute.
  No fields of \<A> nor of \<B> have been initialized yet.
\item \<A>'s constructor completes.  Now, all the fields of \<A> have been
  initialized, and their invariants (such as that field \<a> is non-null) can be
  depended on.  However, because \<B>'s constructor has not yet completed
  executing, the object being constructed is not yet fully initialized.
  When treated as an \<A> (e.g., if only the \<A> fields are accessed), the
  object is initialized (non-raw), but when treated as a \<B>, the object
  is still raw.
\item \<B>'s constructor completes.  The object is fully initialized
  (non-raw), if \<B>'s constructor was invoked via a \<new B()>
  expression.  On the other hand, if there was a \<class C extends B \{
  ... \}>, and \<B>'s constructor had been invoked from that, then the
  object currently under construction would \emph{not} be fully initialized
  --- it would only be initialized when treated as an \<A> or a \<B>, but
  not when treated as a \<C>.
\end{enumerate}

At any moment during initialization, the superclasses of a given class
can be divided into those that have completed initialization and those that
have not yet completed initialization.  More precisely, at any moment there
is a point in the class hierarchy such that all the classes above that
point are fully initialized, and all those below it are not yet
initialized.  As initialization proceeds, this dividing line between the
initialized and raw classes moves down the type hierarchy.

The Nullness Checker lets you indicate where the dividing line is between
the initialized and non-initialized classes.
You have two equivalent ways to indicate the dividing line:  \<@Raw>
indicates the first class \emph{below} the dividing line, or
\<@NonRaw(\emph{classliteral})> indicates the first class \emph{above} the
dividing line.

When you write \refqualclass{checker/nullness/qual}{Raw}\code{ MyClass x;}, that
means that variable \<x> is initialized for all superclasses of \<MyClass>,
and (possibly) uninitialized for \<MyClass> and all subclasses.

When you write \refqualclass{checker/nullness/qual}{NonRaw}\code{(Foo.class) MyClass
  x;}, that means that variable \<x> is initialized for \<Foo> and all its
superclasses, and (possibly) uninitialized for all subclasses of \<Foo>.

If \<A> is a direct superclass of \<B> (as in the example above), then 
\<@Raw A x;> and \<@NonRaw(B.class) A x;> are equivalent declarations.
Neither one is the same as \<@NonRaw A x;>, which indicates that, whatever
the actual class of the object that \<x> refers to, that object is fully
initialized.  Since \<@NonRaw> (with no argument) is the default, you will
rarely see it written.

\label{becoming-non-raw-clarification}

We can now state a clarification of Section~\ref{becoming-non-raw}'s rule
for an object becoming non-raw.
As soon as all of the \refqualclass{checker/nullness/qual}{NonNull} fields
have been initialized, then \code{this} is treated as
\refqualclass{checker/nullness/qual}{NonRaw}\code{(\emph{typeofthis})}, rather than
treated as simply 
\refqualclass{checker/nullness/qual}{NonRaw}.

The example above lists 4 moments during construction.  At those moments,
the type of the object being constructed is:

\begin{enumerate}
\item
  \<@Raw Object>
\item
  \<@Raw Object>
\item
  \<@NonRaw(A.class) A>
  %% Not quite equivalent because the Java (non-qualified) type differs
  % ; equivalently, \<@Raw B>
\item
  \<@NonRaw(B.class) B>
\end{enumerate}

\paragraph{Example}

As another example, consider the following 12 declarations:

\begin{Verbatim}
    @Raw Object rO;
    @NonRaw(Object.class) Object nroO;
    Object o;

    @Raw A rA;
    @NonRaw(Object.class) A nroA;  // same as "@Raw A"
    @NonRaw(A.class) A nraA;
    A a;

    @NonRaw(Object.class) B nroB;
    @Raw B rB;
    @NonRaw(A.class) B nraB;  // same as "@Raw B"
    @NonRaw(B.class) B nrbB;
    B b;
\end{Verbatim}

In the following table, the type in cell C1 is a supertype of the type in
cell C2 if:  C1 is at least as high and at least as far left in the table
as C2 is.  For example, \<nraA>'s type is a supertype of those of \<rB>,
\<nraB>, \<nrbB>, \<a>, and \<b>.  (The empty cells on the top row are real
types, but are not expressible.  The other empty cells are not interesting
types.)

\noindent
\begin{tabular}{|c|c|c|}

\hline
    \<@Raw Object rO;>
&
& 
\\
\hline

    \<@NonRaw(Object.class) Object nroO;>
&
\begin{minipage}{2in}
\begin{Verbatim}
@Raw A rA;
@NonRaw(Object.class) A nroA;
\end{Verbatim}
\end{minipage}
&
    \<@NonRaw(Object.class) B nroB;>
\\
\hline

&
    \<@NonRaw(A.class) A nraA;>
&
\begin{minipage}{1.75in}
\begin{Verbatim}
@Raw B rB;
@NonRaw(A.class) B nraB;
\end{Verbatim}
\end{minipage}
\\
\hline

&
&
    \<@NonRaw(B.class) B nrbB;>
\\
\hline

    \<Object o;>
&
    \<A a;>
&
    \<B b;>
\\
\hline
\end{tabular}



% \urldef{\jlsconstructorbodyurl}{\url}{https://docs.oracle.com/javase/specs/jls/se7/html/jls-8.html#jls-8.8.7}
% (Recall that the superclass constructor is called on the first line, or is
% inserted automatically by the compiler before the first line, see
% \ahref{\jlsconstructorbodyurl}{JLS \S8.8.7}.)



\subsubsection{More details about rawness checking\label{rawness-checking}}


\paragraph{Suppressing warnings}

\begin{sloppypar}
You can suppress warnings related to partially-initialized objects with
\<@SuppressWarnings("rawness")>.  Do not confuse this with the unrelated
\<@SuppressWarnings("rawtypes")> annotation for non-instantiated generic types!
\end{sloppypar}


\paragraph{Checking initialization of all fields, not just \code{@NonNull} ones}

When the \code{-Alint=uninitialized} command-line option is provided, then
an object is considered raw until \emph{all} its fields are assigned, not
just the \code{@NonNull} ones.  See Section~\ref{lint-uninitialized-section}.


\paragraph{Use of method annotations}

A method with a raw receiver often assumes that a few fields (but not all
of them) are non-null, and sometimes sets some more fields to non-null
values.  To express these concepts, use the
\refqualclass{checker/nullness/qual}{RequiresNonNull},
\refqualclass{checker/nullness/qual}{EnsuresNonNull}, and
\refqualclass{checker/nullness/qual}{EnsuresNonNullIf} method annotations;
see Section~\ref{nullness-method-annotations}.


% Should we change the terminology?
\paragraph{The terminology ``raw''}

The name ``raw'' comes from a research paper that proposed this
approach~\cite{FahndrichL2003}.
A better name might have been ``not yet initialized'' or ``partially
initialized'', but the term ``raw'' is now well-known.
The \refqualclass{checker/nullness/qual}{Raw}
annotation has nothing to do with the raw types of Java Generics.


\section{Map Key Checker\label{map-key-checker}}

The Map Key Checker uses the \refqualclass{checker/nullness/qual}{KeyFor}
annotation to declare that a value is a key in a given map.

This is relevant to the Nullness Checker because
Java's
\sunjavadoc{java/util/Map.html#get(java.lang.Object)}{\code{Map.get}}
method always has the possibility to return null, if the key is not in the
map.
In particular,
a call \<mymap.get(mykey)> returns non-null if two conditions are satisfied:
\begin{enumerate}
\item \<mymap>'s values are all non-null; that is, \<mymap> was
  declared as \code{Map<\emph{KeyType}, @NonNull \emph{ValueType}>}.  Note
  that \<@NonNull> is the default type, so it need not be written explicitly.
\item \<mykey> is a key in \<mymap>; that is, \<mymap.containsKey(mykey)>
  returns \<true>.  You express this fact to the Nullness Checker by
  declaring \<mykey> as \<@KeyFor("mymap") \emph{KeyType} mykey>.  For a
  local variable, the \<@KeyFor("mymap")> type qualifier can generally be
  inferred.
\end{enumerate}
\noindent
If either of these two conditions is violated, then \<mymap.get(mykey)> has
the possibility of returning null.


Thus, for the Nullness Checker to guarantee that the value returned from \code{Map.get} is
non-null, it is necessary that the map contains only non-null values,
\emph{and} the key is in the map.
The \refqualclass{checker/nullness/qual}{KeyFor} annotation states the latter
property.

If a type is annotated as \code{@KeyFor("m")}, then any value v with that type
is a key in Map \<m>.  Another way of saying this is that the expression
\code{m.containsKey(v)} evaluates to true.

You usually do not have to write \code{@KeyFor} explicitly, because the
checker infers it based on usage patterns, such as calls to
\code{containsKey} or iteration over a map's
\sunjavadoc{java/util/Map.html#keySet()}{\textrm{key set}}.

One usage pattern where you \emph{do} have to write \<@KeyFor> is for a
user-managed collection that is a subset of the key set:

\begin{Verbatim}
Map<String, Object> m;
Set<@KeyFor("m") String> matchingKeys; // keys that match some criterion
for (@KeyFor("m") String k : matchingKeys) {
  ... m.get(k) ...  // known to be non-null
}
\end{Verbatim}

As with any annotation, use of the \<@KeyFor> annotation may force you to
slightly refactor your code.  For example, this would be illegal:

\begin{Verbatim}
  Map<K,V> m;
  Collection<@KeyFor("m") K> coll;
  coll.add(x);   // compiler error, because the @KeyFor annotation is violated
  m.put(x, ...);
\end{Verbatim}

\noindent
but reordering the two calls this would be OK (no compiler error):

\begin{Verbatim}
  Map<K,V> m;
  Collection<@KeyFor("m") K> coll;
  m.put(x, ...);
  coll.add(x);
\end{Verbatim}


Because the \<@KeyFor> type hierarchy is independent from the nullness and
rawness hierarchies, it uses a different warning suppression key.
You can suppress warnings related to map keys with
\<@SuppressWarnings("keyfor")>.

When you perform a run-time check for map keys, such as \<if (m.containsKey(k))
...>, then the Map Key Checker refines the type of \<k> to
\<@KeyFor("m")> within the scope of the test.  For more details, see
Section~\ref{type-refinement}.

% See these issues:
% https://code.google.com/p/checker-framework/issues/detail?id=221
% https://code.google.com/p/checker-framework/issues/detail?id=241
% https://code.google.com/p/checker-framework/issues/detail?id=273
Currently, the set of expressions allowed in @KeyFor is less expressive
than the expressions described in
Section~\ref{java-expressions-as-arguments}.  The Checker Framework
developers are working to correct this bug.


% LocalWords:  NonNull plugin quals un NonNullExampleWithWarnings java ahndrich
% LocalWords:  NotNull IntelliJ FindBugs Nullable TODO Alint nullable NNEL JSR
% LocalWords:  DefaultLocation Nullness PolyNull nullness AnnotateNullable JLS
% LocalWords:  Daikon JastAdd javac DefaultQualifier boolean MyEnumType NonRaw
% LocalWords:  NullnessAnnotatedTypeFactory NullnessVisitor MonotonicNonNull
% LocalWords:  inferencer Nonnull CheckForNull UnknownNullness rawtypes de ch
% LocalWords:  castNonNull NullnessUtils assertNotNull codebases checkNotNull
% LocalWords:  Nullability typeargs nulltest EnsuresNonNullIf listFiles faq
% LocalWords:  isDirectory AssertionError intraprocedurally SuppressWarnings rB
% LocalWords:  FindBugs's getObject RequiresNonNull EnsuresNonNull KeyFor
% LocalWords:  nonnull EnsuresNonNull ReadOnly arg
% LocalWords:  keySet getField keyfor param TypeName containsValue indexOf nraA
% LocalWords:  lastIndexOf deref getProperty getProperties classliteral MyClass
% LocalWords:  typeofthis nraB nrbB rO nroO nroB 5cm IGJ containsKey enum
% LocalWords:  JUnit's 5in field1 field2 superclasses Foo C1 C2 2in 75in PolyRaw
%  LocalWords:  NullnessChecker redundantNullComparison 5cm instanceof 5in
%  LocalWords:  noInitForMonotonicNonNull UnknownInitialization isArray
%  LocalWords:  UnderInitialization getComponentType 2in
%  LocalWords:  isEmpty AssumeAssertion cleanroom vararg OtherClass 75in
%  LocalWords:  NullnessRawnessChecker mymap mykey KeyType ValueType 5cm
%  LocalWords:  dereferenced dereference Dereferencing initializers 5in
%  LocalWords:  initHelper helperMethod

\htmlhr
\chapter{Interning checker\label{interning-checker}}

If the Interning checker issues no warnings for a given program, then all
reference equality tests (i.e., all uses of ``\code{==}'') are proper;
that is,
\code{==} is not misused where \code{equals()} should have been used instead.

Interning is a design pattern in which the same object is used whenever two
different objects would be considered equal.  Interning is also known as
canonicalization or hash-consing, and it is related to the flyweight design
pattern.
Interning has two benefits:  it can save memory, and it can speed up testing for
equality by permitting use of \code{==}.

The Interning checker prevents two types of errors in your code.  First, 
\code{==} should be used
only on interned values; using \code{==} on
non-interned values can result in subtle bugs.  For example:

\begin{Verbatim}
  Integer x = new Integer(22);
  Integer y = new Integer(22);
  System.out.println(x == y);  // prints false!
\end{Verbatim}

\noindent
The Interning checker helps programmers to prevent such bugs.
Second, 
the Interning checker also helps to prevent performance problems that result
from failure to use interning.
(See Section~\ref{checker-guarantees} for caveats to the checker's guarantees.)

Interning is such an important design pattern that Java builds it in for
strings.  Every string literal in the program is guaranteed to be interned
(\ahref{http://java.sun.com/docs/books/jls/third\_edition/html/lexical.html\#3.10.5}{JLS
  \S3.10.5}), and the
\sunjavadoc{java/lang/String.html#intern()}{String.intern()} method
performs interning for strings that are computed at run time.
Users can also write their own interning methods for other types.

It is a proper optimization to use \code{==}, rather than \code{equals()},
whenever the comparison is guaranteed to produce the same result --- that
is, whenever the comparison is never provided with two different objects
for which \code{equals()} would return true.  Here are three reasons that
this property could hold:

\begin{enumerate}
\item
  Interning.  A factory method ensures that, globally, no two different
  interned objects are \code{equals()} to one another.  (In some cases
  other, non-interned objects of the class might be \code{equals()} to one
  another; in other cases, every object of the class is interned.)
  Interned objects should always be immutable.
\item
  Global control flow.  The program's control flow is such that the
  constructor for class $C$ is called a limited number of times, and with
  specific values that ensure the results are not \code{equals()} to one
  another.  Objects of class $C$ can always be compared with \code{==}.
  Such objects may be mutable or immutable.
\item
  Local control flow.  Even though not all objects of the given type may be
  compared with \code{==}, the specific objects that can reach a given
  comparison may be.  For example, suppose that an array contains no
  duplicates.  Then testing to find the index of a given element that is
  known to be in the array can use \code{==}.
\end{enumerate}

To eliminate Interning Checker warnings, you will need to annotate your
code regarding all legal uses of \code{==}.  Thus, the Interning Checker
could also have been called the Reference Equality Checker.  In the
future, the checker will include annotations that target the non-interning
cases above, but for now you need to use \<@Interned>, \<@UsesObjectEquals>
(which handles a surprising number of cases), and/or
\<@SuppressWarnings>.

To run the Interning Checker, supply the \code{-processor
  checkers.interning.InterningChecker} command-line option to javac.  For
examples, see Section~\ref{interning-example}.


\section{Interning annotations\label{interning-annotations}}

Two qualifiers are part of the Interning type system.

\begin{description}

\item[\code{@\refclass{interning/quals}{Interned}}]
  indicates a type that includes only interned values (no non-interned
  values).

\item[\<@\refclass{interning/quals}{PolyInterned}>]
  indicates qualifier polymorphism.  For a description of
  \<@\refclass{interning/quals}{PolyInterned}>, see
  Section~\ref{qualifier-polymorphism}.

\item[\<@\refclass{interning/quals}{UsesObjectEquals}>]
  is a class (not type) annotation that indicates that this class's
  \<equals> method is the same as that of \<Object>.  In other words,
  neither this class nor any of its superclasses overrides the \<equals>
  method.  Since \<Object.equals> uses reference equality, this means that
  for such as class, \<==> and \<equals> are equivalent, and so the
  Interning Checker does not issue warnings for either one.

\end{description}


\section{Annotating your code with \code{@Interned}\label{annotating-with-interned}}

\begin{figure}
\includeimage{interning}{2.5cm}
\caption{Type hierarchy for the Interning type system.}
\label{fig:interning-hierarchy}
\end{figure}

In order to perform checking, you must annotate your code with the \code{@\refclass{interning/quals}{Interned}}
type annotation, which indicates a type for the canonical representation of an
object:

\begin{Verbatim}
            String s1 = ...;  // type is (uninterned) "String"
  @Interned String s2 = ...;  // Java type is "String", but checker treats it as "Interned String"
\end{Verbatim}

The type system enforced by the checker plugin ensures that only interned
values can be assigned to \code{s2}.

To specify that \emph{all} objects of a given type are interned, annotate the
class declaration:

\begin{Verbatim}
  public @Interned class MyInternedClass { ... }
\end{Verbatim}

This is equivalent to annotating every use of \code{MyInternedClass}, in a
declaration or elsewhere.  For example, \code{enum} classes are implicitly
so annotated.


\subsection{Implicit qualifiers\label{interning-implicit-qualifiers}}

As described in Section~\ref{effective-qualifier}, the Interning checker
adds implicit qualifiers, reducing the number of annotations that must
appear in your code.
For example, String literals and the null literal are always considered interned, and
object creation expressions (using \code{new}) are never considered
\code{@\refclass{interning/quals}{Interned}} unless they are annotated as such, as in

%BEGIN LATEX
\begin{smaller}
%END LATEX
\begin{Verbatim}
@Interned Double internedDoubleZero = new @Interned Double(0); // canonical representation for Double zero
\end{Verbatim}
%BEGIN LATEX
\end{smaller}
%END LATEX

For a complete description of all implicit interning qualifiers, see the
Javadoc for \refclass{interning}{InterningAnnotatedTypeFactory}.


\section{What the Interning checker checks\label{interning-checks}}

Objects of an \code{@\refclass{interning/quals}{Interned}} type may be safely compared using the ``\code{==}''
operator.

The checker issues a warning in two cases:

\begin{enumerate}

\item
  When a reference (in)equality operator (``\code{==}'' or ``\code{!=}'')
  has an operand of non-\code{@\refclass{interning/quals}{Interned}} type.

\item
  When a non-\code{@\refclass{interning/quals}{Interned}} type is used where an \code{@\refclass{interning/quals}{Interned}} type
  is expected.

\end{enumerate}

This example shows both sorts of problems:

\begin{Verbatim}
            Object  obj;
  @Interned Object iobj;
  ...
  if (obj == iobj) { ... }  // checker warning: reference equality test is unsafe
  iobj = obj;               // checker warning: iobj's referent may no longer be interned
\end{Verbatim}

\label{lint-dotequals}

The checker also issues a warning when \code{.equals} is used where
\code{==} could be safely used.  You can disable this behavior via the
javac \code{-Alint} command-line option, like so: \code{-Alint=-dotequals}.

For a complete description of all checks performed by
  the checker, see the Javadoc for
  \refclass{interning}{InterningVisitor}.

\label{checking-class}
You can also restrict which types the checker should examine and type-check,
using the \code{-Acheckclass} option.  For example, to find only the
interning errors related to uses of \code{String}, you can pass
\code{-Acheckclass=java.lang.String}.  The Interning checker always checks all
subclasses and superclasses of the given class.


\section{Examples\label{interning-example}}

To try the Interning checker on a source file that uses the \code{@\refclass{interning/quals}{Interned}} qualifier,
use the following command (where \code{javac} is the JSR 308 compiler that
is distributed with the Checker Framework):

\begin{Verbatim}
  javac -processor checkers.interning.InterningChecker examples/InterningExample.java
\end{Verbatim}

\noindent
Compilation will complete without warnings.

To see the checker warn about incorrect usage of annotations, use the following
command:

\begin{Verbatim}
  javac -processor checkers.interning.InterningChecker examples/InterningExampleWithWarnings.java
\end{Verbatim}

\noindent
The compiler will issue a warning regarding violation of the semantics of
\code{@\refclass{interning/quals}{Interned}}.
% in the \code{InterningExampleWithWarnings.java} file.


The Daikon invariant detector
(\myurl{http://groups.csail.mit.edu/pag/daikon/}) is also annotated with
\code{@\refclass{interning/quals}{Interned}}.  From directory \code{java},
run \code{make check-interning}.



\section{Other interning annotations\label{other-interning-annotations}}

The Checker Framework's interning annotations are similar to annotations used
elsewhere.

If your code is already annotated with a different interning
annotation, you can reuse that effort by converting them to the Checker
Framework's nullness annotations.  Perform the refactoring described in
Figure~\ref{fig:interning-refactoring}.


% These lists should be kept in sync with InterningAnnotatedTypeFactory.java .
\begin{figure}
\begin{center}
% The ~ around the text makes things look better in Hevea (and not terrible
% in LaTeX).
\begin{tabular}{ll}
\begin{tabular}{|l|}
\hline
 ~com.sun.istack.Interned~ \\ \hline
\end{tabular}
&
$\Rightarrow$
~checkers.interning.quals.Interned~
\end{tabular}
\end{center}
%BEGIN LATEX
\vspace{-1.5\baselineskip}
%END LATEX
\caption{Refactoring for converting interning annotations from other tools
  to the Checker Framework.}
\label{fig:interning-refactoring}
\end{figure}

Alternately, the Checker Framework can process those other annotations (as
well as its own, if they also appear in your program).  The Checker
Framework has its own definition of the annotations on the left side of
Figure~\ref{fig:interning-refactoring}, so that they can be used as type
qualifiers.  The Checker Framework interprets them according to the right
side of Figure~\ref{fig:interning-refactoring}.



% LocalWords:  plugin MyInternedClass enum InterningExampleWithWarnings java
% LocalWords:  PolyInterned Alint dotequals quals InterningAnnotatedTypeFactory
% LocalWords:  javac InterningVisitor JLS Acheckclass UsesObjectEquals

\htmlhr
\chapter{IGJ immutability checker\label{igj-checker}}

IGJ is a Java language extension that helps programmers to avoid mutation errors
(unintended side effects).
If the IGJ checker issues no warnings for a given program, then that program
will never change objects that should not be changed.  This guarantee
enables a programmer to detect and prevent mutation-related errors.
(See Section~\ref{checker-guarantees} for caveats to the guarantee.)

To run the IGJ Checker, supply the \code{-processor checkers.igj.IGJChecker}
command-line option to javac.  For examples, see Section~\ref{igj-example}.


\section{IGJ and Mutability\label{igj-and-mutability}}

IGJ~\cite{ZibinPAAKE2007} permits a
programmer to express that a particular object should never be modified via any
reference (object immutability), or that a reference should never be used to
modify its referent (reference immutability). Once a programmer has expressed
these facts, an automatic checker analyzes the code to either locate mutability
bugs or to guarantee that the code contains no such bugs.

\begin{figure}
\includeimage{igj}{3.5cm}
\caption{Type hierarchy for three of IGJ's type qualifiers.}
\label{fig:igj-hierarchy}
\end{figure}

To learn more details of the IGJ language and type system, please see the
ESEC/FSE 2007 paper ``\ahref{http://www.cs.washington.edu/homes/mernst/pubs/immutability-generics-fse2007-abstract.html}{Object and reference immutability using Java
generics}''~\cite{ZibinPAAKE2007}.
The IGJ checker supports Annotation IGJ (Section~\ref{annotation-igj-dialect}),
which is a slightly different dialect
of IGJ than that described in the ESEC/FSE paper.


\section{IGJ Annotations\label{igj-annotations}}

Each object is either immutable (it can never be modified) or mutable (it
can be modified).  The following qualifiers are part of the IGJ type system.

\begin{description}

\item[\code{@\refclass{igj/quals}{Immutable}}]
  An immutable reference always refers to an immutable object.  Neither the
  reference, nor any aliasing reference, may modify the object.

\item[\code{@\refclass{igj/quals}{Mutable}}]
  A mutable reference refers to a mutable object.  The reference, or some
  aliasing mutable reference, may modify the object.

\item[\code{@\refclass{igj/quals}{ReadOnly}}]
  A readonly reference cannot be used to modify its referent.  The referent
  may be an immutable or a mutable object.  In other words, it is possible
  for the referent to change via an aliasing mutable reference, even though
  the referent cannot be changed via the readonly reference.

\item[\code{@\refclass{igj/quals}{Assignable}}]
  The annotated field may be re-assigned regardless of the
  immutability of the enclosing class or object instance.

\item[\code{@\refclass{igj/quals}{AssignsFields}}]
  is similar to \<@Mutable>, but permits only limited mutation ---
  assignment of fields --- and is intended for use by constructor helper
  methods.

\item[\code{@\refclass{igj/quals}{I}}]
  simulates mutability overloading or the template behavior of generics.
  It can be applied to classes, methods, and parameters.  See
  Section~\ref{igj-templating}.

\end{description}

For additional details, see~\cite{ZibinPAAKE2007}.


\section{What the IGJ checker checks\label{igj-checks}}

The IGJ checker issues an error whenever mutation happens through a
readonly reference, when fields of a readonly reference which are not
explicitly marked with \code{@\refclass{igj/quals}{Assignable}} are
reassigned, or when a readonly reference is assigned to a mutable
variable.  The checker also emits a warning when casts increase the
mutability access of a reference.

% There is no visitor to reference!
% For a complete description of all checks performed by
% the checker, see the Javadoc for \refclass{igj}{IGJVisitor}.


\section{Implicit and default qualifiers\label{igj-implicit-qualifiers}}

As described in Section~\ref{effective-qualifier}, the IGJ checker
adds implicit qualifiers, reducing the number of annotations that must
appear in your code.
% For example, ...

For a complete description of all implicit IGJ qualifiers, see the
Javadoc for \refclass{igj}{IGJAnnotatedTypeFactory}.

The default annotation (for types that are unannotated and not given an
implicit qualifier) is as follows:
\begin{itemize}
\item
  \code{@Mutable} for almost all references.  This is backward-compatible
  with Java, since Java permits any reference to be mutated.
\item
  \code{@Readonly} for local variables.  This qualifier may be refined by
  flow-sensitive local type refinement (see Section~\ref{type-refinement}).
\item
  \code{@Readonly} for type parameter and wildcard bounds.  For example,

\begin{Verbatim}
  interface List<T extends Object> { ... }
\end{Verbatim}

\noindent
is defaulted to

\begin{Verbatim}
  interface List<T extends @Readonly Object> { ... }
\end{Verbatim}

This default is not backward-compatible --- that is, you may have to
explicitly add \code{@Mutable} annotations to some type parameter bounds in
order to make unannotated Java code type-check under IGJ\@.  However, this
reduces the number of annotations you must write overall (since most
variables of generic type are in fact not modified), and permits more
client code to type-check (otherwise a client could not write
\code{List<@Readonly Date>}).

\end{itemize}



\section{Annotation IGJ Dialect\label{annotation-igj-dialect}}

The IGJ checker supports the Annotation IGJ dialect of IGJ\@.  The syntax of
Annotation IGJ is based on type annotations.

The syntax of the original IGJ
dialect~\cite{ZibinPAAKE2007} was based on Java 5's generics and annotation mechanisms. The original
IGJ dialect was not backward-compatible with Java (either syntactically or
semantically). The dialect of IGJ checked by the IGJ checker corrects these
problems.

The differences between the Annotation IGJ dialect and the original IGJ dialect
are as follows.

\subsection{Semantic Changes}

\begin{itemize}

\item
  Annotation IGJ does not permit covariant changes in generic type
  arguments, for backward compatibility with Java.  In ordinary Java, types
  with different generic type arguments, such as \code{Vector<Integer>} and
  \code{Vector<Number>}, have no subtype relationship, even if the
  arguments (\code{Integer} and \code{Number}) do. The original IGJ dialect
  changed the Java subtyping rules to permit safely varying a type argument
  covariantly in certain circumstances. For example,

\begin{Verbatim}
  Vector<Mutable, Integer>  <:  Vector<ReadOnly, Integer>
                            <:  Vector<ReadOnly, Number>
                            <:  Vector<ReadOnly, Object>
\end{Verbatim}

is valid in IGJ, but in Annotation IGJ, only

\begin{Verbatim}
  @Mutable Vector<Integer>  <:  @ReadOnly Vector<Integer>
\end{Verbatim}

holds and the other two subtype relations do not hold

\begin{Verbatim}
  @ReadOnly Vector<Integer> </:  @ReadOnly Vector<Number>
                            </:  @ReadOnly Vector<Object>
\end{Verbatim}


\item
  Annotation IGJ supports array immutability. The original IGJ dialect did
  not permit the (im)mutability of array elements to be specified, because
  the generics syntax used by the original IGJ dialect cannot be applied to
  array elements.

\end{itemize}

\subsection{Syntax Changes}

\begin{itemize}

\item  Immutability is specified through
  \ahref{http://types.cs.washington.edu/jsr308/}{type annotations}~\cite{JSR308-2008-09-12} (Section~\ref{igj-annotations}),
not through a combination of generics and annotations.  Use of type
annotations makes Annotation IGJ backward compatible with Java syntax.

\item Templating over Immutability: The annotation \code{@\refclass{igj/quals}{I}(\emph{id})} is used to template
over immutability.  See Section~\ref{igj-templating}.

\end{itemize}


\subsection{Templating Over Immutability: \code{@I}\label{igj-templating}}

\code{@\refclass{igj/quals}{I}} is a template annotation over IGJ Immutability annotations. It acts
similarly to type variables in Java's generic types, and the name
\code{@\refclass{igj/quals}{I}} mimics the standard \code{<I>} type variable name used in code
written in the original IGJ dialect.  The annotation value string is used
to distinguish between multiple instances of \code{@\refclass{igj/quals}{I}} --- in the
generics-based original dialect, these would be expressed as two type
variables \code{<I>} and \code{<J>}.

\paragraph{Usage on classes\label{igj-usage-on-classes}}

A class declaration annotated with \code{@\refclass{igj/quals}{I}} can then be
used with any IGJ Immutability annotation.  The actual immutability that
\code{@\refclass{igj/quals}{I}} is resolved to dictates the immutability type for all the non-static
appearances of \code{@\refclass{igj/quals}{I}} with the same value as the class declaration.

  Example:
\begin{Verbatim}
    @I
    public class FileDescriptor {
       private @Immutable Date creationData;
       private @I Date lastModData;

       public @I Date getLastModDate() @ReadOnly { }
    }

    ...
    void useFileDescriptor() {
       @Mutable FileDescriptor file =
                         new @Mutable FileDescriptor(...);
       ...
       @Mutable Data date = file.getLastModDate();

    }
\end{Verbatim}

In the last example, \code{@\refclass{igj/quals}{I}} was resolved to \code{@\refclass{igj/quals}{Mutable}} for the instance file.

\paragraph{Usage on methods\label{igj-usage-on-methods}}

For example, it could be used for method parameters, return values, and the
actual IGJ immutability value would be resolved based on the method invocation.

For example, the below method \code{getMidpoint} returns a \code{Point} with the same
immutability type as the passed parameters if \code{p1} and \code{p2} match
in immutability, otherwise \code{@\refclass{igj/quals}{I}} is resolved to \code{@\refclass{igj/quals}{ReadOnly}}:

\begin{Verbatim}
  static @I Point getMidpoint(@I Point p1, @I Point p2) { ... }
\end{Verbatim}

The \code{@\refclass{igj/quals}{I}} annotation value distinguishes between \code{@\refclass{igj/quals}{I}}
declarations.  So, the below method \code{findUnion} returns a collection of the same
immutability type as the \emph{first} collection parameter:

\begin{Verbatim}
  static <E> @I("First") Collection<E> findUnion(@I("First") Collection<E> col1,
                                                 @I("Second") Collection<E> col2) { ... }
\end{Verbatim}


\section{Iterators and their abstract state\label{igj-library-annotations}}

This section explains why the receiver of \<Iterator.next()> is annotated
as \<@ReadOnly>.

An iterator conceptually has two pieces of state:
\begin{enumerate}
\item
  the underlying collection
\item
  an index into that collection (indicating the next object to be returned)
\end{enumerate}

We choose to exclude the index from the abstract state of the iterator.
That is, a change to the index does not count as a mutation of the
iterator itself.

Changes to the underlying collection are more important and interesting,
and unintentional changes are much more likely to lead to important
errors.  Therefore, this choice about the iterator's abstract state
appears to be more useful than other choices.  For example, if the
iterator's abstract state included both the underlying collection and
the index, then there would be no way to express, or check, that
\<Iterator.next> does not change the underlying collection.


\section{Examples\label{igj-example}}

To try the IGJ checker on a source file that uses the IGJ qualifier, use
the following command (where \code{javac} is the JSR 308 compiler that
is distributed with the Checker Framework).

\begin{Verbatim}
  javac -processor checkers.igj.IGJChecker examples/IGJExample.java
\end{Verbatim}

The IGJ checker itself is also annotated with IGJ annotations.


% LocalWords:  plugin ReadOnly AssignsFields im templating getMidpoint cp TODO
% LocalWords:  findUnion igj IGJ's quals ESEC readonly covariant
% LocalWords:  NullnessAnnotatedTypeFactory IGJAnnotatedTypeFactory

\htmlhr
\chapter{Javari immutability checker\label{javari-checker}}

\textbf{Note:} The IGJ type-checker has some known bugs and limitations.
Nonetheless, it may still be useful to you.

Javari~\cite{TschantzE2005,QuinonezTE2008} is a Java language extension that helps programmers to avoid mutation
errors that result from unintended side effects.
If the Javari Checker issues no warnings for a given program, then that
program will never change objects that should not be changed.  This
guarantee enables a programmer to detect and prevent mutation-related
errors.  (See Section~\ref{checker-guarantees} for caveats to the guarantee.)
The Javari webpage (\myurl{http://types.cs.washington.edu/javari/}) contains
papers that explain the Javari language and type system.
By contrast to those papers, the Javari Checker uses an annotation-based
dialect of the Javari language.

The Javarifier tool infers Javari types for an existing program; see
Section~\ref{javari-inference}.

Also consider the IGJ Checker (Chapter~\ref{igj-checker}).  The IGJ type
system is more expressive than that of Javari, and the IGJ Checker is a bit
more robust.  However, IGJ lacks a type inference tool such as Javarifier.

To run the Javari Checker, supply the \code{-processor
  checkers.javari.JavariChecker} command-line option to javac.  For
examples, see Section~\ref{javari-examples}.



\begin{figure}
\includeimage{javari}{2.5cm}
\caption{Type hierarchy for Javari's ReadOnly type qualifier.}
\label{fig:javari-hierarchy}
\end{figure}


\section{Javari annotations\label{javari-annotations}}

The following six annotations make up the Javari type system.

\begin{description}

\item[\code{@\refclass{javari/quals}{ReadOnly}}]
  indicates a type that provides only read-only access.  A reference of
  this type may not be used to modify its referent, but aliasing references
  to that object might change it.

\item[\code{@\refclass{javari/quals}{Mutable}}]
  indicates a mutable type.
  
\item[\code{@\refclass{javari/quals}{Assignable}}]
  is a field annotation, not a type qualifier.  It indicates that the given
  field may always be assigned, no matter what the type of the reference
  used to access the field.
  
\item[\code{@\refclass{javari/quals}{QReadOnly}}]
  corresponds to Javari's ``\code{?\ readonly}'' for wildcard types.  An
  example of its use is \code{List<@QReadOnly Date>}.  It allows only the
  operations which are allowed for both readonly and mutable types.

\item[\code{@\refclass{javari/quals}{PolyRead}}]
  (previously named \code{@RoMaybe}) specifies polymorphism over
  mutability; it simulates mutability overloading.  It can be applied to
  methods and parameters.  See Section~\ref{qualifier-polymorphism} and the
  \code{@\refclass{javari/quals}{PolyRead}} Javadoc for more details.

\item[\code{@\refclass{javari/quals}{ThisMutable}}]
  means that the mutability of the field is the same as that of the
  reference that contains it.  \code{@ThisMutable} is the default on
  fields, and does not make sense to write elsewhere.  Therefore,
  \code{@ThisMutable} should never appear in a program.

\end{description}


\section{Writing Javari annotations\label{writing-javari-annotations}}


\subsection{Implicit qualifiers\label{javari-implicit-qualifiers}}

As described in Section~\ref{effective-qualifier}, the Javari Checker
adds implicit qualifiers, reducing the number of annotations that must
appear in your code.
% For example, ...

For a complete description of all implicit Javari qualifiers, see the
Javadoc for \refclass{javari}{JavariAnnotatedTypeFactory}.


\subsection{Inference of Javari annotations\label{javari-inference}}

It can be tedious to write annotations in your code.  The Javarifier tool
(\myurl{http://types.cs.washington.edu/javari/javarifier/}) infers 
Javari types for an existing program.  It 
automatically inserts Javari annotations in your Java program or
in \code{.class} files.

This has two benefits:  it relieves the programmer of the tedium of writing
annotations (though the programmer can always refine the inferred
annotations), and it annotates libraries, permitting checking of programs
that use those libraries.



\section{What the Javari Checker checks\label{javari-checks}}

The checker issues an error whenever mutation happens through a readonly
reference, when fields of a readonly reference which are not explicitly
marked with \code{@\refclass{javari/quals}{Assignable}} are reassigned, or
when a readonly expression is assigned to a mutable variable.  The checker
also emits a warning when casts increase the mutability access of a
reference.

% There is no Javadoc as of 2/2009.
% For a complete description of all checks performed by
% the Nullness Checker, see the Javadoc for
% \refclass{javari}{JavariVisitor}.


\section{Iterators and their abstract state\label{javari-library-annotations}}

For an explanation of why the receiver of \<Iterator.next()> is annotated
as \<@ReadOnly>, see Section~\ref{igj-library-annotations}.


\section{Examples\label{javari-examples}}

To try the Javari Checker on a source file that uses the Javari
qualifier, use the following command (where \code{javac} is the JSR 308
compiler  that
is distributed with the Checker Framework).  Alternately, you may
specify just one of the test files.

\begin{Verbatim}
  javac -processor checkers.javari.JavariChecker tests/javari/*.java
\end{Verbatim}

\noindent
The compiler should issue the errors and warnings (if any) specified in the
\code{.out} files with same name.

To run the test suite for the Javari Checker, use \code{ant javari-tests}.

The Javari Checker itself is also annotated with Javari annotations.


% LocalWords:  PolyRead javari cp plugin ReadOnly QReadOnly romaybe Javarifier
% LocalWords:  readonly wildcard Javadoc javac RoMaybe quals IGJ ThisMutable
% LocalWords:  JavariAnnotatedTypeFactory

\htmlhr
\chapter{Lock Checker\label{lock-checker}}

The Lock Checker prevents certain kinds of concurrency errors.  If the Lock
checker issues no warnings for a given program, then the program holds the
appropriate lock every time that it accesses a variable.

Note:  This does \emph{not} mean that your program has \emph{no} concurrency
errors.  (You might have forgotten to annotate that a particular variable
should only be accessed when a lock is held.  You might release and
re-acquire the lock, when correctness requires you to hold it throughout a
computation.  And, there are other concurrency errors that cannot, or
should not, be solved with locks.)  However, ensuring that your
program obeys its locking discipline is an easy and effective way to
eliminate a common and important class of errors.


To run the Lock Checker, supply the \code{-processor
  checkers.lock.LockChecker} command-line option to javac.


\section{Lock annotations\label{lock-annotations}}

The Lock Checker uses two annotations.  One is a type qualifier, and the
other is a method annotation.

\begin{description}

\item[\code{@\refclass{checkers/lock/quals}{GuardedBy}}]
  indicates a type whose value may be accessed only when the given lock is
  held.
  See the \ahref{api/checkers/lock/quals/GuardedBy.html}{GuardedBy
    Javadoc} for an explanation of the argument and other details.  The lock
  acquisition and the value access may be arbitrarily far in the future;
  or, if the value is never accessed, the lock never need be held.
  Figure~\ref{fig-guardedby-hierarchy} gives the type hierarchy.

\begin{figure}
\includeimage{guardedby}{2.5cm}
\caption{Type hierarchy for the \code{@GuardedBy} annotation of the lock
  type system.  The \<@GuardedByTop> annotation is for internal use by the
  type checker; a programmer cannot write it.}
\label{fig-guardedby-hierarchy}
\end{figure}


\item[\code{@\refclass{checkers/lock/quals}{Holding}}]
  is a method annotation (not a type qualifier).  It indicates that when
  the method is called, the given lock must be held by the caller.
  In other words, the given lock is already held at the time the method is
  called.

\end{description}

\subsection{Examples}

The most common use of \code{@GuardedBy} is to annotate a field declaration
type.  However, other uses of \code{@GuardedBy} are possible.

\paragraph{Return values}

A return value may be annotated with \code{@GuardedBy}:

\begin{Verbatim}
  @GuardedBy("MyClass.myLock") Object myMethod() { ... }

  // reassignments without holding the lock are OK.
  @GuardedBy("MyClass.myLock") Object x = myMethod();
  @GuardedBy("MyClass.myLock") Object y = x;
  Object z = x;  // ILLEGAL (assuming no lock inference),
                 // because z can be freely accessed.
  x.toString() // ILLEGAL because the lock is not held
  synchronized(MyClass.myLock) {
    y.toString();  // OK: the lock is held
  }
\end{Verbatim}

\paragraph{Formal parameters}

A parameter may be annotated with \code{@GuardedBy}, which indicates that
the method body must acquire the lock before accessing the parameter.  A
client may pass a non-\code{@GuardedBy} reference as an argument, since it
is legal to access such a reference after the lock is acquired.

\begin{Verbatim}
  void helper1(@GuardedBy("MyClass.myLock") Object a) {
    a.toString(); // ILLEGAL: the lock is not held
    synchronized(MyClass.myLock) {
      a.toString();  // OK: the lock is held
    }
  }
  @Holding("MyClass.myLock")
  void helper2(@GuardedBy("MyClass.myLock") Object b) {
    b.toString(); // OK: the lock is held
  }
  void helper3(Object c) {
    helper1(c); // OK: passing a subtype in place of a the @GuardedBy supertype
    c.toString(); // OK: no lock constraints
  }
  void helper4(@GuardedBy("MyClass.myLock") Object d) {
    d.toString(); // ILLEGAL: the lock is not held
  }
  void myMethod2(@GuardedBy("MyClass.myLock") Object e) {
    helper1(e);  // OK to pass to another routine without holding the lock
    e.toString(); // ILLEGAL: the lock is not held
    synchronized (MyClass.myLock) {
      helper2(e);
      helper3(e);
      helper4(e); // OK, but helper4's body still does not type-check
    }
  }
\end{Verbatim}


    

\subsection{Discussion of \<@Holding>}

A programmer might choose to use the \code{@Holding} method annotation in
two different ways:  to specify a higher-level protocol, or to summarize
intended usage.  Both of these approaches are useful, and the Lock Checker
supports both.

\paragraph{Higher-level synchronization protocol}

  \code{@Holding} can specify a higher-level synchronization protocol that
  is not expressible as locks over Java objects.  By requiring locks to be
  held, you can create higher-level protocol primitives without giving up
  the benefits of the annotations and checking of them.

\paragraph{Method summary that simplifies reasoning}

  \code{@Holding} can be a method summary that simplifies reasoning.  In
  this case, the \code{@Holding} doesn't necessarily introduce a new
  correctness constraint; the program might be correct even if the lock
  were acquired later in the body of the method or in a method it calls, so
  long as the lock is acquired before accessing the data it protects.

  Rather, here \code{@Holding} expresses a fact about execution:  when
  execution reaches this point, the following locks are already held.  This
  fact enables people and tools to reason intra- rather than
  inter-procedurally.

  In Java, it is always legal to re-acquire a lock that is already held,
  and the re-acquisition always works.  Thus, whenever you write 

\begin{Verbatim}
  @Holding("myLock")
  void myMethod() {
    ...
  }
\end{Verbatim}

\noindent
it would be equivalent, from the point of view of which locks are held
during the body, to write

\begin{Verbatim}
  void myMethod() {
    synchronized (myLock) {   // no-op:  re-aquire a lock that is already held
      ...
    }
  }
\end{Verbatim}

The advantages of the \<@Holding> annotation include:
\begin{itemize}
\item
  The annotation documents the fact that the lock is intended to already be
  held.
\item
  The Lock Checker enforces that the lock is held when the method is
  called, rather than masking a programmer error by silently re-acquiring
  the lock.
\item
  The \<synchronized> statement can deadlock if, due to a programmer error,
  the lock is not already held.  The Lock Checker prevents this type of
  error.
\item
  The annotation has no run-time overhead.  Even if the lock re-acquisition
  succeeds, it still consumes time.
\end{itemize}


\section{Other lock annotations\label{other-lock-annotations}}

The Checker Framework's lock annotations are similar to annotations used
elsewhere.

If your code is already annotated with a different lock
annotation, you can reuse that effort.  The Checker Framework comes with
cleanroom re-implementations of annotations from other tools.  It treats
them exactly as if you had written the corresponding annotation from the
Lock Checker, as described in Figure~\ref{fig-lock-refactoring}.


% These lists should be kept in sync with LockAnnotatedTypeFactory.java .
\begin{figure}
\begin{center}
% The ~ around the text makes things look better in Hevea (and not terrible
% in LaTeX).
\begin{tabular}{ll}
\begin{tabular}{|l|}
\hline
 ~net.jcip.annotations.GuardedBy~ \\ \hline
\end{tabular}
&
$\Rightarrow$
~checkers.lock.quals.GuardedBy~
\end{tabular}
\end{center}
%BEGIN LATEX
\vspace{-1.5\baselineskip}
%END LATEX
\caption{Correspondence between other lock annotations and the
  Checker Framework's annotations.}
\label{fig-lock-refactoring}
\end{figure}

Alternately, the Checker Framework can process those other annotations (as
well as its own, if they also appear in your program).  The Checker
Framework has its own definition of the annotations on the left side of
Figure~\ref{fig-lock-refactoring}, so that they can be used as type
qualifiers.  The Checker Framework interprets them according to the right
side of Figure~\ref{fig-lock-refactoring}.


\subsection{Relationship to annotations in \emph{Java Concurrency in Practice}\label{jcip-annotations}}

The book \ahref{\url{http://jcip.net/}}{\emph{Java Concurrency in Practice}}~\cite{Goetz2006} defines a
\ahref{\url{http://jcip.net.s3-website-us-east-1.amazonaws.com/annotations/doc/net/jcip/annotations/GuardedBy.html}}{\code{@GuardedBy}} annotation that is the inspiration for ours.  The book's
\code{@GuardedBy} serves two related but distinct purposes:

\begin{itemize}
\item
  When applied to a field, it means that the given lock must be held when
  accessing the field.  The lock acquisition and the field access may be
  arbitrarily far in the future.
\item
  When applied to a method, it means that the given lock must be held by
  the caller at the time that the method is called --- in other words, at
  the time that execution passes the \code{@GuardedBy} annotation.
\end{itemize}

The Lock Checker renames the method annotation to
\code{@\refclass{checkers/lock/quals}{Holding}}, and it generalizes the 
\code{@\refclass{checkers/lock/quals}{GuardedBy}} annotation into a type qualifier
that can apply not just to a field but to an arbitrary type (including the
type of a parameter, return value, local variable, generic type parameter,
etc.).  This makes the annotations more expressive and also more amenable
to automated checking.  It also accommodates the distinct
meanings of the two annotations, and resolves ambiguity when \<@GuardedBy>
is written in a location that might apply to either the method or the
return type.

(The JCIP book gives some rationales for reusing the annotation name for
two purposes.  One rationale is
that there are fewer annotations to learn.  Another rationale is
that both variables and methods are ``members'' that can be ``accessed'';
variables can be accessed by reading or writing them (putfield, getfield),
and methods can be accessed by calling them (invokevirtual,
invokeinterface):  in both cases, \code{@GuardedBy} creates preconditions
for accessing so-annotated members.  This informal intuition is
inappropriate for a tool that requires precise semantics.)

% It would not work to retain the name \code{@GuardedBy} but put it on the
% receiver; an annotation on the receiver indicates what lock must be held
% when it is accessed in the future, not what must have already been held
% when the method was called.


\section{Possible extensions\label{lock-extensions}}

The Lock Checker validates some uses of locks, but not all.  It would be
possible to enrich it with additional annotations.  This would increase the
programmer annotation burden, but would provide additional guarantees.

Lock ordering:  Specify that one lock must be acquired before or after
another, or specify a global ordering for all locks.  This would prevent
deadlock.

Not-holding:  Specify that a method must not be called if any of the listed
locks are held.

These features are supported by 
\ahref{\url{http://clang.llvm.org/docs/LanguageExtensions.html#thread-safety-annotation-checking}}{Clang's thread-safety annotations}.


% LocalWords:  quals GuardedBy JCIP putfield getfield invokevirtual 5cm
% LocalWords:  invokeinterface threadsafety Clang's GuardedByTop cleanroom
%%  LocalWords:  api 5cm

\htmlhr
\chapter{Basic checker\label{basic-checker}}

The Basic checker enforces only subtyping rules.  It operates over
annotations specified by a user on the command line.  Thus, users can
create a simple type checker without writing any code beyond definitions of
the type qualifier annotations.

The Basic checker can accommodate all of the type system enhancements that
can be declaratively specified (see Chapter~\ref{writing-a-checker}).
This includes type introduction rules (implicit
annotations, e.g., literals are implicitly considered \code{@\refclass{nullness/quals}{NonNull}}) via
the \code{@\refclass{quals}{ImplicitFor}} meta-annotation, and other features such as
flow-sensitive type qualifier inference (Section~\ref{type-refinement}) and
qualifier polymorphism (Section~\ref{qualifier-polymorphism}).

The Basic checker is also useful to type system designers who wish to
experiment with a checker before writing code; the Basic checker
demonstrates the functionality that a checker inherits from the Checker
Framework.

If you need typestate analysis, then you can extend a typestate checker,
much as you would extend the Basic Checker if you do not need typestate
analysis.  For more details (including a definition of ``typestate''), see
Chapter~\ref{typestate-checker}.

For type systems that require special checks (e.g., warning about
dereferences of possibly-null values), you will need to write code and
extend the framework as discussed in Chapter~\ref{writing-a-checker}.


\section{Using the Basic checker\label{basic-using}}

The Basic checker is used in the same way as other checkers (using the
\code{-processor checkers.basic.BasicChecker} option; see Chapter~\ref{using-a-checker}), except that it
requires an additional annotation processor argument via the standard
``\code{-A}'' switch:

\begin{itemize}

\item
\code{-Aquals}: this option specifies a comma-no-space-separated list of
the fully-qualified class
names of the annotations used as qualifiers in the custom type system.  For
example,

\begin{alltt}
  javac -processor checkers.fenum.BasicChecker
        \textit{-Aquals=myproject.quals.MyQual,myproject.quals.OtherQual} MyFile.java ...
\end{alltt}

It serves the same purpose as the \code{@\refclass{quals}{TypeQualifiers}}
annotation used by other checkers (see section
\ref{writing-compiler-interface}).

The annotations listed in \code{-Aquals} must be accessible to
the compiler during compilation in the classpath.  In other words, they must
already be compiled before you run the Basic checker with \code{javac}; it
is not sufficient to supply their source files on the command line.

\end{itemize}

To suppress a warning issued by the basic checker, use a 
\code{@\sunjavadoc{java/lang/SuppressWarnings.html}{SuppressWarnings}}
annotation, with the argument being the unqualified, uncapitalized name of
any of the annotations passed to \code{-Aquals}.  This will suppress all
warnings, regardless of which of the annotations is involved in the
warning.  (As a matter of style, you should choose one of the annotations
as your \code{@SuppressWarnings} key and stick with it for that entire type
hierarchy.)


\section{Basic checker example\label{basic-example}\label{encrypted-example}}

Consider a hypothetical \code{Encrypted} type qualifier, which denotes that the
representation of an object (such as a \code{String}, \code{CharSequence}, or
\code{byte[]}) is encrypted. To use the Basic checker for the \code{Encrypted}
type system, follow three steps.

\begin{enumerate}
\item
 Define an annotation for the \code{Encrypted} qualifier:

\begin{Verbatim}
package myquals;

import checkers.quals.*;

/**
 * Denotes that the representation of an object is encrypted.
 * ...
 */
@TypeQualifier
@SubtypeOf(Unqualified.class)
@Target({ElementType.TYPE_PARAMETER, ElementType.TYPE_USE})
public @interface Encrypted {}
\end{Verbatim}

Don't forget to compile this class:

\begin{Verbatim}
$ javac myquals/Encrypted.java
\end{Verbatim}

The resulting \<.class> file should either be on your classpath, or on the
processor path (set via the \<-processorpath> command-line option to javac).

\item 
  Write \code{@Encrypted} annotations in your program (YourProgram.java):

\begin{Verbatim}
import myquals.Encrypted;

...

public @Encrypted String encrypt(String text) {
    // ...
}

// Only send encrypted data!
public void sendOverInternet(@Encrypted String msg) {
    // ...
}

void sendText() {
    // ...
    @Encrypted String ciphertext = encrypt(plaintext);
    sendOverInternet(ciphertext);
    // ...
}

void sendPassword() {
    String password = getUserPassword();
    sendOverInternet(password);
}
\end{Verbatim}

You may also need to add \code{@SuppressWarnings} annotations to the
\code{encrypt} and \code{decrypt} methods.  Analyzing them is beyond the
capability of any realistic type system.

\item
  Invoke the compiler with the Basic checker, specifying the
  \code{@Encrypted} annotation using the \code{-Aquals} option.
  You should add the \code{Encrypted} classfile to the processor classpath:

\begin{Verbatim}
$ javac -processorpath myqualspath -processor checkers.basic.BasicChecker \
        -Aquals=myquals.Encrypted YourProgram.java

YourProgram.java:42: incompatible types.
found   : java.lang.String
required: @myquals.Encrypted java.lang.String
    sendOverInternet(password);
                     ^
\end{Verbatim}

\end{enumerate}


\htmlhr
\chapter{Typestate checker\label{typestate-checker}}

In a regular type system, a variable has the same type throughout its
scope.
In a typestate system, a variable's type can change as operations
are performed on it.

The most common example of typestate is for a \<File> object.  Assume a file
can be in two states, \<@Open> and \<@Closed>.  Calling the \<close()> method
changes the file's state.  Any subsequent attempt to read, write, or close
the file will lead to a run-time error.  It would be better for the type
system to warn about such problems, or guarantee their absence, at compile
time.

Just as you can extend the Basic Checker to create a type checker, you can
extend a typestate checker to create a type checker that supports typestate
analysis.  Two extensible typestate analyses that build on the Checker
Framework are available.  One is by Adam Warski:
\myurl{http://www.warski.org/typestate.html}.
The other is by Daniel Wand:
\myurl{http://typestate.ewand.de/}.


\section{Comparison to flow-sensitive type refinement\label{typestate-vs-type-refinement}}

The Checker Framework's flow-sensitive type refinement
(Section~\ref{type-refinement}) implements a form of typestate analysis.
For example, after code that tests a variable against null, the Nullness
Checker (Chapter~\ref{nullness-checker}) treats the variable's type as
\<@NonNull \emph{T}>, for some \<\emph{T}>\@.

For many type systems, flow-sensitive type refinement is sufficient.  But
sometimes, you need full typestate analysis.  This section compares the
two.  (Dependent types and unused variables
(Section~\ref{unused-fields-and-dependent-types}) also have similarities
with typestate analysis and can occasionally substitute for it.  For
brevity, this discussion omits them.)

A typestate analysis is easier for a user to create or extend.
Flow-sensitive type refinement is built into the Checker Framework and is
optionally extended by each checker.  Modifying the rules requires writing
Java code in your checker.  By contrast, it is possible to write a simple
typestate checker declaratively, by writing annotations on the methods
(such as \<close()>) that change a reference's typestate.

A typestate analysis can change a reference's type to something that is not
consistent with its original definition.  For example, suppose that a
programmer decides that the \<@Open> and \<@Closed> qualifiers are
incomparable --- neither is a subtype of the other.  A typestate analysis
can specify that the \<close()> operation converts an \<@Open File> into a
\<@Closed File>.  By contrast, flow-sensitive type refinement can only give
a new type that is a subtype of the declared type --- for flow-sensitive
type refinement to be effective, \<@Closed> would need to be a child of
\<@Open> in the qualifier hierarchy (and \<close()> would need to be
treated specially by the checker).


\htmlhr
\chapter{External checkers\label{external-checkers}}

The checker framework has been used to build other checkers that are not
distributed together with the framework.

If you want a reference to your checker included in this chapter,
send us a link and short description of your checker, 


\section{Units and dimensions checker\label{units-checker}}

A checker for units and dimensions is available at
\url{http://www.lexspoon.org/expannots/}.


\section{Thread locality checker}

Loci, a checker for thread locality, is available at
\url{http://www.it.uu.se/research/upmarc/loci/}.
Developer resources are available at the project page
\url{http://java.net/projects/loci/}.


% In a mail from Amanj Mahmud <amanjpro@gmail.com> on 28.03.2011:

% The plugin name: 
% ``Loci: A Pluggable Type Checker for Expressing Thread Locality in
% Java''

% Project homepage: http://www.it.uu.se/research/upmarc/loci

% Project's developer's page: http://java.net/projects/loci


\section{Safety-Critical Java checker}

A checker for Safety-Critical Java (SCJ, JSR 302) is available at
\url{http://sss.cs.purdue.edu/projects/oscj/checker/checker.html}.
Developer resources are available at the project page
\url{http://code.google.com/p/scj-jsr302/}.


% In a mail from Aleš Plšek <aplsek@gmail.com> on 29.03.2011:

% Name: SCJ Checker
% WWW: http://sss.cs.purdue.edu/projects/oscj/checker/checker.html
% Source-Code Repository: http://code.google.com/p/scj-jsr302/

% Description: The SCJ Checker implements verification of a set of
% annotations defined by the Safety-Critical Java standard (JSR-302).
% The checker mainly focuses on proving memory safety of Java programs
% that use a region-based memory management.

% Publications: Static checking of safety critical Java annotations:
% http://portal.acm.org/citation.cfm?doid=1850771.1850792



% LocalWords:  TODO ImplicitFor Aquals TypeQualifiers sourcepath java NonNull
% LocalWords:  CharSequence classpath nullness quals SuppressWarnings classfile
% LocalWords:  uncapitalized processorpath Warski

\htmlhr
\chapter{Annotating libraries\label{annotating-libraries}}

When annotated code uses an unannotated library, a checker may issue warnings.
As described in Section~\ref{unannotated-code}, the best way to correct
this problem is to add annotations to the library.  (Alternately, you can instead
suppress all warnings related to an unannotated library by use of the
\code{-AskipUses} or \code{-AonlyUses} command-line option; see
Section~\ref{askipuses}.)  If you have source code for the
library, you can easily add the annotations.
This chapter tells you
how to add annotations to a library for which you have no source code,
because the library is distributed only in binary form
(as \code{.class} files, possibly packaged in a \code{.jar} file).
This chapter is also useful if you do not wish to edit the
library's source code.

Note that this chapter is about annotating libraries, not analyzing them.
The Checker Framework analyzes all, and only, the source code that is
passed to it.  The Checker Framework is a plug-in to the javac compiler,
and it never analyzes code that is not being compiled, though it does look
up annotations for code that is not being compiled.

You can make the library's annotations known to the checkers in two ways.

\begin{itemize}

\item
  You can write annotations in a ``stub
  file'' containing classes with no method bodies.
  Section~\ref{stub} describes how to create and use stub files.

\item
  You can insert annotations in the compiled
  \code{.class} files of the library.
  You would express the annotations textually, typically as an annotation index file, and
  then insert them in the library by using the Annotation File Utilities
  (\myurl{http://types.cs.washington.edu/annotation-file-utilities/}).
  See the Annotation File Utilities documentation for full details.

\end{itemize}

The Checker Framework distribution contains annotations for popular
libraries, such as the JDK6 and JDK7\@.  It uses both of the above mechanisms.  The
Nullness, Javari, IGJ, and Interning Checkers use the annotated JDKs
(Section~\ref{skeleton}), and all other checkers use stub files
(Section~\ref{stub}).

If you annotate additional libraries, please share them with us so that we
can distribute the annotations with the Checker Framework; see
Section~\ref{reporting-bugs}.
You can determine the correct annotations for a library either
automatically by running an inference tool, or manually by reading the
documentation.  Presently, type inference tools are available for the
Nullness (Section~\ref{nullness-inference}) and Javari
(Section~\ref{javari-inference}) type systems.


\section{Choosing between stub files and annotated \<.class> files\label{stub-vs-class-files}}

A checker can read annotations either from a stub file or from a library's
\<.class> files.  This section helps you choose between the two alternatives.

Once created, a stub file can be used directly; this makes it an easy way
to get started with library annotations.
When provided by the author of the checker, a stub file is used
automatically, with no need for the user to supply a command-line option.

Inserting annotations in a library's \<.class> files takes an extra step
using an external tool, the Annotation File Utilities
(\myurl{http://types.cs.washington.edu/annotation-file-utilities/}).
However, this process does not suffer the limitations of stub files
(Section~\ref{stub-limitations}).


\section{Using stub classes\label{stub}\label{stub-creating-and-using}}

A stub file contains ``stub classes'' that contain annotated signatures,
but no method bodies.  A
checker uses the annotated signatures at compile time, instead of or in
addition to annotations that appear in the library.

Section~\ref{stub-creating} describes how to create stub classes.
Section~\ref{stub-using} describes how to use stub classes.
These sections illustrate stub classes via the example of creating a \code{@\refclass{checkers/interning/quals}{Interned}}-annotated
version of \code{java.lang.String}.  You don't need to repeat these steps
to handle \code{java.lang.String} for the Interning Checker,
but you might do something similar for a different class and/or checker.

% First, you must install the skeleton class generator
% (Section~\ref{skeleton-installing}).


\subsection{Using a stub file\label{stub-using}}

The \code{-Astubs} argument causes the Checker Framework to read
annotations from annotated stub classes in preference to the unannotated
original library classes.  For example:

%BEGIN LATEX
\begin{smaller}
%END LATEX
\begin{Verbatim}
  javac -processor checkers.interning.InterningChecker -Astubs=String.astub:stubs MyFile.java MyOtherFile.java ...
\end{Verbatim}
%BEGIN LATEX
\end{smaller}
%END LATEX

Each stub path entry is a file or a directory; specifying a directory is
equivalent to specifying every file in it whose name ends with
\code{.astub}.  The stub path entries are delimited by
\<File.pathSeparator> (`\<:>' for Linux and Mac, `\<;>' for Windows).

A checker automatically reads the stub file \code{jdk.astub}.  (The checker
author should place it in the same directory as the Checker class, i.e.,
the subclass of \code{BaseTypeVisitor}.)  Programmers should only use the
\<-Astubs> argument for additional stub files they create themselves.

If a method appears in more than one stub file (or twice in the same 
stub file), then the annotations are merged. If any of the 
methods have different annotations from the same hierarchy on the same type,
then the annotation from the last declaration is used.  

% \textbf{The following is not implemented yet}
% A library writers should create a file \code{library.astub} on the
% classpath (in the resources directory or the binary jars).
% The Checker Framework automatically imports all the stub files named
% \code{library.astub} found in the classpath.  



\subsection{Stub file format\label{stub-format}}

Every Java file is a valid stub file.  However, you can omit information
that is not relevant to pluggable type-checking; this makes the stub file
smaller and easier for people to read and write.

As an illustration, a stub file for the Interning type system
(Chapter~\ref{interning-checker}) could be:

\begin{Verbatim}
  import checkers.interning.quals.Interned;
  package java.lang;
  @Interned class Class<T> { }
  class String {
    @Interned String intern();
  }
\end{Verbatim}

Note, annotations in comments are ignored.

The stub file format is allowed to differ from Java source code in the
following ways:
\begin{description}

\item{\textbf{Method bodies:}}
  The stub class does not require method bodies for classes; any method
  body may be replaced by a semicolon (\code{;}), as in an interface or
  abstract method declaration.

\item{\textbf{Method declarations:}}
  You only have to specify the methods that you need to annotate.
  Any method declaration may be omitted, in which case the checker reads
  its annotations from library's \<.class> files.  (If you are using a stub class, then
  typically the library is unannotated.)

\item{\textbf{Declaration specifiers:}}
  Declaration specifiers (e.g., \<public>, \<final>, \<volatile>)
  may be omitted.

\item{\textbf{Return types:}}
  The return type of a method does not need to match the real method.
  In particular, it is valid to use \<java.lang.Object> for every method.
  This simplifies the creation of stub files.

\item{\textbf{Import statements:}}
  All imports must be at the beginning of the file.
  The only required import statements are the ones to import type
  annotations.  Import statements for types are optional.

  Enum constants in annotations need to be either fully qualified
  or imported.
  For example, one has to either write the enum constant \<ANY> in
  fully-qualified form:

\begin{Verbatim}
@Source(sparta.checkers.quals.FlowPermission.ANY)
\end{Verbatim}

\noindent
or correctly import the enum class:

\begin{Verbatim}
import sparta.checkers.quals.FlowPermission;
...
@Source(FlowPermission.ANY)
\end{Verbatim}

\noindent
or statically import the enum constants:

\begin{Verbatim}
import static sparta.checkers.quals.FlowPermission.*;
...
@Source(ANY)
\end{Verbatim}

  Importing all packages from a class (\<import my.package.*;>) only
  considers annotations from that package; enum types need to be
  explicitly imported.

\item{\textbf{Multiple classes and packages:}}
  The stub file format permits having multiple classes and packages.
  The packages are separated by a package statement:
  \<package my.package;>.  Each package declaration may occur only once; in
  other words, all classes from a package must appear together.

\end{description}



\subsection{Creating a stub file\label{stub-creating}}


\subsubsection{If you have access to the Java source code}

Every Java file is a stub file.  If you have access to the Java file, then
you can use the Java file as the stub file, without removing
any of the parts that the stub file format permits you to.  Just add
annotations to the signatures, leaving the method bodies unchanged.
Optionally (but highly recommended!), run the type-checker to verify that
your annotations are correct.  When you run the type-checker on your
annotations, there should not be any stub file that also contains
annotations for the class.  In particular, if you are type-checking the JDK
itself, then you should use the \<-Aignorejdkastub> command-line option.

This approach retains the original
documentation and source code, making it easier for a programmer to
double-check the annotations.  It also enables creation of diffs, easing
the process of upgrading when a library adds new methods.  And, the
annotations are in a format that the library maintainers can even
incorporate.

The downside of this approach is that the stub files are larger.  This can
slow down parsing.  Furthermore, a programmer must search the stub file
for a given method rather than just skimming a few pages of method signatures.


\subsubsection{If you do not have access to the Java source code}

If you do not have access to the library source code, then you can create a
stub file from the class file (Section~\ref{stub-creating}),
and then annotate it.  The rest of this section describes this approach.


\begin{enumerate}

\item
  Create a stub file by running the stub class generator.  (\<checkers.jar> and \<javac.jar>
  must be on your classpath.)

\begin{Verbatim}
  cd nullness-stub
  java checkers.util.stub.StubGenerator java.lang.String > String.astub
\end{Verbatim}

  Supply it with the fully-qualified name of the class for which you wish to
  generate a stub class.  The stub class generator prints the
  stub class to standard out, so you may wish to redirect its output to a
  file.

\item
  Add import statements for the annotations.  So you would need to
add the following import statement at the beginning of the file:

\begin{Verbatim}
  import checkers.interning.quals.*;
\end{Verbatim}

\noindent
The stub file parser silently ignores any annotations that it cannot
resolve to a type, so don't forget the import statement.
Use the \<-AstubWarnIfNotFound> command-line option to see warnings
if an entry could not be found.

\item
  Add annotations to the stub class.  For example, you might annotate
  the \sunjavadoc{java/lang/String.html#intern()}{String.intern()} method as follows:

\begin{Verbatim}
  @Interned String intern();
\end{Verbatim}

  You may also remove irrelevant parts of the stub file; see
  Section~\ref{stub-format}.

\end{enumerate}


Two command-line options can be used to debug the behavior of stub
files:
\<-AstubWarnIfNotFound> warns if a stub file entry could not be
found. Annotations on unknown classes and methods are silently
ignored. Use this option to ensure that all stub file entries could be
resolved.
\<-AstubDebug> outputs debug messages while parsing stub files.


\subsection{Troubleshooting stub libraries\label{stub-troubleshooting}}

An error is issued if a stub file has a typo or the API method does not
exist.

Fix this error by removing the extra L in the method name:
\begin{Verbatim}
StubParser: Method isLLowerCase(char) not found in type java.lang.Character
\end{Verbatim}

Fix this error by removing the method \<enableForgroundNdefPush(...)> from
the stub file, because it is not defined in class \<android.nfc.NfcAdapter>
in the version of the library you are using:
\begin{Verbatim}
StubParser: Method enableForegroundNdefPush(Activity,NdefPushCallback) 
      not found in type android.nfc.NfcAdapter
\end{Verbatim}


\subsection{Limitations\label{stub-limitations}}

The stub file reader has several limitations.  We will fix these in a
future release.

\begin{itemize}

\item 
  The receiver is written after the method parameter list, instead of as an
  explicit first parameter.  That is, instead of

\begin{Verbatim}
     returntype methodname(@Annotations C this, params);
\end{Verbatim}

\noindent
in a stub file one has to write

\begin{Verbatim}
     returntype methodname(params) @Annotations;
\end{Verbatim}

\item
  The stub file reader
  does not handle nested classes.  To work around this, it permits a
  top-level class to be written with a \<\$> in its name, and applies the
  annotations to the appropriate nested class.

\item
  Annotations must be written before the package name on a fully qualified 
  types rather than directly on the type it qualifies.  However, it is usually not 
  necessary to write the fully qualified name.  
  
  \begin{Verbatim}
          void init(@Nullable java.security.SecureRandom random);
   \end{Verbatim}

\end{itemize}

If these limitations are a problem, then you should insert annotations
in the library's \<.class> files instead.


% Label "skeleton" is for old links from the Javarifier manual, to prevent
% them from being broken links.

\section{Using distributed annotated JDKs\label{skeleton-using}\label{skeleton}}

The Checker Framework distribution contains two
annotated JDKs at the paths \<checkers/binary/jdk6.jar> and
\<checkers/binary/jdk7.jar>.
The \<javac> that is distributed with the Checker Framework and the command
\code{java -jar \$CHECKERS/binary/checkers.jar} both use the appropriate jdk6.jar or jdk7.jar
based on the version of Java used to run them.

The annotated JDKs should \emph{not} be in your classpath at run time, only
at compile time.

% Skeleton classes are inferior to stub classes for two reasons.  First,
% skeleton files must be on the classpath during compilation but must
% \emph{not} be on the classpath during execution; this is inconvenient and
% error-prone.  Second, the skeleton files contain incorrect values for
% certain static final fields.  These incorrect values can lead to
% run-time problems unless the Java code is re-compiled without the skeleton
% classes after type-checking is complete.



% \section{Installing the skeleton class generator\label{skeleton-installing}}
%
% Source code for the skeleton class generator tool is included in the
% Checker Framework
% distribution, but because the tool has additional dependencies, the provided
% build script does not build the tool by default.
%
% Follow these steps to install the skeleton class generator:
%
% \begin{enumerate}
%
% \item
%   Install the annotation file utilities, using the instructions at
%   \myurl{http://types.cs.washington.edu/annotation-file-utilities/}.
%   Per those instructions, the \code{annotation-file-utilities.jar} file
%   should be on your classpath.
%
% % TODO This item should become optional; tell people to install the AFU in
% % the right place.
% \item
%   Update the \code{build.properties} file in the Checker Framework distribution so
%   that the \code{annotation-utils.lib} property specifies the location of
%   the \code{annotation-file-utilities.jar} library.
%
% \item
%   Build the skeleton class generator tool by running \code{ant
%     skeleton-util dist} in the \code{checkers} directory.  This updates the
%   \code{checkers.jar} file to contain the skeleton class generator.
%   \code{checkers.jar} should already be on your classpath (see
%   Section~\ref{installation}).
%
% \end{enumerate}


\section{Troubleshooting/debugging annotated libraries\label{libraries-troubleshooting}}

Sometimes, it may seem that a checker is treating a library as unannotated
even though the library has annotations.  The compiler has two flags that
may help you in determining whether library files are read, and if they are
read whether the library's annotations are parsed.

\begin{description}
\item \<-verbose>
  Outputs info about compile phases --- when the compiler
  reads/parses/attributes/writes any file.  Also outputs the classpath and
  sourcepath paths.
\item \<-XDTA:parser> (which is equivalent to \<-XDTA:reader> plus \<-XDTA:writer>)
  Sets the internal \<debugJSR308> flag, which output information about
  reading and writing.
\end{description}


% LocalWords:  plugin utils util dist RuntimeException NonNull TODO AFU enum
% LocalWords:  sourcepath Nullness javac classpath src quals pathSeparator JDKs
% LocalWords:  IGJ's jdk Astubs skipUses astub AskipUses toArray IGJ JDK6
% LocalWords:  CollectionToArrayHeuristics BaseTypeVisitor Xbootclasspath
% LocalWords:  Interning's UsesObjectEquals Anocheckjdk AonlyUses java
%  LocalWords:  Aignorejdkastub AstubWarnIfNotFound AstubDebug jdk6 jdk7
%  LocalWords:  enableForgroundNdefPush XDTA debugJSR308

\htmlhr
\chapter{How to create a new checker\label{writing-a-checker}}

\newcommand{\TreeAPIBase}{http://types.cs.washington.edu/checker-framework/jdk-api/javac/com/sun/source}
\newcommand{\refTreeclass}[2]{\href{\TreeAPIBase{}/#1/#2.html?is-external=true}{\<#2>}}
\newcommand{\ModelAPIBase}{http://types.cs.washington.edu/checker-framework/jdk-api/javac/javax/lang/model}
\newcommand{\refModelclass}[2]{\href{\ModelAPIBase{}/#1/#2.html?is-external=true}{\<#2>}}

This chapter describes how to create a checker
--- a type-checking compiler plugin that detects bugs or verifies their
absence.  After a programmer annotates a program,
the checker plugin verifies that the code is consistent
with the annotations.
If you only want to \emph{use} a checker, you do not need to read this
chapter.

% TODO: address all the issues and remove this paragraph.
\textbf{Warning:} Due to recent improvements and refactorings, a few parts
of this chapter are out of date as of December 2013.  The Checker Framework
developers are working to update it.  If you notice inaccuracies or can
make suggestions to improve this chapter, please do so.  Thanks!


Writing a simple checker is easy!  For example, here is a complete, useful
type-checker:

\begin{Verbatim}
@TypeQualifier
@SubtypeOf(Unqualified.class)
@Target({ElementType.TYPE_USE, ElementType.TYPE_PARAMETER})
public @interface Encrypted {}
\end{Verbatim}

This checker is so short because it builds on the Subtyping Checker
(Chapter~\ref{subtyping-checker}).
See Section~\ref{subtyping-example} for more details about this particular checker.
When you wish to create a new checker, it is often easiest to begin by
building it declaratively on top of the Subtyping Checker, and then return to
this chapter when you need more expressiveness or power than the Subtyping
Checker affords.

You can also create your own checker by customizing a different existing
checker.  Specific checkers that are designed for extension (besides the Subtyping
Checker) include the Fake Enumeration Checker
(\chapterpageref{fenum-checker}), the Units Checker
(\chapterpageref{units-checker}), and the typestate checkers
(\chapterpageref{typestate-checker}).
Or, you can copy and then modify a different existing checker --- whether
one distributed with the Checker Framework or a third-party one.

\begin{sloppypar}
You can place your checker's source files wherever you like.  When you
compile your checker, \<\$CHECKERFRAMEWORK/framework/dist/framework.jar> and \<\$CHECKERFRAMEWORK/framework/dist/javac.jar>
should be on your classpath.  (If you wish to modify an existing checker in place,
or to place the source code for your new checker in your own private copy of the
Checker Framework source code, then you need to be able to re-compile the
Checker Framework, as described in Section~\ref{build-source}.)
\end{sloppypar}

The rest of this chapter contains many details for people who want to write more powerful
checkers.
You do not need all of the details, at least at first.
In addition to reading this chapter of the manual, you may find it helpful
to examine the implementations of the checkers that are distributed with
the Checker Framework.  You can even create your checker by modifying one
of those.
The Javadoc documentation of the framework and the checkers is in the
distribution and is also available online at
\myurl{http://types.cs.washington.edu/checker-framework/current/api/}.

If you write a new checker and wish to advertise it to the world, let us
know so we can mention it in the Checker Framework Manual, link to
it from the webpages, or include it in the Checker Framework distribution.
For examples, see Chapters~\ref{typestate-checker}
and~\ref{third-party-checkers}.


\section{Relationship of the Checker Framework to other tools\label{tool-relationships}}

This table shows the relationship among various tools.
All of the tools support the Java 8 type annotation syntax.
You use the Checker Framework to build pluggable type systems, and the
Annotation File Utilities to manipulate \code{.java} and \code{.class} files.

\newlength{\bw}
\setlength{\bw}{.5in}

%% Strictly speaking, "Subtyping Checker" should sit on top of Checker
%% Framework and below all the specific checkers.  But omit it for simplicity.

% Unfortunately, Hevea inserts a horizontal line between every pair of rows
% regardless of whether there is a \hline or \cline.  So, make paragraphs.
\begin{center}
\begin{tabular}{|p{\bw}|p{\bw}|p{\bw}|p{\bw}|p{.4\bw}|p{\bw}|p{1.5\bw}|p{1\bw}|}
\cline{1-4} \cline{6-6}
\centering Subtyping \par Checker &
\centering Nullness \par Checker &
\centering Mutation \par Checker &
\centering Tainting \par Checker &
\centering \ldots &
\centering Your \par Checker &
\multicolumn{2}{c}{} 
\\ \hline
\multicolumn{6}{|p{6\bw}|}{\centering Base Checker \par (enforces subtyping rules)} &
\centering Type \par inference &
% Adding "\centering" here causes a LaTeX alignment error
Other \par tools
\\ \hline
\multicolumn{6}{|p{6\bw}|}{\centering Checker Framework \par (enables creation of pluggable type-checkers)} &
\multicolumn{2}{p{3\bw}|}{\centering \href{http://types.cs.washington.edu/annotation-file-utilities/}{Annotation File Utilities} \par (\code{.java} $\leftrightarrow$ \code{.class} files)} 
\\ \hline
\multicolumn{8}{|p{8.5\bw}|}{\centering
  \href{http://types.cs.washington.edu/jsr308/}{Type Annotations} syntax
  and classfile format (``JSR 308'') \par \centering (no built-in semantics)} \\ \hline
\end{tabular}
\end{center}


The Base Checker enforces the standard subtyping rules on extended types.
The Subtyping Checker is a simple use of the Base Checker that supports
providing type qualifiers on the command line.
You usually want to build your checker on the Base Checker.


\section{The parts of a checker\label{parts-of-a-checker}}

The Checker Framework provides abstract base classes (default
implementations), and a specific checker overrides as little or as much of
the default implementations as necessary.
%
Sections~\ref{writing-typequals}--\ref{writing-compiler-interface} describe
the components of a type system as written using the Checker Framework:

\begin{description}

\item{\ref{writing-typequals}}
  \textbf{Type qualifiers and hierarchy.}  You define the annotations for
  the type system and the subtyping relationships among qualified types
  (for instance, that \<@NonNull Object> is a subtype of \<@Nullable
  Object>).

\item{\ref{writing-type-introduction}}
  \textbf{Type introduction rules.}  For some types and
  expressions, a qualifier should be treated as implicitly present even if a
  programmer did not explicitly write it.  For example, in the Nullness
  type system every literal
  other than \<null> has a \refqualclass{checker/nullness/qual}{NonNull} type;
  examples of literals include \<"some string"> and \<java.util.Date.class>.

\item{\ref{extending-visitor}}
  \textbf{Type rules.}  You specify the type system semantics (type
  rules), violation of which yields a type error.  There are two types of
  rules.
\begin{itemize}
\item
  Subtyping rules related to the type hierarchy, such as that every
  assignment and pseudo-assignment satisfies a subtyping relationship.
  Your checker automatically inherits these subtyping rules from the Base
  Checker (Chapter~\ref{subtyping-checker}).
\item
  Additional rules that are specific to your particular checker.  For
  example, in the Nullness type system, only references with a
  \refqualclass{checker/nullness/qual}{NonNull} type may be dereferenced.  You
  write these additional rules yourself.
\end{itemize}

\item{\ref{writing-compiler-interface}}
  \textbf{Interface to the compiler.}  The compiler interface indicates
  which annotations are part of the type system, which command-line options
  and \<@SuppressWarnings> annotations the checker recognizes, etc.
\end{description}


\section{Annotations: Type qualifiers and hierarchy\label{writing-typequals}}

A type system designer specifies the qualifiers in the type system and
the type hierarchy that relates them.

%% True, but seems irrelevant here, so it detracts from the message.
% Each qualifier restricts the values that
% a type can represent.  For example \<@NonNull String> type can only
% represent non-null values, indicating that the variable may not hold
% \<null> values.

Type qualifiers are defined as Java annotations~\cite{JSR269}.  In Java, an
annotation is defined using the Java \code{@interface} keyword.
For example:

\begin{Verbatim}
  // Define an annotation for the @NonNull type qualifier.
  @TypeQualifier
  @Target({ElementType.TYPE_USE, ElementType.TYPE_PARAMETER})
  public @interface NonNull { }
\end{Verbatim}

\noindent
Write the \refqualclass{framework/qual}{TypeQualifier} meta-annotation on the annotation definition
to indicate that the annotation represents a type qualifier
and should be processed by the checker.
Also write a \sunjavadocanno{java/lang/annotation/Target.html}{Target}
meta-annotation to indicate where the annotation may be written.
(An annotation that is written on an annotation
definition, such as \refqualclass{framework/qual}{TypeQualifier}, is called a \emph{meta-annotation}.)

% \noindent
% The \<@TypeQualifier> meta-annotation
% distinguishes it from an ordinary
% annotation that applies to a declaration (e.g., \<@Deprecated> or
% \<@Override>).
% The framework ignores any annotation whose
% declaration does not bear the \<@TypeQualifier> meta-annotation (with minor
% exceptions, such as \<@SuppressWarnings>).

The type hierarchy induced by the qualifiers can be defined either
declaratively via meta-annotations (Section~\ref{declarative-hierarchy}), or procedurally through
subclassing \refclass{framework/type}{QualifierHierarchy} or
\refclass{framework/type}{TypeHierarchy} (Section~\ref{procedural-hierarchy}).


\subsection{Declaratively defining the qualifier and type hierarchy\label{declarative-hierarchy}}

Declaratively, the type system designer uses two meta-annotations (written
on the declaration of qualifier annotations) to specify the qualifier
hierarchy.

\begin{itemize}

\item \refqualclass{framework/qual}{SubtypeOf} denotes that a qualifier is a subtype of
  another qualifier or qualifiers, specified as an array of class
  literals.  For example, for any type $T$,
  \refqualclass{checker/nullness/qual}{NonNull} $T$ is a subtype of \refqualclass{checker/nullness/qual}{Nullable} $T$:

  \begin{Verbatim}
    @TypeQualifier
    @Target({ElementType.TYPE_USE, ElementType.TYPE_PARAMETER})
    @SubtypeOf( { Nullable.class } )
    public @interface NonNull { }
  \end{Verbatim}

  % (The actual definition of \refclass{checker/nullness/qual}{NonNull} is slightly more complex.)


  %% True, but a distraction.  Move to Javadoc?
  % (It would be more natural to use Java subtyping among the qualifier
  % annotations, but Java forbids annotations from subtyping one another.)
  %
  \refqualclass{framework/qual}{SubtypeOf} accepts multiple annotation classes as an argument,
  permitting the type hierarchy to be an arbitrary DAG\@.  For example,
  in the IGJ type system (Section~\ref{igj-annotations}), \refqualclass{checker/igj/qual}{Mutable}
  and \refqualclass{checker/igj/qual}{Immutable} induce two mutually exclusive subtypes of the
  \refqualclass{checker/igj/qual}{ReadOnly} qualifier.
%TODO: In the IGJ hierarchy I didn't find a use of multiple supertypes. Like
% this the previous paragraph is confusing, as it does not give a correct
% example.

% TODO: describe multiple type hierarchies
% TODO: describe multiple polymorphic qualifiers and PolyAll
% TODO: the code consistently uses "top" for type qualifiers and
%       "root" for ASTs, in particular for CompilationUnitTrees.

  All type qualifiers, except for polymorphic qualifiers (see below and
  also Section~\ref{qualifier-polymorphism}), need to be
  properly annotated with \refclass{framework/qual}{SubtypeOf}.

  The top qualifier is annotated with
  \<@SubtypeOf( \{ \} )>.  The top qualifier is the qualifier that is
  a supertype of all other qualifiers.  For example, \refqualclass{checker/nullness/qual}{Nullable}
  is the top qualifier of the Nullness type system, hence is defined as:

  \begin{Verbatim}
    @TypeQualifier
    @Target({ElementType.TYPE_USE, ElementType.TYPE_PARAMETER})
    @SubtypeOf( { } )
    public @interface Nullable { }
  \end{Verbatim}

  If the top qualifier of the hierarchy is the unqualified type, then its children
  will use \code{@SubtypeOf(Unqualified.class)}, but no \code{@SubtypeOf(
    \{ \} )} annotation on the top qualifier is necessary.  For an example, see the
  \<Encrypted> type system of Section~\ref{encrypted-example}.

\item \refqualclass{framework/qual}{PolymorphicQualifier} denotes that a qualifier is a
  polymorphic qualifier.  For example:

  \begin{Verbatim}
    @TypeQualifier
    @Target({ElementType.TYPE_USE, ElementType.TYPE_PARAMETER})
    @PolymorphicQualifier
    public @interface PolyNull { }
  \end{Verbatim}

  For a description of polymorphic qualifiers, see
  Section~\ref{qualifier-polymorphism}.  A polymorphic qualifier needs
  no \refqualclass{framework/qual}{SubtypeOf} meta-annotation and need not be
  mentioned in any other \refqualclass{framework/qual}{SubtypeOf}
  meta-annotation.

\end{itemize}

The declarative and procedural mechanisms for specifying the hierarchy can
be used together.  In particular, when using the \refqualclass{framework/qual}{SubtypeOf}
meta-annotation, further customizations may be
performed procedurally (Section~\ref{procedural-hierarchy})
by overriding the \href{api/org/checkerframework/framework/util/GraphQualifierHierarchy.html#isSubtype-java.util.Collection-java.util.Collection-}{\code{isSubtype}} method in the checker class
(Section~\ref{writing-compiler-interface}).
However, the declarative mechanism is sufficient for most type systems.


\subsection{Procedurally defining the qualifier and type hierarchy\label{procedural-hierarchy}}

While the declarative syntax suffices for many cases, more complex
type hierarchies can be expressed by overriding, in \refclass{common/basetype}{BaseTypeVisitor},
either \href{api/org/checkerframework/framework/type/AnnotatedTypeFactory.html#createQualifierHierarchy--}{\<createQualifierHierarchy>} or \href{api/org/checkerframework/framework/type/AnnotatedTypeFactory.html#createTypeHierarchy--}{\<createTypeHierarchy>} (typically
only one of these needs to be overridden).
For more details, see the Javadoc of those methods and of the classes
\refclass{framework/type}{QualifierHierarchy} and \refclass{framework/type}{TypeHierarchy}.

The \refclass{framework/type}{QualifierHierarchy} class represents the qualifier hierarchy (not the
type hierarchy), e.g., \refclass{checker/igj/qual}{Mutable}
is a subtype of \refclass{checker/igj/qual}{ReadOnly}.  A type-system designer may subclass
\refclass{framework/type}{QualifierHierarchy} to express customized qualifier
relationships (e.g., relationships based on annotation
arguments).

The \refclass{framework/type}{TypeHierarchy} class represents the type hierarchy ---
that is, relationships between
annotated types, rather than merely type qualifiers, e.g., \<@Mutable
Date> is a subtype of \<@ReadOnly Date>.  The default \refclass{framework/type}{TypeHierarchy} uses
\refclass{framework/type}{QualifierHierarchy} to determine all subtyping relationships.
The default \refclass{framework/type}{TypeHierarchy} handles
generic type arguments, array components, type variables, and
wildcards in a similar manner to the Java standard subtype
relationship but with taking qualifiers into consideration.  Some type
systems may need to override that behavior.  For instance, the Java
Language Specification specifies that two generic types are subtypes only
if their type arguments are identical:  for example,
\code{List<Date>} is not a subtype of \code{List<Object>}, or of any other
generic \code{List}.
(In the technical jargon, the generic arguments are ``invariant'' or ``novariant''.)
The Javari type system overrides this
behavior to allow some type arguments to change covariantly in a type-safe
manner (e.g.,
\code{List<@Mutable Date>} is a subtype of \code{List<@QReadOnly Date>}).


\subsection{Defining a default annotation\label{typesystem-defaults}}

% This paragraph is out of place.

A type system designer may set a default annotation.  A user may override
the default; see Section~\ref{defaults}.

The type system designer may specify a default annotation declaratively,
using the \refqualclass{framework/qual}{DefaultQualifierInHierarchy}
meta-annotation.
Note that the default will apply to any source code that the checker reads,
including stub libraries, but will not apply to compiled \code{.class}
files that the checker reads.

\begin{sloppypar}
Alternately, the type system designer may specify a default procedurally,
by calling the
\href{api/org/checkerframework/framework/util/QualifierDefaults.html#addAbsoluteDefault-javax.lang.model.element.AnnotationMirror-org.checkerframework.framework.qual.DefaultLocation-}{\<QualifierDefaults.addAbsoluteDefault>}
method.  You may do this even if you have declaratively defined the
qualifier hierarchy; see the Nullness Checker's implementation for an
example.
\end{sloppypar}


Recall that defaults are distinct
from implicit annotations; see Sections~\ref{effective-qualifier}
and~\ref{writing-type-introduction}.


\subsection{Completeness of the type hierarchy\label{bottom-and-top-qualifier}}

When you define a type system, its type hierarchy must be a
complete lattice --- that is, there must be a top type that is a
supertype of all other types, and there must be a bottom type that is a
subtype of all other types.
Furthermore, it is best if the top type and bottom type are defined
explicitly for the type system, rather than (say) reusing a qualifier from the
Checker Framework such as \<@Unqualified>.

It is possible that a single type-checker checks multiple type hierarchies.
An example is the Nullness Checker that has separate type hierarchies for
nullness, initialization, and map keys.  In this case, each type hierarchy
would have its own top qualifier and its own bottom qualifier; they don't
all have to share a single top qualifier or a single bottom qualifier.


\paragraph{Bottom qualifier\label{bottom-qualifier}}
Your type hierarchy must have a bottom qualifier
--- a qualifier that is a (direct or indirect) subtype of every other
qualifier.

Your type system must give \<null> the bottom type.  (The only exception
is if the type system has special treatment for \<null> values, as the
Nullness Checker does.)  This legal code
will not type-check unless \<null> has the bottom type:
\begin{Verbatim}
<T> T f() {
    return null;
\end{Verbatim}

You don't necessarily have to define a new bottom qualifier.  But, You can
use \<org.checkerframework.framework.qual.Bottom> if your type system does not already have an
appropriate bottom qualifier.

If your type system has a special bottom type that is used \emph{only} for
the \code{null} value, then users should never write the bottom qualifier
explicitly.  To ensure this, write \<@Target(\{\})> on the definition of
the bottom qualifier.


The hierarchy shown in Figure~\ref{fig-igj-hierarchy} lacks
a bottom qualifier, because there is no qualifier that is a subtype of both
\<@Immutable> and \<@Mutable>.
The actual IGJ hierarchy does contain a (non-user-visible) bottom qualifier,
defined like this:

\begin{Verbatim}
  @TypeQualifier
  @SubtypeOf({Mutable.class, Immutable.class, I.class})
  @Target({}) // forbids a programmer from writing it in a program
  @ImplicitFor(trees = { Kind.NULL_LITERAL, Kind.CLASS, Kind.NEW_ARRAY },
               typeClasses = { AnnotatedPrimitiveType.class })
  @interface IGJBottom { }
\end{Verbatim}


\paragraph{Top qualifier\label{top-qualifier}}
Your type hierarchy must have a top qualifier
--- a qualifier that is a (direct or indirect) supertype of every other
qualifier.
The default type for local variables is the top
qualifier (that type is then flow-sensitively
refined depending on what values are stored in the local variable).
If there is no single top qualifier, then there is no
unambiguous choice to make for local variables.

Furthermore, it is most convenient to users if the top qualifier is defined
by the type system.  An example of a type system that does
\emph{not} do that is the \<@Encrypted> type system of
Section~\ref{encrypted-example}.  It lacks its own explicit top qualifier and instead
uses \<@Unqualified>, which is shared across multiple type systems:

\begin{Verbatim}
  @TypeQualifier
  @SubtypeOf(Unqualified.class)
  @Target({ElementType.TYPE_USE, ElementType.TYPE_PARAMETER})
  public @interface Encrypted {}
\end{Verbatim}

% \noindent
% The interning type system of Section~\ref{interning-checker} also lacks a
% top qualifier; there is no \<@Uninterned> qualifier that is a supertype of 
% \refqualclass{checker/interning/qual}{Interned}.

\noindent
It can be convenient to use \<@Unqualified> as the top type to avoid
having to define your own top type.  The disadvantage is that users lose
flexibility in expressing defaults:  there is no
way for a user to change the default qualifier for just that type system.
If a user specifies
\<@DefaultQualifier(Unqualified.class)>,
then the default would apply to every
type system that uses \<@Unqualified>, which is unlikely to be desired.

It is best if a type system as an explicit qualifier for every
possible meaning.  For example,
the Nullness type system has both \refqualclass{checker/nullness/qual}{Nullable} and
\refqualclass{checker/nullness/qual}{NonNull}.  Because it has no built-in meaning
for unannotated types; a user may specify a default qualifier.  This
greater flexibility for the user is usually preferable.

There are reasons to not explicitly define the top qualifier, but to reuse
\<@Unqualified>.
The ability to omit the top qualifier is a convenience
when writing a type system, because it reduces the number of qualifiers
that must be defined; this is especially convenient when using the Subtyping
Checker (Chapter~\ref{subtyping-checker}).
% TODO: Examples of the following?
More importantly, omitting the top qualifier restricts the user in ways
that the type system designer may have intended.


\section{Type factory: Implicit annotations\label{writing-type-introduction}}

For some types and expressions, a qualifier should be treated as present
even if a programmer did not explicitly write it.  For example, every
literal (other than \<null>) has a \refqualclass{checker/nullness/qual}{NonNull} type.

The implicit annotations may be specified declaratively and/or procedurally.


\subsection{Declaratively specifying implicit annotations\label{declarative-type-introduction}}

The \refqualclass{framework/qual}{ImplicitFor} meta-annotation indicates implicit annotations.
When written on a qualifier, \refclass{framework/qual}{ImplicitFor}
specifies the trees (AST nodes) and types for which the framework should
automatically add that qualifier.

In short, the types and trees can be
specified via any combination of five fields in \refclass{framework/qual}{ImplicitFor}:

  \begin{itemize}
  \item \code{trees}: an array of
    \href{\TreeAPIBase{}/tree/Tree.Kind.html?is-external=true}{\code{com.sun.source.tree.Tree.Kind}}, e.g.,
    \code{NEW\_ARRAY} or \code{METHOD\_INVOCATION}
  \item \code{types}: an array of
    \refModelclass{type}{TypeKind}, e.g., \code{ARRAY}
    or \code{BOOLEAN}
  \item
    \begin{sloppypar}
    \code{treeClasses}: an array of class literals for classes
    implementing \refTreeclass{tree}{Tree}, e.g.,
    \code{LiteralTree.class} or \code{ExpressionTree.class}
    \end{sloppypar}
  \item \code{typeClasses}: an array of class literals for classes
    implementing \code{javax.lang.model.type.TypeMirror}, e.g.,
    \code{javax.lang.model.type.PrimitiveType}.  Often you should use
    a subclass of \refclass{framework/type}{AnnotatedTypeMirror}.
  \item \code{stringPatterns}: an array of regular expressions that will
    be matched against
    string literals, e.g., \code{"[01]+"} for a binary number.  Useful
    for annotations that indicate the format of a string.
  \end{itemize}

For example, consider the definitions of the \refqualclass{checker/nullness/qual}{NonNull} and \refqualclass{checker/nullness/qual}{Nullable}
type qualifiers:

%BEGIN LATEX
\begin{smaller}
%END LATEX
\begin{Verbatim}
  @TypeQualifier
  @SubtypeOf( { Nullable.class } )
  @ImplicitFor(
    types={TypeKind.PACKAGE},
    typeClasses={AnnotatedPrimitiveType.class},
    trees={
      Tree.Kind.NEW_CLASS,
      Tree.Kind.NEW_ARRAY,
      Tree.Kind.PLUS,
      // All literals except NULL_LITERAL:
      Tree.Kind.BOOLEAN_LITERAL, Tree.Kind.CHAR_LITERAL, Tree.Kind.DOUBLE_LITERAL, Tree.Kind.FLOAT_LITERAL,
      Tree.Kind.INT_LITERAL, Tree.Kind.LONG_LITERAL, Tree.Kind.STRING_LITERAL
    })
  @Target({ElementType.TYPE_USE, ElementType.TYPE_PARAMETER})
  public @interface NonNull {  }


  @TypeQualifier
  @SubtypeOf({})
  @ImplicitFor(trees={Tree.Kind.NULL_LITERAL})
  @Target({ElementType.TYPE_USE, ElementType.TYPE_PARAMETER})
  public @interface Nullable { }
\end{Verbatim}
%BEGIN LATEX
\end{smaller}
%END LATEX

For more details, see the Javadoc for the \refclass{framework/qual}{ImplicitFor}
  annotation, and the Javadoc for the javac classes that are linked from
it.  You only need to understand a small amount about the javac AST, such
as the
\href{\TreeAPIBase{}/tree/Tree.Kind.html?is-external=true}{\code{Tree.Kind}}
and
\refModelclass{type}{TypeKind}
enums.  All the information you need is in the Javadoc, and
Section~\ref{javac-tips} can help you get started.


\subsection{Procedurally specifying implicit annotations\label{procedurally-specifying-implicit-annotations}}


The Checker Framework provides a representation of annotated types,
\refclass{framework/type}{AnnotatedTypeMirror}, that extends the standard \<TypeMirror>
interface but integrates a representation of the annotations into a
type representation.  A checker's \emph{type factory} class, given an AST
node, returns the annotated type of that expression.  The Checker
Framework's abstract
\emph{base type factory} class, \refclass{framework/type}{AnnotatedTypeFactory},
supplies a uniform, Tree-API-based interface
for querying the annotations on a program element, regardless of
whether that element is declared in a source file or in a class file.
It also handles default annotations, and it optionally performs
flow-sensitive local type inference.

\refclass{framework/type}{AnnotatedTypeFactory} inserts the qualifiers that the programmer
explicitly inserted in the code.  Yet, certain constructs should be
treated as having a type qualifier even when the programmer has not
written one.  The type system designer may subclass
\refclass{framework/type}{AnnotatedTypeFactory} and override
\code{annotateImplicit(Tree,AnnotatedTypeMirror)} and
\code{annotateImplicit(Element,AnnotatedTypeMirror)} to account for
such constructs.


\section{Dataflow: enhancing flow-sensitive type qualifier inference\label{dataflow}}

The Checker Framework provides automatic type refinement as described
in Section~\ref{type-refinement}.

Class
\refclass{common/basetype}{BaseAnnotatedTypeFactory}
provides a 2 parameter constructor that allows subclasses to disable
flow inference.
By default the 1 parameter constructor performs flow inference.
To disable flow inference, call
\code{super(checker, root, false);}
in your subtype of
\refclass{common/basetype}{BaseAnnotatedTypeFactory}.

You can enhance the Checker Framework's built-in flow-sensitive type refinement,
so that it is more powerful and is customized to your type system. In
particular, your enhancement will yield a more refined type for certain
expressions. However, most enhancements to type refinement are based on a
run-time test specific to the type system and not all type-systems have
applicable run-time tests.  See
Section~\refwithpageparen{type-refinement-runtime-tests} to determine if
run-time tests are applicable to your type system.

The Checker Framework's type refinement is implemented with a dataflow algorithm
which can be customized to enhance the built-in type refinement. The next
sections detail dataflow customization.  It would also be helpful to read the
\href{http://types.cs.washington.edu/checker-framework/current/checker-framework-dataflow-manual.pdf}
{Dataflow Manual}, which gives a more in-depth description of the Checker
Framework's dataflow framework.

The steps to customizing type refinement are:
\begin{enumerate}
\item{\ref{dataflow-create-classes}} Create required classes and configure their
    use
\item{\ref{dataflow-override-methods}} Override methods that handle
    \refclass{dataflow/cfg/node}{Node}s of interest
\item{\ref{dataflow-determine-expressions}} Determine which expressions will be
    refined
\item{\ref{dataflow-implement-refinement}} Implement the refinement
\end{enumerate}

The Regex Checker's dataflow customization for the \code{RegexUtil.asRegex}
run-time check is used as an example throughout the steps.

The \code{RegexUtil.asRegex} method is declared as:

%BEGIN LATEX
\begin{smaller}
%END LATEX
\begin{Verbatim}
  @Regex(0) String asRegex(String s, int groups) { ... }
\end{Verbatim}
%BEGIN LATEX
\end{smaller}
%END LATEX

\noindent
which means that an expression such as \code{RegexUtil.asRegex(myString, myInt)}
has type \code{@Regex(0)} String. When \code{int} parameter \code{group} is
known or can be inferred at compile time, a better estimate can be given.  For
example, \code{RegexUtil.asRegex(myString, 2)} has type \code{@Regex(2)} String.

\subsection{Create required classes and configure their
  use\label{dataflow-create-classes}}

The following classes must be created to customize dataflow. These classes must
be included on the classpath like other components of your checker.

\begin{enumerate}
\item \textbf{Create a class that extends
    \refclass{framework/flow}{CFAbstractTransfer}}

  \refclass{framework/flow}{CFAbstractTransfer} performs the default Checker
  Framework type refinement.  The extended class will add functionality by
  overriding superclass methods.

  The Regex Checker's extended \refclass{framework/flow}{CFAbstractTransfer} is
  \refclass{checker/regex}{RegexTransfer}.

\item \textbf{Create a class that extends
    \refclass{framework/flow}{CFAbstractAnalysis} and uses the extended
    \refclass{framework/flow}{CFAbstractTransfer}}

  \refclass{framework/flow}{CFAbstractTransfer} and its superclass,
  \refclass{dataflow/analysis}{Analysis}, are the central coordinating classes
  in the Checker Framework's dataflow algorithm. The
  \code{createTransferFunction} method must be overridden in an extended
  \refclass{framework/flow}{CFAbstractTransfer} to return a new instance of the
  extended \refclass{framework/flow}{CFAbstractTransfer}.

  The Regex Checker's extended \refclass{framework/flow}{CFAbstractAnalysis} is
  \refclass{checker/regex}{RegexAnalysis}, which overrides the
  \code{createTransferFunction} to return a new
  \refclass{checker/regex}{RegexTransfer} instance:

%BEGIN LATEX
\begin{smaller}
%END LATEX
\begin{Verbatim}
  @Override
  public RegexTransfer createTransferFunction() {
      return new RegexTransfer(this);
  }
\end{Verbatim}
%BEGIN LATEX
\end{smaller}
%END LATEX

\item \textbf{Configure the checker's type factory to use the extended
    \refclass{framework/flow}{CFAbstractAnalysis}}

To configure your checker's type factory to use the new extended
\refclass{framework/flow}{CFAbstractAnalysis}, override the
\code{createFlowAnalysis} method in your type factory to return a new instance
of the extended \refclass{framework/flow}{CFAbstractAnalysis}.

%BEGIN LATEX
\begin{smaller}
%END LATEX
\begin{Verbatim}
  @Override
  protected RegexAnalysis createFlowAnalysis(
          List<Pair<VariableElement, CFValue>> fieldValues) {

      return new RegexAnalysis(checker, this, fieldValues);
  }
\end{Verbatim}
%BEGIN LATEX
\end{smaller}
%END LATEX

\end{enumerate}

\subsection{Override methods that handle Nodes of
interest\label{dataflow-override-methods}}

At this point, your checker is configured to use your extended
\refclass{framework/flow}{CFAbstractAnalysis}, but it uses only the default
behavior. Next, in your extended \refclass{framework/flow}{CFAbstractTransfer}
override the visitor method that handles the \refclass{dataflow/cfg/node}{Node}s
relevant to your run-time check or run-time operation can be used to refine
types.

A \refclass{dataflow/cfg/node}{Node} is basically equivalent to a javac compiler
\refTreeclass{tree}{Tree}.  A tree is a node in the abstract syntax tree of the
program being checked. See \ref{javac-tips} for more information about trees.

A \refclass{dataflow/cfg/node}{Node} generally maps one-to-one with a
\refTreeclass{tree}{Tree}. When dataflow processes a method, it translates
\refTreeclass{tree}{Tree}s into \refclass{dataflow/cfg/node}{Node}s and then
calls the appropriate visit method on
\refclass{framework/flow}{CFAbstractTransfer} which then performs the dataflow
analysis for the passed in \refclass{dataflow/cfg/node}{Node}.

Decide what \refclass{dataflow/cfg/node}{Node} kinds are of interest with
respect to the run-time checks or run-time operations you are trying to support.
The \refclass{dataflow/cfg/node}{Node} subclasses can be found in the
\code{org.checkerframework.dataflow.cfg.node} package.  Some examples are
\refclass{dataflow/cfg/node}{EqualToNode},
\refclass{dataflow/cfg/node}{LeftShiftNode},
\refclass{dataflow/cfg/node}{VariableDeclarationNode}.

The Regex Checker refines the type of a run-time test method call, so
\refclass{checker/regex}{RegexTransfer} overrides the method that handles
\refclass{dataflow/cfg/node}{MethodInvocationNode}s,
\code{visitMethodInvocation}.

%BEGIN LATEX
\begin{smaller}
%END LATEX
\begin{Verbatim}
  public TransferResult<CFValue, CFStore> visitMethodInvocation(
    MethodInvocationNode n, TransferInput<CFValue, CFStore> in)  { ... }
\end{Verbatim}
%BEGIN LATEX
\end{smaller}
%END LATEX

\subsection{Determine the expressions to refine the types
of\label{dataflow-determine-expressions}}

There are usually multiple expressions used in a run-time check or run-time
operation; determine which expression the customization will refine.  This is
usually specific to the type system and run-time test.

Expressions are refined by modifying the return value of a visitor method in
\refclass{framework/flow}{CFAbstractTransfer}.
\refclass{framework/flow}{CFAbstractTransfer} visitor methods return a
\refclass{dataflow/analysis}{TransferResult}.  The constructor of a
\refclass{dataflow/analysis}{TransferResult} takes two parameters: the resulting
type for the \refclass{dataflow/cfg/node}{Node} being evaluated (the result
type) and a map from expressions in scopes to estimates of their types (a
\refclass{dataflow/analysis}{Store}).

For the program operation \code{op(a,b)}, an enhancement may improve the Checker
Framework's types by:
\begin{enumerate}
\item Changing the resulting type to refine the estimate of the type of entire
    expression \code{op(a,b)}, or
\item Changing the store to refine the estimate of some other expression, such
    as \code{a} or \code{b}.
\end{enumerate}

Changing the \refclass{dataflow/analysis}{TransferResult}'s result type changes
the type that is returned by the \refclass{framework/type}{AnnotatedTypeFactory}
for the tree corresponding to the \refclass{dataflow/cfg/node}{Node} that was
visited.  (Remember that \refclass{common/basetype}{BaseTypeVisitor} uses the
\refclass{framework/type}{AnnotatedTypeFactory} to look up the type of a
\refTreeclass{tree}{Tree}, and then performs checks on types of one or more
\refTreeclass{tree}{Tree}s).

When \refclass{checker/regex}{RegexTransfer} evaluates a
\code{RegexUtils.asRegex} invocation, it updates the
\refclass{dataflow/analysis}{TransferResult}'s result type. This changes the
type of the \code{RegexUtils.asRegex} invocation when it's
\refTreeclass{tree}{Tree} is looked up by the
\refclass{framework/type}{AnnotatedTypeFactory}.  Regex Checker's
\code{visitMethodInvocation} is shown in more detail in
\ref{dataflow-implement-refinement}.

Updating the \refclass{dataflow/analysis}{Store} treats an expression as a
having a refined type for the remainder of the method or conditional block. For
example, when the Nullness Checker's dataflow evaluates \code{myvar != null}, it
updates the \refclass{dataflow/analysis}{Store} to specify that the variable
\code{myvar} hould be treated as having type \code{@NonNull} for the rest of the
then conditional block.  Not all kinds of expressions can be refined; currently
method return values, local variables, fields, and array values can be stored in
the \refclass{dataflow/analysis}{Store}.  Other kinds of expressions, like
binary expressions or casts, cannot be stored in the
\refclass{dataflow/analysis}{Store}.

Both the \refclass{dataflow/analysis}{Store} and the result type may be updated
in the same \refclass{dataflow/analysis}{TransferResult}.

\subsection{Implement the refinement\label{dataflow-implement-refinement}}

This section details implementing the visitor method
\code{RegexTransfer.visitMethodInvocation} for the \code{RegexUtil.asRegex}
run-time test.  You can find other examples of visitor methods in
\refclass{checker/lock}{LockTransfer} and
\refclass{checker/Formatter}{FormatterTransfer}.

A general outline of the visit method is to:
\begin{enumerate}
\item Determine if the visited \refclass{dataflow/cfg/node}{Node} is of interest
\item Determine the refined type
\item Return a \refclass{dataflow/analysis}{TransferResult} with the refined
    types
\end{enumerate}


\begin{enumerate}
\item \textbf{Determine if the visited \refclass{dataflow/cfg/node}{Node} is of
    interest}

The visitor method for a \refclass{dataflow/cfg/node}{Node} is invoked for all
instances of that \refclass{dataflow/cfg/node}{Node} kind in the program, so the
\refclass{dataflow/cfg/node}{Node} must be inspected to determine if it is an
instance of the desired run-time test or operation. For example,
\code{visitMethodInvocation} is called when dataflow processes any method
invocation, but the \refclass{checker/regex}{RegexTransfer} should only refine
the result of \code{RegexUtils.asRegex} invocations:

%BEGIN LATEX
\begin{smaller}
%END LATEX
\begin{Verbatim}
  @Override
  public TransferResult<CFValue, CFStore> visitMethodInvocation(...)
    ...
    MethodAccessNode target = n.getTarget();
    ExecutableElement method = target.getMethod();
    Node receiver = target.getReceiver();
    if (receiver instanceof ClassNameNode) {
      ClassNameNode cn = (ClassNameNode) receiver;
      String receiverName = cn.getElement().toString();

      // Is this a RegexUtil.isRegex(s, groups) method call?
      if (isRegexUtil(receiverName)) {
        if (ElementUtils.matchesElement(method,
          null, IS_REGEX_METHOD_NAME, String.class, int.class)) {

          ...

\end{Verbatim}
%BEGIN LATEX
\end{smaller}
%END LATEX

\item \textbf{Determine the refined type}

Some run-time tests, like the null comparison test, have a deterministic type
refinement, e.g. the Nullness Checker always refines the argument in the
expression to \code{@NonNull}.  However, sometimes the refined type is dependent
on the parts of run-time test or operation itself, such as arguments passed to
it.

For example, the refined type of \code{RegexUtils.asRegex} is dependent on the
integer argument to the method call. The \refclass{checker/regex}{RegexTransfer}
uses this argument to build the resulting type \code{@Regex(i)}, where \code{i}
is the value of the integer argument.  Note that currently this code only uses
the value of the integer argument if the argument was an integer literal.  It
could be extended to use the value of the argument if it was any compile-time
constant or was inferred at compile time by another analysis, such as the
\ref{constant-value-checker}.

%BEGIN LATEX
\begin{smaller}
%END LATEX
  \begin{Verbatim}
  AnnotationMirror regexAnnotation;
  Node count = n.getArgument(1);
  if (count instanceof IntegerLiteralNode) {
    IntegerLiteralNode iln = (IntegerLiteralNode) count;
    Integer groupCount = iln.getValue();
    regexAnnotation = factory.createRegexAnnotation(groupCount);

\end{Verbatim}
%BEGIN LATEX
\end{smaller}
%END LATEX

If the integer argument was not a literal integer, the
\refclass{checker/regex}{RegexTransfer} falls back to refining the type to just
\code{@Regex(0)}.

%BEGIN LATEX
\begin{smaller}
%END LATEX
\begin{Verbatim}
  } else {
    regexAnnotation = AnnotationUtils.fromClass(factory.getElementUtils(), Regex.class);
  }
\end{Verbatim}
%BEGIN LATEX
\end{smaller}
%END LATEX

\item \textbf{Return a \refclass{dataflow/analysis}{TransferResult} with the
    refined types}

As discussed in section \ref{dataflow-determine-expressions}, the type of an
expression is refined by modifying the
\refclass{dataflow/analysis}{TransferResult}.  Since the
\refclass{checker/regex}{RegexTransfer} is updating the type of the run-time
test itself, it will update the result type and not the
\refclass{dataflow/analysis}{Store}.

A \refclass{framework/flow}{CFValue} is created to hold the type inferred.
\refclass{framework/flow}{CFValue} is a wrapper class for values being inferred
by dataflow:
%BEGIN LATEX
\begin{smaller}
%END LATEX
\begin{Verbatim}
  CFValue newResultValue = analysis.createSingleAnnotationValue(regexAnnotation,
      result.getResultValue().getType().getUnderlyingType());
\end{Verbatim}
%BEGIN LATEX
\end{smaller}
%END LATEX

Then, RegexTransfer's \code{visitMethodInvocation} creates and returns a
\refclass{dataflow/analysis}{TransferResult} using \code{newResultValue} as the
result type.

%BEGIN LATEX
\begin{smaller}
%END LATEX
\begin{Verbatim}
  return new RegularTransferResult<>(newResultValue, result.getRegularStore());
\end{Verbatim}
%BEGIN LATEX
\end{smaller}
%END LATEX

Finally, when the Regex Checker encounters a \code{RegexUtils.asRegex} method
call, the checker will refine the return type of the method if it can determine
the value of the integer parameter at compile time.

\end{enumerate}


\section{Visitor: Type rules\label{extending-visitor}}

A type system's rules define which operations on values of a
particular type are forbidden.
These rules must be defined procedurally, not declaratively.

The Checker Framework provides a \textit{base visitor class},
\refclass{common/basetype}{BaseTypeVisitor}, that performs type-checking at each node of a
source file's AST\@.  It uses the visitor design pattern to traverse
Java syntax trees as provided by Oracle's
\href{http://types.cs.washington.edu/checker-framework/jdk-api/javac/index.html}{Tree
API},
and it issues a warning whenever the type system is violated.

A checker's visitor overrides one method in the base visitor for each special
rule in the type qualifier system.  Most type-checkers
override only a few methods in \refclass{common/basetype}{BaseTypeVisitor}.  For example, the
visitor for the Nullness type system of Chapter~\ref{nullness-checker}
contains a single 4-line method that warns if an expression of nullable type
is dereferenced, as in:
\begin{Verbatim}
  myObject.hashCode();  // invalid dereference
\end{Verbatim}



By default, \refclass{common/basetype}{BaseTypeVisitor} performs subtyping checks that are
similar to Java subtype rules, but taking the type qualifiers into account.
\refclass{common/basetype}{BaseTypeVisitor} issues these errors:

\begin{itemize}

\item invalid assignment (type.incompatible) for an assignment from
  an expression type to an incompatible type.  The assignment may be a
  simple assignment, or pseudo-assignment like return expressions or
  argument passing in a method invocation

  In particular, in every assignment and pseudo-assignment, the
  left-hand side of the assignment is a supertype of (or the same type
  as) the right-hand side.  For example, this assignment is not
  permitted:

  \begin{Verbatim}
    @Nullable Object myObject;
    @NonNull Object myNonNullObject;
    ...
    myNonNullObject = myObject;  // invalid assignment
  \end{Verbatim}

\item invalid generic argument (type.argument.type.incompatible) when a type
  is bound to an incompatible generic type variable

\item invalid method invocation (method.invocation.invalid) when a
  method is invoked on an object whose type is incompatible with the
  method receiver type

\item invalid overriding parameter type (override.parameter.invalid)
  when a parameter in a method declaration is incompatible with that
  parameter in the overridden method's declaration

\item invalid overriding return type (override.return.invalid) when a
  parameter in a method declaration is incompatible with that
  parameter in the overridden method's declaration

\item invalid overriding receiver type (override.receiver.invalid)
  when a receiver in a method declaration is incompatible with that
  receiver in the overridden method's declaration

\end{itemize}


\subsection{AST traversal\label{ast-traversal}}

The Checker Framework needs to do its own traversal of the AST even though
it operates as an ordinary annotation processor~\cite{JSR269}.  Annotation
processors can utilize a visitor for Java code, but that visitor only
visits the public elements of Java code, such as classes, fields, methods,
and method arguments --- it does not visit code bodies or various other
locations.  The Checker Framework hardly uses the built-in visitor --- as
soon as the built-in visitor starts to visit a class, then the Checker
Framework's visitor takes over and visits all of the class's source code.

Because there is no standard API for the AST of Java code\footnote{Actually,
there is a standard API for Java ASTs --- JSR 198 (Extension API for
Integrated Development Environments)~\cite{JSR198}.  If tools were to
implement it
(which would just require writing wrappers or adapters), then the Checker
Framework and similar tools could be portable among different compilers and
IDEs.}, the Checker
Framework uses the javac implementation.  This is why the Checker Framework
is not deeply integrated with Eclipse, but runs as an external tool (see
Section~\ref{eclipse}).


\subsection{Avoid hardcoding\label{avoid-hardcoding}}

It may be tempting to write a type-checking rule for method invocation,
where your rule checks the name of the method being called and then treats
the method in a special way.  This is usually the wrong approach.  It
is better to write annotations, in a stub file
(Chapter~\ref{annotating-libraries}), and leave the work to the standard
type-checking rules.


\section{The checker class:  Compiler interface\label{writing-compiler-interface}}

A checker's entry point is a subclass of \refclass{common/basetype}{BaseTypeChecker}.  This entry
point, which we call the checker class, serves two
roles:  an interface to the compiler and a factory for constructing
type-system classes.

Because the Checker Framework provides reasonable defaults, oftentimes the
checker class has no work to do.  Here are the complete definitions of the
checker classes for the Interning Checker and the Nullness Checker:

\begin{Verbatim}
  @TypeQualifiers({ Interned.class, PolyInterned.class })
  @SupportedLintOptions({"dotequals"})
  public final class InterningChecker extends BaseTypeChecker { }

  @TypeQualifiers({ Nullable.class, Raw.class, NonNull.class, PolyNull.class })
  @SupportedLintOptions({"flow", "cast", "cast:redundant"})
  public class NullnessChecker extends BaseTypeChecker { }
\end{Verbatim}


The checker class must be annotated by
\refqualclass{framework/qual}{TypeQualifiers}, which lists the annotations
that make up the type hierarchy for this checker (including
polymorphic qualifiers), provided as an array of class literals.  Each
one is a type qualifier whose definition bears the
\refqualclass{framework/qual}{TypeQualifier} meta-annotation (or is
returned by the
\href{http://types.cs.washington.edu/checker-framework/api/org/checkerframework/framework/type/AnnotatedTypeFactory.html#getSupportedTypeQualifiers--}{\<BaseTypeChecker\-.getSupportedTypeQualifiers>}
method).

The checker class bridges between the compiler and the rest of the checker.  It
invokes the type-rule check visitor on every Java source file being
compiled, and provides a simple API, \href{http://types.cs.washington.edu/dev/checker-framework/api/org/checkerframework/framework/source/SourceChecker.html#report-org.checkerframework.framework.source.Result-java.lang.Object-}{\<report>}, to issue
errors using the compiler error reporting mechanism.

Also, the checker class follows the factory method pattern to
construct the concrete classes (e.g., visitor, factory) and annotation
hierarchy representation.  It is a convention that, for
a type system named Foo, the compiler
interface (checker), the visitor, and the annotated type factory are
named as \<FooChecker>, \<FooVisitor>, and \<FooAnnotatedTypeFactory>.
\refclass{common/basetype}{BaseTypeChecker} uses the convention to
reflectively construct the components.  Otherwise, the checker writer
must specify the component classes for construction.

\begin{sloppypar}
A checker can customize the default error messages through a
\sunjavadoc{java/util/Properties.html}{Properties}-loadable text file named
\<messages.properties> that appears in the same directory as the checker class.
The property file keys are the strings passed to \href{http://types.cs.washington.edu/dev/checker-framework/api/org/checkerframework/framework/source/SourceChecker.html#report-org.checkerframework.framework.source.Result-java.lang.Object-}{\<report>}
(like \code{type.incompatible}) and the values are the strings to be
printed (\code{"cannot assign ..."}).
The \<messages.properties> file only need to mention the new messages that
the checker defines.
It is also allowed to override messages defined in superclasses, but this
is rarely needed.
For more details about message keys, see Section~\refwithpageparen{compiler-message-keys}.
\end{sloppypar}


\subsection{Bundling multiple checkers\label{bundling-multiple-checkers}}

To run a checker, a user supplies the \<-processor> command-line option.
There are two ways to run multiple related checkers as a unit.

\begin{enumerate}
\item
A user can pass
multiple \<-processor> command-line options, like:

\begin{Verbatim}
  javac -processor DistanceUnitChecker -processor SpeedUnitChecker ... files ...
\end{Verbatim}

\noindent
This is verbose, and it is also error-prone, since a user might omit one of
several related checkers that are designed to be run together.

\item
You can define an aggregate checker class that combines
multiple checkers.  Extend \refclass{framework/source}{AggregateChecker} and override
the \<getSupportedTypeCheckers> method, like the following:

\begin{Verbatim}
  public class UnitCheckers extends AggregateChecker {
    protected Collection<Class<? extends SourceChecker>> getSupportedCheckers() {
      return Arrays.asList(DistanceUnitChecker.class, SpeedUnitChecker.class);
    }
  }
\end{Verbatim}

\noindent
Now, a user can pass a single \<-processor> argument on the command line:

\begin{Verbatim}
  javac -processor UnitCheckers ... files ...
\end{Verbatim}

\end{enumerate}



\subsection{Providing command-line options\label{providing-command-line-options}}

A checker can provide two kinds of command-line options:
boolean flags and
named string values (the standard annotation processor
options).

\subsubsection{Boolean flags\label{providing-command-line-options-boolean-flags}}

To specify a simple boolean flag, add:

\begin{Verbatim}
@SupportedLintOptions({"flag"})
\end{Verbatim}

to your checker subclass.
The value of the flag can be queried using

\begin{Verbatim}
checker.getLintOption("flag", false)
\end{Verbatim}

The second argument sets the default value that should be returned.

To pass a flag on the command line, call javac as follows:

\begin{Verbatim}
javac -processor Mine -Alint=flag
\end{Verbatim}


\subsubsection{Named string values\label{providing-command-line-options-named-string-values}}

For more complicated options, one can use the standard annotation
processing \code{@SupportedOptions} annotation on the checker, as in:

\begin{Verbatim}
@SupportedOptions({"info"})
\end{Verbatim}

The value of the option can be queried using

\begin{Verbatim}
checker.getOption("info")
\end{Verbatim}

To pass an option on the command line, call javac as follows:

\begin{Verbatim}
javac -processor Mine -Ainfo=p1,p2
\end{Verbatim}

The value is returned as a single string and you have to perform the
required parsing of the option.


% TODO: describe -ANullnessChecker_option=value mechanism.



\section{Testing framework\label{testing-framework}}

The Checker Framework comes with a testing framework that is used for
testing the distributed checkers.
It is easy to use this testing framework to ensure correctness of your
checker!

You first need to provide a subclass of
\code{ParameterizedCheckerTest}
that determines the checker to use and all command-line options that
should be provided.
This class can be run as a JUnit test runner.
Note that you always need to use the
\code{-Anomsgtext} option to suppress the substitution of message keys
by human-readable values.
See the test setup classes in directory \code{tests/src/tests/} for examples.

Locate all your test cases in a subdirectory of the \code{tests}
directory.
The individual test cases are normal Java files that use stylized
comments to indicate expected error messages.
For example, consider this test case from the Nullness Checker:

\begin{Verbatim}
  //:: error: (dereference.of.nullable)
  s.toString();
\end{Verbatim}

An expected error message is introduced by the \code{//::} comment.
The next token is either \code{error:} or \code{warning:},
distinguishing what kind of message is expected.
Finally, the message key for the expected message is given.

Multiple expected messages can be given using the "//:: A :: B :: C"
syntax.
This example expects both an error and a warning from the same line of code:

\begin{Verbatim}
  @Regex String s1 = null;
  //:: error: (assignment.type.incompatible) :: warning: (cast.unsafe)
  @Regex(3) String s2 = (@Regex(2) String) s;
\end{Verbatim}

As an alternative, expected errors can be specified in a separate file
using the \code{.out} file extension.
These files are of the following format:

\begin{Verbatim}
:19: error: (dereference.of.nullable)
\end{Verbatim}

The number between the colons is the line number of the expected error
message.
This format is a lot harder to maintain and we suggest using the
in-line comment format.



\section{Debugging options\label{debugging-options}}

The Checker Framework provides debugging options that can be helpful when
writing a checker. These are provided via the standard \code{javac} ``\code{-A}''
switch, which is used to pass options to an annotation processor.

\subsection{Amount of detail in messages\label{debugging-options-detail}}

\begin{itemize}
\item \code{-AprintAllQualifiers}: print all type qualifiers, including
qualifiers like \code{@Unqualified} which are usually not shown.
(Use the \code{@InvisibleQualifier} meta-annotation on a qualifier to hide it.)

\item \code{-Adetailedmsgtext}: Output error/warning messages in a
  stylyized format that is easy for tools to parse.  This is useful for
  tools that run the Checker Framework and parse its output, such as IDE
  plugins.  See the source code of \<SourceChecker.java> for details about
  the format.

\item \code{-AprintErrorStack}: print a stack trace whenever an
internal Checker Framework error occurs.

\item \code{-Anomsgtext}: use message keys (such as ``\code{type.invalid}'')
rather than full message text when reporting errors or warnings.  This is
used by the Checker Framework's own tests, so they do not need to be
changed if the English message is updated.

\end{itemize}

\subsection{Stub and JDK libraries\label{debugging-options-libraries}}

\begin{itemize}

\item \code{-Aignorejdkastub}:
  ignore the \<jdk.astub> file in the checker directory. Files passed
  through the \code{-Astubs} option are still processed. This is useful
  when experimenting with an alternative stub file.

\item \code{-Anocheckjdk}:
  don't issue an error if no annotated JDK can be found.

\item \code{-AstubDebug}:
  Print debugging messages while processing stub files.

\end{itemize}

\subsection{Progress tracing\label{debugging-options-progress}}

\begin{itemize}

\item \code{-Afilenames}: print the name of each file before type-checking it.

\item \code{-Ashowchecks}: print debugging information for each
pseudo-assignment check (as performed by
\refclass{common/basetype}{BaseTypeVisitor}; see
Section~\ref{extending-visitor}.

\end{itemize}

\subsection{Saving the command-line arguments to a file\label{debugging-options-output-args}}

\begin{itemize}

\item \code{-AoutputArgsToFile}:
  This saves the final command-line parameters as passed to the compiler in a file.
  This file can be used as a script (if the file is marked as executable on Unix, or
  if it includes a \code{.bat} extension on Windows) to re-execute the same compilation command.
  This is useful, for example, when debugging problems running the Checker Framework from
  Maven, since normally the command-line parameters used by Maven are not user-visible.
  Note that this argument cannot be included in a file containing command-line arguments
  passed to the compiler using the @argfile syntax.  Please see
  Section~\ref{debugging-maven-args} for more details on how to use this command-line
  parameter to debug compilation using Maven.

  Example usage: \code{-AoutputArgsToFile=/home/username/scriptfile}

\end{itemize}

\subsection{Miscellaneous debugging options\label{debugging-options-misc}}

\begin{itemize}

\item \code{-Aflowdotdir}:
  Directory for .dot files that visualize the control flow graph of all the methods and code fragments
  analyzed by the dataflow analysis.  The graph also contains information about flow-sensitively refined
  types of various expressions at many program points.

\item \code{-AresourceStats}:
  Whether to output resource statistics at JVM shutdown.

\end{itemize}


\subsection{Examples\label{debugging-options-examples}}

The following example demonstrates how these options are used:

%BEGIN LATEX
\begin{smaller}
%END LATEX
\begin{Verbatim}
$ javac -processor org.checkerframework.checker.interning.InterningChecker \
    examples/InternedExampleWithWarnings.java -Ashowchecks -Anomsgtext -Afilenames

[InterningChecker] InterningExampleWithWarnings.java
 success (line  18): STRING_LITERAL "foo"
     actual: DECLARED @org.checkerframework.checker.interning.qual.Interned java.lang.String
   expected: DECLARED @org.checkerframework.checker.interning.qual.Interned java.lang.String
 success (line  19): NEW_CLASS new String("bar")
     actual: DECLARED java.lang.String
   expected: DECLARED java.lang.String
examples/InterningExampleWithWarnings.java:21: (not.interned)
    if (foo == bar)
            ^
 success (line  22): STRING_LITERAL "foo == bar"
     actual: DECLARED @org.checkerframework.checker.interning.qual.Interned java.lang.String
   expected: DECLARED java.lang.String
1 error
\end{Verbatim}
%BEGIN LATEX
\end{smaller}
%END LATEX

You can use any standard debugger to observe the execution of your checker.
Set the execution main class to \code{com.sun.tools.javac.Main}, and insert
the Checker Framework javac.jar (resides in
\code{.../checker-framework/checker/dist/javac.jar}).  If using an IDE, it is
recommended that you add \code{.../jsr308-langtools} as a project, so you
can step into its source code if needed.

You can also set up remote (or local) debugging using the following command as a template:

\begin{Verbatim}
java -jar $CHECKERFRAMEWORK/framework/dist/framework.jar \
    -J-Xdebug -J-Xrunjdwp:transport=dt_socket,server=y,suspend=y,address=5005 \
    -processor org.checkerframework.checker.nullness.NullnessChecker \
    src/sandbox/FileToCheck.java

\end{Verbatim}

% TODO: show example -AprintErrorStack usage. Update text above to
% refer to it.

% $ javac -processor org.checkerframework.checker.fenum.FenumChecker IdentityArrayList.java 
% error: GraphQualifierHierarchy found an unqualified type.  Please ensure that your implicit rules cover all cases and/or use a @DefaulQualifierInHierarchy annotation.
% 1 error

% $ javac -processor org.checkerframework.checker.fenum.FenumChecker -AprintErrorStack IdentityArrayList.java 
%% error: GraphQualifierHierarchy found an unqualified type.  Please ensure that your implicit rules cover all cases and/or use a @DefaulQualifierInHierarchy annotation.
%%   checkers.util.GraphQualifierHierarchy.checkAnnoInGraph(GraphQualifierHierarchy.java:253)
%%   checkers.util.GraphQualifierHierarchy.isSubtype(GraphQualifierHierarchy.java:243)
%%   checkers.fenum.FenumChecker$FenumQualifierHierarchy.isSubtype(FenumChecker.java:129)
%%   checkers.types.QualifierHierarchy.isSubtype(QualifierHierarchy.java:78)
%%   checkers.types.TypeHierarchy.isSubtypeImpl(TypeHierarchy.java:122)
%%   checkers.types.TypeHierarchy.isSubtype(TypeHierarchy.java:67)
%%   checkers.basetype.BaseTypeChecker.isSubtype(BaseTypeChecker.java:323)
%%   checkers.basetype.BaseTypeVisitor.commonAssignmentCheck(BaseTypeVisitor.java:608)
%%   checkers.basetype.BaseTypeVisitor.checkTypeArguments(BaseTypeVisitor.java:680)
%%   checkers.basetype.BaseTypeVisitor.visitMethodInvocation(BaseTypeVisitor.java:299)
%%   checkers.basetype.BaseTypeVisitor.visitMethodInvocation(BaseTypeVisitor.java:1)
%%   com.sun.tools.javac.tree.JCTree$JCMethodInvocation.accept(JCTree.java:1351)
%%   com.sun.source.util.TreePathScanner.scan(TreePathScanner.java:67)
%%   checkers.basetype.BaseTypeVisitor.scan(BaseTypeVisitor.java:122)
%%   checkers.basetype.BaseTypeVisitor.scan(BaseTypeVisitor.java:1)
%%   com.sun.source.util.TreeScanner.visitExpressionStatement(TreeScanner.java:241)
%%   com.sun.tools.javac.tree.JCTree$JCExpressionStatement.accept(JCTree.java:1176)
%%   com.sun.source.util.TreePathScanner.scan(TreePathScanner.java:67)
%%   checkers.basetype.BaseTypeVisitor.scan(BaseTypeVisitor.java:122)
%%   checkers.basetype.BaseTypeVisitor.scan(BaseTypeVisitor.java:1)
%%   com.sun.source.util.TreeScanner.scan(TreeScanner.java:90)
%%   com.sun.source.util.TreeScanner.visitBlock(TreeScanner.java:160)
%%   com.sun.tools.javac.tree.JCTree$JCBlock.accept(JCTree.java:793)
%%   com.sun.source.util.TreePathScanner.scan(TreePathScanner.java:67)
%%   checkers.basetype.BaseTypeVisitor.scan(BaseTypeVisitor.java:122)
%%   checkers.basetype.BaseTypeVisitor.scan(BaseTypeVisitor.java:1)
%%   com.sun.source.util.TreeScanner.scanAndReduce(TreeScanner.java:80)
%%   com.sun.source.util.TreeScanner.visitMethod(TreeScanner.java:143)
%%   checkers.basetype.BaseTypeVisitor.visitMethod(BaseTypeVisitor.java:218)
%%   checkers.basetype.BaseTypeVisitor.visitMethod(BaseTypeVisitor.java:1)
%%   com.sun.tools.javac.tree.JCTree$JCMethodDecl.accept(JCTree.java:693)
%%   com.sun.source.util.TreePathScanner.scan(TreePathScanner.java:67)
%%   checkers.basetype.BaseTypeVisitor.scan(BaseTypeVisitor.java:122)
%%   checkers.basetype.BaseTypeVisitor.scan(BaseTypeVisitor.java:1)
%%   com.sun.source.util.TreeScanner.scanAndReduce(TreeScanner.java:80)
%%   com.sun.source.util.TreeScanner.scan(TreeScanner.java:90)
%%   com.sun.source.util.TreeScanner.scanAndReduce(TreeScanner.java:98)
%%   com.sun.source.util.TreeScanner.visitClass(TreeScanner.java:132)
%%   checkers.basetype.BaseTypeVisitor.visitClass(BaseTypeVisitor.java:158)
%%   checkers.basetype.BaseTypeVisitor.visitClass(BaseTypeVisitor.java:1)
%%   com.sun.tools.javac.tree.JCTree$JCClassDecl.accept(JCTree.java:617)
%%   com.sun.source.util.TreePathScanner.scan(TreePathScanner.java:49)
%%   checkers.source.SourceChecker.typeProcess(SourceChecker.java:337)
%%   com.sun.source.util.AbstractTypeProcessor$AttributionTaskListener.finished(AbstractTypeProcessor.java:211)
%%   com.sun.tools.javac.main.JavaCompiler.flow(JavaCompiler.java:1272)
%%   com.sun.tools.javac.main.JavaCompiler.flow(JavaCompiler.java:1231)
%%   com.sun.tools.javac.main.JavaCompiler.compile2(JavaCompiler.java:885)
%%   com.sun.tools.javac.main.JavaCompiler.compile(JavaCompiler.java:844)
%%   com.sun.tools.javac.main.Main.compile(Main.java:419)
%%   com.sun.tools.javac.main.Main.compile(Main.java:333)
%%   com.sun.tools.javac.main.Main.compile(Main.java:324)
%%   com.sun.tools.javac.Main.compile(Main.java:76)
%%   com.sun.tools.javac.Main.main(Main.java:61)
%% 1 error



%%% Rather out of date!
%% Not relevant to most readers.  Can go in a README file in our repository.
% \section{Putting your checker in the repository\label{writing-repository}}
%
% This section is relevant only if you wish to add your checker to the source code
% repository for the Checker Framework --- for example, to include your
% checker in the Checker Framework distribution.
%
% The checkers appear in directory \code{annotations/checkers/} of
% the \code{annotations} repository.  It contains the following relevant
% subdirectories:
% \begin{itemize}
% \item
%   \code{manual/}: Documentation for your checker goes here.
% \item
%   \code{src/checkers/\emph{annotation\_name}/}: Code for the checker,
%   in a directory that is a sibling of \code{quals/}, \code{nonnull/},
%   etc.
% \item
%   \code{jdk/\emph{annotation\_name}/}: Annotated ``skeleton class''
%   versions of the JDK and other libraries (see Section~\ref{skeleton}).
% \item
%   \code{tests/\emph{annotation\_name}/}: Inputs and outputs for the test
%   suite for the checker.  A single top-level test suite class goes in
%   \code{tests/src/tests/}.
% \end{itemize}


\section{Documenting the checker\label{documenting-a-checker}}

This section describes how to write a chapter for this manual that
describes a new type-checker.  This is a prerequisite to having your
type-checker distributed with the Checker Framework, which is the best way
for users to find it and for it to be kept up to date with Checker
Framework changes.  Even if you do not want your checker distributed with
the Checker Framework, these guidelines may help you write better
documentation.

When writing a chapter about a new type-checker, see the existing chapters
for inspiration.  (But recognize that the existing chapters aren't perfect:
maybe they can be improved too.)

A chapter in the Checker Framework manual should generally have the
following sections:

\begin{description}
\item[Chapter: Belly Rub Checker]
  The text before the first section in the chapter should state the
  guarantee that the checker provides and why it is important.  It should
  give an overview of the concepts.  It should state how to run the checker.
\item[Section: Belly Rub Annotations]
  This section includes descriptions of the annotations with links to the
  Javadoc and a diagram of the type hierarchy.  A textual description of
  the hieraarchy is not sufficient; the diagram really helps readers to
  understand the system.

  The Javadoc for the annotations deserves the same care as the manual
  chapter.  Each annotation's Javadoc comment should use the
  \<@checker\_framework.manual> Javadoc taglet to refer to the chapter that
  describes the checker; see \refclass{javacutil/dist}{ManualTaglet}.
\item[Section: What the Belly Rub Checker checks]
  This section gives more details about when an error is issued, with examples.
\item[Section: Examples]
  Code examples.
\end{description}

Sometimes you can omit some of the above sections.  Sometimes there are
additional sections, such as tips on suppressing warnings, comparisons to
other tools, and run-time support.

Don't forget to add the checker to a logical place in the manual (not
necessarily as the last checker-related chapter).  Also add two references
to the checker's chapter:  one at the beginning of
chapter~\ref{introduction}, and identical text in
Section~\ref{type-refinement-runtime-tests} (both of these lists appear in
the same order as the manual chapters, to help us notice if anything is
missing).

Every chapter and (sub)section should have a label defined \emph{within} the 
\verb|\section| command.  Section labels should starting with the checker
name (as in \verb|\label{bellyrub-examples}|) and not with ``\<sec:>''.
These conventions are for the benefit of the Hevea program that produces
the HTML version of the manual.


\section{javac implementation survival guide\label{javac-tips}}

Since this section of the manual was written, the useful ``The Hitchhiker's
Guide to javac'' has become available at
\url{http://openjdk.java.net/groups/compiler/doc/hhgtjavac/index.html}.
See it first, and then refer to this section.  (This section of the manual
should be revised, or parts eliminated, in light of that document.)


A checker built using the Checker Framework makes use of a few interfaces
from the underlying compiler (Oracle's OpenJDK javac).
This section describes those interfaces.




\subsection{Checker access to compiler information\label{compiler-information}}

The compiler uses and exposes three hierarchies to model the Java
source code and classfiles.


\subsubsection{Types - Java Language Model API\label{javac-types}}

A \refModelclass{type}{TypeMirror} represents a Java type.
% Java declaration, statement, or expression.

\begin{sloppypar}
There is a \code{TypeMirror} interface to represent each type kind,
e.g., \code{PrimitiveType} for primitive types, \code{ExecutableType}
for method types, and \code{NullType} for the type of the \code{null} literal.
\end{sloppypar}

\code{TypeMirror} does not represent annotated types though.  A checker
should use the Checker Framework types API,
\refclass{framework/type}{AnnotatedTypeMirror}, instead.  \code{AnnotatedTypeMirror}
parallels the \code{TypeMirror} API, but also present the type annotations
associated with the type.

The Checker Framework and the checkers use the types API extensively.


\subsubsection{Elements - Java Language Model API\label{javac-elements}}

An \refModelclass{element}{Element} represents a potentially-public
declaration that can be accessed from elsewhere:  classes, interfaces, methods, constructors, and
fields.  \<Element> represents elements found in both source
code and bytecode.

There is an \code{Element} interface to represent each construct, e.g.,
\code{TypeElement} for class/interfaces, \code{ExecutableElement} for
methods/constructors, \code{VariableElement} for local variables and
method parameters.

If you need to operate on the declaration level, always use elements rather
than trees
% in same subsection, which is the limit of the numbering.
% (Section~\ref{javac-trees})
(see below).  This allows the code to work on
both source and bytecode elements.

Example: retrieve declaration annotations, check variable
modifiers (e.g., \code{strictfp}, \code{synchronized})


\subsubsection{Trees - Compiler Tree API\label{javac-trees}}

A \refTreeclass{tree}{Tree} represents a syntactic unit in the source code,
like a method declaration, statement, block, \<for> loop, etc. Trees only
represent source code to be compiled (or found in \code{-sourcepath});
no tree is available for classes read from bytecode.

There is a Tree interface for each Java source structure, e.g.,
\code{ClassTree} for class declaration, \code{MethodInvocationTree}
for a method invocation, and \code{ForEachTree} for an enhanced-for-loop
statement.

You should limit your use of trees. A checker uses Trees mainly to
traverse the source code and retrieve the types/elements corresponding to
them.  Then, the checker performs any needed checks on the types/elements instead.


\subsubsection{Using the APIs\label{using-the-apis}}

The three APIs use some common idioms and conventions; knowing them will
help you to create your checker.

\emph{Type-checking}:
Do not use \code{instanceof} to determine the class of the object,
because you cannot necessarily predict the run-time type of the object that
implements an interface.  Instead, use the \code{getKind()} method.  The
method returns \refModelclass{type}{TypeKind},
\refModelclass{element}{ElementKind}, and \refTreeclass{tree}{Tree.Kind}
for the three interfaces, respectively.

\emph{Visitors and Scanners}:
The compiler and the Checker Framework use the visitor pattern
extensively. For example, visitors are used to traverse the source tree
(\refclass{common/basetype}{BaseTypeVisitor} extends
\refTreeclass{util}{TreePathScanner}) and for type
checking (\refclass{framework/type}{TreeAnnotator} implements
\refTreeclass{tree}{TreeVisitor}).

\emph{Utility classes}:
Some useful methods appear in a utility class.  The Oracle convention is that
the utility class for a \code{Foo} hierarchy is \code{Foos} (e.g.,
\refModelclass{util}{Types}, \refModelclass{util}{Elements}, and
\refTreeclass{util}{Trees}).  The Checker Framework uses a common
\code{Utils} suffix instead (e.g., \refclass{javacutil}{TypesUtils},
\refclass{javacutil}{TreeUtils}, \refclass{javacutil}{ElementUtils}), with one
notable exception: \refclass{framework/util}{AnnotatedTypes}.


\subsection{How a checker fits in the compiler as an annotation processor\label{checker-as-annotation-processor}}

The Checker Framework builds on the Annotation Processing API
introduced in Java 6.  A type annotation processor is one that extends
\refclass{javacutil}{AbstractTypeProcessor}; these get run on each class
source file after the compiler confirms that the class is valid Java code.

The most important methods of \refclass{javacutil}{AbstractTypeProcessor}
are \code{typeProcess} and \code{getSupportedSourceVersion}. The former
class is where you would insert any sort of method call to walk the AST\@,
and the latter just returns a constant indicating that we are targeting
version 8 of the compiler. Implementing these two methods should be enough
for a basic plugin; see the Javadoc for the class for other methods that
you may find useful later on.

The Checker Framework uses Oracle's Tree API to access a program's AST\@.
The Tree API is specific to the Oracle OpenJDK, so the Checker Framework only
works with the OpenJDK javac, not with Eclipse's compiler ecj or with
\href{http://gcc.gnu.org/java/}{gcj}.  This also limits the tightness of
the integration of the Checker Framework into other IDEs such as \href{http://www.jetbrains.com/idea/}{IntelliJ IDEA}\@.
An implementation-neutral API would be preferable.
In the future, the Checker Framework
can be migrated to use the Java Model AST of JSR 198 (Extension API for
Integrated Development Environments)~\cite{JSR198}, which gives access to
the source code of a method.  But, at present no tools
implement JSR~198.  Also see Section~\ref{ast-traversal}.



\subsubsection{Learning more about javac\label{learning-more-about-javac}}

Sun's javac compiler interfaces can be daunting to a
newcomer, and its documentation is a bit sparse. The Checker Framework
aims to abstract a lot of these complexities.
You do not have to understand the implementation of javac to
build powerful and useful checkers.
Beyond this document, 
other useful resources include the Java Infrastructure
Developer's guide at
\url{http://wiki.netbeans.org/Java_DevelopersGuide} and the compiler
mailing list archives at
\url{http://news.gmane.org/gmane.comp.java.openjdk.compiler.devel}
(subscribe at
\url{http://mail.openjdk.java.net/mailman/listinfo/compiler-dev}).



% LocalWords:  plugin javac's SourceChecker AbstractProcessor getMessages quals
% LocalWords:  getSourceVisitor SourceVisitor getFactory AnnotatedTypeFactory
% LocalWords:  SupportedAnnotationTypes SupportedSourceVersion TreePathScanner
% LocalWords:  TreeScanner visitAssignment AssignmentTree AnnotatedClassTypes
% LocalWords:  SubtypeChecker SubtypeVisitor NonNull isSubtype getClass nonnull
% LocalWords:  AnnotatedClassType isAnnotatedWith hasAnnotationAt TODO src jdk
% LocalWords:  processor NullnessChecker InterningChecker Nullness Nullable igj
% LocalWords:  AnnotatedTypeMirrors BaseTypeChecker BaseTypeVisitor basetype
% LocalWords:  Aqual Anqual java CharSequence getAnnotatedType UseLovely IGJ
% LocalWords:  AnnotatedTypeMirror LovelyChecker Anomsgtext Ashowchecks enums
% LocalWords:  Afilenames dereferenced SuppressWarnings declaratively SubtypeOf
% LocalWords:  TypeQualifier TypeHierarchy GraphQualifierHierarchy ReadOnly Foo
% LocalWords:  QualifierHierarchy QualifierRoot createQualifierHierarchy util
% LocalWords:  createTypeHierarchy QReadOnly ImplicitFor treeClasses TypeMirror
% LocalWords:  LiteralTree ExpressionTree typeClasses annotateImplicit nullable
% LocalWords:  TypeQualifiers getSupportedTypeQualifiers FooChecker nullness
% LocalWords:  FooVisitor FooAnnotatedTypeFactory basicstyle InterningVisitor
% LocalWords:  InterningAnnotatedTypeFactory QualifierDefaults TypeKind getKind
% LocalWords:  setAbsoluteDefaults PolymorphicQualifier TreeVisitor subnodes
% LocalWords:  SimpleTreeVisitor TreePath instanceof subinterfaces TypeElement
% LocalWords:  ExecutableElement PackageElement DeclaredType VariableElement
% LocalWords:  TypeParameterElement ElementVisitor javax getElementUtils NoType
% LocalWords:  ProcessingEnvironment ExecutableType MethodTree ArrayType Warski
% LocalWords:  MethodInvocationTree PrimitiveType BlockTree TypeVisitor blog
% LocalWords:  AnnotatedTypeVisitor SimpleAnnotatedTypeVisitor html langtools
% LocalWords:  AnnotatedTypeScanner bootclasspath asType stringPatterns Foos
% LocalWords:  DefaultQualifierInHierarchy invocable wildcards novariant Utils
% LocalWords:  AggregateChecker getSupportedTypeCheckers Uninterned sourcepath
% LocalWords:  DefaultQualifier bytecode NullType strictfp ClassTree TypesUtils
% LocalWords:  ForEachTree ElementKind TreeAnnotator TreeUtils ElementUtils ecj
% LocalWords:  AnnotatedTypes AbstractTypeProcessor gcj hardcoding jsr
% LocalWords:  typeProcess getSupportedSourceVersion fenum classpath astub
%%  LocalWords:  addAbsoluteDefault BaseAnnotatedTypeFactory superclasses
%%  LocalWords:  SupportedOptions AprintAllQualifiers InvisibleQualifier
%%  LocalWords:  Adetailedmsgtext AprintErrorStack Aignorejdkastub Astubs
%%  LocalWords:  Anocheckjdk AstubDebug Aflowdotdir AresourceStats
%%  LocalWords:  classfiles

\htmlhr
\chapter{Troubleshooting and getting help\label{troubleshooting}}

Please read the entire manual, including this chapter and the FAQ
(Chapter~\ref{faq}), because the manual might already answer your question.
If not, you can use the mailing list,
\code{checker-framework-discuss@googlegroups.com}, to ask other users for
help.  For archives and to subscribe, see \url{http://groups.google.com/group/checker-framework-discuss}.
To report bugs, use the issue tracker at
\url{http://code.google.com/p/checker-framework/issues/list}.
If you want to help out, you can choose a bug and fix it, or select a
project from the ideas list at
\url{http://code.google.com/p/checker-framework/wiki/Ideas}.


\section{Common problems and solutions\label{common-problems}}

\begin{itemize}
\item
To verify that you are using the compiler you think you are, you can add
\code{-version} to the command line.  For instance, instead of running
\code{javac -g MyFile.java}, you can run \code{javac \underline{-version} -g
  MyFile.java}.  Then, javac will print out its version number in addition
to doing its normal processing.

\end{itemize}



\subsection{Unable to run the checker, or checker crashes\label{common-problems-running}}

If you are unable to run the checker, or if the checker or the compiler
crashes, then the problem may be a problem with your environment.
This section describes some possible problems and solutions.

\begin{itemize}
\item
If you get the error

%BEGIN LATEX
\begin{smaller}
%END LATEX
\begin{Verbatim}
com.sun.tools.javac.code.Symbol$CompletionFailure: class file for com.sun.source.tree.Tree not found
\end{Verbatim}
% Unconfuse Emacs by matching the "$" in the above Verbatim
%BEGIN LATEX
\end{smaller}
%END LATEX

\noindent
then you are using the source installation and file \code{tools.jar} is not
on your classpath.  See the installation instructions
(Section~\ref{installation}).

\item
If you get an error such as

\begin{Verbatim}
package checkers.nullness.quals does not exist
\end{Verbatim}

  \noindent
  despite no apparent use of \code{import checkers.nullness.quals.*;} in
  the source code, then perhaps
  \code{jsr308\_imports} is set as a Java system property, a shell
  environment variable, or a command-line option (see
  Section~\ref{jsr308_imports}).  You can solve this by unsetting the
  variable/option, or by ensuring that the \code{checkers.jar} file is on
  your classpath.

If the error is 

\begin{Verbatim}
package 'checkers.nullness.quals does not exist
\end{Verbatim}

\noindent
(note the extra apostrophe!), then you have probably misused quoting when
supplying the \code{jsr308\_imports} environment variable.

\item
If you get an error like the following when using the Ant task
(Section~\ref{ant-task}),

%BEGIN LATEX
\begin{smaller}
%END LATEX
\begin{Verbatim}
...\build.xml:59: Error running ${env.CHECKERS}\binary\javac.bat compiler
\end{Verbatim}
% Unconfuse Emacs by matching the "$" in the above Verbatim
%BEGIN LATEX
\end{smaller}
%END LATEX

\noindent
then the problem may be that you have not set the CHECKERS environment
variable, as described in Section~\ref{windows-installation}.  Or, maybe
you made it a user variable instead of a system variable.

\item
If you get one of these errors:

\begin{alltt}
The hierarchy of the type \emph{ClassName} is inconsistent

The type com.sun.source.util.AbstractTypeProcessor cannot be resolved.
  It is indirectly referenced from required .class files", 
\end{alltt}

\noindent
then you are missing \code{jsr308-all.jar} from your classpath.

\item
If you get the error

\begin{Verbatim}
  java.lang.ArrayStoreException: sun.reflect.annotation.TypeNotPresentExceptionProxy
\end{Verbatim}

\noindent
% I'm not 100% sure of the following explanation and solution.
then an annotation is not present at run time that was present at compile
time.  For example, maybe when you compiled the code, the \<@Nullable>
annotation was available, but it was not available at run time.
You can use JDK 8 at run time, or compile
with a Java 7 compiler that will ignore the annotations in comments.

\item
A ``class file not found'' error may be due to a JDK version mismatch.
For instance, you might be using JDK 7, but you get an error that refers to a class that was in a
previous version of the JDK but has subsequently been removed, such as:

\begin{Verbatim}
  class file for java.io.File$LazyInitialization not found
  class file for java.util.Hashtable$EmptyIterator not found
  java.lang.NoClassDefFoundError: java/util/Hashtable$EmptyEnumerator
\end{Verbatim}

Or, you might be using JDK 6, but you get an error that refers to a class
that has been introduced in a newer version of the JDK, such as:

\begin{Verbatim}
  class file for java.util.Vector$Itr not found
\end{Verbatim}
% To unconfuse Emacs's LaTeX mode: $

This problem occurs when your classpath contains code that was compiled
with one version of the JDK and refers to its implementation details, but
your classpath does not contain that version of the JDK itself.

You can solve the problem by re-generating \code{jdk/jdk.jar} and
\code{binary/jdk.jar}.  You can do this by running

\begin{Verbatim}
  cd checkers
  ant jdk.jar bindist
\end{Verbatim}


\item
A \<NoSuchFieldError> such as this:

\begin{Verbatim}
java.lang.NoSuchFieldError: NATIVE_HEADER_OUTPUT
\end{Verbatim}

\noindent
Field \<NATIVE\_HEADER\_OUTPUT> was added in JDK 8.
The error message suggests that
you're not executing with the right bootclasspath: some classes were
compiled with the JDK 8 version and expect the field, but you're
executing the compiler on a JDK without the field.

One possibility is that you are not running the Type Annotations compiler
--- use \<javac -version> to check this, then use the right one.  (Maybe
the Type Annotations javac is at the end rather than the beginning of your
path.)

If you are using Ant, then one possibility
is that the javac compiler is using the same JDK as Ant is using.  You can
correct this by being sure to use \<fork="yes"> (see
Section~\ref{ant-task}) and/or setting the \<build.compiler> property to
\<extJavac>.

If you are building from source, you might need to rebuild the Annotation
File Utilities before recompiling or using the Checker Framework.


\item
If you get an error that contains lines like these:

\begin{Verbatim}
Caused by: java.util.zip.ZipException: error in opening zip file
	at java.util.zip.ZipFile.open(Native Method)
	at java.util.zip.ZipFile.<init>(ZipFile.java:131)
\end{Verbatim}

\noindent
then one possibility is that you have installed the Checker Framework in a
directory that contains special characters that Java's ZipFile
implementation cannot handle.  For instance, if the directory name contains
``\<+>'', then Java 1.6 throws a ZipException, and Java 1.7 throws a
FileNotFoundException and prints out the directory name with ``\<+>''
replaced by blanks.

\end{itemize}


\subsection{Unexpected type-checking results\label{common-problems-typechecking}}

This section describes possible problems that can lead the type-checker to
give unexpected results.


\begin{itemize}
\item
  If the Checker Framework is unable to verify a property that you know is
  true, then it is helpful to formulate an argument about why the property
  is true.  Recall that the Checker Framework does modular verification,
  one procedure at a time; it observes the specifications, but not the
  implementations, of other methods.

  If any aspects of your argument are not expressed as annotations, then
  you may need to write more annotations.  If any aspects of your argument
  are not expressible as annotations, then you may need to extend the
  type-checker.

\item
If a checker seems to be ignoring the annotation on a method, then it is
possible that the checker is reading the method's signature from its
\code{.class} file, but the \code{.class} file was not created by the JSR
308 compiler.  You can check whether the annotations actually appear in the
\code{.class} file by using the \code{javap} tool.

If the annotations do not appear in the \code{.class} file, here are two
ways to solve the problem:
\begin{itemize}
\item
  Re-compile the method's class with the Type Annotations compiler.  This will
  ensure that the type annotations are written to the class file, even if
  no type-checking happens during that execution.
\item
  Pass the method's file explicitly on the command line when type-checking,
  so that the compiler reads its source code instead of its \code{.class}
  file.
\end{itemize}

\item
If the compiler reports that it cannot find a method from the
JDK or another external library, then maybe the stub/skeleton file for that
class is incomplete.  You can edit it to add the missing method.  The
libraries appear, for example, at \code{checkers/jdk/nullness/src/} for the
Nullness checker.

The error might take one of these forms:

\begin{Verbatim}
method sleep in class Thread cannot be applied to given types
cannot find symbol: constructor StringBuffer(StringBuffer)
\end{Verbatim}

\item
If you get an error related to a bounded type parameter and a literal such
as \<null>, the problem may be missing defaulting.  Here is an example:

\begin{Verbatim}
mypackage/MyClass.java:2044: warning: incompatible types in assignment.
      T retval = null;
                 ^
  found   : null
  required: T extends @MyQualifier Object
\end{Verbatim}

\noindent
A value that can be assigned to a variable of type \<T extends @MyQualifier
Object> only if that value is of the bottom type, since the bottom type is
the only one that is a subtype of every subtype of \<T extends @MyQualifier
Object>.  The value \<null> satisfies this for the Java type system, and it
must be made to satisfy it for the pluggable type system as well.  The
typical way to address this is to write the meta-annotation
\<@ImplicitFor(trees={Tree.Kind.NULL\_LITERAL})> on the definition of the
bottom type qualifier.

\end{itemize}


\subsection{Unable to build the checker, or to run programs\label{common-problems-running-java}}

An error like this

\begin{Verbatim}
Unsupported major.minor version 51.0
\end{Verbatim}

means that you have compiled some files into the Java 7 format (version
51.0), but you are trying to run them with Java 6.  Run \<java -version> to
determine the version of Java you are using and use a newer version,
and/or use the \<-target>
command-line option to \<javac> to specify the version of the class files
that are created, such as \<javac -target 6 ...>.



%% Commented out because these are not very helpful in terms of helping
%% a reader understand whether he/she is encountering the specific problem,
%% and because the issue tracker is a better way to find out about current
%% known problems.
%
% \subsection{Known problems in the framework\label{known-problems}}
% 
% \begin{itemize}
% 
% \item
%   The framework may not parse annotations from skeleton files if the
%   skeleton files are older than the classfiles.  Running \code{ant
%     touch-jdk} solves this problem, by applying the 
%   \code{touch} program to each distributed skeleton file.
% 
% % Mahmood will address.  -MDE 3/19/2009
% \item The framework is missing a check for type argument subtyping in
%   method invocations if the type arguments are inferred.
% 
% % Mahmood will address.  -MDE 3/19/2009
% \item The checks for enclosed types are not yet fully tested.
% 
% \end{itemize}
% 
% \subsection{Known problems in the Nullness checker}
% 
% \begin{itemize}
% \item
%   The Nullness checker is often able to determine that a call to
%   \code{Map.get()} will not return null.  This enables the checker to avoid
%   issuing false positive warnings, in circumstances like the following.
% 
% \begin{Verbatim}
%     @NonNull String value;
%     if (myMap.containsKey(key)) {
%       value = myMap.get(key);
%     }
%     for (String keyInMap : myMap.keySet()) {
%         value = myMap.get(keyInMap);
%     }
% \end{Verbatim}
% 
%   The Nullness checker can sometimes fail to issue a warning if the map is
%   modified or re-assigned between the check of \code{containsKey} and the
%   call to \code{get}.
% 
% % This description really needs an example of a case where the checker
% % fails.  Right now, it is impossible for a reader to tell what the problem
% % is or whether a particular piece of code triggers it.
% % Do we have a test case?
% 
% 
% % The solution is to merge flow with the Map.get heuristics.
% % And to do forward instead of backward analysis.
% 
% 
% \end{itemize}


\section{How to report problems (bug reporting)\label{reporting-bugs}}

If you have a problem with any checker, or with the Checker Framework,
please file a bug at 
\url{http://code.google.com/p/checker-framework/issues/list}.
(First, check whether there is an existing bug report for that issue.)

Alternately (especially if your communication is not a bug report), you can
send mail to checker-framework-dev@googlegroups.com.
We welcome suggestions, annotated libraries, bug fixes, new
features, new checker plugins, and other improvements.

Please ensure that your bug report is clear and that it is complete.
Otherwise, we may be unable to understand it or to reproduce it, either of
which would prevent us from fixing the bug.  Your bug report will be most
helpful if you:

\begin{itemize}
\item
  Add \code{-version -verbose -AprintErrorStack -printAllQualifiers} to the javac options.  This causes the compiler to output
  debugging information, including its version number.
\item
  Indicate exactly what you did.  Don't skip any steps, and don't merely
  describe your actions in words.  Show the exact commands by attaching a
  file or using cut-and-paste from your command shell;
\item
  Include all files that are necessary to reproduce the problem.  This
  includes every file that is used by any of the commands you reported, and
  possibly other files as well.
\item
  Indicate exactly what the result was by attaching a file or using
  cut-and-paste from your command shell (don't merely describe it in
  words).  Also indicate what you expected the result to be --- remember, a
  bug is a difference between desired and actual outcomes.
\end{itemize}

A particularly useful format for a test case is as a new file, or a diff to
an existing file, for the existing Checker Framework test suite.  For
instance, for the Nullness
Checker, see directory \<checker-framework/checkers/tests/nullness/>.
But, please report your bug even if you do not report it in this format.


\section{Building from source\label{build-source}}

The Checker Framework release (Section~\ref{installation}) contains
everything that most users need, both to use the distributed checkers and
to write your own checkers.  This section describes how to compile its
binaries from source.  You will be using the latest development version of
the Checker Framework, rather than an official release.

% Doing
% so permits you to examine and modify the implementation of the distributed
% checkers and of the checker framework.  It may also help you to debug
% problems more effectively.


\subsection{Obtain the source}

Obtain the latest source code from the version control repository:

\begin{Verbatim}
export JSR308=$HOME/jsr308
mkdir -p $JSR308
cd $JSR308
hg clone https://code.google.com/p/jsr308-langtools/ jsr308-langtools
hg clone https://code.google.com/p/checker-framework/ checker-framework
hg clone https://code.google.com/p/annotation-tools/ annotation-tools
\end{Verbatim}
% $ to unconfuse Emacs LaTeX mode

\noindent
(Alternately, you could use the version of the source code that is packaged
in the Checker Framework release.)

% TODO: the AFU and JSR 308 source code is not included in the
% checkers.zip file!


\subsection{Build the Type Annotations compiler}

\begin{enumerate}
\item
% Why is this necessary?  What goes wrong if it is not set?  Can I avoid
% the need to set it?  It's used for:
%  * the location of tools.jar, below.
%  * the default location of RTJAR, in checkers/jdk/Makefile.
Set the \<JAVA\_HOME> environment variable to the location of your JDK 
7 installation (not the JRE installation, and not JDK 6).
This needs to be an Oracle JDK.
(The \<JAVA\_HOME> environment
variable might already be set, because it is needed for Ant to work.)

In the bash shell, the following command \emph{sometimes} works (it might
not because \<java> might be the version in the JDK or in the JRE):
% Can someone give a simpler command?
\begin{Verbatim}
  export JAVA_HOME=${JAVA_HOME:-$(dirname $(dirname $(dirname $(readlink -f $(/usr/bin/which java)))))}
\end{Verbatim}

% # To build your own JDK:
% mkdir -p /scratch/$USER/jdk7
% cd /scratch/$USER/jdk7
% wget http://docs.oracle.com/otn-pub/java/jdk/7/jdk-7-linux-x64.tar.gz
% tar xzf jdk-7-linux-x64.tar.gz
% export JAVA_HOME=/scratch/$USER/jdk7/jdk1.7.0
% export PATH=${JAVA_HOME}/bin:${PATH}

\item
Compile the Type Annotations javac compiler and the javap tool:

\begin{Verbatim}
  cd $JSR308/jsr308-langtools/make
  ant clean build-javac build-javap
\end{Verbatim}

\item
 Add the \<jsr308-langtools/dist/bin> directory to the front of your PATH environment variable.
  Example command:

\begin{Verbatim}
  export PATH=$JSR308/jsr308-langtools/dist/bin:${PATH}
\end{Verbatim}

\end{enumerate}

% JSR 308 extends the Java language to permit annotations to appear on types,
% as in \code{List<@NonNull String>} (see Section~\ref{writing-annotations}).
% This change will be part of the Java 8 language.  We recommend that you
% write annotations in comments, as in \code{List</*@NonNull*/ String>} (see
% Section~\ref{annotations-in-comments}).  The JSR 308 compiler still reads
% such annotations, but this syntax permits you to use a compiler other than
% the JSR 308 compiler.  For example, you can compile your code with a Java 5
% compiler, and you can use a checker as an external tool in an IDE.



\subsection{Build the Annotation File Utilities\label{afu-building}}

This is simply done by:

\begin{Verbatim}
  cd $JSR308/annotation-tools
  ant
\end{Verbatim}

You do not need to add the Annotation File Utilities to the path, as
the Checker Framework build finds it using relative paths.


\subsection{Build the Checker Framework\label{building}}

% Building (compiling) the checkers and framework from source creates the
% \code{checkers.jar} file.  A pre-compiled \code{checkers.jar} is included
% in the distribution, so building it is optional.  It is mostly useful for
% people who are developing compiler plug-ins (type-checkers).  If you only
% want to \emph{use} the compiler and existing plug-ins, it is sufficient to
% use the pre-compiled version.

\begin{enumerate}
% \item
% Edit \code{checkers/build.properties} file so that the
% \code{compiler.lib} property specifies the location of the JSR 308
% \code{javac.jar} library.  (If you also installed the JSR 308 compiler from
% source, and you made the \code{checkers} and \code{jsr308-langtools} directories
% siblings, then you don't need to edit \code{checkers/build.properties}.)

\item
Run \code{ant} to create \<checkers.jar>:

\begin{Verbatim}
  cd $JSR308/checker-framework/checkers
  ant
\end{Verbatim}
% $ to unconfuse Emacs LaTeX mode

\item Add \code{tools.jar} and \code{checkers.jar} to your classpath.
  (If you do not do this, you will have to supply the \code{-cp} option
  whenever you run \code{javac} and use a checker plugin.)
  Example command:

%BEGIN LATEX
\begin{smaller}
%END LATEX
\begin{Verbatim}
  export CLASSPATH=${CLASSPATH}:$JAVA_HOME/lib/tools.jar:$JSR308/checker-framework/checkers/checkers.jar
\end{Verbatim}
% $ to unconfuse Emacs LaTeX mode
%BEGIN LATEX
\end{smaller}
%END LATEX
  %% In Cygwin, are reversed slashes required?

\item Test that everything works:

  \begin{itemize}

  \item Run \code{ant all-tests} in the \code{checkers} directory:
\begin{Verbatim}
  cd $JSR308/checker-framework/checkers
  ant all-tests
\end{Verbatim}
% $ to unconfuse Emacs LaTeX mode

  \item Run the Nullness checker examples (see
    Section~\refwithpage{nullness-example}).

  \end{itemize}

\end{enumerate}

\subsection{Build the Checker Framework manual (this document)}

\begin{enumerate}
\item
To build the manual you will need plume-bib (\myurl{http://code.google.com/p/plume-bib/}) and {\hevea} (\myurl{http://hevea.inria.fr/}) installed.

\item
Run \code{make} in the \code{checkers/manual} directory to build both the PDF and HTML versions of the manual.
\end{enumerate}

% \subsection{Adjust classpath}

% Building the Checker Framework requires use of a Java 8 compiler.  You may
% use either the OpenJDK compiler or the JSR 308 compiler.  The latter has a
% few extra features and tends to get bug fixes more quickly.

% The following instructions give detailed steps for installing the source
% release of the Checker Framework.


% \item Download and install the JSR 308 implementation; follow the instructions at
% % alternative: \urldef{\JsrInstallingUrl}\url{http://types.cs.washington.edu/checker-framework/current/README-jsr308.html#installing}
% {\codesize\url{http://types.cs.washington.edu/checker-framework/current/README-jsr308.html#installing}}.
% This creates a \code{jsr308-langtools} directory.
% 
% \item Download the Checker Framework distribution zipfile from
% \myurl{http://types.cs.washington.edu/checker-framework/current/checkers.zip},
% and unzip it to create a \code{checkers} directory.  We recommend that the
% \code{checkers} directory and the \code{jsr308-langtools} directory be siblings.
% Example commands:
% 
% \begin{Verbatim}
%   cd $JSR308
%   wget http://types.cs.washington.edu/checker-framework/current/checkers.zip
%   unzip checkers.zip
% \end{Verbatim}
% 
% You will also need to adjust the path to \<javac> in any Ant buildfiles,
% etc.

% \item Optionally edit property \code{compiler.lib} in file
%   \code{checkers/build.properties}.  You don't have to do this if the
%   \code{checkers} directory and the \code{jsr308-langtools} directory are
%   siblings.


% (A checkers implementation builds on
% standard mechanisms such as JSR 269 annotation processing, but also
% accesses the compiler's AST. In the long run, a checker built using the
% Checker Framework should not be dependent on any compiler specifics.)
% If you do not place the annotations in 
% then you should also disable Eclipse's on-the-fly syntax checking.




% \subsection{TO DO:  The short instructions (for Linux only)}
% 
% %%% This comment does not seem to be correct any longer.
% %% This text is identically reproduced at ../../jsr308-langtools/README-jsr308.html
% %% so if you change either one, change the other also!
% 
% The following commands install
% the JSR 308 \code{javac} compiler and the Checker
% Framework, or update an existing installation.
% It currently works only on \textbf{Linux}.
% For more details, or if anything goes wrong, see the comments in the 
% \code{Makefile-jsr308-install} file.
% 
% \begin{enumerate}
% 
% \item
%   Execute the following commands:
% 
% \begin{Verbatim}
%   cd
%   wget -nv -N http://types.cs.washington.edu/jsr308/Makefile-jsr308-install
%   make -f Makefile-jsr308-install
% \end{Verbatim}
% 
% \item
% Set some environment variables according to the instructions at the top of file
% \code{Makefile-jsr308-install}.
% 
% \end{enumerate}



\section{Learning more\label{learning-more}}

The technical paper ``Practical pluggable types for Java''~\cite{PapiACPE2008}
(\myurl{http://www.cs.washington.edu/homes/mernst/pubs/pluggable-checkers-issta2008.pdf})
gives more technical detail about many
aspects of the Checker Framework and its implementation.
%
The technical
paper also describes case
studies in which each of the checkers found
previously-unknown errors in real software.

The paper ``Building and using pluggable type-checkers''~\cite{DietlDEMS2011}
(\myurl{http://www.cs.washington.edu/homes/mernst/pubs/pluggable-checkers-icse2011.pdf})
discusses further experience with the Checker Framework, increasing the
number of lines of verified code to 3 million.

In addition to these papers that discuss use the Checker Framework
directly, other academic papers use the Checker Framework in their
implementation or evaluation.  Most educational use of the Checker
Framework is never published, and most commercial use of the Checker
Framework is never discussed publicly.


\section{Comparison to other tools\label{other-tools}}

A pluggable type-checker, such as those created by the Checker Framework,
aims to help you prevent or detect all errors of a given variety.  An
alternate approach is to use a bug detector such as
\ahref{http://findbugs.sourceforge.net/}{FindBugs},
\ahref{http://jlint.sourceforge.net/}{Jlint}, or
\ahref{http://pmd.sourceforge.net/}{PMD}.

A pluggable type-checker
differs from a bug detector in several ways:
\begin{itemize}
\item
  A type-checker aims to find \emph{all} errors.  Thus, it can verify the
  \emph{absence} of errors:  if the type checker says there are no null
  pointer errors in your code, then there are none.  (This guarantee only
  holds for the code it checks, of course; see
  Section~\ref{checker-guarantees}.)

  A bug detector aims to find \emph{some} of the most obvious errors.  Even
  if it reports no errors, then there may still be errors in your code.

  Both types of tools may issue false positive warnings; see
  Section~\ref{suppressing-warnings}.

\item
  A type-checker requires you to annotate your code with type qualifiers,
  or to run an inference tool that does so for you.  A bug detector may not
  require annotations.  This means that it may be easier to get started
  running a bug detector.

\item
  A type-checker may use a more sophisticated and complete analysis.
  A bug detector typically does a more lightweight analysis, coupled with
  heuristics to suppress false positives.

  As one example, a type-checker can take advantage of annotations on
  generic type parameters, such as \code{List<@NonNull String>}, permitting
  it to be much more precise for code that uses generics.

\end{itemize}

A case study~\cite[\S6]{PapiACPE2008} compared the Checker Framework's nullness
checker with those of FindBugs, Jlint, and PMD\@.  The case study was on a
well-tested program in daily use.  The Checker Framework tool found 8
nullness errors (that is, null pointer dereferences).  None of the other
tools found any errors.

Also see the
\ahref{http://types.cs.washington.edu/jsr308/}{JSR 308}~\cite{JSR308-2008-09-12}
documentation for a detailed discussion of related work.



\section{Credits and changelog\label{credits}}

The key developers of the Checker Framework are Mahmood Ali, Telmo Correa,
Werner M. Dietl, Michael D. Ernst, and Matthew M. Papi.
Many other developers have also contributed, for example by writing
the checkers that are distributed with the Checker Framework.
Many, many users to list have provided valuable feedback, for which we are
grateful.

%% Not so accurate, since Mahmood is really an author of the nullness and
%% interned checkers too.
% The Checker Framework was implemented by 
% The nullness checker was implemented by Matthew M. Papi.
% The interning checker was implemented by Matthew M. Papi.
% The Javari checker was implemented by Telmo Correa.
% The IGJ checker was implemented by Mahmood Ali.
% The basic checker was implemented by Matthew M. Papi.
% The Fake enum checker was written by Werner M. Dietl.
% ... many others ...

Differences from previous versions of the checkers and framework can be found
in the \code{changelog-checkers.txt} file.  This file is included in the
Checker Framework distribution and is also available on the web at
\myurl{http://types.cs.washington.edu/checker-framework/current/changelog-checkers.txt}.






% LocalWords:  jsr unsetting plugins langtools zipfile cp plugin Nullness txt
% LocalWords:  nullness classpath NonNull MyObject javac uref changelog MyEnum
% LocalWords:  subtyping containsKey proc classfiles SourceChecker javap jdk
% LocalWords:  MyFile buildfiles ClassName JRE java jsr308 bootclasspath
%  LocalWords:  extJavac ZipFile AprintErrorStack printAllQualifiers Jlint
%  LocalWords:  Telmo Correa Papi


\htmlhr
\bibliographystyle{alpha}
\bibliography{bibstring-unabbrev,types,ernst,invariants,generals,alias,concurrency}

\end{document}

% LocalWords:  pt TODO JavaDocs Arg api HEVEA html ernst
