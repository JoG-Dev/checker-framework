\htmlhr
\chapter{Advanced type system features\label{advanced-type-system-features}}

This section describes features that are automatically supported by every
checker written with the Checker Framework.
You may wish to skim or skip this section on first reading.  After you have
used a checker for a little while and want to be able to express more
sophisticated and useful types, or to understand more about how the Checker
Framework works, you can return to it.


\section{Polymorphism and generics\label{polymorphism}}

\subsection{Generics (parametric polymorphism or type polymorphism)\label{generics}}

The Checker Framework fully supports
type-qualified Java generic types (also known in the research literature as ``parametric
polymorphism'').  Before running any checker, we recommend that you eliminate
raw types from your code (e.g., your code should use \code{List<...>} as
opposed to \code{List}).
Using generics helps prevent type errors just as using a pluggable
type-checker does.
% Should say why, or what are the consequences of violating this.

When instantiating a generic type,
clients supply the qualifier along with the type argument, as in
\code{List<@NonNull String>}.


\paragraph{Restricting instantiation of a generic class}

There are two ways to restrict the type qualifiers that may be used on
the actual type argument when instantiating a generic class.

The first technique is the standard Java approach of using the
\code{extends} or \code{super} clause to supply an upper or lower bound.
For example:

\begin{Verbatim}
  MyClass<T extends @NonNull Object> { ... }

  MyClass<@NonNull String> m1;       // OK
  MyClass<@Nullable String> m2;      // error
\end{Verbatim}

The second technique is to write a type annotation on the declaration of a
generic type parameter, which specifies the exact annotation that is
required on the actual type argument, rather than just a bound.  For example:

\begin{Verbatim}
  class MyClassNN<@NonNull T> { ... }
  class MyClassNble<@Nullable T> { ... }

  MyClassNN<@NonNull Number> v1;     // OK
  MyClassNN<@Nullable Number> v2;    // error
  MyClassNble<@NonNull Number> v4;   // error
  MyClassNble<@Nullable Number> v3;  // OK
\end{Verbatim}

A way to view a type annotation on a generic type parameter declaration is
as syntactic sugar for the annotation on both the \<extends> and the
\<super> clauses of the declaration.  For example, these two declarations
have the same effect:

\begin{Verbatim}
  class MyClassNN<@NonNull T> { ... }
  class MyClassNN<T extends @NonNull Object super @NonNull void> { ... }
\end{Verbatim}

\noindent
except that the latter is not legal Java syntax.  The syntactic sugar is
necessary because of two limitations in Java syntax:  it is illegal to
specify both the upper and the
lower bound, and it is impossible to specify a type annotation for a lower
bound without also specifying a type (use of \<void> is illegal).

If a type parameter declaration is annotated with \code{@A}, and a bound is
also given, then the annotation applies everywhere that there is no
explicit annotation.  For example, the following pairs of declarations are
identical.

\begin{Verbatim}
  class MyClassNN<@A T> { ... }
  class MyClassNN<T extends @A Object super @A void> { ... }

  class MyClassNN<@A T extends Number> { ... }
  class MyClassNN<T extends @A Number super @A void> { ... }

  class MyClassNN<@A T extends @B Number> { ... }
  class MyClassNN<T extends @B Number super @A void> { ... }

  class MyClassNN<@A T super Number> { ... }
  class MyClassNN<T extends @A Object super @A Number> { ... }

  class MyClassNN<@A T super @B Number> { ... }
  class MyClassNN<T extends @A Object super @B Number> { ... }
\end{Verbatim}

We can see from the above that almost all of the following types mean
different things:

\begin{Verbatim}
  class MyList1<@Nullable T> { ... }
  class MyList2<@NonNull T> { ... }
  class MyList3<T extends @Nullable Object> { ... }
  class MyList4<T extends @NonNull Object> { ... } // same as MyList2
\end{Verbatim}

One way to express the difference is by comparing what expressions are
legal in the implementation of the list --- that is, what expressions may
appear in the ellipsis, such as inside a method's body.  Suppose each class
has, in the ellipsis, these declarations:

\begin{Verbatim}
  T t;
  @Nullable T nble;
  @NonNull T nn;
  void add(T arg) { ... }
  T get(int i) { ... }
\end{Verbatim}

\noindent
Then the following expressions would be legal, inside a given implementation.

\begin{tabular}{|l|c|c|c|c|} \hline
                        & MyList1 & MyList2 & MyList3 & MyList4 \\ \hline
  t = null;             & OK      & error   & error   & error   \\ \hline
  nble = null;          & OK      & OK      & OK      & OK      \\ \hline
  nn = null;            & error   & error   & error   & error   \\ \hline
  t = this.get(0);      & OK      & OK      & OK      & OK      \\ \hline
  nble = this.get(0);   & OK      & OK      & OK      & OK      \\ \hline
  nn = this.get(0);     & error   & OK      & error   & OK      \\ \hline
  this.add(t);          & OK      & OK      & OK      & OK      \\ \hline
  this.add(nble);       & OK      & error   & error   & error   \\ \hline
  this.add(nn);         & OK      & OK      & OK      & OK      \\ \hline
\end{tabular}

(This complete example appears as file
\<checker-framework/checkers/tests/nullness/GenericsExample.java>.)

%% This text is not very helpful.
% The
% implementation of \code{MyList2} may only place non-null objects in the
% list and may assume that retrieved elements are non-null.  The
% implementation of \code{MyList3} is similar in that it may only place
% non-null objects in the list, because it might be instantiated as, say,
% \code{MyList3<@NonNull Date>}.  When retrieving elements from the list,
% the implementation of \code{MyList3} must account for the fact that
% elements of \code{MyList3} may be null, because it might be instantiated
% as, say, \code{MyList3<@Nullable Date>}.
The differences are more
significant when the qualifier hierarchy is more complicated than just
\<@Nullable> and \<@NonNull>.


\paragraph{Defaults for bounds}
Ordinarily, a type parameter declaration with no extends clause means the
type parameter can be instantiated with any type argument at all.  For
example:

\begin{Verbatim}
  class C<T> { ... }
  class C<T extends Object> { ... }  // identical to previous line
\end{Verbatim}

\noindent
However, instantiation may be restricted if a default qualifier is in
effect (see Section~\ref{defaults}).  For example, the Nullness checker
(Chapter~\ref{nullness-checker}) uses a (configurable) default of
\<@NonNull> (see Section~\ref{null-defaults}).  That means that either
declaration above is interpreted as

\begin{Verbatim}
  class C<T extends @NonNull Object> { ... }
\end{Verbatim}

\noindent
and an instantiation such as \code{C<@Nullable Number>} is illegal.
In such a case, to permit all type arguments, the programmer would write

\begin{Verbatim}
  class C<T extends @Nullable Object> { ... }
\end{Verbatim}


It is possible to set the default qualifier for upper bounds separately
from other default qualifiers, by writing an annotation such as
\<@DefaultQualifier(value="Nullable", locations={DefaultLocation.UPPER\_BOUNDS})>.


\paragraph{Type annotations on a use of a generic type variable}

A type annotation on a generic type variable overrides/ignores any type
qualifier (in the same type hierarchy) on the corresponding actual type
argument.  For example, suppose that \code{T} is a formal type parameter.
Then using \code{@Nullable T} within the scope of \code{T} applies the type
qualifier \code{@Nullable} to the (unqualified) Java type of \code{T}.

Here is an example of applying a type annotation to a generic type
variable:

\begin{Verbatim}
  class MyClass2<T> {
    ...
    @Nullable T = null;
    ...
  }
\end{Verbatim}

\noindent
The type annotation does not restrict how \code{MyClass2} may be instantiated
(only the optional \code{extends} clause on the declaration of type
variable \code{T} would do so).  In other words, both
\code{MyClass2<@NonNull String>} and \code{MyClass2<@Nullable String>} are
legal, and in both cases \code{@Nullable T} means \code{@Nullable String}.
In \code{MyClass2<@Interned String>},
\code{@Nullable T} means \code{@Nullable @Interned String}.

% Note that a type annotation on a generic type variable does not act like
% other type qualifiers.  In both cases the type annotation acts as a type
% constructor, but as noted above they act slightly differently.


% %% This isn't quite right because a type qualifier is itself a type
% %% constructor.
% More formally, a type annotation on a generic type variable acts as a type
% constructor rather than a type qualifier.  Another example of a type
% constructor is \code{[]}.  Just as \code{T[]} is not the same type as
% \code{T}, \code{@Nullable T} is not (necessarily) the same type as
% \code{T}.


\subsection{Qualifier polymorphism\label{qualifier-polymorphism}}

The Checker Framework also supports type \emph{qualifier} polymorphism for
methods, which permits a single method to have multiple different qualified
type signatures.

To \emph{define} a polymorphic qualifier, mark the definition with
\<@\refclass{quals}{PolymorphicQualifier}>.  For example,
\<@\refclass{nullness/quals}{PolyNull}> is a polymorphic type
qualifier for the Nullness type system:

\begin{Verbatim}
  @PolymorphicQualifier
  @Target(ElementType.TYPE_USE)
  public @interface PolyNull { }
\end{Verbatim}

To \emph{use} a polymorphic qualifier, just write it on a type.
For example, you can write \<@PolyNull> anywhere that you would write
\<@NonNull> or \<@Nullable>.

A method written using a polymorphic qualifier conceptually has multiple
versions, somewhat like a template in C++ or the generics feature of Java.
In each version, each instance of the polymorphic qualifier has been
replaced by the same other qualifier from the hierarchy.  See the examples
below in Section~\ref{qualifier-polymorphism-examples}.

The method body must type-check with all signatures.  A method call is
type-correct if it type-checks under any one of the signatures.  If a call
matches multiple signatures, then the compiler uses the most specific
matching signature for the purpose of type-checking.  This is just like
Java's rule for resolving overriding methods, though there is no effect on
run-time dispatch or behavior.

Polymorphic qualifiers can be used on a method signature or body.
They may not be used on classes or fields.

%% I don't see why this is necessarily true; one could define @PolyNull1
%% and @PolyNull2.  It's not so relevant to the manual anyway, and raising
%% the point just makes type system bigots criticize the Checker Framework.
% Qualifier polymorphism is limited to a single qualifier variable per method.


\paragraph{Examples of using polymorphic qualifiers\label{qualifier-polymorphism-examples}}

As an example of the use of \<@PolyNull>, method \sunjavadoc{java/lang/Class.html#cast(java.lang.Object)}{Class.cast}
returns null if and only if its argument is \<null>:

\begin{Verbatim}
  @PolyNull T cast(@PolyNull Object obj) { ... }
\end{Verbatim}

\noindent
This is like writing:

\begin{Verbatim}
   @NonNull T cast( @NonNull Object obj) { ... }
  @Nullable T cast(@Nullable Object obj) { ... }
\end{Verbatim}

\noindent
except that the latter is not legal Java, since it defines two
methods with the same Java signature.


As another example, consider

\begin{Verbatim}
  @PolyNull T max(@PolyNull T x, @PolyNull T y);
\end{Verbatim}

\noindent
which is like writing

\begin{Verbatim}
   @NonNull T max( @NonNull T x,  @NonNull T y);
  @Nullable T max(@Nullable T x, @Nullable T y);
\end{Verbatim}

\noindent
Another way of thinking about which one of the two \code{max} variants is
selected is that the nullness annotations of (the declared types of) both
arguments are \emph{unified} to a type that is a supertype of both, also
known as the \emph{least upper bound} or lub.  If both
arguments are \code{@NonNull}, their unification (lub) is \<@NonNull>, and the
method return type is \<@NonNull>.  But if even one of the arguments is \<@Nullable>,
then the unification (lub) is \<@Nullable>, and so is the return type.


\paragraph{Use multiple polymorphic qualifiers in a method signature\label{qualifier-polymorphism-multiple-qualifiers}}

%% I can't think of a non-clumsy way to say this.
% Each method containing a polymorphic qualifier is (conceptually) expanded
% into multiple versions completely independently.

Usually, it does not make sense to write only a single instance of a polymorphic
qualifier in a method definition:  if you write one instance of (say)
\<@PolyNull>, then you should use at least two.  (An exception is a
polymorphic qualifier on an array element type; this section ignores that
case, but see below for further details.)

For example, there is no point to writing

\begin{Verbatim}
  void m(@PolyNull Object obj)
\end{Verbatim}

\noindent
which expands to

\begin{Verbatim}
  void m(@NonNull Object obj)
  void m(@Nullable Object obj)
\end{Verbatim}

This is no different (in terms of which calls to the method will
type-check) than writing just

\begin{Verbatim}
  void m(@Nullable Object obj)
\end{Verbatim}

The benefit of polymorphic qualifiers comes when one is used multiple times
in a method, since then each instance turns into the same type qualifier.
Most frequently, the polymorphic qualifier appears on at least one formal
parameter and also on the return type.  It can also be useful to have
polymorphic qualifiers on (only) multiple formal parameters, especially if
the method side-effects one of its arguments.
For example, consider

\begin{Verbatim}
void moveBetweenStacks(Stack<@PolyNull Object> s1, Stack<@PolyNull Object> s2) {
  s1.push(s2.pop());
}
\end{Verbatim}

\noindent
In this example, if it is acceptable to rewrite your code to use Java
generics, the code can be even cleaner:

\begin{Verbatim}
<T> void moveBetweenStacks(Stack<T> s1, Stack<T> s2) {
  s1.push(s2.pop());
}
\end{Verbatim}


%% It would be nice to give an example that isn't too contrived.


\paragraph{Using a single polymorphic qualifier on an element type\label{qualifier-polymorphism-element-types}}

There is an exception to the general rule that a polymorphic qualifier
should be used multiple times in a signature.  It can make sense to use a
polymorphic qualifier just once, if it is on an array or generic element
type.

For example, consider a routine that returns the first index, in an array
or collection, of a given element:

\begin{Verbatim}
  public static int indexOf(@PolyNull Object[] a, Object elt) { ... }

  public static int indexOf(Collection<@PolyNull Object> a, Object elt) { ... }
\end{Verbatim}

If \<@PolyNull> were replaced with either \<@Nullable> or \<@NonNull>, then
some safe client calls would be rejected.

Of course, it would be better style to use a generic method, as in either
of these signatures (and likewise for the \<Collection> version):

\begin{Verbatim}
 public static <T> int indexOf(T[] a, /*@Nullable*/ Object elt) { ... }
 public static <T> int indexOf(T[] a, T elt) { ... }
\end{Verbatim}

In conclusion, use of a single polymorphic qualifier may be necessary in
legacy code, but can be avoided by use of better code style.


\section{Unused fields and dependent types\label{unused-fields-and-dependent-types}}

Sometimes, the type of a field depends on the qualifier on the receiver.
The Checker Framework supports two varieties of such a field:  a field that
may not be used if the receiver has a given qualifier, and a fields whose
qualifier changes based on the qualifier of the receiver.
(Also see the discussion of typestate checkers, in
Chapter~\ref{typestate-checker}.)


\subsection{Unused fields\label{unused-fields}}

A Java subtype can have more fields than its supertype.  You can simulate
the same effect for type qualifiers:  a given field may not be accessed via
a reference with a supertype qualifier, but can be accessed via a reference
with a subtype qualifier.

This permits you to restrict use of a field to certain contexts.

The \code{@\refclass{quals}{Unused}} annotation
on a field declares that the field may not be accessed via a receiver of
the given qualified type (or any supertype).


\subsection{Dependent types\label{dependent-types}}

A variable has a \emph{dependent type} if its type depends on some other
value or type.
%  --- the type is dynamically, not statically, determined.
% (Type-safety can still be statically determined, though.)

The Checker Framework supports a form of dependent types, via the
\code{@\refclass{quals}{Dependent}} annotation.
This annotation changes the type of a field or variable, based on the
qualified type of the receiver (\code{this}).  This can be viewed as a more
expressive form of polymorphism (see Section~\ref{polymorphism}).  It can
also be seen as a way of linking the meanings of two type qualifier
hierarchies.

When the \code{@\refclass{quals}{Unused}} annotation is sufficient, you
should use it instead of \code{@Dependent}.


\subsection{Example\label{dependent-types-example}}

Suppose we have a class \code{Person} and a field \code{spouse} that is
non-\code{null} if the person is married.  We could declare this as

\begin{Verbatim}
  class Person {
    ...
    // non-null if this person is married
    @Nullable Person spouse;
    ...
  }
\end{Verbatim}

Now, suppose that we have defined the qualifier hierarchy in which
\code{@Single} (meaning ``not married'') is a supertype of \code{@Married}.
A more informative declaration for \<Person> would be

\begin{Verbatim}
  class Person {
    ...
    @Nullable @Dependent(result=NonNull.class, when=Married.class) Person spouse;
    ...
  }
\end{Verbatim}

If a person is known to be \code{@Married}, the
\code{spouse} field is known to be non-\code{null}:

\begin{Verbatim}
  class Person {
    ...

    void celebrateWeddingAnniversary() @Married {
      System.out.println("Happy anniversary, "
                         + spouse.toString()); // no possible null pointer exception
    }

    ...
  }
\end{Verbatim}

\noindent
Without the \code{@\refclass{quals}{Dependent}} annotation on the
declaration of the \code{spouse} variable, the Nullness Checker would
complain that \code{toString} was invoked on a possibly-\code{null}
value.

An even better declaration is

\begin{Verbatim}
  class Person {
    ...
    @Unused(when=Single.class) @NonNull Person spouse;
    ...
  }
\end{Verbatim}

Then, if a person is known to be \code{@Married} (or more
appropriately non-\code{@Single}), the \code{spouse} field is known to
be non-\code{null}.  Also, if a person is known to be \code{@Single},
the \code{spouse} field may not be accessed:

\begin{Verbatim}
  @Single Person person = ...;
  Person spouse = person.spouse;  // invalid field access
  ...
\end{Verbatim}


\section{The effective qualifier on a type (defaults and inference)\label{effective-qualifier}}

A checker sometimes treats a type as having a slightly different qualifier
than what is written on the type --- especially if the programmer wrote no
qualifier at all.
Most readers can skip this section on first reading, because you will
probably find the system simply ``does what you mean'', without forcing
you to write too many qualifiers in your program.
In particular, qualifiers in method bodies are extremely rare.

  The following steps determine the effective
qualifier on a type --- the qualifier that the checkers treat as being present.

\begin{enumerate}
\item
  The type system adds implicit qualifiers.  Implicit qualifiers can be
  built into a type system (Section~\ref{writing-type-introduction}), in
  which case the type system's documentation should explain all of the type
  system's implicit qualifiers.  Or, a programmer may introduce an implicit
  annotation on each use of class $C$ by writing a qualifier on the
  declaration of class $C$.

\begin{itemize}
\item
  Example 1 (built-in):  In the Nullness type system,
  \<enum> values are never null, nor is a method receiver.
\item
  Example 2 (built-in):  In the Interning type system, string literals
  and \<enum> values are always interned.
\end{itemize}

\item
  If a type qualifier is present in the source code, that qualifier is used.

  If the type has an implicit qualifier, then it is an error to write an
  explicit qualifier that is equal to (redundant with) or a supertype of
  (weaker than) the implicit qualifier.  A programmer may strengthen
  (write a subtype of) an implicit qualifier, however.

\item
  If there is no implicit or explicit qualifier on a type, then a default
  qualifier may be applied; see Section~\ref{defaults}.

  \smallskip

  At this point, every type has a qualifier.

\item
  The type system may refine a qualified type on a local variable --- that
  is, treat it as a subtype of how it was declared or defaulted.  This
  refinement is always sound and has the effect of eliminating false
  positive error messages.  See Section~\ref{type-refinement}.

  % Type
  % qualifier refinement is implemented by the \refclass{flow}{Flow} class.

\end{enumerate}



\subsection{Default qualifier for unannotated types\label{defaults}}

A type system designer, or an end-user programmer, can cause unannotated
references to be treated as if they had a default annotation.

There are several defaulting mechanisms, for convenience and flexibility.
When determining the default qualifier for a use of a type, the following
rules are used in order, until one applies.
\begin{itemize}
\item
  Use the innermost user-written \code{@DefaultQualifier}, as explained in
  this section.
\item
  Use the default specified by the type system designer
  (Section~\ref{typesystem-defaults}).
\item
  Use \code{@\refclass{quals}{Unqualified}}, which the framework
  inserts to avoid ambiguity and simplify the programming interface for
  type system designers.  Users do not have to worry about this detail,
  but type system implementers can rely on the fact that some
  qualifier is present.
\end{itemize}

% (Implementation detail:  setting defaults is implemented by the
% \refclass{util}{QualifierDefaults} class.)


The end-user programmer specifies a default qualifier by writing the \code{@\refclass{quals}{DefaultQualifier}}
annotation on a package, class, method, or variable declaration.  The
argument to \<@\refclass{quals}{DefaultQualifier}> is the \code{String}
name of an annotation.  It may be a short name like \code{"NonNull"}, if an
appropriate import statement exists.  Otherwise, it should be
fully-qualified, like \code{"checkers.nullness.quals.NonNull"}.
The optional second argument indicates where the default
applies.  If the second argument is omitted, the specified annotation is
the default in all locations.  See the Javadoc of \refclass{quals}{DefaultQualifier} for details.

For example, using the Nullness type system (Chapter~\ref{nullness-checker}):

\begin{Verbatim}
import checkers.quals.*;        // for DefaultQualifier[s]
import checkers.nullness.quals.NonNull;

@DefaultQualifier("NonNull"),
class MyClass {

  public boolean compile(File myFile) { // myFile has type "@NonNull File"
    if (!myFile.exists())          // no warning: myFile is non-null
      return false;
    @Nullable File srcPath = ...;  // must annotate to specify "@Nullable File"
    ...
    if (srcPath.exists())          // warning: srcPath might be null
      ...
  }

  @DefaultQualifier("Mutable")
  public boolean isJavaFile(File myfile) {  // myFile has type "@Mutable File"
    ...
  }
}
\end{Verbatim}

If you wish to write multiple
\<@\refclass{quals}{DefaultQualifier}> annotations at a single location,
use
\<@\refclass{quals}{DefaultQualifiers}> instead.  For example:

\begin{Verbatim}
@DefaultQualifiers({
  @DefaultQualifier("NonNull"),
  @DefaultQualifier("Mutable")
})
\end{Verbatim}


If \code{@DefaultQualifier}[\code{s}] is placed on a package (via the
\<package-info.java> file), then it applies to the given package \emph{and}
all subpackages.
% This is slightly at odds with Java's treatment of packages of different
% names as essentially unrelated, but is more intuitive and useful.

Recall that an annotation on a class definition indicates an implicit
qualifier (Section~\ref{effective-qualifier}) that can only be
strengthened, not weakened.  This can lead to unexpected results if
the default qualifier applies to a class definition.  Thus, you may want to
put explicit qualifiers on class declarations (which prevents the default
from taking effect), or exclude class declarations from defaulting.


%% Don't even bother to bring this up; it will just sow confusion without
%% being helpful.
% For some type systems, a user may not specify a default qualifier, or doing
% so prevents giving any other qualifier to any reference.  This is a
% consequence of the design of the type system; see
% Section~\ref{bottom-qualifier}.


When a programmer omits an \<extends> clause at a declaration of a type
parameter, the default still applies to the implicit upper bound.  For
example, consider these two declarations:

\begin{Verbatim}
  class C<T> { ... }
  class C<T extends Object> { ... }  // identical to previous line
\end{Verbatim}

\noindent
The two declarations are treated identically by Java, and the default
qualifier applies to the \<Object> upper bound whether it is implicit or
explicit.  (The @NonNull default annotation applies only to the upper bound
in the \<extends> clause, not to the lower bound in the inexpressible
implicit \<super void> clause.)


\subsection{Automatic type refinement (flow-sensitive type qualifier inference)\label{type-refinement}}

In order to reduce the burden of annotating types in your program, the
checkers soundly treat certain variables and expressions as having a
subtype of their declared or defaulted (Section~\ref{defaults})
type.  This functionality
never introduces unsoundness or causes an error to be missed:  it merely
suppresses false positive warnings.

By default, all checkers, including new checkers that you write, can take
advantage of this functionality.  Most of the time, users don't have to
think about, and may not even notice, this feature of the framework.  The
checkers simply do the right thing even when a programmer forgets an
annotation on a local variable, or when a programmers writes an
unnecessarily general type in a declaration.

If you are curious or want more details about this feature, then read on.

As an example, the Nullness checker (Chapter~\ref{nullness-checker}) can automatically
determine that certain variables are non-null, even if they were explicitly
or by default annotated as nullable.
The checker treats a variable or expression as \code{@\refclass{nullness/quals}{NonNull}}
\begin{itemize}
\item
starting at the time that it is either
assigned a non-null value or checked against null (e.g., via an assertion,
\code{if} statement, or being dereferenced)
\item
until it might be re-assigned (e.g.,
via an assignment that might affect this variable, or via a method call
that might affect this variable).
\end{itemize}

As with explicit annotations, the implicitly non-null types permit
dereferences and assignments to non-null types, without
compiler warnings.

Consider this code, along with comments indicating whether the
Nullness checker (Chapter~\ref{nullness-checker}) issues a warning.  Note that the same expression may yield a
warning or not depending on its context.

\begin{Verbatim}
  // Requires an argument of type @NonNull String
  void parse(@NonNull String toParse) { ... }

  // Argument does NOT have a @NonNull type
  void lex(@Nullable String toLex) {
    parse(toLex);        // warning:  toLex might be null
    if (toLex != null) {
      parse(toLex);      // no warning:  toLex is known to be non-null
    }
    parse(toLex);        // warning:  toLex might be null
    toLex = new String(...);
    parse(toLex);        // no warning:  toLex is known to be non-null
  }
\end{Verbatim}

If you find examples where you think a value should be inferred to have
(or not have) a
given annotation, but the checker does not do so, please submit a bug
report (see Section~\ref{reporting-bugs}) that includes a small piece of
Java code that reproduces the problem.

% Flow-sensitive non-null inference has been implemented for the following
% varieties of expressions:
%
% \begin{itemize}
% \item null checks in if/else statements
% \item null checks in assert statements
% \item null checks that result in a return or thrown exception, or call System.exit
% \item assignments from new class/array expressions
% \end{itemize}
%
% \emph{Note:} The items in the above list exclude complex null checks, i.e., not
% of the form \code{x != null}. Support for these types of checks will be available in a
% future release.


% TODO:  Is NonNull inferred for any parameters or fields, or just for locals?

Type inference is never performed for method parameters of non-private
methods and for non-private fields, because unknown client code could use
them in arbitrary ways.  The inferred information is never written to the
\code{.class} file as user-written annotations are.

The inference indicates when a variable can be treated as having a subtype
of its declared type --- for instance, when an otherwise nullable type can be
treated as a \code{@\refclass{nullness/quals}{NonNull}} one.  The inference never treats a variable as
a supertype of its declared type (e.g., an expression of \code{@\refclass{nullness/quals}{NonNull}}
type is never inferred to be treated as possibly-null).

\subsection{Fields and flow sensitivity analysis}

Flow sensitivity analysis infers the type of fields in some restricted cases:

\begin{itemize}

\item
A final initialized field:
Type inference is performed for final fields that are initialized to a
compile-time constant at the declaration site; so the type of \code{protocol}
is \code{@NonNull String} in the following declaration:

\begin{Verbatim}
    public final String protocol = "https";
\end{Verbatim}

Please note that such inferred type may leak to the public interface of the
class.  To override such behavior, you can explicitly insert the desired
annotation, e.g.

\begin{Verbatim}
    public final @Nullable String protocol = "https";
\end{Verbatim}

\item
Within method bodies:
Type inference is performed for fields in the context of method bodies,
like local variables, but method invocations invalidate any inferred
information.  Consider the following example, where \code{name} is a nullable
field:

\begin{Verbatim}
class DBObject {
  @Nullable Date updatedAt;

  void update() {
    if (updatedAt == null)
        updatedAt = new Date();
    // updatedAt is nonnull
    log("Updating object at " + updatedAt.getTime());

    persistData();
    // updatedAt is nullable again
    log.debug("Saved object updated at " + updatedAt.getTime()); // invalid!
  }
}
\end{Verbatim}

Here the call to \code{persistData()} invalidates the inferred non-null type
of \code{updatedAt}.

When methods do not modify any object state or have any identity side-effects
(e.g. \code{log()} method here), you can annotate these methods as
\code{Pure}.  Annotating them as \code{Pure}, would cause the flow analyzer to
carry the inferred types across the method invocation boundary.

\end{itemize}


\subsection{Inherited defaults}

In certain situations, it would be convenient for an annotation on a
superclass member to be automatically inherited by subclasses that override
it.  This feature would reduce both annotation effort and program
comprehensibility.  In general, a program is read more often than it is
edited/annotated, so the Checker Framework does not currently support this
feature.  Here are more detailed justifications:

\begin{itemize}

\item
  Currently, a user can determine the annotation on a parameter or return
  value by looking at a single file.  If annotations could be inherited
  from supertypes, then a user would have to examine all supertypes to
  understand the meaning of an unannotated type in a given file.

\item
  Different annotations might be inherited from a supertype and an
  interface, or from two interfaces.  Presumably, the subtype's annotations
  would be stronger than either (the greatest lower bound in the type
  system), or an error would be thrown if no such annotations existed.

\end{itemize}

If these issues can be resolved, then the feature may be added in the
future.  Or, it may be added optionally, and each type-checker
implementation can enable it if desired.


\section{Inexpressible types\label{inexpressible-types}}

The Type Annotations syntax~\cite{jsr308} is designed to be easy to read.  As a result,
there are types that it cannot express.  An example is the type of
\<Collection.toArray()>, which returns an array of objects, where the
objects have the same annotation as the elements of the receiver.

A possible annotation would be

\begin{Verbatim}
public @Polynull Object [] toArray() ArrayList<@PolyNull E> { ... }
\end{Verbatim}

\noindent
except that this is illegal syntax:  ``\code{ArrayList<@PolyNull E>}'' is
not legal in the receiver position.  (This is a motivation for
\ahref{http://types.cs.washington.edu/jsr308/specification/java-annotation-design.html#receiver-type-parameter-annotations}{extending}
the Type Annotations syntax.)

The annotated libraries (Section~\ref{annotating-libraries}) contain a less-precise annotation for
\code{toArray}.  The Nullness Checker special-cases \code{toArray} to
act as if it had the above annotation.  The cases that
are currently being handled are described in
\refclass{nullness}{CollectionToArrayHeuristics}.
This approach would be possible for other checkers and other methods as the
need arises.


% LocalWords:  MyClass quals PolymorphicQualifier DefaultQualifier subpackages
% LocalWords:  DefaultQualifiers actuals toArray CollectionToArrayHeuristics nn
% LocalWords:  MyList Nullness DefaultLocation nullness PolyNull util java TODO
% LocalWords:  QualifierDefaults nullable lub persistData updatedAt nble
