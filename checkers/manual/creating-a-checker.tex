\htmlhr
\section{How to create a new checker\label{writing-a-checker}}

\newcommand{\TreeAPIBase}{http://java.sun.com/javase/6/docs/jdk/api/javac/tree/com/sun/source}
\newcommand{\refTreeclass}[2]{\ahref{\TreeAPIBase{}/#1/#2.html?is-external=true}{\<#2>}}
\newcommand{\ModelAPIBase}{http://java.sun.com/javase/6/docs/api/javax/lang/model}
\newcommand{\refModelclass}[2]{\ahref{\ModelAPIBase{}/#1/#2.html?is-external=true}{\<#2>}}

This section describes how to extend the Checker Framework to create a checker
--- a type-checking compiler plugin that detects bugs or verifies their
absence.  After a programmer annotates a program,
the checker plugin verifies that the code is consistent
with the annotations.
If you only want to \emph{use} a checker, you do not need to read this
section.

Writing a simple checker is easy!  For example, here is a complete, useful
type checker:

\begin{Verbatim}
@TypeQualifier
@SubtypeOf(Unqualified.class)
public @interface Encrypted {}
\end{Verbatim}

This checker is so short because it builds on the Basic Checker
(Section~\ref{basic-checker}).
See Section~\ref{basic-example} for more details about this particular checker.

You can also customize a typestate checker.
Two of these are available.  One is by Adam Warski:  
\myurl{http://www.warski.org/typestate.html}.
The other is by Daniel Wand:
\myurl{http://typestate.ewand.de/}.


The rest of this section contains many details for people who want to more write powerful
checkers.
You won't need all of the details, at least at first.
In addition to reading this section of the manual, you may find it helpful
to examine the implementations of the checkers that are distributed with
the Checker Framework, or to create your checker by modifying another one.
The Javadoc documentation of the framework and the checkers is in the
distribution and is also available online at
\myurl{http://types.cs.washington.edu/checker-framework/current/doc/}.

If you write a new checker, let us know so we can mention it here, link to
it from our webpages, or include it in the Checker Framework distribution.


\subsection{The parts of a checker\label{parts-of-a-checker}}

The Checker Framework provides abstract base classes (default
implementations), and a specific checker overrides as little or as much of
the default implementations as necessary.
%
Sections~\ref{writing-typequals}--\ref{writing-compiler-interface} describe
the components of a type system as written using the Checker Framework:

\begin{description}

\item{\ref{writing-typequals}}
  \textbf{Type qualifiers and hierarchy.}  You define the annotations for
  the type system and the subtyping relationships among qualified types
  (for instance, that \<@NonNull Object> is a subtype of \<@Nullable
  Object>).

\item{\ref{writing-type-introduction}}
  \textbf{Type introduction rules.}  For some types and
  expressions, a qualifier should be treated as present even if a
  programmer did not explicitly write it.  For example, in the Nullness
  type system every literal
  other than \<null> has a \<@\refclass{nullness/quals}{NonNull}> type; examples of literals include \<"some
  string"> and \<java.util.Date.class>.

\item{\ref{extending-visitor}}
  \textbf{Type rules.}  You specify the the type system semantics (type
  rules), violation of which yields a type error.  There are two types of
  rules.  Your checker automatically inherits rules related to the type
  hierarchy, such as that every assignment and
  pseudo-assignment satisfies a subtyping relationship.  You write any
  additional rules.  For example, in the
  Nullness type system, only references with a \<@\refclass{nullness/quals}{NonNull}> type may be
  dereferenced.

\item{\ref{writing-compiler-interface}}
  \textbf{Interface to the compiler.}  The compiler interface indicates
  which annotations are part of the type system, which command-line options
  and \<@SuppressWarnings> annotations the checker recognizes, etc.
\end{description}


\subsection{Annotations: Type qualifiers and hierarchy\label{writing-typequals}}

A type system designer specifies the qualifiers in the type system and
the type hierarchy that relates them.

%% True, but seems irrelevant here, so it detracts from the message.
% Each qualifier restricts the values that
% a type can represent.  For example \<@NonNull String> type can only
% represent non-null values, indicating that the variable may not hold
% \<null> values.

Type qualifiers are defined as Java annotations~\cite{JSR269}.  In Java, an
annotation is defined using the Java \code{@interface} keyword.
Write the \<@\refclass{quals}{TypeQualifier}> annotation on the annotation definition
to indicate that the annotation represents a type qualifier (e.g.,
\<@\refclass{nullness/quals}{NonNull}> or \<@\refclass{interning/quals}{Interned}>) and should be processed by the checker.  For example:

\begin{Verbatim}
  // Define an annotation for the @NonNull type qualifier.
  @TypeQualifier
  public @interface NonNull { }
\end{Verbatim}

\noindent
(An annotation that is written on an annotation
definition, such as \<@\refclass{quals}{TypeQualifier}>, is called a \emph{meta-annotation}.)

% \noindent
% The \<@TypeQualifier> meta-annotation
% distinguishes it from an ordinary
% annotation that applies to a declaration (e.g., \<@Deprecated> or
% \<@Override>).
% The framework ignores any annotation whose
% declaration does not bear the \<@TypeQualifier> meta-annotation (with minor
% exceptions, such as \<@SuppressWarnings>).

The type hierarchy induced by the qualifiers can be defined either
declaratively via meta-annotations (Section~\ref{declarative-hierarchy}), or procedurally through
subclassing \refclass{types}{QualifierHierarchy} or
\refclass{types}{TypeHierarchy} (Section~\ref{procedural-hierarchy}).


\subsubsection{Declaratively defining the qualifier and type hierarchy\label{declarative-hierarchy}}

Declaratively, the type system designer uses two meta-annotations (written
on the declaration of qualifier annotations) to specify the qualifier
hierarchy.

\begin{itemize}

\item \code{@\refclass{quals}{SubtypeOf}} denotes that a qualifier is the subtype of
  another qualifier or qualifiers, specified as an array of class
  literals.  For example, for any type $T$,
  \code{@\refclass{nullness/quals}{NonNull}} $T$ is a subtype of \code{@\refclass{nullness/quals}{Nullable}} $T$:

  \begin{Verbatim}
    @TypeQualifier
    @SubtypeOf( { Nullable.class } )
    public @interface NonNull { }
  \end{Verbatim}

  (The actual definition of \refclass{nullness/quals}{NonNull} is slightly more complex.)


  %% True, but a distraction.  Move to Javadoc?
  % (It would be more natural to use Java subtyping among the qualifier
  % annotations, but Java forbids annotations from subtyping one another.)
  %
  \<@\refclass{quals}{SubtypeOf}> accepts multiple annotation classes as an argument,
  permitting the type hierarchy to be an arbitrary DAG\@.  For example,
  in the IGJ type system (Section~\ref{igj-annotations}), \<@\refclass{igj/quals}{Mutable}>
  and \<@\refclass{igj/quals}{Immutable}> induce two mutually exclusive subtypes of the
  \<@\refclass{igj/quals}{ReadOnly}> qualifier.

  As a special case, the root qualifier needs to be annotated with
  \<@Subtype( \{ \} )>.  The root qualifier is the qualifier that is
  a supertype of all other qualifiers.  \refclass{nullness/quals}{Nullable}
  is the root of the Nullness type system, hence is defined as:

  \begin{Verbatim}
    @TypeQualifier
    @SubtypeOf( { } )
    public @interface Nullable { }
  \end{Verbatim}

  All type qualifiers, except for polymorphic qualifiers, need to be
  properly annotated with \refclass{quals}{SubtypeOf}.

  If the root of the hierarchy is the unqualified type, then its children
  will use \code{@SubtypeOf(Unqualified.class)}, but no \code{@SubtypeOf(
    \{ \} )} annotation on the root is necessary.  For an example, see the
  \<Encrypted> type system of Section~\ref{writing-a-checker}.

\item \<@\refclass{quals}{PolymorphicQualifier}> denotes that a qualifier is a
  polymorphic qualifier.  For example:

  \begin{Verbatim}
    @TypeQualifier
    @PolymorphicQualifier
    public @interface PolyNull { }
  \end{Verbatim}

  For a description of polymorphic qualifiers, see
  Section~\ref{qualifier-polymorphism}.  A polymorphic qualifier needs
  no \<@\refclass{quals}{SubtypeOf}> meta-annotation and need not be
  mentioned in any other \<@\refclass{quals}{SubtypeOf}>
  meta-annotation.

\end{itemize}

\urldef{\isSubtypeURL}\url{doc/checkers/basetype/BaseTypeChecker.html#isSubtype(checkers.types.AnnotatedTypeMirror,%20checkers.types.AnnotatedTypeMirror)}

The declarative and procedural mechanisms for specifying the hierarchy can
be used together.  In particular, when using the \<@\refclass{quals}{SubtypeOf}>
meta-annotation, further customizations may be
performed procedurally (Section~\ref{procedural-hierarchy})
by overriding the \ahref{\isSubtypeURL}{\code{isSubtype}} method in the checker class
(Section~\ref{writing-compiler-interface}).
However, the declarative mechanism is sufficient for most type systems.


\subsubsection{Procedurally defining the qualifier and type hierarchy\label{procedural-hierarchy}}

\urldef{\createQualifierHierarchyURL}\url{doc/checkers/basetype/BaseTypeChecker.html#createQualifierHierarchy()}
\urldef{\createTypeHierarchyURL}\url{doc/checkers/basetype/BaseTypeChecker.html#createTypeHierarchy()}

While the declarative syntax suffices for many cases, more complex
type hierarchies can be expressed by overriding, in \refclass{basetype}{BaseTypeChecker},
either \ahref{\createQualifierHierarchyURL}{\<createQualifierHierarchy>} or \ahref{\createTypeHierarchyURL}{\<createTypeHierarchy>} (typically
only one of these needs to be overridden).
For more details, see the Javadoc of those methods and of the classes
\refclass{types}{QualifierHierarchy} and \refclass{types}{TypeHierarchy}.

The \refclass{types}{QualifierHierarchy} class represents the qualifier hierarchy (not the
type hierarchy), e.g., \refclass{igj/quals}{Mutable}
is a subtype of \refclass{igj/quals}{ReadOnly}.  A type-system designer may subclass
\refclass{types}{QualifierHierarchy} to express customized qualifier
relationships (e.g., relationships based on annotation
arguments).

The \refclass{types}{TypeHierarchy} class represents relationships between
annotated types, rather than merely type qualifiers, e.g., \<@Mutable
Date> is a subtype of \<@ReadOnly Date>.  The default \refclass{types}{TypeHierarchy} uses
\refclass{types}{QualifierHierarchy} to determine all subtyping relationships.
The default \refclass{types}{TypeHierarchy} handles
generic type arguments, array components, type variables, and
wild-cards in a similar manner to the Java standard subtype
relationship but with taking qualifiers into consideration.  Some type
systems may need to override that behavior.  For instance, the Java
Language Specification specifies that two generic types are subtypes only
if their type arguments are identical:  for example,
\code{List<Date>} is not a subtype of \code{List<Object>}, or of any other
generic \code{List}.
(In the technical jargon, the generic arguments are ``invariant''.)
The Javari type system overrides this
behavior to allow some type arguments to change covariantly in a type-safe
manner (e.g.,
\code{List<@Mutable Date>} is a subtype of \code{List<@QReadOnly Date>}).


\subsubsection{Defining a default annotation\label{typesystem-defaults}}

% This paragraph is out of place.

\urldef{\setAbsoluteDefaultsURL}\url{doc/checkers/util/QualifierDefaults.html#setAbsoluteDefaults(javax.lang.model.element.AnnotationMirror,%20java.util.Set)}

A type system designer may set a default annotation.  A user may override
the default; see Section~\ref{defaults}.

The type system designer may specify a default annotation declaratively,
using the \code{@\refclass{quals}{DefaultQualifierInHierarchy}}
meta-annotation.
Note that the default will apply to any source code that the checker reads,
including stub libraries, but will not apply to compiled \code{.class}
files that the checker reads.

Alternately, the type system designer may specify a default procedurally,
by calling the
\ahref{\setAbsoluteDefaultsURL}{\<QualifierDefaults.setAbsoluteDefaults>}
method.  You may do this even if you have declaratively defined the
qualifier hierarchy; see the Nullness checker's implementation for an
example.


Recall that defaults are distinct
from implicit annotations; see Sections~\ref{effective-qualifier}
and~\ref{writing-type-introduction}.


\subsubsection{Bottom qualifier\label{bottom-qualifier}}

It is usually a good idea to have a bottom qualifier in your type hierarchy
--- a qualifier that is a (direct or indirect) subtype of every other
qualifier.  The reason is that this is the natural type for the \code{null}
value, which is can be viewed as having any type at all.

Users should never write the bottom qualifier explicitly; it is merely used
for the \code{null} value.

You might write the bottom qualifier like this:

\begin{Verbatim}
  package myTypeQuals;

  import checkers.quals.*;
  import com.sun.source.tree.Tree;

  @TypeQualifier
  @SubtypeOf({Prototype.class, NonPrototype.class})
  @ImplicitFor(trees={Tree.Kind.NULL_LITERAL})
  public @interface PrototypeBottom {}
\end{Verbatim}



\subsection{Type Factory: Implicit annotations\label{writing-type-introduction}}

For some types and expressions, a qualifier should be treated as present
even if a programmer did not explicitly write it.  For example, every
literal (other than \<null>) has a \<@\refclass{nullness/quals}{NonNull}> type.

The implicit annotations may be specified declaratively and/or procedurally.


\subsubsection{Declaratively specifying implicit annotations}

The \<@\refclass{quals}{ImplicitFor}> meta-annotation indicates implicit annotations.
When written on a qualifier, \refclass{quals}{ImplicitFor}
specifies the trees (AST nodes) and types for which the framework should
automatically add that qualifier.

In short, the types and trees can be
specified via any combination of five fields:

  \begin{itemize}
  \item \code{trees}: an array of
    \ahref{\TreeAPIBase{}/tree/Tree.Kind.html?is-external=true}{\code{com.sun.source.tree.Tree.Kind}}, e.g.,
    \code{NEW\_ARRAY} or \code{METHOD\_INVOCATION}
  \item \code{types}: an array of
    \refModelclass{type}{TypeKind}, e.g., \code{ARRAY}
    or \code{BOOLEAN}
  \item \code{treeClasses}: an array of class literals for classes
    implementing \refTreeclass{tree}{Tree}, e.g.,
    \code{LiteralTree.class} or \code{ExpressionTree.class}
  \item \code{typeClasses}: an array of class literals for classes
    implementing \code{javax.lang.model.type.TypeMirror}, e.g.,
    \code{javax.lang.model.type.PrimitiveType}.  Often you should use
    a subclass of \refclass{types}{AnnotatedTypeMirror}
  \item \code{stringPatterns}: an array of regular expressions that will
    be matched against
    string literals, e.g., \code{"[01]+"} for a binary number.  Useful
    for annotations that indicate the format of a string.
  \end{itemize}

For example, consider the definitions of the \<@\refclass{nullness/quals}{NonNull}> and \<@\refclass{nullness/quals}{Nullable}>
type qualifiers:

%BEGIN LATEX
\begin{smaller}
%END LATEX
\begin{Verbatim}
  @TypeQualifier
  @SubtypeOf( { Nullable.class } )
  @ImplicitFor(
    types={TypeKind.PACKAGE},
    typeClasses={AnnotatedPrimitiveType.class},
    trees={
      Tree.Kind.NEW_CLASS,
      Tree.Kind.NEW_ARRAY,
      Tree.Kind.PLUS,
      // All literals except NULL_LITERAL:
      Tree.Kind.BOOLEAN_LITERAL, Tree.Kind.CHAR_LITERAL, Tree.Kind.DOUBLE_LITERAL, Tree.Kind.FLOAT_LITERAL,
      Tree.Kind.INT_LITERAL, Tree.Kind.LONG_LITERAL, Tree.Kind.STRING_LITERAL
    })
  public @interface NonNull {  }


  @TypeQualifier
  @SubtypeOf({})
  @ImplicitFor(trees={Tree.Kind.NULL_LITERAL})
  public @interface Nullable { }
\end{Verbatim}
%BEGIN LATEX
\end{smaller}
%END LATEX

For more details, see the Javadoc for the \refclass{quals}{ImplicitFor}
  annotation, and the Javadoc for the javac classes that are linked from
it.  (You only need to understand a small amount about the javac AST, such
as the
\ahref{\TreeAPIBase{}/tree/Tree.Kind.html?is-external=true}{\code{Tree.Kind}}
and
\refModelclass{type}{TypeKind}
enums.  All the information you need is in the Javadoc, and
Section~\ref{javac-tips} can help you get started.)


\subsubsection{Procedurally specifying implicit annotations}


The Checker Framework provides a representation of annotated types,
\refclass{types}{AnnotatedTypeMirror}, that extends the standard \<TypeMirror>
interface but integrates a representation of the annotations into a
type representation.  A checker's \emph{type factory} class, given an AST
node, returns the annotated type of that expression.  The Checker
Framework's abstract
\emph{base type factory} class, \refclass{types}{AnnotatedTypeFactory},
supplies a uniform, Tree-API-based interface
for querying the annotations on a program element, regardless of
whether that element is declared in a source file or in a class file.
It also handles default annotations, and it optionally performs
flow-sensitive local type inference.

\refclass{types}{AnnotatedTypeFactory} inserts the qualifiers that the programmer
explicitly inserted in the code.  Yet, certain constructs should be
treated as having a type qualifier even when the programmer has not
written one.  The type system designer may subclass
\refclass{types}{AnnotatedTypeFactory} and override
\code{annotateImplicit(Tree,AnnotatedTypeMirror)} and
\code{annotateImplicit(Element,AnnotatedTypeMirror)} to account for
such constructs.


\subsection{Visitor: Type rules\label{extending-visitor}}

A type system's rules define which operations on values of a
particular type are forbidden.

The framework provides a \textit{base visitor class},
\refclass{basetype}{BaseTypeVisitor}, that performs type-checking at each node of a
source file's AST\@.  It uses the visitor design pattern to traverse
Java syntax trees as provided by Sun's
\ahref{http://java.sun.com/javase/6/docs/jdk/api/javac/tree/index.html}{Tree
API},
and issues a warning whenever the type system induced by the type
qualifier is violated.

A checker's visitor overrides one method in the base visitor for each special
rule in the type qualifier system.  Most type-checkers
override only a few methods in \refclass{basetype}{BaseTypeVisitor}.  For example, the
visitor for the Nullness type system of Section~\ref{nullness-checker} consists
of a single 4-line method that warns if an expression of nullable type
is dereferenced, as in:
\begin{Verbatim}
  myObject.hashCode();  // invalid dereference
\end{Verbatim}



By default, \refclass{basetype}{BaseTypeVisitor} performs subtyping checks that are
similar to Java subtype rules, but taking the type qualifiers into account.
\refclass{basetype}{BaseTypeVisitor} issues these errors:

\begin{itemize}

\item invalid assignment (type.incompatible) when an assignment from
  an expression type to an incompatible type.  The assignment may be a
  simple assignment, or pseudo-assignment like return expressions or
  argument passing in a method invocation

  In particular, in every assignment and pseudo-assignment, the
  left-hand side of the assignment is a supertype of (or the same type
  as) the right-hand side.  For example, this assignment is not
  permitted:

  \begin{Verbatim}
    @Nullable Object myObject;
    @NonNull Object myNonNullObject;
    ...
    myNonNullObject = myObject;  // invalid assignment
  \end{Verbatim}

\item invalid generic argument (generic.argument.invalid) when a type
  is bound to an incompatible generic type variable

\item invalid method invocation (method.invocation.invalid) when a
  method is invoked on an object whose type is incompatible with the
  method receiver type

\item invalid overriding parameter type (override.parameter.invalid)
  when a parameter in a method declaration is incompatible with that
  parameter in the overridden method's declaration

\item invalid overriding return type (override.return.invalid) when a
  parameter in a method declaration is incompatible with that
  parameter in the overridden method's declaration

\item invalid overriding receiver type (override.receiver.invalid)
  when a receiver in a method declaration is incompatible with that
  receiver in the overridden method's declaration

\end{itemize}


\subsection{The checker class:  Compiler interface\label{writing-compiler-interface}}

A checker's entry point is a subclass of \refclass{basetype}{BaseTypeChecker}.  This entry
point, which we call the checker class, serves two
roles:  an interface to the compiler and a factory for constructing
type-system classes.

Because the Checker Framework provides reasonable defaults, oftentimes the
checker class has no work to do.  Here are the complete definitions of the
checker classes for the Interning and Nullness checkers:

\begin{Verbatim}
  @TypeQualifiers({ Interned.class, PolyInterned.class })
  @SupportedLintOptions({"dotequals"})
  public final class InterningChecker extends BaseTypeChecker { }

  @TypeQualifiers({ Nullable.class, Raw.class, NonNull.class, PolyNull.class })
  @SupportedLintOptions({"flow", "cast", "cast:redundant"})
  public class NullnessChecker extends BaseTypeChecker { }
\end{Verbatim}


\urldef{\getSupportedTypeQualifiersURL}\url{http://types.cs.washington.edu/checker-framework/current/doc/checkers/basetype/BaseTypeChecker.html#getSupportedTypeQualifiers()}

The checker class must be annotated by
\code{@\refclass{quals}{TypeQualifiers}}, which lists the annotations
that make up the type hierarchy for this checker (including
polymorphic qualifiers), provided as an array of class literals.  Each
one is a type qualifier whose definition bears the
\code{@\refclass{quals}{TypeQualifier}} meta-annotation (or is
returned by the
\ahref{\getSupportedTypeQualifiersURL}{\<BaseTypeChecker\-.getSupportedTypeQualifiers>}
method).

\urldef{\reportURL}\url{http://types.cs.washington.edu/checker-framework/current/doc/checkers/source/SourceChecker.html#report(checkers.source.Result,%20java.lang.Object)}

The checker class bridges between the compiler and the checker plugin.  It
invokes the type-rule check visitor on every Java source file being
compiler, and provides a simple API, \ahref{\reportURL}{\<report>}, to issue
errors using the compiler error reporting mechanism.

Also, the checker class follows the factory method pattern to
construct the concrete classes (e.g., visitor, factory) and annotation
hierarchy representation.  It is a convention that, for
a type system named Foo, the compiler
interface (checker), the visitor, and the annotated type factory are
named as \<FooChecker>, \<FooVisitor>, and \<FooAnnotatedTypeFactory>.
\refclass{basetype}{BaseTypeChecker} uses the convention to
reflectively construct the components.  Otherwise, the checker writer
must specify the component classes for construction.

A checker can customize the default error messages through a
\sunjavadoc{java/util/Properties.html}{Properties}-loadable text file named
\<messages.properties> that appears in the same directory as the checker class.
The property file keys are the strings passed to \ahref{\reportURL}{\<report>}
(like \code{type.incompatible}) and the values are the strings to be
printed (\code{cannot assign ...}).
The \<messages.properties> file only need to mention the new messages that
the checker defines.
It is also allowed to override messages defined in superclasses, but this
is rarely needed.

\subsubsection{Bundling multiple checkers}

% TODO: Have better explanation on why one would want to bundle checkers
% and why passing checkers in commandline is error-prone

Users need to specify the checker class name in command line
\<-processor> flag to invoke each checker.  When multiple related
checkers need to be run together as a unit, users will have to pass
each checker class name, like:

\begin{Verbatim}
  javac -processor DistanceUnitChecker -processor SpeedUnitChecker ... files ...
\end{Verbatim}

Alternatively, an aggregate checker class is declared to combine these
multiple checkers.  \refclass{util}{AggregateChecker} forms a
convenient base class for such situation, where the checkers can be
declared in one method, like the following:

\begin{Verbatim}
  public class UnitCheckers extends AggregateChecker {
    protected abstract Collection<Class<? extends SourceChecker>>
    getSupportedCheckers() {
      return Arrays.asList(DistanceUnitChecker.class, SpeedUnitChecker);
    }
  }
\end{Verbatim}

Now, users can simply pass \<UnitCheckers> a single argument to the commandline:

\begin{Verbatim}
  javac -processor UnitCheckers ... files ...
\end{Verbatim}

\subsection{Testing framework\label{testing-framework}}

[This section should discuss the testing framework that is used for
testing the distributed checkers.]


\subsection{Debugging options\label{debugging-options}}

The Checker Framework provides debugging options that can be helpful when
writing checker. These are provided via the standard \code{javac} ``\code{-A}''
switch, which is used to pass options to an annotation processor.

\begin{itemize}

\item \code{-Anomsgtext}: use message keys (such as ``\code{type.invalid}'')
rather than full message text when reporting errors or warnings

\item \code{-Ashowchecks}: print debugging information for each
pseudo-assignment check (as performed by \refclass{basetype}{BaseTypeVisitor}; see Section
\ref{extending-visitor} above)

\item \code{-Afilenames}: prints the name of each file before type-checking it

\end{itemize}

The following example demonstrates how these options are used:

%BEGIN LATEX
\begin{smaller}
%END LATEX
\begin{Verbatim}
$ javac -processor checkers.interning.InterningChecker \
    examples/InternedExampleWithWarnings.java -Ashowchecks -Anomsgtext -Afilenames

[InterningChecker] InterningExampleWithWarnings.java
 success (line  18): STRING_LITERAL "foo"
     actual: DECLARED @checkers.interning.quals.Interned java.lang.String
   expected: DECLARED @checkers.interning.quals.Interned java.lang.String
 success (line  19): NEW_CLASS new String("bar")
     actual: DECLARED java.lang.String
   expected: DECLARED java.lang.String
examples/InterningExampleWithWarnings.java:21: (not.interned)
    if (foo == bar)
            ^
 success (line  22): STRING_LITERAL "foo == bar"
     actual: DECLARED @checkers.interning.quals.Interned java.lang.String
   expected: DECLARED java.lang.String
1 error
\end{Verbatim}
%BEGIN LATEX
\end{smaller}
%END LATEX

You can use any standard debugger to observe the execution of your checker.
Set the main class to \code{com.sun.tools.javac.Main} and the bootclasspath
to include the JSR308 langtools.


%% Not relevant to most readers.  Can go in a README file in our repository.
% \subsection{Putting your checker in the repository\label{writing-repository}}
%
% This section is relevant only if you wish to add your checker to the source code
% repository for the Checker Framework --- for example, to include your
% checker in the Checker Framework distribution.
%
% The checkers appear in directory \code{annotations/checkers/} of
% the \code{annotations} repository.  It contains the following relevant
% subdirectories:
% \begin{itemize}
% \item
%   \code{manual/}: Documentation for your checker goes here.
% \item
%   \code{src/checkers/\emph{annotation\_name}/}: Code for the checker,
%   in a directory that is a sibling of of \code{quals/}, \code{nonnull/},
%   etc.
% \item
%   \code{jdk/\emph{annotation\_name}/}: Annotated ``skeleton class''
%   versions of the JDK and other libraries (see Section~\ref{skeleton}).
% \item
%   \code{tests/\emph{annotation\_name}/}: Inputs and outputs for the test
%   suite for the checker.  A single top-level test suite class goes in
%   \code{tests/src/tests/}.
% \end{itemize}


\subsection{javac implementation survival guide\label{javac-tips}}

The implementation of Sun's javac compiler can be a bit daunting to a
newcomer, and its documentation does not particularly help a newcomer to
get oriented.  But do not lose heart!
This section helps you to understand the small part of javac
that you need in order to write a checker.
Other useful resources include the Java Infrastructure Developer's guide at
\url{http://wiki.netbeans.org/Java_DevelopersGuide} and the
compiler mailing list archives at
\url{http://news.gmane.org/gmane.comp.java.openjdk.compiler.devel}
(subscribe at \url{http://mail.openjdk.java.net/mailman/listinfo/compiler-dev}).


The Checker Framework uses Sun's Tree API to access a program's AST\@.
This is specific to the Sun JDK\@.  In the future, the Checker Framework
can be migrated to use the Java Model AST of JSR 198 (Extension API for
Integrated Development Environments)~\cite{JSR198}, which gives access to
the entire source code of a method in an implementation-neutral way.


A \refTreeclass{tree}{Tree} is an AST node; it represents an arbitrary code snippet such
as a method definition, a block, a statement, etc.

The \<Tree> interface has many subinterfaces, that specify what
kind of node is being handled. Trees are usually processed by a class
implementing the
\refTreeclass{tree}{TreeVisitor}
interface, through the \<accept> method on \<Tree>. Common
implementations of \<TreeVisitor> that you may want to extend are
\refTreeclass{util}{SimpleTreeVisitor},
that visits a single node based on its type,
\refTreeclass{util}{TreeScanner},
that visits all subnodes recursively, and
\refTreeclass{util}{TreePathScanner},
that visits all subnodes recursively and stores the
\refTreeclass{util}{TreePath}
corresponding the the currently visited \<Tree>. (Also note that the
iterator given by \<TreePath> used to have an implementation
\ahref{http://bugs.sun.com/view_bug.do?bug_id=6473148}{bug}.)

In order to determine the kind of an object that extends \<Tree>, use
the
\ahref{\TreeAPIBase{}/tree/Tree.html#getKind()}{getKind}
method, as opposed to the \<instanceof> operator, since a \<Tree>
implementation might opt to implement more than one interface from
this API\@.  There is a utility class to perform operations on trees,
\refTreeclass{util}{Trees},
but the framework is intended to do all the low-level tree processing,
so you probably should not need to use this class.

An
\refModelclass{element}{Element}
represents a program element such as packages, classes or methods.
\<Element> has 5 subinterfaces:
\refModelclass{element}{ExecutableElement}
represents methods, constructors or initializers (anything invocable);
\refModelclass{element}{PackageElement}
represents package elements, and contain package information;
\refModelclass{element}{TypeElement}
represents the element of a class or an interface (note that
\<TypeElement> is an \<Element>, not a \<Type>; the corresponding
\<Type> is represented by
\refModelclass{type}{DeclaredType};
\refModelclass{element}{TypeParameterElement}
represents an element of a formal type parameter of a something with
generics, and
\refModelclass{element}{VariableElement}
represents the element associated with a variable. There is an
\refModelclass{element}{ElementVisitor}
interface for visiting objects that \<Element>, in a similar manner to the \<Tree> visitors, with similar
provided implementations. Use the \<asType> method from \<Element> to obtain a \<TypeMirror> for the element.

Again, \<Element> is an interface, so use getKind() to obtain the kind
of an \<Element>, as opposed to the instanceof operator, since an
implementation of \<Element> might also implement other element
interfaces. There is an utility class for handling elements,
\refModelclass{util}{Elements};
the appropriate instance can be obtained by using the
\sunjavadoc{javax/annotation/processing/ProcessingEnvironment.html#getElementUtils()}{getElementUtils{}}
method on the \<ProcessingEnvironment> object visible on factories and
checkers. The framework should do most of the element processing that
requires \<Elements>, unless you are doing something non-trivial.

A
\refModelclass{type}{TypeMirror}
represents a Java type. It is yet another interface you should be
familiar with, with various subinterfaces, notable ones being
\refModelclass{type}{DeclaredType}
for class and interface types, and
\refModelclass{type}{ExecutableType}
for method, constructor and initializer types.

Note that a
\refTreeclass{tree}{MethodTree}
resolves into a \<ExecutableType>, while a
\refTreeclass{tree}{MethodInvocationTree}
resolves into a \<DeclaredType> if the return type is a class or an
interface, an
\refModelclass{type}{ArrayType}
if the return type is an array, a
\refModelclass{type}{NoType}
if the return type is void, or a
\refModelclass{type}{PrimitiveType}
if the return type is primitive.

Not every \<Tree> corresponds to an \<Element> (such as a
\<BlockTree>), not every \<Tree> corresponds to a \<TypeMirror>
(again, such as a \<BlockTree>), and not every \<TypeMirror> has a
corresponding \<Element> (such as primitive types or arrays).

As one could expect by this point, \<TypeMirror> is an interface, so
use the appropriate getKind() method to distinguish the types, as
opposed to the instanceof operator, since those are interfaces, and
more than one can be implemented by a same object.

Note that the \<TypeMirror> API makes no guarantees that the same type
will always be represented by the same object; use the method
recommended on the API if you need to compare two types.

\refModelclass{type}{TypeVisitor}
and implementations of visitors for \<TypeMirror> are provided, but
those classes should not be used or extended directly on the
framework, since all checker plugin classes are meant to visit
\<AnnotatedTypeMirror> instead, modifying the annotations as needed.
A
\refModelclass{util}{Types}
utility class is provided by the \<ProcessingEnvironment> as well, if
you need to do more complex operations with types. In general, you
should use \<AnnotatedTypeMirror> and its subclasses as opposed to
using \<TypeMirror> and its subinterfaces.

An \refclass{types}{AnnotatedTypeMirror}
(defined in the Checker Framework, not in javac) represents an
annotated type --- a type along with all its annotations.  It is
modeled after Sun's
\refModelclass{type}{TypeMirror}.
Similarly modeled visitors are presented: a
\refclass{types/visitors}{AnnotatedTypeVisitor} interface, implemented by
\refclass{types/visitors}{SimpleAnnotatedTypeVisitor}
for visiting just one node,
\refclass{types/visitors}{AnnotatedTypeScanner}
for visiting every node recursively.

In short: a
\refTreeclass{tree}{Tree}
represents some snippet of code, an
\refModelclass{element}{Element}
represents some program element, and a
\refModelclass{type}{TypeMirror}
represents a Java type, but you usually should use
\refclass{types}{AnnotatedTypeMirror},
provided by the Checker Framework, instead of \<TypeMirror>, as our
implementation carries along with the types the annotation information
at every node level.  The
\refclass{types}{AnnotatedTypeFactory}
(or its extension on your framework plugin) is responsible for
producing \<AnnotatedTypeMirror> objects for \<Tree> and \<Element>
parameters it receives; those \<AnnotatedTypeMirror> objects are then
processed by the visitor class and checked by the checker class on
your checker plugin.


% LocalWords:  plugin javac's SourceChecker AbstractProcessor getMessages quals
% LocalWords:  getSourceVisitor SourceVisitor getFactory AnnotatedTypeFactory
% LocalWords:  SupportedAnnotationTypes SupportedSourceVersion TreePathScanner
% LocalWords:  TreeScanner visitAssignment AssignmentTree AnnotatedClassTypes
% LocalWords:  SubtypeChecker SubtypeVisitor NonNull isSubtype getClass nonnull
% LocalWords:  AnnotatedClassType isAnnotatedWith hasAnnotationAt TODO src jdk
% LocalWords:  processor NullnessChecker InterningChecker Nullness Nullable igj
% LocalWords:  AnnotatedTypeMirrors BaseTypeChecker BaseTypeVisitor basetype
% LocalWords:  Aqual Anqual java CharSequence getAnnotatedType UseLovely IGJ
% LocalWords:  AnnotatedTypeMirror LovelyChecker Anomsgtext Ashowchecks enums
% LocalWords:  Afilenames dereferenced SuppressWarnings declaratively SubtypeOf
% LocalWords:  TypeQualifier TypeHierarchy GraphQualifierHierarchy ReadOnly Foo
% LocalWords:  QualifierHierarchy QualifierRoot createQualifierHierarchy util
% LocalWords:  createTypeHierarchy QReadOnly ImplicitFor treeClasses TypeMirror
% LocalWords:  LiteralTree ExpressionTree typeClasses annotateImplicit nullable
% LocalWords:  TypeQualifiers getSupportedTypeQualifiers FooChecker nullness
% LocalWords:  FooVisitor FooAnnotatedTypeFactory basicstyle InterningVisitor
% LocalWords:  InterningAnnotatedTypeFactory QualifierDefaults TypeKind getKind
% LocalWords:  setAbsoluteDefaults PolymorphicQualifier TreeVisitor subnodes
% LocalWords:  SimpleTreeVisitor TreePath instanceof subinterfaces TypeElement
% LocalWords:  ExecutableElement PackageElement DeclaredType VariableElement
% LocalWords:  TypeParameterElement ElementVisitor javax getElementUtils NoType
% LocalWords:  ProcessingEnvironment ExecutableType MethodTree ArrayType Warski
% LocalWords:  MethodInvocationTree PrimitiveType BlockTree TypeVisitor blog
% LocalWords:  AnnotatedTypeVisitor SimpleAnnotatedTypeVisitor html langtools
% LocalWords:  AnnotatedTypeScanner bootclasspath asType stringPatterns
% LocalWords:  DefaultQualifierInHierarchy invocable
