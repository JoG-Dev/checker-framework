\htmlhr
\chapter{Property File Checker\label{propkey-checker}}

The Property File Checker ensures that a property file or resource bundle (both
of which act like maps from keys to values) is only accessed with valid keys.
Accesses without a valid key either return \code{null} or a default value, which
can lead to a \code{NullPointerException} or hard-to-trace behavior.
The Property File Checker (Section \refwithpage{genpropkey-checker}) ensures
that the used keys are found in the corresponding property file or resource
bundle.

We also provide two specialized checkers.
An Internationalization Checker (Section \refwithpage{i18n-checker})
verifies that code is properly internationalized.
A Compiler Message Key Checker (Section \refwithpage{compilermsgs-checker})
verifies that compiler message keys used in the Checker Framework are
declared in a property file;
This is an example of a simple specialization of the property
file checker, and the Checker Framework source code shows how it is used.

It is easy to customize the property key checker for other related purposes.
Take a look at the source code of the Compiler Message Key Checker and adapt it for
your purposes.



\section{General Property File Checker\label{genpropkey-checker}}

The general Property File Checker ensures that a resource key is located
in a specified property file or resource bundle.


The annotation \code{@\refclass{propkey/quals}{PropertyKey}}
indicates that the qualified \code{String} is a valid key
found in the property file or resource bundle.
You do not need to annotate \code{String} literals.
The checker looks up every \code{String} literal in the specified
property file or resource bundle, and adds annotations as appropriate.

If you pass a \code{String} variable to be eventually used as a key, you
also need to annotate all these variables with \code{@PropertyKey}.


The checker can be invoked by running the following
command:

\begin{Verbatim}
  javac -processor checkers.propkey.PropertyKeyChecker
        -Abundlenames=MyResource MyFile.java ...
\end{Verbatim}

You must specify the resources, which map keys to strings.
The checker supports two types of resource:
resource bundles and property files.  You can specify one or both of the
following two command-line options:

\begin{enumerate}

\item \code{-Abundlenames=\emph{resource\_name}}

  \emph{resource\_name} is the name of the resource to be used with
  \sunjavadoc{java/util/ResourceBundle.html#getBundle(java.lang.String, java.util.Locale, java.lang.ClassLoader)}{ResourceBundle.getBundle()}.
  The checker uses the default \code{Locale} and \code{ClassLoader} in the
  compilation system.
  (For a tutorial about \code{ResourceBundle}s, see
  \myurl{http://docs.oracle.com/javase/tutorial/i18n/resbundle/concept.html}.)
  % WAS:  http://java.sun.com/developer/technicalArticles/Intl/ResourceBundles/
  % http://docs.oracle.com/javase/tutorial/ui/features/i18n.html is not what
  % I am looking for to replace it.
  Multiple resource bundle names are separated by colons '\code{:}'.

\item \code{-Apropfiles=\emph{prop\_file}}

  \emph{prop\_file} is the name of a properties file that maps
  keys to values.  The file format is described in
  the Javadoc for 
  \sunjavadoc{java/util/Properties.html#load(java.io.Reader)}{Properties.load()}.
  Multiple files are separated by colons '\code{:}'.

\end{enumerate}



\section{Internationalization Checker\label{i18n-checker}}

The Internationalization Checker, or I18n Checker, verifies that your code is properly
internationalized.  Internationalization is the process of adapting
software to different languages and locales.  Internationalization is
sometimes called localization (though the terms are not
identical), and is sometimes called i18n (because the word starts with ``i'',
ends with ``n'', and has 18 characters in between; localization is similarly
sometimes abbreviated as l10n).

The checker focuses on one aspect of internationalization:  user-visible strings
should be presented in the user's own language, such as English, French, or
German.  This is achieved by looking up keys in a localization resource,
which maps keys to user-visible strings.  For instance, one version of a
resource might map \code{"CANCEL\_STRING"} to
\code{"Cancel"}, and another version of the same resource might map
\code{"CANCEL\_STRING"} to \code{"Abbrechen"}.

There are other aspects to localization, such as formatting of dates (3/5
vs.~5/3 for March 5), that the checker does not check.

The Internationalization Checker verifies these two properties:

\begin{enumerate}

\item
  Any user-visible text should be obtained from a localization resource.
  For example, \code{String} literals should not be output to the user.

\item
  When looking up keys in a localization resource, the key should exist in
  that resource.  This check catches incorrect or misspelled localization
  keys.

\end{enumerate}


\subsection{Internationalization annotations\label{i18n-annotations}}

The Internationalization Checker supports two annotations:

\begin{enumerate}
\item \code{@\refclass{i18n/quals}{Localized}}: indicates that the qualified
\code{String} is a message that has been localized and/or formatted with
respect to the used locale.

\item \code{@\refclass{i18n/quals}{LocalizableKey}}: indicates that the
qualified \code{String} or \code{Object} is a valid key found in the
localization resource.
This annotation is a specialization of the \code{@PropertyKey} annotation, that
gets checked by the general Property Key Checker.
\end{enumerate}

You may need to add the \code{@Localized} annotation to more methods in the
JDK or other libraries, or in your own code.


\subsection{Running the Internationalization Checker\label{i18n-running}}

The Internationalization Checker can be invoked by running the following
command:

\begin{Verbatim}
  javac -processor checkers.i18n.I18nChecker -Abundlenames=MyResource MyFile.java ...
\end{Verbatim}

You must specify the localization resource, which maps keys to user-visible
strings.  Like the general Property Key Checker, the Internationalization Checker
supports two types of localization resource:
\code{ResourceBundle}s using the 
\code{-Abundlenames=\emph{resource\_name}} option
or property files using the
\code{-Apropfiles=\emph{prop\_file}} option.



\section{Compiler Message Key Checker\label{compilermsgs-checker}}

The Checker Framework uses compiler message keys to output error messages.
These keys are substituted by localized strings for user-visible error messages.
Using keys instead of the localized strings in the source code enables easier
testing, as the expected error keys can stay unchanged while the localized
strings can still be modified. 
We use the Compiler Message Key Checker to ensure that all internal
keys are correctly localized.
Instead of using the Property File Checker, we use a specialized checker,
giving us more precise documentation of the intended use of \code{String}s.

The single annotation used by this checker is 
\code{@\refclass{compilermsgs/quals}{CompilerMessageKey}}.
The Checker Framework is completely annotated;
for example, class \code{checkers.source.Result}
uses \code{@CompilerMessageKey} in methods \code{failure} and \code{warning}.
For most users of the Checker Framework there will be no need to annotate any
\code{String}s, as the checker looks up all \code{String} literals and adds
annotations as appropriate.

The Compiler Message Key Checker can be invoked by running the following
command:

\begin{Verbatim}
  javac -processor checkers.compilermsgs.CompilerMessagesChecker
        -Apropfiles=messages.properties MyFile.java ...
\end{Verbatim}

You must specify the resource, which maps compiler message keys to user-visible
strings.  The checker supports the same options as the general property key checker.
Within the Checker Framework we only use property files,
so the \code{-Apropfiles=\emph{prop\_file}} option should be used.

% LocalWords:  NullPointerException genpropkey compilermsgs propkey PropertyKey
% LocalWords:  Abundlenames getBundle ResourceBundle ClassLoader Apropfiles
% LocalWords:  Abbrechen LocalizableKey CompilerMessageKey
