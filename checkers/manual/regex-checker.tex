\htmlhr
\chapter{Regex checker for regular expression syntax\label{regex-checker}}

The Regex Checker prevents, at compile-time, use of syntactically invalid
regular expressions.

A regular expression, or regex, is a pattern for matching certain strings
of text.  In Java, a programmer writes a regular expression as a string.
At run time, the string is ``compiled'' into an efficient internal form
(\sunjavadoc{java/util/regex/Pattern.html}{Pattern}) that is used for
text-matching.

The syntax of regular expressions is complex, so it is easy to make a
mistake.  It is also easy to accidentally use a regex feature from another
language that is not supported by Java (see section ``Comparison to Perl
5'' in the \sunjavadoc{java/util/regex/Pattern.html}{Pattern} Javadoc).
Ordinarily, the programmer does not learn of these errors until run time.
The Regex checker warns about these problems at compile time.

To run the Regex Checker, supply the \code{-processor
  checkers.regex.RegexChecker} command-line option to javac.


\section{Regex annotations\label{regex-annotations}}

These qualifiers make up the Regex type system:

\begin{description}

\item[\<@\refclass{regex/quals}{Regex}>]
  indicates valid regular expression \code{String}s.

\item[\<@\refclass{regex/quals}{PolyRegex}>]
  indicates qualifier polymorphism. For a description of
  \<@\refclass{regex/quals}{PolyRegex}>,
  see Section~\ref{qualifier-polymorphism}.

\end{description}

\section{Annotating your code with \code{@Regex}\label{annotating-with-regex}}


\subsection{Implicit qualifiers\label{regex-implicit-qualifiers}}

As described in Section~\ref{effective-qualifier}, the Regex checker adds
implicit qualifiers, reducing the number of annotations that must appear
in your code. The checker implicitly adds the \code{Regex} qualifier to
any \code{String} literal that is a valid regex and to the \code{null}
literal.


\subsection{Testing whether a string is a regular expression\label{regexutil-methods}}

Sometimes, the Regex Checker cannot infer whether a particular expression
is a regular expression --- and sometimes your code cannot either!  In
these cases, you can use the \<isRegex> method to perform such a test, and
other helper methods to provide useful error messages.  A
common use is for user-provided regular expressions (such as ones passed
on the command-line).
Figure~\ref{fig:regex-util-example} gives an
example of the intended use of the \code{RegexUtil} methods.

\begin{description}

\item[\refmethod{regex}{RegexUtil}{isRegex}{(java.lang.String)}]
  returns \<true> if its argument is a valid regular expression.

\item[\refmethod{regex}{RegexUtil}{regexError}{(java.lang.String)}]
  returns a \<String> error message if its argument is not a valid regular
  expression, or \<null> if its argument is a valid regular expression.

\item[\refmethod{regex}{RegexUtil}{regexException}{(java.lang.String)}]
  returns the
  \sunjavadoc{java/util/regex/PatternSyntaxException.html}{PatternSyntaxException}
  that \sunjavadoc{java/util/regex/Pattern.html#compile(java.lang.String)}{Pattern.compile(String)}
  throws when compiling an invalid regular expression.  It returns \<null>
  if its argument is a valid regular expression.

\end{description}

A potential disadvantage of using these methods is that your code becomes
dependent on the Checker Framework at run time as well as at compile time.
You can avoid this by copying the implementation of these methods into
your own code.

\begin{figure}
%BEGIN LATEX
\begin{smaller}
%END LATEX
\begin{Verbatim}
String regex = getRegexFromUser();
if (! RegexUtil.isRegex(regex)) {
   throw new RuntimeException("Error parsing regex " + regex, RegexUtil.regexException(regex));
   // or: System.out.println("Error parsing regex \"" + regex + "\": " + RegexUtil.regexError(regex));
}
// The following line suppresses a Regex Checker warning and is only necessary until the 
// Regex Checker supports flow-sensitivity, after which time it can be removed from the code.
regex = RegexUtil.asRegex(regex);   // @SuppressWarnings("regex") // flow-sensitivity
Pattern p = Pattern.compile(regex);
\end{Verbatim}
%BEGIN LATEX
\end{smaller}
%END LATEX
\caption{Example use of \code{RegexUtil} methods.}
\label{fig:regex-util-example}
\end{figure}


\subsection{Suppressing warnings}

If you are positive that a particular string that is being used as a
regular expression is syntactically valid, but the Regex Checker cannot
conclude this and issues a warning about possible use of an invalid regular
expression, then you can use the
\refmethod{regex}{RegexUtil}{asRegex}{(java.lang.String)} method to suppress the
warning.

You can think of this method 
is a cast:  it returns its argument unchanged, but with the type
\code{@Regex String} if it is a valid regular expression.  It throws an
Error if its argument is not a valid regular expression, but you should
only use it when you are sure it will not throw an error.

This method is
mainly a workaround until the Regex Checker supports flow-sensitivity (see
Section~\ref{type-refinement}) and should be used rarely once the Regex
Checker supports flow-sensitivity.




% LocalWords:  Regex regex quals PolyRegex isRegex RegexUtil regexError asRegex
% LocalWords:  regexException PatternSyntaxException
