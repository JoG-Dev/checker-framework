\htmlhr
\chapter{Integration with external tools\label{external-tools}}

This chapter discusses how to run a checker from your favorite IDE\@.

Or, if your favorite isn't here, you should customize how it runs the
javac command on your behalf.  See the IDE documentation to learn how to
customize it, adapting the instructions for javac in Section~\ref{running}.
If you make another tool support running a checker, please
inform us via the
\ahref{\url{http://groups.google.com/group/checker-framework-discuss}}{mailing
  list} or
\ahref{\url{http://code.google.com/p/checker-framework/issues/list}}{issue tracker} so
we can add it to this manual.

This chapter also discusses type inference tools (see
Section~\ref{type-inference-tools}).

All examples in this chapter are in the public domain, with no copyright nor
licensing restrictions.


\section{Javac Compiler\label{javac-installation}}

If you use \code{javac} compiler from the command line, then you can
instead use the Checker Framework compiler that is bundled with the Checker
Framework.  The bundled \code{javac} is a variant of the OpenJDK
\code{javac} that recognizes type annotations.  Eventually, the OpenJDK
\code{javac} will recognize type annotations.  The bundled \code{javac}
also supports annotations in comments (see
Section~\ref{annotations-in-comments}), which the OpenJDK \code{javac} will
not.

This section describes how you can install and use the bundled
\code{javac}, using either Unix/Linux/MacOS (see
Section~\ref{unix-installation}) or Windows (see
Section~\ref{windows-installation}).
The instructions are identical to those of Section~\ref{installation},
but are given as commands that you can cut and paste into your command shell.


%%% *****
%%% UPDATE
%%% *****

%%% Note that much of this section is duplicated with the "Windows
%%% installation" section.  That is better for users, even though it is
%%% longer and makes the maintainers keep two versions in sync.
\subsection{Unix/Linux/MacOS installation\label{unix-installation}}

These instructions assume that you use the bash or sh shell.  If you use a
different shell, you may need to slightly adjust the commands.

\begin{enumerate}

\item
  Download the latest Checker Framework distribution
  % (\ahrefurl{http://types.cs.washington.edu/checker-framework/current/checkers.zip})
  and unzip it.  You can put it anywhere you like by changing the
  definition of environment variable \code{JSR308} below; a standard place
  is in a
  new directory named \code{jsr308}.

\begin{Verbatim}
  export JSR308=$HOME/jsr308
  mkdir -p ${JSR308}
  cd ${JSR308}
  # or:  wget http://types.cs.washington.edu/checker-framework/current/checkers.zip
  curl -O http://types.cs.washington.edu/checker-framework/current/checkers.zip
  unzip checkers.zip
  chmod +x checker-framework/checkers/binary/javac
  checker-framework/checkers/binary/javac -version
\end{Verbatim}

The output of the last command should be:

\begin{Verbatim}
  javac 1.7.0-jsr308-1.7.1
\end{Verbatim}


\item
  Place the following commands in your \code{.bashrc} file:
\begin{Verbatim}
  export JSR308=$HOME/jsr308
  export CHECKERS=$JSR308/checker-framework/checkers
  export PATH=$CHECKERS/binary:${PATH}
\end{Verbatim}

% It is not necessary to add checkers.jar to your classpath, because the
% shipped compiler already does so.
%   export CLASSPATH=$JSR308/checker-framework/checkers/binary/checkers.jar:${CLASSPATH}

Also execute them on the command line, or log out and back in.  Then,
verify that the installation works.  From the command line, run:

\begin{Verbatim}
  javac -version
\end{Verbatim}

The output should be:

\begin{Verbatim}
  javac 1.7.0-jsr308-1.7.1
\end{Verbatim}

\end{enumerate}

That's all there is to it!  Now you are ready to start using the checkers with
the new \code{javac} compiler.

\subsection{Windows installation\label{windows-installation}}

\begin{enumerate}

\item
  Download the latest Checker Framework distribution
  % (\ahrefurl{http://types.cs.washington.edu/checker-framework/current/checkers.zip})
  and unzip it to create a \<checkers> directory.  You can put it anywhere
  you like; a standard place is in a new directory under \<C:\ttbs{}Program
  Files>.

\begin{enumerate}
\item
  Save the file
  \ahrefurl{http://types.cs.washington.edu/checker-framework/current/checkers.zip}
  to your Desktop.
\item
  Double-click the \<checkers.zip> file on your computer.  Click on
  the \<checkers> directory, then Select \<Extract all files>, and use
  \<C:\ttbs{}Program Files> as the destination.  You will obtain a new
  \<C:\ttbs{}Program Files\ttbs{}checker-framework> folder.
\item
  Verify that the installation works.  From a Windows command prompt, run:

% Do I need to quote the space in "Program Files"?
\begin{Verbatim}
  set CHECKERS = C:\Program Files\checker-framework\checkers
  java -jar C:%CHECKERS%\binary\checkers.jar -version
\end{Verbatim}

The output should be:

\begin{Verbatim}
  javac 1.7.0-jsr308-1.7.1
\end{Verbatim}

\end{enumerate}


\item
  In order to use the updated compiler when you type \code{javac}, add the
  directory \<C:\ttbs{}Program Files\ttbs{}checker-framework\ttbs{}checkers\ttbs{}binary> to the
  beginning of your path variable.  Also set a \code{CHECKERS} variable.

% Instructions stolen from http://www.webreference.com/js/tips/020429.html

To set an environment variable, you have two options:  make the change
temporarily or permanently.
\begin{itemize}
\item
To make the change \textbf{temporarily}, type at the command shell prompt:

\begin{alltt}
path = \emph{newdir};%PATH%
\end{alltt}

For example:

\begin{Verbatim}
set CHECKERS = C:\Program Files\checker-framework\checkers
path = %CHECKERS%\binary;%PATH%
\end{Verbatim}

This is a temporary change that endures until the window is closed, and you
must re-do it every time you start a new command shell.

\item
To make the change \textbf{permanently},
Right-click the \<My Computer> icon and
select \<Properties>. Select the \<Advanced> tab and click the
\<Environment Variables> button. You can set the variable as a ``System
Variable'' (visible to all users) or as a ``User Variable'' (visible to
just this user).  Both work; the instructions below show how to set as a
``System Variable''.
In the \<System Variables> pane, select
\<Path> from the list and click \<Edit>. In the \<Edit System Variable>
dialog box, move the cursor to the beginning of the string in the
\<Variable Value> field and type the full directory name (not using the
\verb|%CHECKERS%| environment variable) followed by a
semicolon (\<;>).

% This is for the benefit of the Ant task.
Similarly, set the \code{CHECKERS} variable.

This is a permanent change that only needs to be done once ever.
\end{itemize}


% It is not necessary to add checkers.jar to your classpath, because the
% shipped compiler already does so.
%   export CLASSPATH=$JSR308/checker-framework/checkers/binary/checkers.jar:${CLASSPATH}

Now, verify that the installation works.  From the command line, run:

\begin{Verbatim}
  javac -version
\end{Verbatim}

The output should be:

\begin{Verbatim}
  javac 1.7.0-jsr308-1.7.1
\end{Verbatim}

\end{enumerate}


\section{Ant task\label{ant-task}}

If you use the \ahref{\url{http://ant.apache.org/}}{Ant} build tool to compile
your software, then you can add an Ant task that runs a checker.  We assume
that your Ant file already contains a compilation target that uses the
\code{javac} task.

\begin{enumerate}
\item
Set the \code{jsr308javac} property:

%BEGIN LATEX
\begin{smaller}
%END LATEX
\begin{Verbatim}
  <property environment="env"/>

  <property name="checkers" value="${env.CHECKERS}" />
  
  <!-- On Mac/Linux, use the javac shell script; on Windows, use javac.bat -->
  <condition property="cfJavac" value="javac.bat" else="javac">
      <os family="windows" />
  </condition>

  <presetdef name="jsr308.javac">
    <javac fork="yes" executable="${checkers}/binary/${cfJavac}" >
      <!-- JSR-308-related compiler arguments -->
      <compilerarg value="-version"/>
      <!-- optional, so .class files work with older JVMs: <compilerarg line="-target 5"/> -->
      <compilerarg value="-implicit:class"/>
    </javac>
  </presetdef>
\end{Verbatim}
%BEGIN LATEX
\end{smaller}
%END LATEX

\item \textbf{Duplicate} the compilation target, then \textbf{modify} it slightly as
indicated in this example:

%BEGIN LATEX
\begin{smaller}
%END LATEX
\begin{Verbatim}
  <target name="check-nullness"
          description="Check for null pointer dereferences"
          depends="clean,...">
    <!-- use jsr308.javac instead of javac -->
    <jsr308.javac ... >
      <compilerarg line="-processor checkers.nullness.NullnessChecker"/>
      <!-- optional, for implicit imports: <compilerarg value="-J-Djsr308_imports=checkers.nullness.quals.*:dataflow.quals.*"/> -->
      <!-- optional, to not check uses of library methods: <compilerarg value="-AskipUses=^(java\.awt\.|javax\.swing\.)"/> -->
      <compilerarg line="-Xmaxerrs 10000"/>
      ...
    </jsr308.javac>
  </target>
\end{Verbatim}
%BEGIN LATEX
\end{smaller}
%END LATEX

Fill in each ellipsis (\ldots) from the original compilation target.  But,
don't include any \code{-source} argument with value other than \code{1.8}
or \code{8}.  Doing so will disable the annotations in
comments feature (see Section~\refwithpage{annotations-in-comments}).

In the example, the target is named \code{check-nullness}, but you can
name it whatever you like.
\end{enumerate}

\subsection{Explanation\label{ant-task-explanation}}

This section explains each part of the Ant task.

\begin{enumerate}
\item Definition of \code{jsr308.javac}:

The \code{fork} field of the \code{javac} task
ensures that an external javac program is called.  Otherwise, Ant will run
javac via a Java method call, and there is no guarantee that it will get
the JSR 308 version that is distributed with the Checker Framework.

The \code{-version} compiler argument is just for debugging; you may omit
it.

The \code{-target 5} compiler argument is optional, if you use Java 5 in
ordinary compilation when not performing pluggable type-checking (see
Section~\refwithpage{java5-class-files}).

The \code{-implicit:class} compiler argument causes annotation processing
to be performed on implicitly compiled files.  (An implicitly compiled file
is one that was not specified on the command line, but for which the source
code is newer than the \code{.class} file.)  This is the default, but
supplying the argument explicitly suppresses a compiler warning.

%% -Awarns was removed above without removing it here.
% The \code{-Awarns} compiler argument is optional, and causes the checker to
% treat errors as warnings so that compilation does not fail even if
% pluggable type-checking fails; see Section~\ref{checker-options}.

\item The \code{check-nullness} target:

The target assumes the existence of a \code{clean} target that removes all
\code{.class} files.  That is necessary because Ant's \code{javac} target
doesn't re-compile \code{.java} files for which a \code{.class} file
already exists.

The \code{-processor ...} compiler argument indicates which checker to
run.  You can supply additional arguments to the checker as well.

\end{enumerate}


\section{Maven plugin\label{maven-plugin}}

If you use the \ahref{\url{http://maven.apache.org/}}{Maven} tool,
then you can specify pluggable type-checking as part of your build
process.

\begin{enumerate}

\item Add the repositories to your \code{pom.xml} file:

\begin{Verbatim}
    <repositories>
        <repository>
            <id>checker-framework-repo</id>
            <url>http://types.cs.washington.edu/m2-repo</url>
        </repository>
    </repositories>
    <pluginRepositories>
        <pluginRepository>
            <id>checker-framework-repo</id>
            <url>http://types.cs.washington.edu/m2-repo</url>
        </pluginRepository>
    </pluginRepositories>
\end{Verbatim}

As an alternative to the repository 
\ahref{\url{http://types.cs.washington.edu/m2-repo}},
you can find the Checker Framework in Maven Central (MC):
\ahref{\url{http://search.maven.org/#search|ga|1|a%3A%22checker-framework%22}}


\item Declare a dependency on the type qualifier annotations.  Find the
  existing \code{<dependencies>} section and add a new
  \code{<dependencies>} item:

\begin{alltt}
    <dependencies>
        ... existing <dependency> items ...

        <!-- annotations from the Checker Framework: nullness, interning, locking, ... -->
        <dependency>
            <groupId>types.checkers</groupId>
            <artifactId>checkers-quals</artifactId>
            <version>\ReleaseVersion{}</version>
        </dependency>

    </dependencies>
\end{alltt}

\item Attach the plugin to your build lifecycle by adding a new
  \code{<plugin>} item within \code{<plugins>}, or a new \code{<plugins>}
  item if necessary.

\begin{alltt}
    <build>
        <plugins>
            ... existing <plugin> items ...

            <plugin>
                <groupId>types.checkers</groupId>
                <artifactId>checkers-maven-plugin</artifactId>
                <version>\ReleaseVersion{}</version>
                <executions>
                    <execution>
                        <!-- run the checkers after compilation; this can also be any later phase -->
                        <phase>process-classes</phase>
                        <goals>
                            <goal>check</goal>
                        </goals>
                    </execution>
                </executions>
                <configuration>
                    <!-- required configuration options -->
                    <!-- a list of processors to run -->
                    <processors>
                        <processor>checkers.nullness.NullnessChecker</processor>
                        <processor>checkers.interning.InterningChecker</processor>
                    </processors>
                    <!-- optional configuration options go here -->
                </configuration>
              </plugin>
        </plugins>
    </build>
\end{alltt}

The above example attaches the "check" goal to the "process-lifecycle" phase which occurs directly after
the "compile" phase (see \url{http://maven.apache.org/guides/introduction/introduction-to-the-lifecycle.html}).
This means that any maven target that exercises that phase will also run the Checker Framework
(e.g., \<mvn package>, \<mvn verify>).

To run only checking:
\code{mvn checkers:check}

\item Add optional parameters

Within the configuration element you can add a number of parameters.

\begin{Verbatim}
    <procOnly>false</procOnly>
\end{Verbatim}
By default, the Checker Framework Maven plugin will only check source files but will not produce
class files for the source files it checks (see \<-proc:only> at \url{http://docs.oracle.com/javase/6/docs/technotes/tools/solaris/javac.html}).
Setting \code{procOnly} to false will cause the Checker Framework compiler to generate class files.
Note: If your Maven project's output directory has not been created then the plugin will attempt to
create it and all required directories preceding it.  The check task will fail
in the event that these directories cannot be created.

You may want to check only a subset of source files in your project.  To do so you can use the \code{includes} and \code{excludes}
elements using the standard Maven syntax.  The semantics of including/excluding files are also described
by the DirectoryScanner documentation found at:
\url{http://plexus.codehaus.org/plexus-utils/apidocs/org/codehaus/plexus/util/DirectoryScanner.html} (see setIncludes/setExcludes)

\begin{Verbatim}
    <!-- a list of patterns to include, in the standard maven syntax; defaults to **/*.java -->
    <includes>
        <include>org/company/important/**/*.java</include>
    </includes>
    <!-- a list of patterns to exclude, in the standard maven syntax; defaults to an empty list -->
    <excludes>
        <exclude>org/company/notimportant/**/*.java</exclude>
    </excludes>
\end{Verbatim}

By default, an error reported by the Checker Framework Maven plugin will cause your build to fail.  To prevent
a build from failing and still produce error and warning output set \code{failOnError} to false.

\begin{Verbatim}
    <failOnError>false</failOnError>
\end{Verbatim}

You may want to run the Checker Framework using a Java version that differs from the version running Maven.  The
\code{executable} element specifies a Java executable to use when running the Checker Framework.

\begin{Verbatim}
    <executable>/usr/bin/java</executable>
\end{Verbatim}

The version of the Checker Framework run by the Maven plugin will share the same version as the Maven plugin
by default.  That is, if you are using the Checker Framework Maven plugin version \ReleaseVersion{} then it will run the
Checker Framework version \ReleaseVersion{} when checking source code.  However, you can specify an earlier
or later version of the Checker Framework using the \code{checkerFrameworkVersion} element.
Note: The Checker Framework Maven plugin will only work with Checker Framework versions 1.5.0 and higher.

\begin{Verbatim}
    <checkerFrameworkVersion>1.6.2-johndoe</checkerFrameworkVersion>
\end{Verbatim}

To disable the Checker Framework check without removing it from your pom file you can use the \code{skip} option.
\begin{Verbatim}
    <skip>true</skip>
\end{Verbatim}

Finally, the Checker Framework compiler is an extension of the javac compiler and therefore excepts any parameter
that can be provided to javac or the JVM.  To provide javac parameters you can use the \code{javacParams}
element.  To provide JVM arguments you can use the \code{javaParams} element.

\begin{Verbatim}
    <!-- additional parameters passed to the Java compiler -->
    <javacParams>-Alint</javacParams>
    <!-- additional parameters to pass to the forked JVM -->
    <javaParams>-Xdebug</javaParams>
\end{Verbatim}

\end{enumerate}

You can find an example Maven project that uses the Checker Framework Maven plugin at:
\url{http://types.cs.washington.edu/checker-framework/current/mvn-examples.zip}

The plugin was contributed by Adam Warski.
% [Thanks Adam!]


\section{Gradle\label{gradle}}

% This information came from:
% http://jira.codehaus.org/browse/GRADLE-342
% http://docs.codehaus.org/display/GRADLE/Gradle+0.8+Breaking+Changes

If you fork the compilation task, \ahref{\url{http://www.gradle.org/}}{Gradle}
lets you specify the executable to compile java programs.

To specify the appropriate executable, set
\code{options.fork = true} and
\code{compile.options.fork.executable = "\$CHECKERS/binary/javac"}

To specify command-line arguments, set
\code{compile.options.compilerArgs}.  Here is a possible example:

\begin{Verbatim}
allprojects {
  tasks.withType(JavaCompile).all { JavaCompile compile ->
    compile.options.debug = true
    compile.options.compilerArgs = [
      '-version',
      '-implicit:class',
      '-processor', 'checkers.nullness.NullnessChecker'
    ]
    options.fork = true
    options.forkOptions.executable = "$CHECKERS/binary/javac"
  }
}
\end{Verbatim}

% extra $ to unconfuse Emacs's LaTeX mode


\section{IntelliJ IDEA\label{intellij}}

IntelliJ IDEA (Maia release)
\ahref{\url{http://blogs.jetbrains.com/idea/2009/07/type-annotations-jsr-308-support/}}{supports}
the Type Annotations (JSR-308) syntax.
See \url{http://blogs.jetbrains.com/idea/2009/07/type-annotations-jsr-308-support/}.

\section{Eclipse\label{eclipse}}

An early access (prerelease) version of Eclipse that supports type
annotations is available; see
\url{http://dev.eclipse.org/mhonarc/lists/eclipse-dev/msg09679.html}.  It
does not enable running the Checker Framework, nor is it necessary for
running the Checker Framework.

There are two ways to run a checker from within the Eclipse IDE:  via Ant
or using an Eclipse plug-in.  These two methods are described below.

No matter what method you choose, we suggest that
all Checker Framework annotations be written in the comments
if you are using a version of Eclipse that
does not support Java 8.  This will avoid many
text highlighting errors with versions of Eclipse that don't support Java 8 and the
new JSR 308 syntax changes.

Even in a version of Eclipse that supports Java 8's type annotations, you
still need to run the Checker Framework via Ant or via the plug-in, rather
than by supplying the \<-processor> command-line option to the ejc
compiler.  The reason is that the Checker Framework is built upon javac,
and ejc represents the Java program differently.  (If both javac and ejc
implemented JSR 198~\cite{JSR198}, then it would be possible to build
type-checking plug-ins that work with both compilers.)


\paragraph{Using an Ant task}

Add an Ant target as described in Section~\ref{ant-task}.  You can
run the Ant target by executing the following steps
(instructions copied from
\myurl{http://www.eclipse.org/documentation/?topic=/org.eclipse.platform.doc.user/gettingStarted/qs-84_run_ant.htm}):

\begin{enumerate}

\item
  Select \code{build.xml} in one of the navigation views and choose
  {\bf Run As $>$ Ant Build...} from its context menu.

\item
  A launch configuration dialog is opened on a launch configuration
  for this Ant buildfile.

\item
  In the {\bf Targets} tab, select the new ant task (e.g., check-interning).

\item
  Click {\bf Run}.

\item
  The Ant buildfile is run, and the output is sent to the Console view.

\end{enumerate}

\paragraph{Eclipse plug-in for the Checker Framework}

The Checker Plugin is an Eclipse plugin that enables the use of the Checker
Framework.
Its website (\myurl{http://types.cs.washington.edu/checker-framework/eclipse/}).
The website contains instructions for installing and using the plugin.
% The plugin has been substantially improved through a Google Summer of Code 2010 project
% and supports all checkers that are distributed with the Checker Framework.


\section{tIDE\label{tide}}

tIDE, an open-source Java IDE, supports the Checker Framework.  See its
documentation at \myurl{http://tide.olympe-network.com/}.


\section{Type inference tools\label{type-inference-tools}}

\subsection{Varieties of type inference}

There are two different tasks that are commonly called ``type inference''.

\begin{enumerate}
\item
  Type inference during type-checking (Section~\ref{type-refinement}):
  During type-checking, if certain variables have no type qualifier, the
  type-checker determines whether there is some type qualifier that would
  permit the program to type-check.  If so, the type-checker uses that type
  qualifier, but never tells the programmer what it was.  Each time the
  type-checker runs, it re-infers the type qualifier for that variable.  If
  no type qualifier exists that permits the program to type-check, the
  type-checker issues a type warning.

  This variety of type inference is built into the Checker Framework.  Every
  checker can take advantage of it at no extra effort.  However, it only
  works within a method, not across method boundaries.

  Advantages of this variety of type inference include:
  \begin{itemize}
  \item
    If the type qualifier is obvious to the programmer, then omitting it
    can reduce annotation clutter in the program.
  \item
    The type inference can take advantage of only the code currently being
    compiled, rather than having to be correct for all possible calls.
    Additionally, if the code changes, then there is no old annotation to
    update.
  \end{itemize}


\item
  Type inference to annotate a program (Section~\ref{type-inference-to-annotate}):
  As a separate step before type-checking, a type inference tool takes the
  program as input, and outputs a set of type qualifiers that would
  type-check.  These qualifiers are inserted into the source code or the
  class file.  They can be viewed and adjusted by the programmer, and can
  be used by tools such as the type-checker.

  This variety of type inference must be provided by a separate tool.  It
  is not built into the Checker Framework.

  Advantages of this variety of type inference include:
  \begin{itemize}
  \item
    The program contains documentation in the form of type qualifiers,
    which can aid programmer understanding.
  \item
    Error messages may be more comprehensible.  With type inference
    during type-checking, error messages can be obscure, because the
    compiler has already inferred (possibly incorrect) types for a number
    of variables.
  \item
    A minor advantage is speed:  type-checking can be modular, which can be
    faster than re-doing type inference every time the
    program is type-checked.
  \end{itemize}

\end{enumerate}

Advantages of both varieties of inference include:
\begin{itemize}
\item
  Less work for the programmer.
\item
  The tool chooses the most general type, whereas a programmer might
  accidentally write a more specific, less generally-useful annotation.
\end{itemize}


Each variety of type inference has its place.  When using the Checker
Framework, type inference during type-checking is performed only
\emph{within} a method (Section~\ref{type-refinement}).  Every method
signature (arguments and return values) and field must be explicitly annotated,
either by the programmer or by a separate type-checking tool
(Section~\ref{type-inference-to-annotate}).  This choice reduces programmer
effort (typically, a programmer does not have to write any qualifiers
inside the body of a method) while still retaining modular checking and
documentation benefits.


\subsection{Type inference to annotate a program\label{type-inference-to-annotate}}

This section lists tools that take a program and output a set of
annotations for it.

Section~\ref{nullness-inference} lists several tools that infer
annotations for the Nullness Checker.

Section~\ref{javari-inference} lists a tool that infers
annotations for the Javari Checker, which detects mutation errors.


% LocalWords:  jsr plugin Warski xml buildfile tIDE java Awarns pom lifecycle
% LocalWords:  IntelliJ Maia newdir classpath Unconfuse nullness Gradle
% LocalWords:  compilerArgs Xbootclasspath JSR308 jsr308 jsr308javac mvn
%  LocalWords:  plugins proc procOnly DirectoryScanner setIncludes
%  LocalWords:  setExcludes checkerFrameworkVersion javacParams javaParams
