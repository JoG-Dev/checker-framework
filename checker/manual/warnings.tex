\htmlhr
\chapter{Suppressing warnings\label{suppressing-warnings}}

%% This feels redundant.
% The Checker Framework is sound:  whenever your code contains an error, the
% Checker Framework will warn you about the error.  The Checker Framework is
% conservative:  it may issue warnings when your code is safe and never
% misbehaves at run time.

You may wish to suppress checker warnings because of unannotated libraries
or un-annotated portions of your own code, because of application
invariants that are beyond the capabilities of the type system, because of
checker limitations, because you are interested in only some of the
guarantees provided by a checker, or for other reasons.
Suppressing a warning is similar to writing a cast in a Java
program:  the programmer knows more about the type than the type system does
and uses the warning suppression or cast to convey that information to the
type system.

You can suppress a single warning message (or those in a single method or
class) by using the following mechanisms:

\newcounter{lastsinglesuppression}
\begin{itemize}
\item
  the \code{@SuppressWarnings} annotation
  (Section~\ref{suppresswarnings-annotation}), or
\item
  the \code{@AssumeAssertion} string in an \<assert> message (Section~\ref{assumeassertion}).
\end{itemize}

You can suppress warnings throughout the codebase by using the following mechanisms:

\begin{itemize}
\item
  the \code{-AsuppressWarnings} command-line option (Section~\ref{suppresswarnings-command-line}),
\item
  the \code{-AskipUses} and \code{-AonlyUses} command-line options (Section~\ref{askipuses}),
\item
  the \code{-AskipDefs} and \code{-AonlyDefs} command-line options (Section~\ref{askipdefs}),
\item
  the \code{-Alint} command-line option (Section~\ref{alint}), or
\item
  not using the \code{-processor} command-line option
  (Section~\ref{no-processor}).
\end{itemize}

Some type checkers can suppress warnings via
\begin{itemize}
\item
  checker-specific mechanisms (Section~\ref{checker-specific-suppression}).
\end{itemize}

\noindent
We now explain these mechanisms in turn.

% See the @SuppressWarningsKey annotation and the getSuppressWarningsKey method.

\section{\code{@SuppressWarnings} annotation\label{suppresswarnings-annotation}}

You can suppress specific errors and warnings by use of the
\code{@SuppressWarnings("\emph{checkername}")} annotation, for example
\code{@SuppressWarnings("interning")} or \code{@SuppressWarnings("nullness")}.
The argument \emph{checkername} is in lower case and is derived from the
way you invoke the checker; for example, if you invoke a checker as
\code{javac -processor MyNiftyChecker ...}, then you would suppress its
error messages with \code{@SuppressWarnings("mynifty")}.  (An exception is
the Subtyping Checker, for which you use the annotation name; see
Section~\ref{subtyping-using}).

While not recommended, using \code{@SuppressWarnings("all")} will
suppress warnings for all checkers.

A \sunjavadocanno{java/lang/SuppressWarnings.html}{SuppressWarnings}
annotation may be placed on program declarations such as a local
variable declaration, a method, or a class.  It suppresses all warnings
related to the given checker, for that program element.
\<@SuppressWarnings> is a declaration annotation, and it cannot be used on
statements, expressions, or types.

For instance, one common use is
to suppress warnings at a cast that you know is safe.  Here is an example
that uses the Tainting Checker (Section~\ref{tainting-checker}); assume
that \<expr> has type \<@Tainted String>:

\begin{Verbatim}
  @SuppressWarnings("tainting") // Explain why the suppression is sound.
  @Untainted String myvar = expr;
\end{Verbatim}

\noindent
It would have been \emph{illegal} to write

\begin{Verbatim}
  @Untainted String myvar;
  // This does not work; @SuppressWarnings goes on a declaration, not a statement
  @SuppressWarnings("tainting") // Explain why the suppression is sound.
  myvar = expr;
\end{Verbatim}

\noindent
because Java does not permit annotations (such as \<@SuppressWarnings>) on
assignments or other statements or expressions.


% TODO: improve description

Each warning from the compiler also issues the most concrete
suppression key that can be used to suppress that warning.
Additionally, the \code{-AshowSuppressWarningKeys} command-line option
can be used to show all applicable suppression keys.


\subsection{Good practices when suppressing warnings\label{suppresswarnings-best-practices}}

\subsubsection{Suppress warnings in the smallest possible scope\label{suppresswarnings-best-practices-smallest-scope}}

If a particular expression causes a
false positive warning, you should extract that expression into a local variable
and place a \code{@SuppressWarnings} annotation on the variable
declaration, rather than suppressing warnings for a larger expression or an
entire method body.

%% I'm not sure how this is related to the smallest possible scope.
% As another example, if you have annotated the signatures but not the bodies
% of the methods in a class or package, put a \code{@SuppressWarnings}
% annotation on the class declaration or on the package's
% \code{package-info.java} file.

\subsubsection{Justify why the warning is a false positive\label{suppresswarnings-best-practices-justification}}

A \<@SuppressWarnings> annotation asserts that the code is actually
correct, even though the type system is unable to prove that the code is
correct.

Whenever you write a \<@SuppressWarnings> annotation, you should also
write, typically on the same line or on the preceding line, a code comment
explaining why the code is actually correct.  In some cases you might also
justify why the code cannot be rewritten in a simpler way that would be
amenable to type-checking.

This documentation will help you and others to understand the reason for
the \<@SuppressWarnings> annotation.  It will also help if you decide to
audit your code to verify all the warning suppressions.

\subsubsection{Use a specific argument to \code{@SuppressWarnings}\label{suppresswarnings-best-practices-specific-argument}}

\label{compiler-message-keys}

The \<@SuppressWarnings> argument string can be of the form
\emph{checkername} or
or \emph{checkername:messagekey}.  The \emph{checkername} part is as
described above.  The \emph{messagekey} part suppresses only
errors/warnings relating to the given message key.  For example,
\code{cast.unsafe} is the key for warnings about an unsafe cast, and
\code{cast.redundant} to the key for warnings about a redundant cast.

Thus, the above example could have been written as any one of the
following, which would have suppressed the specific error:

\begin{Verbatim}
  @SuppressWarnings("tainting")              // suppresses all tainting-related warnings
  @SuppressWarnings("tainting:cast.unsafe")  // suppresses tainting warnings about unsafe casts
  @SuppressWarnings("tainting:cast")         // suppresses tainting warnings about casts
\end{Verbatim}

\noindent
For a list of the message keys, 
see the \code{messages.properties} files in 
%BEGIN LATEX
\\
%END LATEX
\code{checker-framework/checker/src/org/checkerframework/checker/\emph{checkername}/messages.properties}.
Each checker is built on the \code{basetype} checker and inherits its
properties.  Thus, to find all the error keys for a checker, you usually
need to examine its own \code{messages.properties} file and that of
\code{basetype}.

If a checker produces a warning/error and you want to determine its message
key, you can re-run the checker, passing the the \code{-Anomsgtext}
command-line option (Section~\ref{debugging-options}).


\section{\code{@AssumeAssertion} string in an \<assert> message\label{assumeassertion}}

\begin{sloppypar}
You can suppress a warning by \<assert>ing that some property is true, and
placing the string \<@AssumeAssertion(\emph{warningkey})> in the assertion
message.
\end{sloppypar}

For example, in this code:

\begin{Verbatim}
  assert x != null : "@AssumeAssertion(nullness)";
  ... x.f ...
\end{Verbatim}

\noindent
the Nullness Checker assumes that \<x> is non-null from the \<assert>
statement forward, and so the expression \<x.f> cannot throw a null pointer
exception.

The \<assert> expression must be an expression that would affect flow-sensitive
type qualifier refinement (Section~\ref{type-refinement}), if the
expression appeared in a conditional test.  Each type system has its own
rules about what type refinement it performs.

The warning key is exactly as in the \<@SuppressWarnings> annotation
(Section~\ref{suppresswarnings-annotation}).  The same good practices apply
as for \<@SuppressWarnings> annotations, such as writing a comment
justifying why the assumption is safe
(Section~\ref{suppresswarnings-best-practices}).

%% Redundant.
% If the string \<@AssumeAssertion(\emph{warningkey})> does not appear in the
% assertion message, then the Checker Framework treats the assertion as
% being used for defensive programming.  That is, the programmer believes
% that the assertion might fail at run time, so the Checker Framework should
% not make any inference, which would not be justified.

%% Users should never see assertions anyway -- they are for programmers.
% A downside of putting the string in the assertion message is that if the
% assertion ever fails, then a user might see the string and be confused.
% This should never be a problem, since
% the programmer should write the string should only if the programmer has
% reasoned that the
% assertion can never fail.

% (Another way of stating the Nullness Checker's use of assertions is as an
% additional caveat to the guarantees provided by a checker
% (Section~\ref{checker-guarantees}).  The Nullness Checker prevents null
% pointer errors in your code under the assumption that assertions are
% enabled, and it does not guarantee that all of your assertions succeed.)


\subsection{Suppressing warnings and defensive programming\label{defensive-programming}}

This section explains the distinction between two different uses for
assertions (and for related methods like JUnit's \<Assert.assertNotNull>).

Assertions are commonly used for two distinct purposes:  documenting how
the program works and debugging the program when it does not work
correctly.  By default, the Checker Framework assumes that each assertion
is used for debugging:  the assertion might fail at run time, and the programmer
wishes to be informed at compile time about such run-time errors.  On the
other hand, if you write the \<@AssumeAssertion> string in the \<assert>
message, then the Checker Framework assumes that you have used some other
technique to verify that the assertion can never fail at run time, so the
checker assumes the assertion passes and does not issue a warning.

Distinguishing the purpose of each assertion is important for precise
type-checking.  
% In particular, the Checker Framework would miss many errors
% if it assumed that every assertion succeeds at run time.
Suppose that a
programmer encounters a failing test, adds an assertion to aid debugging, and fixes the
test.  The programmer leaves the assertion in the program if the programmer
is worried that the program might fail in a similar way in the future.  
% The assertion indicates the potential for failure at this point in the code.
The Checker Framework should not assume that the assertion succeeds ---
doing so would defeat the very purpose of the Checker Framework, which is
to detect errors at compile time and prevent them from occurring at run
time.

On the other hand, assertions sometimes document facts that a programmer
has independently verified to be true, and the Checker Framework can
leverage these assertions in order to avoid issuing false positive
warnings.  The progammer marks such assertions with the \<@AssumeAssertion>
string in the \<assert> message.  Only do so if you are sure
that the assertion always succeeds at run time.


% In each case, the assertion or method indicates an application invariant --- a
% fact that should always be true.  There are two distinct reasons a
% programmer may have written the invariant, depending on whether the
% programmer is 100\% sure that the application invariant holds.
% 
% \begin{enumerate}
% \item
%   A programmer might write it as \textbf{defensive programming}.  This causes
%   the program to throw an exception, which is useful for debugging because
%   it gives an earlier run-time indication of the error.
%   A programmer would use an assertion in this way if the programmer is not
%   100\% sure that the application invariant holds.
% 
%   % , or even to document what the program
%   % is intended to do.
% 
% \item
%   A programmer might write it to \textbf{suppress} false positive
%   \textbf{warning messages} from a checker.  A programmer would use an
%   assertion this way if the programmer is 100\% sure that the application
%   invariant holds, and the reference can never be null at run time.
% 
% \end{enumerate}


Sometimes methods such as
\refmethod{checker/nullness}{NullnessUtils}{castNonNull}{-T-} are used
instead of assertions.  Just as for assertions, you can treat them as
debugging aids or as documentation.
If you know that a particular codebase uses
a nullness-checking method not for defensive programming but to indicate
facts that are guaranteed to be true (that is, these assertions will never
fail at run time), then you can suppress
warnings related to it.
Annotate its definition just as
\refmethod{checker/nullness}{NullnessUtils}{castNonNull}{-T-} is annotated (see the
source code for the Checker Framework).
% TODO:
% For an assert statement, XXXXX.
Also, be sure to document the intention in the method's Javadoc, so that
programmers do not
accidentally misuse it for defensive programming.


If you are annotating a codebase that already contains precondition checks,
such as:

\begin{Verbatim}
  public String get(String key, String def) {
    checkNotNull(key, "key"); //NOI18N
    ...
  }
\end{Verbatim}

\noindent
then you should mark the appropriate parameter as \<@NonNull> (which is the
default).  This will prevent the checker from issuing a warning about the
\<checkNotNull> call.



\section{\code{-AsuppressWarnings} command-line option\label{suppresswarnings-command-line}}

Supplying the \<-AsuppressWarnings> command-line option is equivalent to
writing a \<@SuppressWarnings> annotation on every class that the compiler
type-checks.  The argument to \<-AsuppressWarnings> is a comma-separated
list of warning suppression keys, as in
\<-AsuppressWarnings=purity,uninitialized>.

When possible, it is better to write a \<@SuppressWarnings> annotation with a
smaller scope, rather than using the \<-AsuppressWarnings> command-line option.


\section{\code{-AskipUses} and \code{-AonlyUses} command-line options\label{askipuses}}

You can suppress all errors and warnings at all \emph{uses} of a given
class, or suppress all errors and warnings except those at uses of a given
class.  (The class itself is still type-checked, unless you also use
the \code{-AskipDefs} or \code{-AonlyDefs} command-line option, see~\ref{askipdefs}).

Set the \code{-AskipUses} command-line option to a
regular expression that matches class names (not file names) for which warnings and errors
should be suppressed.
Or, set the \code{-AonlyUses} command-line option to a
regular expression that matches class names (not file names) for which warnings and errors
should be emitted; warnings about uses of all other classes will be suppressed.

For example, suppose that you use
``{\codesize\verb|-AskipUses=^java\.|}'' on the command line
(with appropriate quoting) when invoking
\code{javac}.  Then the checkers will suppress all warnings related to
classes whose fully-qualified name starts with \codesize\verb|java.|, such
as all warnings relating to invalid arguments and all warnings relating to
incorrect use of the return value.

To suppress all errors and warnings related to multiple classes, you can use
the regular expression alternative operator ``\code{|}'', as in
``{\codesize\verb+-AskipUses="java\.lang\.|java\.util\."+}'' to suppress
all warnings related to uses of classes belong to the \code{java.lang} or
\code{java.util} packages.

You can supply both \code{-AskipUses} and \code{-AonlyUses}, in which case
the \code{-AskipUses} argument takes precedence, and \code{-AonlyUses} does
further filtering but does not add anything that \code{-AskipUses} removed.

Warning:  Use the \code{-AonlyUses} command-line option with care,
because it can have unexpected results.  For example, if the
given regular expression does not match classes in the JDK, then the
Checker Framework will suppress every warning that involves a JDK class
such as \<Object> or \<String>.  The meaning of \code{-AonlyUses} may be
refined in the future.  Oftentimes \code{-AskipUses} is more useful.

% The desired meaning of -AonlyUses is tricky, because what is a "use"?
% Maybe only check calls of methods on the class (though don't check
% argument expressions) and field accesses, but nothing else (such as
% extends clauses that happen to use the class).  But then we would also
% want to suppress warnings related to assignments where a method use or
% field access is the right-hand side.  I'm going to punt on this for now.


\section{\code{-AskipDefs} and \code{-AonlyDefs} command-line options\label{askipdefs}}

You can suppress all errors and warnings in the \emph{definition} of a given
class, or suppress all errors and warnings except those in the definition
of a given class.  (Uses of the class are still type-checked, unless you also use
the \code{-AskipUses} or \code{-AonlyUses} command-line option,
see~\ref{askipuses}).

Set the \code{-AskipDefs} command-line option to a
regular expression that matches class names (not file names) in whose definition warnings and errors
should be suppressed.
Or, set the \code{-AonlyDefs} command-line option to a
regular expression that matches class names (not file names) whose
definitions should be type-checked.

For example, if you use
``{\codesize\verb|-AskipDefs=^mypackage\.|}'' on the command line
(with appropriate quoting) when invoking
\code{javac}, then the definitions of 
classes whose fully-qualified name starts with \codesize\verb|mypackage.|
will not be checked.

If you supply both \code{-AskipDefs} and \code{-AonlyDefs}, then
\code{-AskipDefs} takes precedence.

Another way not to type-check a file is not to pass it on the compiler
command-line:  the Checker Framework type-checks only files that are passed
to the compiler on the command line, and does not type-check any file that
is not passed to the compiler.  The \code{-AskipDefs} and \code{-AonlyDefs}
command-line options
are intended for situations in which the build system is hard to understand
or change.  In such a situation, a programmer may find it easier to supply
an extra command-line argument, than to change the set of files that is
compiled.

A common scenario for using the arguments is when you are starting out by
type-checking only part of a legacy codebase.  After you have verified the
most important parts, you can incrementally check more classes until you
are type-checking the whole thing.


\section{\code{-Alint} command-line option\label{alint}}

\label{lint-options}

The \code{-Alint} option enables or disables optional checks, analogously to
javac's \code{-Xlint} option.
Each of the distributed checkers supports at least the following lint options:

% For the current list of lint options supported by all checkers, see
% method BaseTypeChecker.getSupportedLintOptions().

% For the per-checker list, search for "@SupportedLintOptions" in the
% checker implementations.


\begin{itemize}

\item
  \code{cast:unsafe} (default: on) warn about unsafe casts that are not
  checked at run time, as in \code{((@NonNull String) myref)}.  Such casts
  are generally not necessary when flow-sensitive local type refinement is
  enabled.

\item
  \code{cast:redundant} (default: on) warn about redundant
  casts that are guaranteed to succeed at run time,
  as in \code{((@NonNull String) "m")}.  Such casts are not necessary,
  because the target expression of the cast already has the given type
  qualifier.

\item
  \code{cast} Enable or disable all cast-related warnings.

\item
\begin{sloppypar}
  \code{all} Enable or disable all lint warnings, including
  checker-specific ones if any.  Examples include \code{redundantNullComparison} for the
  Nullness Checker (see Section~\ref{lint-nulltest-section}) and \<dotequals> for
  the Interning Checker (see Section~\ref{lint-dotequals}).  This option
  does not enable/disable the checker's standard checks, just its optional
  ones.
\end{sloppypar}

\item
  \code{none} The inverse of \<all>:  disable or enable all lint warnings,
  including checker-specific ones if any.

\end{itemize}

% This syntax is different from -Xlint that uses a colon instead of an
% equals sign, because javac forces the use of the equals sign.

\noindent
To activate a lint option, write \code{-Alint=} followed by a
comma-delimited list of check names.  If the option is preceded by a
hyphen (\code{-}), the warning is disabled.  For example, to disable all
lint options except redundant casts, you can pass
\code{-Alint=-all,cast:redundant} on the command line.

Only the last \code{-Alint} option is used; all previous \code{-Alint}
options are silently ignored.  In particular, this means that \<-Alint=all
-Alint=cast:redundant> is \emph{not} equivalent to
\code{-Alint=-all,cast:redundant}.


\section{No \code{-processor} command-line option\label{no-processor}}

You can also compile parts of your code without use of the
\code{-processor} switch to \code{javac}.  No checking is done during
such compilations.

\section{Checker-specific mechanisms\label{checker-specific-suppression}}

Finally, some checkers have special rules.  For example, the Nullness
checker (Chapter~\ref{nullness-checker}) uses
the special \<castNonNull> method to suppress warnings
(Section~\ref{suppressing-warnings-with-assertions}).
This manual also explains special mechanisms for
suppressing warnings issued by the Fenum Checker
(Section~\ref{fenum-suppressing}) and the Units Checker
(Section~\ref{units-suppressing}).



\htmlhr
\chapter{Handling legacy code\label{legacy-code}}

Section~\ref{get-started-with-legacy-code} describes a methodology for
applying annotations to legacy code.  This chapter tells you what to do if,
for some reason, you cannot change your code in such a way as to eliminate
a checker warning.

Also recall that you can convert checker errors into warnings via the
\code{-Awarns} command-line option; see Section~\ref{checker-options}.


\section{Checking partially-annotated programs:  handling unannotated code\label{unannotated-code}}

Sometimes, you wish to type-check only part of your program.
You might focus on the most mission-critical or error-prone part of your
code.  When you start to use a checker, you may not wish to annotate
your entire program right away.
% Not having source code is *not* a reason.
You may not have
enough knowledge to annotate poorly-documented libraries that your program uses.

If annotated code uses unannotated code, then the checker may issue
warnings.  For example, the Nullness Checker (Chapter~\ref{nullness-checker}) will
warn whenever an unannotated method result is used in a non-null context:

\begin{Verbatim}
  @NonNull myvar = unannotated_method();   // WARNING: unannotated_method may return null
\end{Verbatim}

If the call \emph{can} return null, you should fix the bug in your program by
removing the \refqualclass{checker/nullness/qual}{NonNull} annotation in your own program.

If the library call \emph{never} returns null,
there are several ways to eliminate the compiler warnings.
\begin{enumerate}
\item Annotate \code{unannotated\_method} in full.  This approach provides
  the strongest guarantees, but may require you to annotate additional
  methods that \code{unannotated\_method} calls.  See
  Chapter~\ref{annotating-libraries} for a discussion of how to annotate
  libraries for which you have no source code.
\item Annotate only the signature of \code{unannotated\_method}, and
  suppress warnings in its body.  Two ways to suppress the warnings are via a
  \code{@SuppressWarnings} annotation or by not running the checker on that
  file (see Section~\ref{suppressing-warnings}).
\item Suppress all warnings related to uses of \code{unannotated\_method}
  via the \code{skipUses} processor option
  (see Section~\ref{askipuses}).
  Since this can suppress more warnings than you may expect,
  it is usually better to annotate at least the method's signature.  If you
  choose the boundary between the annotated and unannotated code wisely,
  then you only have to annotate the signatures of a limited number of
  classes/methods
  (e.g., the public interface to a library or package).

\end{enumerate}

Chapter~\ref{annotating-libraries} discusses adding annotations to
signatures when you do not have source code available.
Section~\ref{suppressing-warnings} discusses suppressing warnings.


If you annotate a third-party library, please share it with us so that we
can distribute the annotations with the Checker Framework; see
Section~\ref{reporting-bugs}.


\section{Backward compatibility with earlier versions of Java\label{backward-compatibility}}

Sometimes, your code needs to be \emph{compiled} by people who are using a
Java 5/6/7 compiler, which does not support type annotations.
You can handle this situation by writing annotations in comments (Sections~\ref{annotations-in-comments}--\ref{uncommenting-annotations}).

If your code just needs to be \emph{run} by people who are not using a Java
8 JVM, supply an appropriate \<-target> command-line option to javac.  As
discussed in Section~\ref{no-modular-type-checking-java7-jvm}, the
disadvantage is that this makes it more difficult for clients of your
library to use pluggable type-checking to verify their own code against the
\<.class> or \<.jar> files that you supply;
Section~\ref{declaration-annotations-for-java7} gives a partial solution.


\subsection{Annotations in comments\label{annotations-in-comments}}

A Java 4 compiler does not permit use of
annotations.
A Java 5/6/7 compiler only permits annotations on
declarations --- it does not permit annotations on generic arguments,
casts, \<extends> clauses, method receivers, etc.

So that your code can be compiled by any Java compiler (for any version of
the Java language), you may write any single annotation inside a
\code{/*}\ldots\code{*/} Java comment, as in \code{List</*@NonNull*/ String>}.
The Checker Framework compiler treats the code exactly as if you had not written the
\code{/*} and \code{*/}.
In other words, the Checker Framework compiler will recognize the
annotation (when it is targeting a Java 8 or later JVM),
but your code will still compile with any Java compiler.

%% This is true, but obvious and not important enough to take up space in
%% the manual.
% In a single program, you may write some annotations in comments, and others
% outside of comments.  There is not much point in doing so, however.

By default, the Checker Framework compiler ignores any comment that contains spaces at the
beginning or end, or between the \code{@} and the annotation name.
In other words, it reads \code{/*@NonNull*/} as an annotation but ignores
\code{/* @NonNull*/} and \code{/*@ NonNull*/} and \code{/*@NonNull */}.
This
feature enables backward compatibility with code that contains comments
that start with \code{@} but are not annotations.  (The
ESC/Java~\cite{FlanaganLLNSS02}, JML~\cite{LeavensBR2006:JML}, and
Splint~\cite{Evans96} tools all use ``\code{/*@}'' or ``\code{/*~@}'' as a
comment marker.)
Compiler flag
\code{-XDTA:spacesincomments} causes the compiler to parse annotation comments
even when they contain spaces.  You may need to use
\code{-XDTA:spacesincomments} if you use Eclipse's ``Source $>$ Correct
Indentation'' command, since it inserts space in comments.  But the
annotation comments are less readable with spaces, so it's even better to disable
inserting spaces:  in the Formatter preferences, in the Comments tab,
unselect the ``enable block comment formatting'' checkbox.

Compiler flag \code{-XDTA:noannotationsincomments} causes the compiler
to ignore annotation comments.  With this compiler flag, the 
Checker Framework compiler behaves like a standard Java 8 compiler that does
not support annotations in comments.  If your code already contains
comments of the form \</*@...*/> that look like type annotations, and
you want the Checker Framework compiler not to try to interpret them,
then you can either selectively add spaces to the comments or use
\code{-XDTA:noannotationsincomments} to turn off all annotation
comments.

\textbf{Note:} Annotations in comments is a feature of the javac compiler
that is
distributed along with the Checker Framework.  It is \emph{not}
supported by the mainline OpenJDK javac.  This is the key
difference between the Checker Framework compiler and the OpenJDK compiler.


\subsubsection{Annotations in comments do not appear in Java 5/6/7 \code{.class} files\label{annotations-in-java7-class-files}}

The Checker Framework compiler ignores annotations in comments when
targeting a Java 5/6/7 JVM, for example when the \<-target 7> command-line
option is supplied.

It would be possible for the Checker Framework compiler to read the
annotations in comments and place them in the Java 5/6/7 \<.class> file so
that they are available when type-checking client code.  However, this
would have two problems.  First, it would only be use useful to the Checker
Framework compiler, because a standard Java 8 compiler will not look for
type annotations in Java 5/6/7 bytecode.  Second, the type annotations
make reference to parts of the Java 8 JDK, such as
\sunjavadoc{java/lang/annotation/ElementType.html\#TYPE\_USE}{ElementType.TYPE\_USE}.
% What are the exact consequences?  A warning or a crash?
Therefore, trying to run the \<.class> file on a Java 5/6/7 JVM
would cause warnings or crashes.


\subsection{Import statements and receiver parameters in comments\label{receivers-and-imports-in-comments}}

There is a more powerful mechanism that permits arbitrary code to be
written in a comment.  Format the comment as ``\code{/*>>>}\ldots\code{*/}'',
with the first three characters of the comment being greater-than signs. As
with annotations in comments, the commented code is ignored by ordinary
compilers but is treated like code by the
Checker Framework compiler.

This mechanism is intended for two purposes.
First, it supports the receiver (\<this> parameter) syntax.  For example,
to specify a method that does not modify its receiver:

\begin{Verbatim}
public boolean method1(/*>>> @ReadOnly MyClass this*/) { ... }
public boolean method2(/*>>> @ReadOnly MyClass this, */ String argument) { ... }
\end{Verbatim}

Second, it can be used for import statements:

\begin{Verbatim}
/*>>>
import org.checkerframework.checker.nullness.qual.*;
import org.checkerframework.checker.regex.qual.*;
*/
\end{Verbatim}

\noindent
If the import statements are \emph{not} commented out, then when compiling
it is necessary to have the annotation definitions (e.g., the
\code{checker.jar} or \code{checker-qual.jar} file) on the classpath.
Commenting out the import statements also eliminates Eclipse
warnings about unused import statements, if all uses of the imported
qualifier are themselves in comments and thus invisible to Eclipse.

A third use is for writing multiple annotations inside one
comment, as in \code{/*>>> @NonNull @Interned */ String s;}.
However, it is better style to write multiple annotations each
inside its own comment, as in \</*@NonNull*/ /*@Interned*/ String s;>.

It would be possible to abuse the \code{/*>>>...*/} mechanism to inject
code only when using
the Checker Framework compiler.  Doing so is not a sanctioned use of the
mechanism.


\subsection{Migrating away from annotations in comments\label{uncommenting-annotations}}

Suppose that your codebase currently uses annotations in comments, but you
wish to remove the comment characters around your annotations, because in
the future you will use only compilers that support type annotations and
your code will only run on Java 8 or later JVMs.
This Unix command removes
the comment characters, for all Java files in the current
working directory or any subdirectory.

\begin{Verbatim}
   find . -type f -name '*.java' -print \
     | xargs grep -l -P '/\*\s*@([^ */]+)\s*\*/' \
     | xargs perl -pi.bak -e 's|/\*\s*@([^ */]+)\s*\*/|@\1|g'
\end{Verbatim}

You can customize this command:
\begin{itemize}
\item
To process comments with embedded spaces and asterisks, change
two instances of ``\verb|[^ */]|'' to ``\verb|[^/]|''.
\item
To ignore comments with leading or trailing spaces, remove the four
instances of ``\verb|\s*|''.
\item
  To not make backups, remove ``\verb|.bak|''.
\end{itemize}

% TODO: adapt it to do so.
The command does not handle the \code{>>>} comments; you will need to
adapt the above command to do so, or remove them in another way.


\subsection{No modular type-checking when targeting Java 5/6/7\label{no-modular-type-checking-java7-jvm}}

The Checker Framework's type annotations utilize a Java 8 feature that
allows them to be placed on any type use, including generic type parameters
as in \code{List<@NonNull String>}.  A downside is that use of these type
annotations creates a dependency on Java 8, which means that the compiled
program requires a Java 8 or later JDK at run time. 

To ensure that your program can run on a Java 5/6/7 JVM, use a command-line
option such as \<-target 7> when doing normal compilation to produce
classfiles.  Before doing so, you will do pluggable typechecking, using the
\<-target 8> command-line option (or no \<-target> command-line option) to
javac; you may wish to supply the \<-proc:only> command-line argument so
that the type-checking step does not overwrite existing classfiles.

Here are the disadvantages of this approach:

\begin{itemize}
\item
It produces classfiles that contain no trace of your type annotations.
This means that modular typechecking (also known as separate compilation)
is not possible.

You need to compile your entire application every time you
do pluggable type-checking, rather than just compiling a subset of the
files.  Furthermore, clients of your code cannot do pluggable
type-checking to verify that they are using your code correctly, unless
they re-compile your code (or at least all the interfaces that they use)
every time that they compile their own.

\item
It makes pluggable type-checking a
different step than ``real'' compilation, rather than both happening at the
same time.  You will do pluggable type-checking first, and when it works or
when you want to create a binary to distribute to others, you will compile
with an ordinary Java compiler.
\end{itemize}

One way to enable clients to do pluggable type-checking is to provide a
version of your library compiled for Java 8 or later, with the type
annotations.  Clients will do type-checking against this version of the
library, but will do normal compilation and execution using the Java 5, 6,
or 7 version of your library.

Section~\ref{declaration-annotations-for-java7} gives an alternative
approach with its own advantages and disadvantages.


\subsection{Distributing declaration annotations instead of type annotations\label{declaration-annotations-for-java7}}

If it is important to you to distribute Java 5/6/7 classfiles against which
clients can do some type-checking, this section gives a way to do so.

The idea is to 
use annotations that are Java 7 declaration annotations.
This approach requires you to use annotations that are declared in
different packages than usual and that have slightly different names.

\begin{itemize}
\item
At code locations that are legal for both declaration and type
annotations (such as for fields, method returns, and method parameters),
write annotations normally (not in comments).
\item
At locations where a declaration annotation is not permitted
(such as generic type parameters and \<extends> clauses), write
annotations in comments.
\end{itemize}

Here are some disadvantages of this approach:

\begin{itemize}
\item
  You need to use nonstandard names for
  some annotations, and to remember which annotations to write in comments
  and which to write normally.
\item
  It produces classfiles that contain only some of your type annotations
  --- the ones that were not written in comments.
  If your code uses type annotations at 
  locations such as generic type parameters and \<extends> clauses, then
  modular type-checking will not observe them;
  the implications of that were described above.
\end{itemize}


Here are more details about the approach.
Suppose you wish to run the Nullness Checker using Java 6 or 7
declaration annotations rather than type annotations.  You have two options.

\begin{enumerate}
\item
At locations where declaration annotations are possible,
use aliased annotations from other projects.  For example, the aliased
annotations for the Nullness Checker are listed in
Section~\ref{nullness-related-work}.

At locations where only type annotations are possible, use the
``\<*Type>'' compatibility annotations from package
\<org.checkerframework.checker.nullness.compatqual>
in comments.  For example, the Nullness Checker
declares these declaration annotations:
\refqualclass{checker/nullness/compatqual}{NullableType},
\refqualclass{checker/nullness/compatqual}{NonNullType},
\refqualclass{checker/nullness/compatqual}{PolyNullType},
\refqualclass{checker/nullness/compatqual}{MonotonicNonNullType}, and
\refqualclass{checker/nullness/compatqual}{KeyForType}.

\item
At locations where declaration annotations are possible,
use ``\<*Decl>'' compatibility annotations from package
\<org.checkerframework.checker.nullness.compatqual>.
For example, the Nullness Checker
declares these declaration annotations:
\refqualclass{checker/nullness/compatqual}{NullableDecl},
\refqualclass{checker/nullness/compatqual}{NonNullDecl},
\refqualclass{checker/nullness/compatqual}{PolyNullDecl},
\refqualclass{checker/nullness/compatqual}{MonotonicNonNullDecl}, and
\refqualclass{checker/nullness/compatqual}{KeyForDecl}.

At locations where only type annotations are possible, use the regular
Checker Framework type annotations in comments.
\end{enumerate}

Notice that in each case, the declaration annotations and type annotations
have distinct names.  This enables a programmer to import both sets of
annotations without a name conflict.  But, you must remember to use the
correct name, depending on where the annotations are written.

Eventually, when backward compatibility with Java 7 and earlier is not important,
you should refactor your codebase to use only the regular Checker Framework
annotations, and not to write them in comments.




% LocalWords:  quals skipUses un AskipUses Alint annotationname javac's Awarns
% LocalWords:  Xlint dotequals castNonNull XDTA spacesincomments Formatter jsr
% LocalWords:  unselect checkbox classpath Djsr bak Nullness nullness java lang
% LocalWords:  checkername util myref nulltest html ESC buildfile mynifty Fenum
% LocalWords:  MyNiftyChecker messagekey basetype uncommenting Anomsgtext '
% LocalWords:  AskipDefs mypackage jsr308 Djsr308 Makefile PLXCOMP expr '
%  LocalWords:  TODO AsuppressWarnings AssumeAssertion AonlyUses AonlyDefs
%  LocalWords:  ing warningkey redundantNullComparison ' ' qual proc Decl
%  LocalWords:  noannotationsincomments lastsinglesuppression classfiles
%%  LocalWords:  AshowSuppressWarningKeys NullableType NonNullType
%%  LocalWords:  PolyNullType MonotonicNonNullType KeyForType NullableDecl
%%  LocalWords:  NonNullDecl PolyNullDecl MonotonicNonNullDecl KeyForDecl
%%  LocalWords:  refactored
