\htmlhr
\chapter{Frequently Asked Questions (FAQs)\label{faq}}

These are some common questions about the Checker Framework and about
pluggable type-checking in general.  Feel free to suggest improvements to
the answers, or other questions to include here.

% Not supported by Hevea, so don't bother; instead do by hand:
% \minitoc

%BEGIN LATEX
~
%END LATEX

%BEGIN LATEX
\newcommand{\faqtocpara}[1]{\paragraph{#1} ~}
%END LATEX
%HEVEA \newcommand{\faqtocpara}[1]{\textbf{#1}}


\noindent
\textbf{Contents:}

\faqtocpara{\ref{faq-motivation-section}: Motivation for pluggable type-checking}
\\ \ref{never-make-type-errors}: I don't make type errors, so would pluggable type-checking help me?
\\ \ref{faq-qualifiers-vs-subclasses}: When should I use type qualifiers, and when should I use subclasses?

\faqtocpara{\ref{faq-getting-started-section}: Getting started}
\\ \ref{faq-annotate-existing-program}: How do I get started annotating an existing program?
\\ \ref{faq-first-checker}: Which checker should I start with?

\faqtocpara{\ref{faq-usability-section}: Usability of pluggable type-checking}
\\ \ref{faq-ease-of-use}: Are type annotations easy to read and write?
\\ \ref{faq-code-clutter}: Will my code become cluttered with type annotations?
\\ \ref{faq-slowdown}: Will using the Checker Framework slow down my program?  Will it slow down the compiler?
\\ \ref{faq-shorten-command-line}: How do I shorten the command line when invoking a checker?

\faqtocpara{\ref{faq-warnings-section}: How to handle warnings}
\\ \ref{faq-handling-warnings}: What should I do if a checker issues a warning about my code?
\\ \ref{faq-interpreting-warnings}: What does a certain Checker Framework warning message mean?
\\ \ref{faq-no-absolute-guarantee}: Can a pluggable type-checker guarantee that my code is correct?
\\ \ref{faq-concurrency}: What guarantee does the Checker Framework give for concurrent code?
\\ \ref{faq-awarns}: How do I make compilation succeed even if a checker issues errors?
\\ \ref{faq-100-warnings}: Why does the checker always say there are 100 errors or warnings?
\\ \ref{faq-type-i-did-not-write}: Why does the Checker Framework report an error regarding a type I have not written in my program?
\\ \ref{faq-run-time-checking}: How can I do run-time monitoring of properties that were not statically checked?

\faqtocpara{\ref{faq-syntax-section}: Syntax of type annotations}
\\ \ref{faq-receiver}: What is a ``receiver''?
\\ \ref{faq-annotation-after-type}: What is the meaning of an annotation after a type, such as \<@NonNull Object @Nullable>?
\\ \ref{faq-array-syntax-meaning}: What is the meaning of array annotations such as \<@NonNull Object @Nullable []>?
\\ \ref{faq-type-qualifier-on-class-declaration}: What is the meaning of a type qualifier at a class declaration?
\\ \ref{faq-no-annotation-on-types-and-declarations}: Why shouldn't a qualifier apply to both types and declarations?

\faqtocpara{\ref{faq-semantics-section}: Semantics of type annotations}
\\ \ref{faq-list-map-nonnull-typeargs}: Why are the type parameters to \<List> and \<Map> annotated as \<@NonNull>?
\\ \ref{faq-typestate}: How can I handle typestate, or phases of my program with different data properties?
\\ \ref{faq-implicit-bounds}: Why are explicit and implicit bounds defaulted differently?

\faqtocpara{\ref{faq-create-a-checker-section}: Creating a new checker}
\\ \ref{faq-create-a-checker}: How do I create a new checker?
\\ \ref{faq-declarative-syntax-for-type-rules}: Why is there no declarative syntax for writing type rules?

\faqtocpara{\ref{faq-other-tools-section}: Relationship to other tools}
\\ \ref{faq-type-checking-vs-bug-detectors}: Why not just use a bug detector (like FindBugs)?
\\ \ref{faq-jml}: How does pluggable type-checking compare with JML?
\\ \ref{faq-checker-framework-part-of-java}: Is the Checker Framework an official part of Java?
\\ \ref{faq-jsr-305}: What is the relationship between the Checker Framework and JSR 305?
\\ \ref{faq-jsr-308}: What is the relationship between the Checker Framework and JSR 308?


\section{Motivation for pluggable type-checking\label{faq-motivation-section}}

\subsection{I don't make type errors, so would pluggable type-checking help me?\label{never-make-type-errors}}

Occasionally, a developer says that he makes no errors that type-checking
could catch, or that any such errors are unimportant because they have low
impact and are easy to fix.  When I investigate the claim, I invariably
find that the developer is mistaken.

Very frequently, the developer has underestimated what type-checking can
discover.  Not every type error leads to an exception being thrown; and
even if an exception is thrown, it may not seem related to classical types.
Remember that a type system can discover
null pointer dereferences,
incorrect side effects, 
security errors such as information leakage or SQL injection,
partially-initialized data,
wrong units of measurement,
and many other errors.
Every programmer makes errors sometimes and works with other people
who do.
Even where type-checking does not discover a
problem directly, it can indicate code with bad smells, thus revealing
problems, improving documentation, and making future maintenance easier.

There are other ways to discover errors, including extensive testing and
debugging.  You should continue to use these.
But type-checking is a good complement to these.  Type-checking is more
effective for some problems, and less effective for other problems.  It can
reduce (but not eliminate) the time and effort that you spend on other
approaches.  There are many important errors that type-checking and other
automated approaches cannot find; pluggable type-checking gives you more
time to focus on those.


\subsection{When should I use type qualifiers, and when should I use subclasses?\label{faq-qualifiers-vs-subclasses}}

In brief, use subtypes when you can, and use type qualifiers when you cannot
use subtypes.
For more details, see Section~\ref{when-to-use-type-qualifiers}.



\section{Getting started\label{faq-getting-started-section}}

\subsection{How do I get started annotating an existing program?\label{faq-annotate-existing-program}}

See Section~\ref{get-started-with-legacy-code}.


\subsection{Which checker should I start with?\label{faq-first-checker}}

You should start with a property that matters to you.  Think about what
aspects of your code cause the most errors, or cost the most time during
maintenance, or are the most common to be incorrectly-documented.  Focusing
on what you care about will give you the best benefits.

When you first start out with the Checker Framework, it's usually best to
get experience with an existing type-checker before you write your own new
checker.

Many users are tempted to start with the
\ahrefloc{nullness-checker}{Nullness Checker} (see
\chapterpageref{nullness-checker}), since null pointer errors are common
and familiar.  The Nullness Checker works very well, but be warned of three
facts that make the absence of null pointer exceptions challenging to
verify.

\begin{enumerate}
\item
  Dereferences happen throughout your codebase, so there are a lot of
  potential problems.  By contrast, fewer lines of code are related to
  locking, regular expressions, etc., so those properties are easier to
  check.
\item
  Programmers use \<null> for many different purposes.  More seriously,
  programmers write run-time tests against \<null>, and those are difficult
  for any static analysis to capture.
\item
  The Nullness Checker interacts with initialization and map keys.
\end{enumerate}

If null pointer exceptions are most important to you, then by all means use
the Nullness Checker.  But if you just want to try \emph{some}
type-checker, there are others that are easier to use.

we do not recommend indiscriminately running all the checkers on your code.
The reason is that each one has a cost --- not just at compile time, but
also in terms of code clutter and human time to maintain the annotations.
If the property is important to you, is difficult for people to reason
about, or has caused problems in the past, then you should run that
checker.  For other properties, the benefits may not repay the effort to
use it.  You will be the best judge of this for your own code, of course.

The \ahrefloc{linear-checker}{Linear Checker} (see
\chapterpageref{linear-checker}) has not been extensively tested.
The 
\ahrefloc{igj-checker}{IGJ Checker} (see \chapterpageref{igj-checker}),
\ahrefloc{javari-checker}{Javari Checker} (see
\chapterpageref{javari-checker}), and some of the third-party checkers (see
\chapterpageref{third-party-checkers})
have known bugs that limit their
usability.  (Report the ones that affect you, and the Checker Framework
developers will prioritize fixing them.)


\section{Usability of pluggable type-checking\label{faq-usability-section}}

\subsection{Are type annotations easy to read and write?\label{faq-ease-of-use}}

% This FAQ also appears in the JSR 308 FAQ.
% When I update one, also update the other.

The papers
\href{http://homes.cs.washington.edu/~mernst/pubs/pluggable-checkers-issta2008-abstract.html}{``Practical
  pluggable types for Java''}~\cite{PapiACPE2008}
and
\href{http://homes.cs.washington.edu/~mernst/pubs/pluggable-checkers-icse2011-abstract.html}{``Building
  and using pluggable type-checkers''}~\cite{DietlDEMS2011}
discuss case studies in
which programmers
found type annotations to be natural to read and write.  The code
continued to feel like Java, and the type-checking errors were easy to
comprehend and often led to real bugs.

You don't have to take our word for it, though.  You can try the
Checker Framework for yourself.

The difficulty of adding and verifying annotations depends on your program.
If your program is well-designed and -documented, then skimming the
existing documentation and writing type annotations is extremely easy.
Otherwise, you may find yourself spending a lot of time trying to
understand, reverse-engineer, or fix bugs in your program, and then just a
moment writing a type annotation that describes what you discovered.  This
process inevitably improves your code.  You must decide whether it is a
good use of your time.  For code that is not causing trouble now and is
unlikely to do so in the future (the code is bug-free, and you do not
anticipate changing it or using it in new contexts), then the
effort of writing type annotations for it may not be justified.


\subsection{Will my code become cluttered with type annotations?\label{faq-code-clutter}}

% This FAQ also appears in the JSR 308 FAQ.
% When I update one, also update the other.

In summary:  annotations do not clutter code; they are used much
less frequently than generic types, which Java programmers find acceptable;
and they reduce the overall volume of documentation that a codebase needs.

As with any language feature, it is possible to write ugly code that
over-uses annotations.  However, in normal use, very few annotations need
to be written.  Figure 1 of the paper
\href{http://homes.cs.washington.edu/~mernst/pubs/pluggable-checkers-issta2008-abstract.html}{Practical
  pluggable types for Java}~\cite{PapiACPE2008} reports data for over
350,000 lines of type-annotated code:

\begin{itemize}
\item
    1 annotation per 62 lines for nullness annotations (\<@NonNull>, \<@Nullable>, etc.)
    % (/ (+ 4640 3961 10798) (+ 107 35 167))
\item
    1 annotation per 1736 lines for interning annotations (\<@Interned>)
    % (/ 224048 129)
\item
    1 annotation per 27 lines for immutability annotations (IGJ type system)
    % (/ (+ 6236 18159 30507 8691 59221 26828) (+ 315 1125 1386 384 1815 450))
\end{itemize}

% ICSE 2011 paper says:
% Signature String Checker: less than 1 annotation per 500 lines of code

These numbers are for annotating existing code.  New code that
is written with the type annotation system in mind is cleaner and more
correct, so it requires even fewer annotations.

Each annotation that a programmer writes replaces a sentence or phrase of
English descriptive text that would otherwise have been written in the
Javadoc.  So, use of annotations actually reduces the overall size of the
documentation, at the same time as making it machine-processable
and less ambiguous.


\subsection{Will using the Checker Framework slow down my program?  Will it slow down the compiler?\label{faq-slowdown}}

Using the Checker Framework has no impact on the execution of your program:
the compiler emits the identical bytecodes as the Java 8
compiler and so there is no run-time effect.  Because there is no run-time
representation of type qualifiers, there is no way to use reflection to
query the qualifier on a given object, though you can use reflection to
examine a class/method/field declaration.

Using the Checker Framework does increase compilation time.  In theory it
should only add a few percent overhead, but our current implementation
can double the compilation time --- or more, if you run many pluggable
type-checkers at once.  This is especially true if you run pluggable
type-checking on every file (as we recommend) instead of just on the ones
that have recently changed.
Nonetheless, compilation with pluggable type-checking still feels like
compilation, and you can do it as part of your normal development process.


\subsection{How do I shorten the command line when invoking a checker?\label{faq-shorten-command-line}}

\begin{sloppypar}
The compile options to javac can be a pain to type; for example,
\code{javac -processor org.checkerframework.checker.nullness.NullnessChecker ...}.
See Section~\ref{checker-auto-discovery} for a way to avoid the need for
the \code{-processor} command-line option.
\end{sloppypar}


\section{How to handle warnings and errors\label{faq-warnings-section}}

\subsection{What should I do if a checker issues a warning about my code?\label{faq-handling-warnings}}

For a discussion of this issue, see Section~\ref{handling-warnings}.


\subsection{What does a certain Checker Framework warning message mean?\label{faq-interpreting-warnings}}

Search through this manual for the text of the warning message.
Oftentimes the manual explains it.  If not, ask on the \href{http://groups.google.com/group/checker-framework-discuss}{mailing list}.


\subsection{Can a pluggable type-checker guarantee that my code is correct?\label{faq-no-absolute-guarantee}}

Each checker looks for certain errors.  You can use multiple checkers to
detect more errors in your code, but you will never have a guarantee that
your code is completely bug-free.

If the type-checker issues no warning, then you have a guarantee that your
code is free of some particular error.  There are some limitations to the
guarantee.

Most importantly, if you run a pluggable checker on only part of a program, then
you only get a guarantee that those parts of the program are error-free.
For example, suppose you have type-checked a framework that clients
are intended to extend.  You should recommend that clients
run the pluggable checker.  There is no way to force users to do so, so you
may want to retain dynamic checks or use other mechanisms to detect errors.

Section~\ref{checker-guarantees} states other limitations to a checker's
guarantee, such as regarding concurrency.  Java's type system is also
unsound in certain situations, such as for arrays and casts (however, the
Checker Framework is sound for arrays and casts).  Java uses dynamic checks
is some places it is unsound, so that errors are thrown at run time.  The
pluggable type-checkers do not currently have built-in dynamic checkers to
check for the places they are unsound.
Writing dynamic checkers would be an interesting and valuable project.

Other types of dynamism in a Java application do not jeopardize the
guarantee, because the type-checker is conservative.  For example, at a
method call, dynamic dispatch chooses some implementation of the method,
but it is impossible to know at compile time which one it will be.  The
type-checker gives a guarantee no matter what implementation of the method
is invoked.

% This paragraph is weak.

Even if a pluggable checker cannot give an ironclad
guarantee of correctness, it is still useful.  It can find errors, 
exclude certain types of possible problems (e.g., restricting the
possible class of problems), improve documentation, and increase confidence
in your software.


\subsection{What guarantee does the Checker Framework give for concurrent code?\label{faq-concurrency}}

The Lock Checker (see Chapter~\ref{lock-checker}) offers a way to detect
and prevent certain concurrency errors.


By default, the Checker Framework assumes that the code that it is checking
is sequential:  that is, there are no concurrent accesses from another
thread.  This means that the Checker Framework is unsound for concurrent
code, in the sense that it may fail to warn about a possible null
dereference.  For example, the Nullness Checker issues no warning for this
code:

\begin{Verbatim}
  if (myobject.myfield != null) {
    myobject.myfield.toString();
  }
\end{Verbatim}

\noindent
This code is safe when run on its own.
However, in the presence of multithreading, the call to \<toString> may
fail because another thread may set \<myobject.myfield> to \<null> after
the nullness check in the \<if> condition, but before the \<if> body is
executed.

If you supply the \<-AconcurrentSemantics> command-line option, then the
Checker Framework assumes that any field can be changed at any time.  This
limits the amount of flow-sensitive type qualifier refinement
(Section~\ref{type-refinement}) that the Checker Framework can do.


% If you are concerned about concurrency, then the ``fix''
% of putting data in a local variable doesn't fix the problem,
% just masks it from one particular checker.  This is bad style and may make
% debugging harder rather than easier.
% 
% For instance, suppose you you have
% 
% if (x.val != null) {
%   x.val = x.val + 1;
% }
% 
% which can suffer a null pointer exception if another thread nulls out
% x.val.  The underlying problem is the possible concurrency error:  the user
% should have used locks or some other mechanism to protect access to x.val.
% 
% Changing this to
% 
% myval = x.val;
% if (myval != null) {
%   x.val = myval + 1;
% }
% 
% does not fix the concurrency error, because no locking has been introduced.
% The code still has a data race that can lose updates or corrupt data
% structures.  The code has been transformed so that the Nullness Checker
% does not issue a warning, but this is scant comfort since the code is no
% more correct than it was before.
% 
% If you want to detect concurrency errors, then it is better to use a
% correct checker that is concurrency-aware, rather than an unsound one that
% encourages incorrect workarounds.  Another way to put this is that a static
% checker should encourage better overall design, not just different bad
% designs.


\subsection{How do I make compilation succeed even if a checker issues errors?\label{faq-awarns}}

Section~\ref{running} describes the \<-Awarns> command-line
option that turns checker errors into warnings, so type-checking errors
will not cause \<javac> to exit with a failure status.


\subsection{Why does the checker always say there are 100 errors or warnings?\label{faq-100-warnings}}

By default, javac only reports the first 100 errors or warnings.
Furthermore, once javac encounters an error, it doesn't try compiling any
more files (but does complete compilation of all the ones that it has
started so far).

To see more than 100 errors or warnings, use the javac options \<-Xmaxerrs>
and \<-Xmaxwarns>.  To convert Checker Framework errors into warnings so
that javac will process all your source files, use the option \<-Awarns>.
See Section~\ref{running} for more details.


\subsection{Why does the Checker Framework report an error regarding a type I have not written in my program?\label{faq-type-i-did-not-write}}


Sometimes, a Checker Framework warning message will mention a type you have
not written in your program.  This is typically because a default has been
applied where you did not write a type; see Section~\ref{defaults}.  In
other cases, this is because flow-sensitive type refinement has given an
expression a more specific type than you wrote or than was defaulted; see
Section~\ref{type-refinement}.


\subsection{How can I do run-time monitoring of properties that were not statically checked?\label{faq-run-time-checking}}

Some properties are not checked statically (see
Chapter~\ref{warnings-and-legacy} for reasons that code might not be
statically checked).  In such cases, it would be desirable to check the
property dynamically, at run time.
Currently, the Checker Framework has no support for adding code to perform
run-time checking.

Adding such support would be an interesting and valuable project.
An example would be an option that causes the Checker Framework to
automatically insert a run-time check anywhere that static checking is
suppressed.
% such as casts
If you
are able to add run-time verification functionality, we would gladly
welcome it as a contribution to the Checker Framework.

Some checkers have library methods that you can explicitly insert in your
source code.
Examples include the Nullness Checker's
\refmethod{checker/nullness}{NullnessUtils}{castNonNull}{-T-} method (see
Section~\ref{suppressing-warnings-with-assertions}) and the Regex Checker's
\<RegexUtil> class (see Section~\ref{regexutil-methods}).
But, it would be better to have more general support that does not require
the user to explicitly insert method calls.


\section{Syntax of type annotations\label{faq-syntax-section}}

There is also a separate FAQ for the type annotations syntax
(\url{http://types.cs.washington.edu/jsr308/current/jsr308-faq.html}).


\subsection{What is a ``receiver''?\label{faq-receiver}}

The \emph{receiver} of a method is the \<this> formal parameter, sometimes
also called the ``current object''.  Within the method declaration, \<this>
is used to refer to the receiver formal parameter.  At a method call, the
receiver actual argument is written before the method name.

The method \<compareTo> takes \emph{two} formal parameters.  At a call site
like \<x.compareTo(y)>, the two arguments are \<x> and \<y>.  It is
desirable to be able to annotate the types of both of the formal
parameters, and doing so is supported by both Java's type annotations
syntax and by the Checker Framework.

A type annotation on the receiver is treated exactly like a type annotation
on any other formal parameter.  At each call site, the type of the argument
must be a consistent with (a subtype of or equal to) the declaration of the
corresponding formal parameter.  If not, the type-checker issues a warning.

Here is an example.  Suppose that \<@A Object> is a supertype of \<@B
Object> in the following declaration:

\begin{Verbatim}
  class MyClass {
    void requiresA(@A MyClass this) { ... }
    void requiresB(@B MyClass this) { ... }
  }
\end{Verbatim}

\noindent
Then the behavior of four different invocations is as follows:

\begin{Verbatim}
  @A MyClass myA = ...;
  @B MyClass myB = ...;

  myA.requiresA()    // OK
  myA.requiresB()    // compile-time error
  myB.requiresA()    // OK
  myB.requiresB()    // OK
\end{Verbatim}

The invocation \<myA.requiresB()> does not type-check because the actual
argument's type is not a subtype of the formal parameter's type.

A top-level constructor does not have a receiver.  An inner class
constructor does have a receiver, whose type is the same as the containing
outer class.  The receiver is distinct from the object being constructed.
In a method of a top-level class, the receiver is named \<this>.  In a
constructor of an inner class, the receiver is named \<Outer.this> and the
result is named \<this>.


\subsection{What is the meaning of an annotation after a type, such as \<@NonNull Object @Nullable>?\label{faq-annotation-after-type}}

In a type such as \<@NonNull Object @Nullable []>, it may appear that the
\<@Nullable> annotation is written \emph{after} the type \<Object>.  In
fact, \<@Nullable> modifies \<[]>.  See the next FAQ, about array
annotations (Section~\ref{faq-array-syntax-meaning}).


\subsection{What is the meaning of array annotations such as \<@NonNull Object @Nullable []>?\label{faq-array-syntax-meaning}}

You should parse this as:
(\textbf{\<@NonNull Object>}) (\textbf{\<@Nullable []>}).
Each annotation precedes the component of the type that it qualifies.

Thus,
\<@NonNull Object @Nullable []> is a possibly-null array of non-null
objects.  Note that the first token in the type,
``\<@NonNull>'', applies to the element
type \<Object>, not to the array type as a whole.  The annotation \<@Nullable> applies to the
array (\<[]>).

Similarly,
\<@Nullable Object @NonNull []> is a non-null array of possibly-null
objects.


Some older tools interpret a declaration like \<@NonEmpty String[] var> as
``non-empty array of strings''.  This is in conflict with the Java type
annotations specification, which defines it as meaning ``array of
non-empty strings''.
% (and has since October 2007)
If you use one of these
older tools, you will find this incompatibility confusing.
You will have to live with it until the older
tool is updated to conform to the Java specification, or until you
transition to a newer tool that conforms to the Java specification.


\subsection{What is the meaning of a type qualifier at a class declaration?\label{faq-type-qualifier-on-class-declaration}}

% TODO: use a more realistic example.

Writing an annotation on a class declaration makes that annotation implicit
for all uses of the class (see Section~\ref{effective-qualifier}).  If you
write \<class @MyQual MyClass \ttlcb\ ... \ttrcb>, then every unannotated
use of \<MyClass> is \<@MyQual MyClass>.  A user is permitted to strengthen
the type by writing a more restrictive annotation on a use of MyClass, such
as \<@MyMoreRestrictiveQual MyClass>.


\subsection{Why shouldn't a qualifier apply to both types and declarations?\label{faq-no-annotation-on-types-and-declarations}}

It is bad style for an annotation to apply to both types and declarations.
In other words, every annotation should have a \<@Target> meta-annotation,
and the \<@Target> meta-annotation should list either only declaration
locations or only type annotations.  (It's OK for an annotation to target
both \<ElementType.TYPE\_PARAMETER> and \<ElementType.TYPE\_USE>, but no
other declaration location along with \<ElementType.TYPE\_USE>.)

Sometimes, it may seem tempting for an annotation to apply to both type
uses and (say) method declarations.  Here is a hypothetical example:

\begin{quote}
  ``Each \<Widget> type may have a \<@Version> annotation.
  I wish to prove that versions of widgets don't get assigned to
  incompatible variables, and that older code does not call newer code (to
  avoid problems when backporting).

  A \<@Version> annotation could be written like so:

\begin{Verbatim}
  @Version("2.0") Widget createWidget(String value) { ... }
\end{Verbatim}

\<@Version("2.0")> on the method could mean that the \<createWidget> method
only appears in the 2.0 version.  \<@Version("2.0")> on the return type
could mean that the returned \<Widget> should only be used by code that
uses the 2.0 API of \<Widget>.  It should be possible to specify these
independently, such as a 2.0 method that returns a value that allows the
1.0 API method invocations.''
\end{quote}

Both of these are type properties and should be specified with type
annotations.  No method annotation is necessary or desirable.  The best way
to require that the receiver has a certain property is to use a type
annotation on the receiver of the method.  (Slightly more formally, the
property being checked is compatibility between the annotation on the type
of the formal parameter receiver and the annotation on the type of the
actual receiver.)  If you do not know what ``receiver'' means, see the next
question.


Another example of a type-and-declaration annotation that represents poor
design is JCIP's \<@GuardedBy> annotation~\cite{Goetz2006}.  As discussed
in Section~\ref{jcip-annotations}, it means two different things when
applied to a field or a method.  To reduce confusion and increase
expressiveness, the Lock Checker (see Chapter~\ref{lock-checker}) uses the
\<@Holding> annotation for one of these meanings, rather than overloading
\<@GuardedBy> with two distinct meanings.


\section{Semantics of type annotations\label{faq-semantics-section}}


\subsection{Why are the type parameters to \<List> and \<Map> annotated as \<@NonNull>?\label{faq-list-map-nonnull-typeargs}}

The annotation on \<java.util.Collection> only allows non-null elements:

\begin{Verbatim}
  public interface Collection<E extends @NonNull Object> {
    ...
  }
\end{Verbatim}

\noindent
Thus, you will get a type error if you write code like
\code{Collection<@Nullable Object>}.
A nullable
type parameter is also forbidden for certain other collections, including
\<AbstractCollection>, \<List>, \<Map>, and \<Queue>.

% AbstractCollection has no documentation of its own regarding nullness,
% but it implements Collection.

% The JML specifications of the add() method says
%       @   signals (NullPointerException)
%       @             (* not allowed to add null *);
%       ...
%       @   signals (NullPointerException)
%       @             (* not allowed to add null *);
% In other words, the method might throw NullPointerException, but the JML
% spec does not say under what circumstances.

The \<extends @NonNull Object> bound is a direct consequence of the design
of the collections classes; it merely formalizes the Javadoc specification.
The Javadoc for \<Collection> states:

\begin{quote}
  Some list implementations have restrictions on the elements that they may
  contain. For example, some implementations prohibit null elements, \ldots
\end{quote}

Here are some consequences of the requirement to detect all nullness errors
at compile time.  If even one subclass of a given collection class may
prohibit null, then the collection class and all its subclasses must
prohibit null.  Conversely, if a collection class is specified to accept
null, then all its subclasses must honor that specification.

The Checker Framework's annotations make apparent a flaw in the JDK
design, and helps you to avoid problems that might be caused by that flaw.


\paragraph{Justification from type theory\label{faq-list-map-nonnull-typeargs-junification-from-type-theory}}
Suppose \<B> is a subtype of \<A>.
Then an overriding method in \<B> must have a stronger (or equal) signature
than the overridden method in~\<A>.  In a stronger signature, the formal
parameter types may be supertypes, and the return type may be a subtype.
Here are examples:

\begin{Verbatim}
  class A           {  @NonNull Object Number m1( @NonNull Object arg) { ... } }
  class B extends A { @Nullable Object Number m1( @NonNull Object arg) { ... } } // error!
  class C extends A {  @NonNull Object Number m1(@Nullable Object arg) { ... } } // OK
  class D           { @Nullable Object Number m2(@Nullable Object arg) { ... } }
  class E extends D {  @NonNull Object Number m2(@Nullable Object arg) { ... } } // OK
  class F extends D { @Nullable Object Number m2( @NonNull Object arg) { ... } } // error!
\end{Verbatim}

According to these rules, since some subclasses of \<Collection> do not
permit nulls, then \<Collection> cannot either:

\begin{Verbatim}
  // does not permit null elements
  class PriorityQueue<E> implements Collection<E> {
    boolean add(E);
    ...
  }
  // must not permit null elements, or PriorityQueue would not be a subtype of Collection
  interface Collection<E> {
    boolean add(E);    
    ...
  }
\end{Verbatim}


\paragraph{Justification from checker behavior\label{faq-list-map-nonnull-typeargs-justification-from-behavior}}

Suppose that you changed the bound in the \<Collection> declaration to
\<extends @Nullable Object>.  Then, the checker would issue no warning for
this method:

\begin{Verbatim}
  static void addNull(Collection l) {
    l.add(null);
  }
\end{Verbatim}

\noindent
However, calling this method \emph{can} result in a null pointer exception,
for instance caused by the following code:

\begin{Verbatim}
  addNull(new PriorityQueue());
\end{Verbatim}

\noindent
Therefore, the bound must remain as \<extends @NonNull Object>.

By contrast, this code is OK because \<ArrayList> is documented to support
null elements:

\begin{Verbatim}
  static void addNull(ArrayList l) {
    l.add(null);
  }
\end{Verbatim}

\noindent
Therefore, the upper bound in \<ArrayList> is \<extends @Nullable Object>.
Any subclass of \<ArrayList> must also support null elements.




% Every implementation of List seems to permit null.
% Examples of Collection that do not permit null:
% BlockingQueue family:
%   BlockingQueue, BlockingDeque, ArrayBlockingQueue, DelayQueue, LinkedBlockingDeque, LinkedBlockingQueue, PriorityBlockingQueue, SynchronousQueue 
% PriorityQueue
% probably lots of other queues.

% A similar argument applies to \<Map>.
% For example, \<ConcurrentHashMap> and \<Hashtable> implement \<Map> but do
% not permit \<null> to be used as a key or value.  Therefore, \<Map> must
% not permit \<null> to be used as a key or value


% The Checker Framework is designed to warn you whenever your code might
% throw a null pointer exception.  If you want to be safe, you will never put
% \<null> in a \<List> of unknown provenance, because that \<List> might not
% accept null.



\paragraph{Suppressing warnings\label{faq-list-map-nonnull-typeargs-suppressing-warnings}}

Suppose your program has a list variable, and you know that any list referenced
by that variable will definitely support null.  Then, you can suppress the
warning:

\begin{Verbatim}
  @SuppressWarnings("nullness:generic.argument")
  static void addNull(List l) {
    l.add(null);
  }
\end{Verbatim}

\noindent
You need to use \<@SuppressWarnings("nullness:generic.argument")>
whenever you use a collection that may contain \<null> elements in
contradiction to its documentation.  Fortunately, such uses are relatively
rare.


For more details on suppressing nullness warnings, see
Section~\ref{suppressing-warnings-nullness}.


\subsection{How can I handle typestate, or phases of my program with different data properties?\label{faq-typestate}}

Sometimes, your program works in phases that have different behavior.  For
example, you might have a field that starts out null and becomes non-null
at some point during execution, such as after a method is called.  You can
express this property as follows:

\begin{enumerate}
\item
Annotate the field type as \refqualclass{checker/nullness/qual}{MonotonicNonNull}.
\item
Annotate the method that sets the field as \refqualclass{checker/nullness/qual}{EnsuresNonNull}\<(">\emph{\<myFieldName>}\<")>.
(If method \<m1> calls method \<m2>, which actually sets the field, then
you would probably write this annotation on both \<m1> and \<m2>.)
\item
Annotate any method that depends on the field being non-null as
\refqualclass{checker/nullness/qual}{RequiresNonNull}\<(">\emph{\<myFieldName>}\<")>.
The type-checker will verify that such a method is only called when the
field isn't null --- that is, the method is only called after the setting
method.
\end{enumerate}

You can also use a typestate checker (see
\chapterpageref{typestate-checker}), but they have not been as extensively
tested.


\subsection{Why are explicit and implicit bounds defaulted differently?\label{faq-implicit-bounds}}

The following two bits of code have the same semantics under Java, but are
treated differently by the Checker Framework's CLIMB-to-top defaulting
rules (Section~\ref{climb-to-top}):

\begin{Verbatim}
class MyClass<T> { ... }
class MyClass<T extends Object> { ... }
\end{Verbatim}

The difference is the annotation on the upper bound of the type argument
\<T>.  They are treated in the following.

\begin{Verbatim}
class MyClass<T>  ==  class MyClass<T extends @TOPTYPEANNO Object> { ... }
class MyClass<T extends Object>  ==  class MyClass<T extends @DEFAULTANNO Object>
\end{Verbatim}

\noindent
\<@TOPTYPEANNO> is the top annotation in the type qualifier hierarchy.  For
example, for the nullness type system, the top type annotation is
\<@Nullable>; as shown in Figure~\ref{fig-nullness-hierarchy}.
\<@DEFAULTANNO> is the default annotation for the type system.  For
example, for the nullness type system, the default type annotation is
\<@NonNull>.

In some type systems, the top qualifier and the default are the same.  For
such type systems, the two code snippets shown above are treated the same.
An example is the regular expression type system; see
Figure~\ref{fig-regex-hierarchy}.

The CLIMB-to-top rule reduces the code edits required to annotate an
existing program, and it treats types written in the program consistently.

When a user writes no upper bound, as in
\code{class C<T> \ttlcb\ ... \ttrcb}, 
then Java permits the class to be instantiated with any type parameter.
The Checker Framework behaves exactly the same, no matter what the default
is for a particular type system -- and no matter whether the user has
changed the default locally.

When a user writes an upper bound, as in 
\code{class C<T extends OtherClass> \ttlcb\ ... \ttrcb},
then the Checker Framework treats this occurrence of \<OtherClass> exactly
like any other occurrence, and applies the usual defaulting rules.  Use of
\<Object> is treated consistently with all other types in this location and
all other occurrences of \<Object> in the program.

It is uncommon for a user to write \<Object> as an upper bound with no type
qualifier:
\code{class C<T extends Object> \ttlcb\ ... \ttrcb}.
It is better style to write no upper bound or to write an explicit type
annotation on \<Object>.


\section{Creating a new checker\label{faq-create-a-checker-section}}

\subsection{How do I create a new checker?\label{faq-create-a-checker}}

In addition to using the checkers that are distributed with the Checker
Framework, you can write your own checker to check specific properties that
you care about.  Thus, you can find and prevent the bugs that are most
important to you.

Chapter~\ref{writing-a-checker} gives
complete details regarding how to write a checker.  It also suggests places
to look for more help, such as the \href{api/}{Checker Framework
API documentation (Javadoc)} and the source code of the distributed
checkers.

To whet your interest and demonstrate how easy it is to get started, here
is an example of a complete, useful type-checker.

\begin{Verbatim}
  @TypeQualifier
  @SubtypeOf(Unqualified.class)
  @Target({ElementType.TYPE_USE, ElementType.TYPE_PARAMETER})
  public @interface Encrypted { }
\end{Verbatim}

Section~\ref{subtyping-example} explains this checker and tells
you how to run it.


\subsection{Why is there no declarative syntax for writing type rules?\label{faq-declarative-syntax-for-type-rules}}

A type system implementer can declaratively specify the type qualifier
hierarchy (Section~\ref{declarative-hierarchy}) and the type introduction rules
(Section~\ref{declarative-type-introduction}).  However, the Checker
Framework uses a procedural syntax for specifying type-checking
rules (Section~\ref{extending-visitor}).
A declarative syntax might be more concise, more readable, and more
verifiable than a procedural syntax.

We have not found the procedural syntax to be the most important impediment
to writing a checker.

Previous attempts to devise a declarative syntax 
for realistic type systems have failed; see a technical
paper~\cite{PapiACPE2008} for a discussion.  When an
adequate syntax exists, then the Checker Framework can be extended to
support it.


\section{Relationship to other tools\label{faq-other-tools-section}}


\subsection{Why not just use a bug detector (like FindBugs)?\label{faq-type-checking-vs-bug-detectors}}

Pluggable type-checking finds more bugs than a bug detector does, for any
given variety of bug.

A bug detector like \href{http://findbugs.sourceforge.net/}{FindBugs}~\cite{HovemeyerP2004,HovemeyerSP2005},
\href{http://jlint.sourceforge.net/}{Jlint}~\cite{Artho2001}, or
\href{http://pmd.sourceforge.net/}{PMD}~\cite{Copeland2005} aims to find \emph{some}
of the most obvious bugs in your program.  It uses a lightweight analysis,
then uses heuristics to discard some of its warnings.  Thus, even if the tool
prints no warnings, your code might still have errors --- maybe the
analysis was too weak to find them, or the tool's heuristics classified the
warnings as likely false positives and discarded them.

A type-checker aims to find \emph{all} the bugs (of certain varieties).
It requires you to write type qualifiers in your program, or to use a tool
that infers types.  Thus, it requires more work from the programmer, and in
return it gives stronger guarantees.

Each tool is useful in different circumstances, depending on how important
your code is and your desired level of confidence in your code.  For more
details on the comparison, see Section~\ref{other-tools}.  For a case study
that compared the nullness analysis of FindBugs, Jlint, PMD, and the
Checker Framework, see section 6 of the paper
\href{http://homes.cs.washington.edu/~mernst/pubs/pluggable-checkers-issta2008.pdf}{``Practical pluggable types for Java''}~\cite{PapiACPE2008}.


\subsection{How does pluggable type-checking compare with JML?\label{faq-jml}}

\href{http://www.cs.ucf.edu/~leavens/JML/}{JML}, the Java Modeling
Language~\cite{LeavensBR2006:JML}, is a language for writing formal
specifications.

\textbf{JML aims to be more expressive than pluggable type-checking.}
A programmer can write a JML specification that
describes arbitrary facts about program behavior.  Then, the programmer can
use formal reasoning or a theorem-proving tool to verify that the code
meets the specification.  Run-time checking is also possible.
By contrast, pluggable type-checking can express a more limited set of
properties about your program.  Pluggable type-checking annotations are
more concise and easier to understand.

\textbf{JML is not as practical as pluggable type-checking.}
The JML toolset is less mature.  For instance, if your code uses
generics or other features of Java 5, then you cannot use JML.  
However, JML has a run-time checker, which the Checker Framework currently
lacks.


\subsection{Is the Checker Framework an official part of Java?\label{faq-checker-framework-part-of-java}}

The Checker Framework is not an official part of Java.
The Checker Framework relies on
type annotations, which are part of Java 8.  See the
\href{http://types.cs.washington.edu/jsr308/current/jsr308-faq.html#pluggable-type-checking-in-java}{Type
  Annotations (JSR 308) FAQ} for more details.


\subsection{What is the relationship between the Checker Framework and JSR 305?\label{faq-jsr-305}}

JSR 305 aimed to define official Java names for some annotations, such as
\<@NonNull> and \<@Nullable>.  However, it did not aim to precisely define
the semantics of those annotations nor to provide a reference
implementation of an annotation processor that validated their use.

By contrast, the Checker Framework precisely defines the meaning of a set
of annotations and provides powerful type-checkers that validate them.
However, the Checker Framework is not an official part of the Java
language; it chooses one set of names, but another tool might choose other
names.

JSR 305 has been abandoned; there has been
no activity by its expert group since
% January
2009.
In the future, the Java Community Process might standardize the names and
meanings of specific annotations, after there is more experience with their
use in practice.

The Checker Framework defines annotations \<@NonNull> and \<@Nullable> that
are compatible with annotations defined by JSR 305, FindBugs, IntelliJ, and
other tools; see Section~\ref{nullness-related-work}.


\subsection{What is the relationship between the Checker Framework and JSR 308?\label{faq-jsr-308}}

JSR 308, also known as the Type Annotations specification, dictates the
syntax of type annotations in Java SE 8:  how they are expressed in the
Java language.

JSR 308 does not define any type annotations such as \<@NonNull>, and it does
not specify the semantics of any annotations.  Those tasks are left to
third-party tools.  The Checker Framework is one such tool.

The Checker Framework makes use of Java SE 8's type annotation syntax, but
the Checker Framework can
be used with previous versions of the Java language via the
annotations-in-comments feature (Section~\ref{annotations-in-comments}).


% LocalWords:  IGJ toolset AbstractCollection ConcurrentHashMap NullnessUtils
% LocalWords:  castNonNull createWidget backporting JCIP's GuardedBy Awarns PMD
% LocalWords:  ElementType nullness bytecodes Jlint Hashtable SuppressWarnings
%  LocalWords:  RegexUtil compareTo myA requiresB nullable java myobject
%  LocalWords:  multithreading myfield Regex NonEmpty Xmaxerrs Xmaxwarns
%  LocalWords:  8's igj javari MonotonicNonNull EnsuresNonNull myFieldName
%  LocalWords:  m1 m2 RequiresNonNull AconcurrentSemantics 8's 8's MyQual
%  LocalWords:  MyClass MyMoreRestrictiveQual TOPTYPEANNO DEFAULTANNO api
%%  LocalWords:  OtherClass 8's
